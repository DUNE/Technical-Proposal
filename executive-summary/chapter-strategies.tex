%%%%%%%%%%%%%%%%%%%%%%%%%%%%%%%%%%%%%%%%%%%%%%%%%%%%%%%%%%%%%%%%%%%%
\section{Simulation and Reconstruction}
\label{sec:exec-summ-strat-simreco}

Liquid argon time projection chambers (LAr TPCs) provide a robust and elegant method for measuring the properties of neutrino interactions above a few tens of MeV by providing 3D event imaging with excellent spatial and energy resolution.  The state of the art in LAr TPC event reconstruction and particle identification is evolving rapidly and will continue to do so for many years.  The adoption of the common framework LArSoft by several LAr TPC experiments facilitates the exchange of tools and ideas.

The DUNE experimental design and physics program to be presented in the Technical Design Report (TDR) will be, in the main, based on a realistic end-to-end simulation and reconstruction chain.  This is in contrast to the highly parametrized methods used in the Conceptual Design Report (CDR).  Note that the science case summarized in Chapter~\ref{ch:exec-summ-physics} of this Technical Proposal is still based on CDR-era studies, as we intend to carry out the full refresh of the DUNE science case using our modern tools on the TDR timeline (2019).  The primary exception to this strategy are sensitivity studies for Beyond the Standard Model (BSM) physics, which will largely continue to use parametrized analyses with updated assumptions to reflect our latest understanding.  A full description of DUNE simulation and reconstruction tools will be included in the TDR.  In this section, we give a brief summary of the techniques now in use.

\subsection{Simulation Chain}
Simulated events are created in four stages: event generation, {\sc GEANT4} tracking, TPC and PDS signal simulation, and digitization.  The first step is unique to each sample type while the remaining steps are common for all samples. Beam neutrino, atmospheric neutrino, and nucleon decay events are generated using {\sc GENIE} appropriately configured for each.  Supernova events are generated using the new low-energy, argon-specific MARLEY generator~\cite{marley}.  Cosmogenic events at depth are generated using MUSIC (Muon Simulation Code) \cite{MUSICPaper} and MUSUN (Muon Simulations Underground) \cite{Kudryavtsev:musun}.

The truth particles generated in the event generator step are passed to a {\sc GEANT4}-based detector simulation.  Energy depositions are converted to ionization electrons and scintillation photons, with recombination, electron attenuation, and diffusion effects included.  The response of the photon detectors is simulated using a ``photon library'' that has precalculated the likelihood for propagation of photons from any point in the detector to any PDS element.  The response of the TPC induction and collection wires is based on a detailed GARFIELD~\cite{garfield} simulation.  Throughout, measurements from test stands or from operating LAr TPC experiments are used to establish simulation parameters.

The raw signals on each wire are converted into ADC versus time traces by convolution with the field response and electronics response.  ASIC electronics response is simulated with the BNL SPICE~\cite{spice} simulation.  The photon detector electronics simulation separately generates waveforms for each channel of a photon detector that has been hit by photons, with dark noise and line noise added.  The raw data are passed through hit finding algorithms that handle deconvolution and disambiguation to produce the basic data used by the downstream event reconstruction. PDS signals are reconstructed by searching for peaks on individual channels and then forming coincidences across channels. Techniques for matching the correct PDS signal to TPC signals to reconstruction $t_0$ are being developed, and early results from these tools can be seen in Chapter~\ref{PDS sim physics chapter}.

\subsection{Reconstruction and Event Identification}
Several approaches to LAr TPC reconstruction are under active development in DUNE and in the community at large.  En route to the TDR, efforts on all fronts have been supported.  The most traditional reconstruction path (``TrajCluster'' + ``Projection Matching'') forms 2D trajectory clusters in each detector view and then stitches these together into 3D objects.  Resulting objects are further characterized by, for instance, extracting $dE/dx$ information or comparing to EM shower profiles.  An alternative approach is provided by the Pandora system~\cite{Marshall:2015rfa}, in which the reconstruction and pattern recognition task is broken down into a large number of decoupled algorithms, where each algorithm addresses a specific task or targets a particular topology.  Two algorithms (``WireCell'' and ``SpacePointSolver'') take a different approach and create 3D maps of energy depositions directly by solving a constrained system of equations governed by the geometry of the TPC wires.  Finally, several analyses are using deep learning and convolutional neural networks with great success, as these techniques are well suited to the type of data produced by LAr TPCs.

Energy reconstruction is based on electron-lifetime-corrected calorimetry except in the case of muons where energy is determined from track range or (for uncontained muons) multiple Coulomb scattering.  Moving forward, more particle-specific energy estimators will be developed.

The output from all reconstruction algorithms is processed into standard ntuple files for use by analysis developers.  In the special case of long-baseline oscillation measurements, the CAFAna fitting toolkit developed originally for NOvA is used to combine far detector and near detector information, to assess the impact of systematic uncertainties, and to ultimately produce oscillation sensitivities.
