%%%%%%%%%%%%%%%%%%%%%%%%%%%%%
\subsection{Laser Systems}\label{sec:laser}
\fixme{electric field, E field or E-field? We have to decide for entire TP!}

The Calibration task force has considered multiple systems that use a laser to extract the electric field map.  They fall into two categories: photo-electron and direct ionization of the \dword{lar}, both driven by a \SI{266}{\nano\m} laser system.

The system considered to be the primary reference design 
%\fixme{ref design?} 
uses direct ionization and multiple laser paths, as it has the largest potential benefit as described in Section~\ref{sec:laser:motiv}. % the next section. 
An ionization-based system has been used in the ARGONTUBE~\cite{Zeller:2013sva}, \dword{microboone}, CAPTAIN and SBND experiments.
%{\it Add references to: JINST 4 (2009) P07011, New J.Phys. 12 (2010) 113024}

%Photoelectron-based system  and direct ionization system (MicroBooNE/CAPTAIN). Prioritization of SBND style laser.
%%%%%%%%%%%%%%%
\subsubsection{Motivation and Possible Measurements}
\label{sec:laser:motiv}
The main purpose of a laser calibration system is to provide a measurement of the electric field in the detector with high statistics. %There are m
Many useful secondary uses of laser %such as 
include alignment, stability monitoring, and diagnosing detector failures (e.g., \dword{hv}). As discussed in earlier sections, several detector parameters such as drift velocity and recombination which ultimately impact the spatial resolution and energy response of the detector have critical dependence on the electric field. 
%\fixme{the E field or the parameters ultimately impact the resolution?} the parameters; resolved -- SG
Approximately, a \num{1}\% distortion in the E field %will 
corresponds to a \num{0.25}\% effect on $dQ/dx$. E field distortions can occur globally or locally, and are caused by a variety of effects. Although the space charge due to ionization sources such as cosmic rays or ${}^{39}$Ar is expected to be small, fluid flow pattern make the effect significant in both SP and DP modules. Also, as discussed in Section~\ref{sec:DP}, this effect can get further amplified significantly in the \dword{dp} due to ion accumulation at the liquid-gas interface. 

Additionally, %there are several 
other sources in the detector (especially detector imperfections) %that 
can cause E-field distortions. For example, \dword{cpa} misalignment,  \dword{cpa} structural deformations, APA and CPA offsets and deviations from flatness %and offsets 
can create localized E-field distortions. % localized in space. 
Non-uniform resistivity in the voltage dividers that produce the E-field may create a net E-field distortion localized in space, and a failure of a \dword{fc} resistor will make the distortions time dependent. 
%\fixme{timing?} done -SG&KM
Each individual E-field distortion may add in quadrature with other effects, and can exceed a \numrange{1}{4}\,\% overall E-field distortion. Understanding all these effects require in-situ measurement of E-field for proper calibration. %Laser 
A laser system also has the intrinsic advantage of %not being impacted by 
being immune to recombination, thus eliminating particle-dependent effects.  

Assuming multiple, steerable laser entry points as discussed in Section~\ref{sec:FTs}, the ionization-based system can extract the electric field with fewer dependencies compared to other systems. An unambiguous field map requires crossing laser tracks in every relevant “voxel” of the detector. If two tracks that enter the same spatial voxel\footnote{Finer sampling in certain regions may be desirable, but \dword{daq} requirements prevent much finer sampling for overall E-field mapping.} ($10 \times 10 \times 10 \textrm{cm}^3$ volume) in the \dword{detmodule}, the relative position of the tracks provides an estimate of the local 3D E-field. %The ionization signal is not sensitive to recombination effects as well. \fixme{"either"?} 
Assuming a single, steerable laser track, the apparent curvature of the track can also be used to assess (more limited) information about  the electric field. 
%The laser track location placement precision is limited solely by the optics design and self-focusing effects can impact the practical range of the track (theoretically, the Rayleigh scattering length of 266~nm light is about 40~m). As discussed in Section~\ref{sec:FTs}, this is mitigated by spacing lasers about 15~m apart which is very close to the laser track range demonstrated by the MicroBooNE experiment. 
 
%\fixme{Clarify the extra degeneracy in this case, related to overall parameterization of TPC response model?}
%\fixme{Need to reference optics of system?}

Even if the laser is not intense enough to ionize the \dword{lar}, electrons may be liberated from material on the cathode, which provides many useful measurements. The total drift time can be  assessed from laser pulse to readout of charge on the anode. A photo-electron-based calibration system was used in the T2K gaseous (predominantly Ar), TPCs~\cite{Abgrall:2010hi}. Targets placed on the cathode provided dots and lines that were then imaged by the electronics, and relative distortions of the surveyed positions could be used. The T2K photo-electron system provided measurements of adjacent electronics modules' relative timing response, drift velocity with few \si{\nano\s} resolution of \SI{870}{\milli\m} drift distance, electronics gain, transverse diffusion, and an integrated measurement of the electric field along the drift direction. For DUNE, the system would be similarly used as on T2K to diagnose electronics or TPC response issues on demand, and provide an integral field measurement and relative distortions of $y$, $z$ positions with time, and of either $x$ or drift velocity. Ejection of photo-electrons from the direct ionization laser system has also been observed.

%%%%%%%%%%%%%%%
\subsubsection{Design Considerations}

% Mini workshop: https://indico.fnal.gov/event/14909/
For either system, a \SI{266}{\nano\m} laser would be mounted on the top of the cryostat, and service two adjacent feedthroughs. A steerable head and fiber interface would be mounted in the feedthrough, which is coated in a insulator. Two options are under investigation: (1) the \dword{fc} (but not the \dword{gp}) is penetrated, and (2) the \dword{fc} is not penetrated. In the former case, the \dword{fc} penetration has been shown to create a small distortion to the E-field, for the benefit of full volume E-field mapping. When the \dword{fc} is not penetrated, the laser shines through the \dword{fc} tubes, producing some regions that are not mappable by the laser. Unlike the ports that are inwards of the cryostat, the lasers through penetrations that are outside the \dword{fc} on the far east and west side of the cryostat will not penetrate the field cage. The photo-electron system would include a fiber and no steering; the necessity of penetrating the \dword{fc} is unlikely but has not been assessed yet.

The current feedthrough penetrations are spaced at a \SI{15}{\m}, a plausible distance for the laser beam to travel; the maximum distance light would travel would be to the bottom corner of the detector, approximately \SI{20}{\m}. Direct-ionization tracks have been demonstrated at a maximum possible distance in \microboone of \SI{10}{\m}. While the Rayleigh scattering of the laser light is about \SI{40}{\m}, additional optics effects, including self-focusing (Kerr) effects may limit the maximum practical range.
%DocDB: 4769
%At this point in time a maximum usable track length is unknown. ''

%On SBND and MicroBooNE, the spatial footprint of the laser system is modest.
%%%%%%%%%%%%%%%
\subsubsection{Remaining Studies}

The remaining studies for the laser systems to be done prior to the TDR are:

\begin{itemize}
\item Determine %What would be 
a nominal design for photoelectric targets on the cathode, and whether %would 
such targets would provide sufficient survey-like information.
\item Determine if the known classes of possible E-field distortions require penetration of the \dword{fc} (versus reduced sampling from shining between the field cage). 
\item Further understand limitations on laser location precision, practical range of propagation due to optics design and Rayleigh scattering.
\item Continue to quantify the range of possible E-field distortions in the DUNE FD to further refine the estimation of overall variation of E-field (both locally and globally) in the detector.
\end{itemize}
%% 

