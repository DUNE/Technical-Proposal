
\section{Calibration Summary}
%- Kendall and Sowjanya; 1.5 pages}
\label{sec:calibsum}
The calibration needs for the DUNE \dwords{detmodule} are presented along with a strategy to meet them. An overview of existing sources and limitations in using those sources are also discussed. A set of proposed external calibration systems needed to meet DUNE's physics goals are presented. As listed, the calibration group 
%\fixme{task force? (and other instances in this section)}
will continue to pursue remaining studies with an aim to finalize the design considerations for calibration systems. Feedthrough accommodations are already made for the \dword{spmod} to accommodate the proposed UV laser and radioactive source systems. The CRT and the neutron generator systems require more understanding in terms of the space constraints around the cryostat. More studies are also needed for the CRT system to finalize the size and placement of the system. In addition to the proposed systems, %there are 
other systems are under %being 
consideration at an exploratory level. The need for these systems will be critically reviewed in the immediate future with a goal of finalizing the list of proposed systems in the near future. In the coming months, the calibration group will aim to finalize calibration penetrations for the \dword{dpmod}. 

The calibration group has also started working closely with the consortia %by identifying 
and identified liaisons for each. %group 
It is important to ensure that calibration needs are considered as each consortium develops its %their 
designs. A preliminary list of DAQ %needs for 
calibration requirements are already in process and %being identified and these needs 
will be finalized in the coming months. Additionally, more physics-driven studies are %also 
being launched to understand the impact of calibrations on \lbl physics. As a first step, the calibration group has started collaborating with the \lbl physics group to develop the necessary tools and techniques to propagate detector physics effects into oscillation analyses. Calibration tools developed in other experiments such as \microboone will also be adopted for DUNE in the near future. The calibration group will also aim to increase dual phase participation to better understand calibration challenges for the \dword{dp} design.  

\subsection{Path to the TDR}
\label{sec:TDR}
The calibration systems for DUNE, as presented in this document, will be further discussed and developed for the TDR within DUNE's management %the consortium 
structure. Two options are currently favored for %the consortium structure for 
calibration, (1) formation of a new Calibration consortium, or (2) inclusion of calibration in the Cryogenic Instrumentation and Slow Controls (CISC) consortium. This decision will depend on the scope of the proposed calibration systems presented in this document. The goal is to make and execute this decision in June 2018, shortly after the final Technical Proposal submission. % and move calibrations into a consortium. 
The full design development for each calibration system, along with costing and risk mitigation, will follow. %be pursued under the future calibration consortium. 
For the TDR, the overall detector calibration strategy for the \dword{sp} design (low-level calibration systems and physics-based) will be presented in a \textit{Calibration Strategy} chapter of Volume 3 of the TDR. Details of the hardware will be presented in the corresponding consortium volume of the TDR. A similar structure is envisioned for the \dword{dpmod}. %There will also be a section in t
The Physics Volume of the TDR will contain a section discussing the physics-process based calibration measurements and the assumed systematic uncertainties that will be propagated to the physics sensitivities. Table~\ref{tab:TDRsteps} shows some high-level key milestones for calibration leading to the TDR.

\begin{dunetable}[Key Calibration milestones leading to TDR]{ll}{tab:TDRsteps}{Key Calibration milestones leading to TDR.}
Date & \textbf{Milestone}\\ \toprowrule
Aug. 2017& Formation of Calibration Task Force\\ \colhline
Dec. 2017& Finalize calibration penetrations for \dword{spmod}\\ \colhline
Jan. 2018 & First proposal of calibration systems to the collaboration \\ \colhline
Feb. 2018 & Address calibration key questions and concerns\\ \colhline
Mar. 2018 & Calibration workshop -- agree on proposed systems\\ \colhline
May 2018 & Technical Proposal\\ \colhline
June 2018 & Finalize process of integrating calibration into Consortium structure\\ \colhline
%Aug. 2018 & Finalize calibration penetrations for \dword{dpmod}?\\ \colhline (SG: remove for now until we figure the actual deadline from Marzio)
Nov. 2018 & Internal Review of Final Calibration System Design\\ \colhline
Dec. 2018 & Design validation using \dword{protodune} and SBN data  \\
&(where applicable) and incorporate lessons learned into designs \\ \colhline
Feb. 2019 & Documentation of Final Design of all Calibration Systems\\ \colhline
Apr. 2019 & Technical Design Report\\ \colhline
Oct. 2019 & CD-2 DOE Review\\
\end{dunetable}    