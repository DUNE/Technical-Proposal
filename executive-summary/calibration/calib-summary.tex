\section{Summary} % 1.5 pages
\label{sec:calibsum}

The physics requirements for the broad DUNE physics program places stringent requirements on calibration systems and sources. The aim of the upcoming TDR will be to demonstrate that the proposed calibration systems, in conjunction with existing sources, external measurements and monitors, will be sufficient for DUNE's calibration requirements. The proposed systems discussed here have been identified as important to DUNE's physics program; the use of multi-purpose ports will enable  in principle other calibration proposals not described here to participate in DUNE in the future. %Inclusion of such systems will require a strong physics case, however, and additional systems not discussed here were viewed in the calibration group as either not essential or exploratory at time of writing.

The calibration group has started establishing relevant connections to physics working groups and consortia as appropriate. The calibration group has started collaborating with the \lbl physics group to develop the necessary tools and techniques to propagate detector physics effects into oscillation analyses and similar effort will occur to connect with other physics groups. The calibration group has also started working closely with the consortia and identified liaisons for each 
to ensure that calibration needs are strongly considered as each consortium develops its designs. For example, a preliminary list of DAQ calibration requirements are already in process and  will be finalized in the coming months.


%\fixme{Make a single punchy paragraph for now. Relationship to other consortia and other groups. We need to have a better connection with LBL group. But, the bulk of the work is refining design of the proposed systems. }

%\fixme{Are the core remaining studies summarized and where?}

%The calibration needs for the DUNE \dwords{detmodule} are presented along with a strategy to meet them. An overview of existing sources and limitations in using those sources are also discussed. A set of proposed external calibration systems needed to meet DUNE's physics goals are presented. As listed, the calibration task force will continue to pursue remaining studies with an aim to finalize the design considerations for calibration systems. Feedthrough accommodations are already made for the \dword{spmod} to accommodate the proposed UV laser and radioactive source systems. 

%The CRT and the neutron generator systems require more understanding in terms of the space constraints around the cryostat. More studies are also needed for the CRT system to finalize the size and placement of the system.
%In addition to the proposed systems,  other systems are under consideration at an exploratory level. The need for these systems will be critically reviewed in the immediate future with a goal of finalizing the list of proposed systems in the near future. In the coming months, the calibration task force will aim to finalize calibration penetrations for the \dword{dpmod}. 

%\fixme{Not much explicitly said on DP. } %In the case of \dword{dp}, one of the biggest challenges is the \SI{12}{\m} long single drift path and ion accumulation at the liquid-gas interface. 

\subsection{Path to the TDR}
\label{sec:TDR}
The calibration systems for DUNE, as presented in this document, will be further discussed and developed for the TDR within DUNE's management %the consortium 
structure. Two options are currently favored for %the consortium structure for 
calibration, (1) formation of a new calibration consortium, or (2) inclusion of calibration in the \dword{cisc} consortium. This decision will depend on the scope of the proposed calibration systems presented in this document. The goal is to make and execute this decision in June 2018, shortly after the final \dword{tp} submission. % and move calibrations into a consortium. 
The full design development for each calibration system, along with costing and risk mitigation, will follow. %be pursued under the future calibration consortium. 


\begin{dunetable}[Key calibration milestones leading to first detector installation]{ll}{tab:TDRsteps}{Key calibration milestones leading to first detector installation.}
Date & \textbf{Milestone}\\ \toprowrule
May 2018 & \dword{tp} \\ \colhline
June 2018 & Finalize process of integrating calibration into consortium structure\\ \colhline
Jan. 2019 & Design validation of calibration systems using \dword{protodune}\slash SBN data  \\
&(where applicable) and incorporate lessons learned into designs \\ \colhline
Apr. 2019 & \dword{tdr} \\ \colhline
Sep. 2022 & Finish construction of calibration systems for Cryostat \#1 \\ \colhline
May 2023 & Cryostat 1 ready for TPC installation \\ \colhline
Oct. 2023 & All calibration systems installed in Cryostat \#1 \\
\end{dunetable} 

%\begin{dunetable}[Key Calibration milestones leading to TDR]{ll}{tab:TDRsteps}{Key Calibration milestones leading to TDR.}
%Date & \textbf{Milestone}\\ \toprowrule
%Aug. 2017& Formation of Calibration Task Force\\ \colhline
%Dec. 2017& Finalize calibration penetrations for \dword{spmod}\\ \colhline
%Jan. 2018 & First proposal of calibration systems to the collaboration \\ \colhline
%Feb. 2018 & Address calibration key questions and concerns\\ \colhline
%Mar. 2018 & Calibration workshop -- agree on proposed systems\\ \colhline
%May 2018 & Technical Proposal\\ \colhline
%June 2018 & Finalize process of integrating calibration into Consortium structure\\ \colhline
%Aug. 2018 & Finalize calibration penetrations for \dword{dpmod}?\\ \colhline (SG: remove for now until we figure the actual deadline from Marzio)
%Nov. 2018 & Internal Review of Final Calibration System Design\\ \colhline
%Dec. 2018 & Design validation using \dword{protodune} and SBN data  \\
%&(where applicable) and incorporate lessons learned into designs \\ \colhline
%Feb. 2019 & Documentation of Final Design of all Calibration Systems\\ \colhline
%Apr. 2019 & Technical Design Report\\ \colhline
%Oct. 2019 & CD-2 DOE Review\\
%\end{dunetable} 

