\chapter{Summary}
\label{ch:project-summary}

LBNF/DUNE will be a world-leading facility for pursuing a cutting-edge program of neutrino physics and astroparticle physics. The 
combination of the intense wide-band neutrino beam, the massive LArTPC far detector and the highly capable near detector will provide 
the opportunity to discover CP violation in the neutrino sector as well as to determine the neutrino mass ordering and provide a 
precision test of the three-flavor oscillation paradigm. The massive, deep-underground far detector will offer unprecedented sensitivity for  
theoretically favored proton decay modes 
and for observation of electron neutrinos from a core-collapse supernova, should one occur in our galaxy during the operation of the experiment.

In addition to summarizing the compelling scientific case for LBNF/DUNE, this document presents an overview of the technical 
designs of the facility and experiment and the strategy for their implementation. 
This strategy delivers the science goals described in the 2014 report of the Particle Physics Project Prioritisation Panel (P5) on a 
competitive timescale. Furthermore, a detailed management plan for the 
organization of LBNF as a U.S.-hosted facility and the DUNE experiment 
as a broad international scientific collaboration has been developed, thus satisfying the goal of internationalizing the project as highlighted in the 
P5 report.

