\documentclass{article}
\usepackage[letterpaper,margin=2cm]{geometry}

\usepackage{xspace}
\usepackage{siunitx}
% This holds definitions of macros to enforce consistency in units.

% This file is the sole location for such definitions.  Check here to
% learn what there is and add new ones only here.  

% also see defs.tex for names.


% see
%  http://ctan.org/pkg/siunitx
%  http://mirrors.ctan.org/macros/latex/contrib/siunitx/siunitx.pdf

% Examples:
%  % angles
%  \ang{1.5} off-axis
%
%  % just a unit
%  \si{\kilo\tonne}
%
%  % with a value:
%  \SI{10}{\mega\electronvolt}

%  range of values:
% \SIrange{60}{120}{\GeV}

% some shorthand notation
\DeclareSIUnit \kton {\kilo\tonne}
\DeclareSIUnit \kt {\kilo\tonne}
\DeclareSIUnit \Mt {\mega\tonne}
\DeclareSIUnit \eV {\electronvolt}
\DeclareSIUnit \keV {\kilo\electronvolt}
\DeclareSIUnit \MeV {\mega\electronvolt}
\DeclareSIUnit \GeV {\giga\electronvolt}
\DeclareSIUnit \km {\kilo\meter}
\DeclareSIUnit \kW {\kilo\watt}
\DeclareSIUnit \MW {\mega\watt}
\DeclareSIUnit \MHz {\mega\hertz}
\DeclareSIUnit \mrad {\milli\radian}
\DeclareSIUnit \year {year}
\DeclareSIUnit \POT {POT}
\DeclareSIUnit \sig {$\sigma$}
\DeclareSIUnit\parsec{pc}
\DeclareSIUnit\lightyear{ly}
\DeclareSIUnit\foot{ft}
\DeclareSIUnit\ft{ft}
% for a bare kt-year
\def\ktyr{\si[inter-unit-product=\ensuremath{{}\cdot{}}]{\kt\year}\xspace}
\def\Mtyr{\si[inter-unit-product=\ensuremath{{}\cdot{}}]{\Mt\year}\xspace}
\def\msr{\si[inter-unit-product=\ensuremath{{}\cdot{}}]{\meter\steradian}\xspace}
\def\ktMWyr{\si[inter-unit-product=\ensuremath{{}\cdot{}}]{\kt\MW\year}\xspace}

% used for hyphen, obsolete now: \newcommand{\SIadj}[2]{\SI[number-unit-product = -]{#1}{#2}}
% change command definition Nov 2017 in case people copy e.g., \ktadj from CDR text.
% E.g., \ktadj{10} now renders the same as \SI{10}{\kt}
\newcommand{\SIadj}[2]{\SI{#1}{#2}}

% Adjective form of some common units (Nov 2107 changed to be same as normal form, no hyphen)
% "the 10-kt detector"

\newcommand{\ktadj}[1]{\SIadj{#1}{\kt}}
% "the 1,300-km baseline"
\newcommand{\kmadj}[1]{\SIadj{#1}{\km}}
% "a 567-keV endpoint"
\newcommand{\keVadj}[1]{\SIadj{#1}{\keV}}
% "Typical 20-MeV event"
\newcommand{\MeVadj}[1]{\SIadj{#1}{\MeV}}
% "Typical 2-GeV event"
\newcommand{\GeVadj}[1]{\SIadj{#1}{\GeV}}
% "the 1.2-MW beam"
\newcommand{\MWadj}[1]{\SIadj{#1}{\MW}}
% "the 700-kW beam"
\newcommand{\kWadj}[1]{\SIadj{#1}{\kW}}
% "the 100-tonne beam"
\newcommand{\tonneadj}[1]{\SIadj{#1}{\tonne}}
% "the 4,850-foot depth beam"
\newcommand{\ftadj}[1]{\SIadj{#1}{\ft}}
%

% Mass exposure, people like to put dots between the units
% \newcommand{\ktyr}[1]{\SI[inter-unit-product=\ensuremath{{}\cdot{}}]{#1}{\kt\year}}
% must make usage of \ktyr above consistent with this one before turning on

% Beam x mass exposure, people like to put dots between the units
\newcommand{\ktmwyr}[1]{\SI[inter-unit-product=\ensuremath{{}\cdot{}}]{#1}{\kt\MW\year}}


\usepackage[hidelinks]{hyperref}

% This holds definitions of macros to enforce consistency in names.

% This file is the sole location for such definitions.  Check here to
% learn what there is and add new ones only here.  

% also see units.tex for units.  Units can be used here.

%%% Common terms

% Check here first, don't reinvent existing ones, add any novel ones.
% Use \xspace.

%%%%% Anne adding macros for referencing TDR volumes and annexes Apr 20, 2015 %%%%%
\def\expshort{DUNE\xspace}
\def\explong{The Deep Underground Neutrino Experiment\xspace}

%\def\thedocsubtitle{LBNF/DUNE Technical Design Report (DRAFT)}
\def\thedocsubtitle{Deep Underground Neutrino Experiment (DUNE)} 
\def\tdrtitle{Technical Proposal}
\def\thevolumenumber{0}

% For the document titles (not italicized)
\def\voltitleexecsumm{Volume 1: Executive Summary\xspace}
\def\voltitlespfd{Volume 2: The Single-Phase Far Detector\xspace}
\def\voltitledpfd{Volume 3: The Dual-Phase Far Detector\xspace}
%\def\voltitleswcomputing{Volume 4: Software and Computing\xspace}



% For use within volumes (italicized)
\def\volexecsumm{\textbf{Executive Summary\xspace}}

\def\volspfd{\textbf{The Single-Phase Far Detector Module Design\xspace}  
}

\def\voldpfd{\textbf{The Dual-Phase Far Detector Module Design\xspace} }

\def\volswcomputing{\textbf{Software and Computing\xspace}}

% Things about oscillation
%
\newcommand{\numu}{$\nu_\mu$\xspace}
\newcommand{\nue}{$\nu_e$\xspace}
\newcommand{\nutau}{$\nu_\tau$\xspace}

\newcommand{\anumu}{$\bar\nu_\mu$\xspace}
\newcommand{\anue}{$\bar\nu_e$\xspace}
\newcommand{\anutau}{$\bar\nu_\tau$\xspace}

\newcommand{\dm}[1]{$\Delta m^2_{#1}$\xspace} % example: \dm{12}

\newcommand{\sinst}[1]{$\sin^2\theta_{#1}$\xspace} % example \sinst{12}
\newcommand{\sinstt}[1]{$\sin^22\theta_{#1}$\xspace}  % example \sinstt{12}

\newcommand{\deltacp}{$\delta_{\rm CP}$\xspace}   % example \deltacp
\newcommand{\mdeltacp}{$\delta_{\rm CP}$}   %%%%%%%%%%  <--- missing something; what's the m for?

\newcommand{\nuxtonux}[2]{$\nu_{#1} \to \nu_{#2}$\xspace}  % example \nuxtonux23 (no {...} )
\newcommand{\numutonumu}{\nuxtonux{\mu}{\mu}}
\newcommand{\numutonue}{\nuxtonux{\mu}{e}}
% Add chi sqd MH?  avg delta chi sqd?

% atmospheric neutrinos and PDK
\newcommand{\ptoknubar}{$p \rightarrow K^+ \overline{\nu}$\xspace}
\newcommand{\ptoepizero}{$p^+ \rightarrow e^+ \pi^0$\xspace}

% Names of expts or detectors
\newcommand{\cherenkov}{Cherenkov\xspace}
\newcommand{\kamland}{KamLAND\xspace}
\newcommand{\kkande}{Kamiokande\xspace}  %%%% <---- changed to make shorter
\newcommand{\superk}{Super--Kamiokande\xspace}
\newcommand{\hyperk}{Hyper--Kamiokande\xspace}
\newcommand{\miniboone}{MiniBooNE\xspace}
\newcommand{\microboone}{MicroBooNE\xspace}
\newcommand{\minerva}{MINER$\nu$A\xspace}
\newcommand{\nova}{NO$\nu$A\xspace}
\newcommand{\numi}{NuMI\xspace}
\newcommand{\lariat}{LArIAT\xspace}
\newcommand{\argoneut}{ArgoNeuT\xspace}

% Random
\newcommand{\lartpc}{LArTPC\xspace}
\newcommand{\globes}{GLoBES\xspace}
\newcommand{\larsoft}{LArSoft\xspace}
\newcommand{\snowglobes}{SNOwGLoBES\xspace}
\newcommand{\docdb}{DUNE DocDB\xspace}
% Isotopes
\def\argon40{$^{40}$Ar}  %%%%%       <--- \Ar40 doesn't work; don't know why
\def\Ar39{$^{39}$Ar}
\def\Cl40{$^{40}$Cl}
\def\K40{$^{40}$K}
\def\B8{$^{8}$B}
\newcommand\isotope[2]{\textsuperscript{#2}#1} % use as, e.g.,: \isotope{Si}{28}

% Parameters common to SP DP
\def\fdfiducialmass{\SI{40}{\kt}\xspace}
\def\driftvelocity{\SI{1.6}{\milli\meter/\micro\second}\xspace} % same for sp and dp?
\def\lartemp{\SI{88}\,K\xspace}
\def\larmass{\SI{17.5}{\kt}\xspace} % full mass in cryostat
\def\tpcheight{\SI{12}{\meter}\xspace} % height of SP TPC, APA, CPA and of DP TPC
\def\cryostatht{\SI{14.1}{\meter}\xspace} % height of cryostat
\def\cryostatlen{\SI{62.0}{\meter}\xspace} % height of cryostat
\def\cryostatwdth{\SI{14.0}{\meter}\xspace} % height of cryostat
\def\nominalmodsize{\SI{10}{kt}\xspace} % nominal module size 10 kt

\def\dunelifetime{\SI{20}{year}\xspace} % nominal operational life time of DUNE experiment
\def\beamturnon{{2026}\xspace} % the year we expect beam to turn on
\def\firstfdmodstartinstall{{2022}\xspace} % the year we expect to start install of 1st FD moc
\def\maincavernstartexc{{2019}\xspace} % the year we expect to start cavern excavation
\def\pipiibeampower{\SI{1.2}{MW}\xspace} 


% Parameters SP
\def\spmaxfield{\SI{500}{\volt/\centi\meter}} % SPfield strength
\def\spactivelarmass{\SI{10}{\kt}\xspace} % active mass in cryostat
\def\spmaxdrift{\SI{3.53}{\m}\xspace}
\def\sptpclen{\SI{58}{\meter}\xspace} % length of SP TPC, APA, CPA
\def\apacpapitch{\SI{2.32}{\meter}\xspace} % pitch of SP CPAs and APAs
\def\spfcmodlen{\SI{3.5}{\m}} % length of SP FC module
\def\spnumch{\num{384000}\xspace} % total number of APA readout channels 
\def\spnumpdch{\num{6000}\xspace} % total number of PD readout channels 
\def\uvpitch{\SI{4.669}{\milli\meter}\xspace}
\def\xgpitch{\SI{4.790}{\milli\meter}\xspace}
\def\planespace{\SI{4.75}{\milli\meter}\xspace}
\def\sptargetdriftvoltpos{\SI{180}{kV}\xspace} % target drift voltage - positive

% Parameters DP
\def\dpactivelarmass{\SI{12.096}{\kt}\xspace} % active mass in cryostat
\def\dpfidlarmass{\SI{10.643}{\kt}\xspace} % fiducial mass in cryostat
\def\dpmaxdrift{\SI{12}{\m}\xspace} % max drift length
\def\dptpclen{\SI{60}{\meter}\xspace} % length of TPC
\def\dptpcwdth{\SI{12}{\meter}\xspace} % width of TPC
\def\dpswchpercrp{\num{36}\xspace} % number of anode/lem sandwiches per CRP 
\def\dpnumswch{\num{2880}\xspace} % total number of anode sandwiches in module
\def\dptotcrp{\num{80}\xspace} % total number of CRPs in module
\def\dpchpercrp{\num{1920}\xspace} %  channels per CRP
\def\dpnumcrpch{\num{153600}\xspace} % total number of CRP channels in module
\def\dpchperchimney{\num{6400}\xspace} %  channels per chimney
\def\dpnumpmtch{\num{720}\xspace} % number of PMT channels
\def\dpstrippitch{\SI{3.125}{\milli\meter}\xspace} % pitch of anode strips
\def\dpnumfcmod{\num{244}\xspace} % number of FC modules
\def\dpnumfcres{\num{240}\xspace} % number of FC resistors
\def\dpnumfcrings{\num{60}\xspace} % number of FC rings
\def\dpnominaldriftfield{\SI{500}{V/cm}\xspace} % nominal drift voltage per cm
\def\dptargetdriftvoltpos{\SI{600}{kV}\xspace} % target drift voltage - positive
\def\dptargetdriftvoltneg{\SI{-600}{kV}\xspace} % target drift voltage - negative

% Nominal readout window time
%% SP has 2.25ms drift time.  The readout is 2*dt + 20%*dt extra.
\def\spreadout{\SI{5.4}{\ms}\xspace}
%% DP has 7.5 ms drift time.  The same (over generous) rule gives 16.5ms
\def\dpreadout{\SI{16.4}{\ms}\xspace}
% Supernova Neutrino Burst buffer and readout window time
\def\snbtime{\SI{30}{\s}\xspace}
% interesting amount of time we might have SNB neutrinos but not yet
% enough to trigger.
\def\snbpretime{\SI{10}{\s}\xspace}
% SP SNB dump size
\def\spsnbsize{\SI{45}{\PB}\xspace}

% keep these three numerically in sync
\def\offsitepbpy{\SI{30}{\PB/\year}\xspace}
\def\offsitegbyteps{\SI{1}{\GB/\s}\xspace}
\def\offsitegbps{\SI{8}{\Gbps}\xspace}

\def\surffnalbw{\SI{100}{\Gbps}\xspace}
\newcommand{\fnal}{Fermilab\xspace}
\newcommand{\surf}{SURF\xspace}
\newcommand{\bnl}{BNL\xspace}
\newcommand{\anl}{ANL\xspace}

% New from Anne March/April 2018
%physics terms
\newcommand{\efield}{E field\xspace}
\newcommand{\lbl}{long-baseline\xspace}
\newcommand{\Lbl}{Long-baseline\xspace}
\newcommand{\rms}{RMS\xspace} % Might want this small caps?
\newcommand{\threed}{3D\xspace}
\newcommand{\twod}{2D\xspace}

%detectors and modules
\newcommand{\fardet}{Far Detector\xspace}
\newcommand{\detmodule}{detector module\xspace}
\newcommand{\dual}{DP\xspace}
\newcommand{\Dual}{DP\xspace}
\newcommand{\single}{SP\xspace}
\newcommand{\Single}{SP\xspace}
\newcommand{\dpmod}{DP detector module\xspace}
\newcommand{\spmod}{SP detector module\xspace}

\newcommand{\lar}{LAr\xspace}

%detector components SP and DP
\newcommand{\dss}{DSS\xspace}
\newcommand{\hv}{high voltage\xspace}
\newcommand{\fcage}{field cage\xspace}
\newcommand{\fc}{FC\xspace}
\newcommand{\fdth}{feedthrough\xspace}
\newcommand{\fcmod}{FC module\xspace}  %%%   eon't need?
\newcommand{\topfc}{top FC\xspace}
\newcommand{\botfc}{bottom FC\xspace}
\newcommand{\ewfc}{endwall FC\xspace}
\newcommand{\pdsys}{PD system\xspace}
\newcommand{\phdet}{photon detector\xspace}
\newcommand{\sipm}{SiPM\xspace}
\newcommand{\pmt}{PMT\xspace}
\newcommand{\phel}{photoelectron\xspace}
\newcommand{\pwrsupp}{power supply\xspace}
\newcommand{\pwrsupps}{power supplies\xspace}

%detector components SP only

%detector components DP only

%%%%%%     START ADDING WORDS BELOW LINE 50    %%%%%%%%

% see dune-words.tex for explanation.

\usepackage[acronyms,toc]{glossaries}
\makeglossaries


\newcommand{\dshort}[1]{\acrshort{abbrev-#1}}
\newcommand{\dlong}[1]{\acrlong{abbrev-#1}}

% force the "first time" behavior
\newcommand{\dfirst}[1]{\glsfirst{abbrev-#1}}

\newcommand{\dword}[1]{\gls{#1}}
\newcommand{\dwords}[1]{\glspl{#1}}
\newcommand{\Dword}[1]{\Gls{#1}}
\newcommand{\Dwords}[1]{\Glspl{#1}}
%\newcommand{\dwordies}[1]{\glspl{#4}} %Anne -- need to do more research on this!!


% \newduneword{label}{term}{description}
\newcommand{\newduneword}[3]{
    \newglossaryentry{#1}{
        text={#2},
        long={#2},
        name={\glsentrylong{#1}},
        first={\glsentryname{#1}},
        firstplural={\glsentrylong{#1}\glspluralsuffix},
        description={#3}
    }
}

%                 1      2     3       4
% \newduneabbrev{label}{abbrev}{term}{description}
\newcommand{\newduneabbrev}[4]{
  % this makes acronym entries even if they neither term or abbrev is
  % referenced in the text.
  % \newacronym[see={[Glossary:]{#1}}]{abbrev-#1}{#2}{#3}
  \newacronym{abbrev-#1}{#2}{#3}
  \newglossaryentry{#1}{
    text={#2},
    long={#3},
    name={\glsentrylong{#1}{} (\glsentrytext{#1}{})},
    first={\glsentryname{#1}},
    firstplural={\glsentrylong{#1}\glspluralsuffix{} (\glsentrytext{#1}\glspluralsuffix{})},
    description={#4}
  }
}

%  (Anne trying)    1      2     3       4       5
% \newduneabbrev2{label}{abbrev}{term}{termpl}{description}
\newcommand{\newduneabries}[5]{
  % this makes acronym entries even if they neither term or abbrev is
  % referenced in the text.
  % \newacronym[see={[Glossary:]{#1}}]{abbrev-#1}{#2}{#3}
  \newacronym{abbrev-#1}{#2}{#3}{#4}
  \newglossaryentry{#1}{
    text={#2},
    long={#3},
    name={\glsentrylong{#1}{} (\glsentrytext{#1}{})},
    first={\glsentryname{#1}},
    firstplural={\glsentrylong{#1}\glspluralsuffix{} (\glsentrytext{#1}\glspluralsuffix{#4})},
    description={#5}
  }
}


%%%%%%     START ADDING WORDS, IN ALPHABETICAL ORDER IF POSSIBLE!    %%%%%%%%

% Model to copy

%\newduneword{abc}{ABC's real name}{Sentence or two describing ABC. It can use other dunewords defined here, e.g., \gls{fe}.}

%\newduneabries{abc}{ABC}{annes beecy}{annes beecies}{a made-up thing}

\newduneabbrev{nd}{ND}{near detector}{Refers to the detectors or more
  generally the experimental site at Fermilab}

\newduneabbrev{fd}{FD}{far detector}{Refers to the detector or more
  generally the experimental site in or above the Homestake mine in
  Lead, SD}

\newduneabbrev{sp}{SP}{single-phase}{Distinguishes one of the four
  \SI{10}{\kton} \glspl{detmodule} of the DUNE far detector by the
  fact that it operates using argon in just its liquid phase.}


\newduneabbrev{dp}{DP}{dual-phase}{Distinguishes one of the four
  \SI{10}{\kton} \glspl{detmodule} of the DUNE far detector by the
  fact that it operates using argon in both gas and liquid phases.}

\newduneabbrev{pds}{PDS}{photon detection system}{The \gls{submodule}
  system sensitive to light produced in the LAr}

\newduneabbrev{tpc}{TPC}{time projection chamber}{The portion of the
  various DUNE \glspl{submodule} which record ionization electrons
  after they drift away from a cathode, through LAr and potentially
  through GAr as well. 
  The activity is recorded by digitizing the waveforms of current
  induced on anode as the distribution of ionization charge passes by
  or is collected on the electrode}

\newduneabbrev{apa}{APA}{anode plane assembly}{One unit the SP
  detector containing the elements sensitive to activity in the LAr. 
  It contains two faces each of three planes of wires, cold
  electronics and photo detection system.} 

\newduneabbrev{cro}{CRO}{charge readout}{The system for detecting
  ionization charge distributions in the DP detector module}

\newduneabbrev{lro}{LRO}{light readout}{The system for detecting
  scintillation photons in the DP detector module}

% front-end
\newduneabbrev{fe}{FE}{front-end}{The front-end refers a point which is
  ``upstream'' of the data flow for a particular subsystem. 
  For example the front-end electronics is where the cold electronics
  meet the sense wires of the TPC and the front-end DAQ is where the
  DAQ meets the output of the electronics}

% analog digital converter
\newduneabbrev{adc}{ADC}{analog digital converter}{A sampling of a voltage
  resulting in a discrete integer count corresponding in some way to
  the input}

% data aquisition
\newduneabbrev{daq}{DAQ}{data aquisition}{The data acquisition system
  accepts data from the detector FE electronics, buffers
  it, performs a \gls{trigdecision}, builds events from the selected
  data and delivers the result to the offline \gls{diskbuffer}}

% detector module
\newduneword{detmodule}{detector module}{The entire DUNE far detector is
  segmented into four modules, each with a nominal \SI{10}{\kton}
  fiducial mass}

% detector unit
\newduneword{detunit}{detector unit}{A \gls{submodule} may be partitioned
  into a number of similar parts. 
  For example the single-phase TPC \gls{submodule} is made up of APA
  units}

% secondary DAQ buffer
\newduneword{diskbuffer}{secondary DAQ buffer}{A secondary
  \dshort{daq} buffer holds a small subset of the full rate as
  selected by a \gls{trigcommand}. 
  This buffer also marks the interface with the DUNE Offline}

% data quality monitoring
\newduneabbrev{dqm}{DQM}{data quality monitoring}{Analysis of the raw
  data to monitor the integrity of the data and the performance of the
  detectors and their electronics. This type of monitoring may be
  performed in real time, within the \gls{daq} system, or in later
  stages of processing, using disk files as input}

% DAQ dump buffer
\newduneword{dumpbuffer}{DAQ dump buffer}{This \dshort{daq} buffer
  accepts a high-rate data stream, in aggregate, from an associated
  \gls{submodule} sufficient to collect all data likely relevant to
  a potential Supernova Burst.}


% Global Trigger Logic
\newduneabbrev{gtl}{ETL}{External Trigger Logic}{Trigger processing
  which consumes \gls{detmodule} level \gls{trignote} information
  and other global sources of trigger input and emits
  \gls{trigcommand} information back to the \glspl{mtl}}

\newduneword{trignote}{trigger notification}{Information provided by
  \gls{mtl} to \gls{gtl} about \gls{trigdecision} its processing}

% trigger primitive
\newduneword{trigprimitive}{trigger primitive}{Information derived by
  the DAQ \gls{fe} hardware and which describes a region of space (eg,
  one or several neighboring channels) and time (eg, a contiguous set
  of ADC sample ticks) associated with some activity}

\newduneword{externtrigger}{external trigger candidate}{Information
  provided to the \gls{mtl} about events external to a
  \gls{detmodule} so that it may be considered in forming
  \glspl{trigcommand}}

\newduneabbrev{daqoob}{OOB dispatcher}{out-of-band trigger command
  dispatcher}{This component is responsible for dispatching a SNB dump
  command to all \glspl{daqfer} in the \gls{detmodule}.}

\newduneabbrev{mtl}{MTL}{Module Trigger Logic}{Trigger processing
  which consumes \gls{detunit}-level \gls{trigcommand} information
  and emits \glspl{trigcommand}. 
  It provides the \gls{gtl} with \glspl{trignote} and receives back any
  \glspl{externtrigger}}

% octant
\newduneword{octant}{octant}{Any of the eight parts into which 4$\pi$
  is divided by three mutually perpendicular axes. 
  In particular in referencing the value for the mixing angle
  $\theta_{23}$}

% primary DAQ buffer
\newduneword{ringbuffer}{primary DAQ buffer}{A buffer in the
  DAQ with sufficient size to store data long enough for a
  trigger decision to be made and with sufficient endurance and
  throughput to allow constant flow of full-stream data}

% sub-detector ??? %%%%%%%%%%%%%%%%%%%      Why not ``subdet''? (Anne)  %%%%%% ???????
\newduneword{submodule}{sub-detector}{A detector unit of granularity less
  than one \gls{detmodule} such as the TPC of the single-phase
  \gls{detmodule}}

\newduneword{trigcandidate}{trigger candidate}{Summary information derived
  from the full data stream and representing a contribution toward
  forming a \gls{trigdecision}}

% trigger command
\newduneword{trigcommand}{trigger command}{Information derived from
  one or more \glspl{trigcandidate} and which directs elements of the
  \gls{detmodule} to read out a portion of the data stream}

% trigger command message
\newduneabbrev{tcm}{TCM}{trigger command message}{A message flowing
  down the trigger hierarchy global to local context}

% trigger decision
\newduneword{trigdecision}{trigger decision}{The process by which
  \glspl{trigcandidate} are converted into \glspl{trigcommand}}

% trigger primitive message
\newduneabbrev{tpm}{TPM}{trigger primitive message}{A message flowing
  up the trigger hierarchy from local to global context}

\newduneabbrev{eb}{EB}{event builder}{A software agent servicing one
  \gls{detmodule} by executing \glspl{trigcommand} by reading out
  the requested data}


% \fixme{Needs improvement}
\newduneabbrev{cob}{COB}{Cluster On Board}{Four \glspl{rce} together
  with networking an other hardware}

% \fixme{Needs improvement}
\newduneabbrev{rce}{RCE}{Reconfigurable Computing Element}{One of four
  nodes in a \gls{cob} which consists of ARM CPU and FPGA resources}

% \fixme{Needs improvement}
\newduneabbrev{atca}{ATCA}{ATCA}{A computer platform} 

\newduneabbrev{rf}{RF}{radio frequency}{Electromagnetic emissions
  which are within the frequency band of sensitivity of the detector
  electronics.}

% \fixme{Needs improvement}
\newduneabbrev{fpga}{FPGA}{Field Programmable Gate Array}{An
  integrated circuit technology which the hardware to be reconfigured
  to execute different algorithms after manufacture}

\newduneabbrev{felix}{FELIX}{Front-End Link eXchange}{A
  high-throughput interface between front-end and trigger electronics
  and the standard PCIe computer bus}

\newduneabbrev{daqpart}{partition}{DAQ partition}{A cohesive and
 coherent collection of DAQ hardware and software working together to trigger and readout some portion of one detector module consisting of some integral number of\glspl{daqfrag}. 
 Multiple DAQ partitions may operate simultaneously but each instance operates independently.}
 
\newduneabbrev{fec}{FEC}{front-end computer}{The portion of one
  \gls{daqpart} which hosts the \gls{daqdr}, \gls{daqbuf} and
  \gls{daqds}.  It is connected to the \gls{daqfer} via fiber optic. 
Each \gls{detunit} of a  certain granularity, such as two SP APAs, has one front-end computer
  which receives data from the readout hardware, hosts the primary DAQ
  memory buffer for that data, emits trigger candidates derived from
  that data and satisfies requests for producing subsets of that data
  for egress.}

\newduneabbrev{daqfrag}{fragment}{DAQ front-end fragment}{The portion of one
  \gls{daqpart} relating to a single \gls{fec} and corresponding to an
  integral number of \glspl{detunit}.  See also \gls{datafrag}}

\newduneabbrev{datafrag}{fragment}{data fragment}{A block of data read
  out from a single \gls{daqfrag} which covers a contiguous period of time
  as requested by a \gls{trigcommand}}

\newduneabbrev{daqfer}{FER}{DAQ front-end readout}{The portion of a
  \gls{daqfrag} which accepts data from the detector electronics and
  provides it to the \gls{fec}. 
  In the nominal design it is also responsible for generating channel
  level \glspl{trigprimitive}}

\newduneabbrev{daqdr}{DDR}{DAQ data receiver}{The portion of the
  \gls{daqfrag} which accepts data from the \gls{daqfer}, emits
  trigger candidates produced from the input trigger primitives and
  forwards the full data stream to the \gls{daqbuf}}

\newduneabbrev{daqbuf}{primary buffer}{DAQ primary buffer}{The portion
  of the \gls{daqfrag} which accepts full data stream from the
  corresponding \gls{detunit} and retains it sufficiently long for it
  to be available to produce a \gls{datafrag}}

\newduneword{daqds}{data selector}{The portion of the
  \gls{daqfrag} which accepts \glspl{trigcommand} and returns the  corresponding \gls{datafrag}.}
  
\newduneabbrev{femb}{FEMB}{Front-End Mother Board}{Refers a unit of
  the \gls{sp} cold electronics which contains the front-end amplifier
  and ADC ASICs covering 128 channels}

\newduneword{protodune}{ProtoDUNE}{Two prototype detectors operated in
  a CERN beam test. 
  One prototyping \gls{sp} and the other \gls{dp} technology}
\newduneword{pdsp}{ProtoDUNE-SP}{The single-phase ProtoDUNE detector.}
\newduneword{pddp}{ProtoDUNE-DP}{The dual-phase ProtoDUNE detector.}

% oh boy, here's that dirty word: "event"
\newduneword{rawevent}{DAQ event block}{The unit of data output by the
  DAQ. 
  It contains trigger and detector data spanning a unique, contiguous
  time period and a subset of the detector channels}

\newduneabbrev{ssd}{SSD}{solid-state disk}{Any storage device which
  may provide sufficient write throughput to receive, collectively and
  distributed, the sustained full rate of data from a \gls{detmodule}
  for many seconds}

% fixme: this needs improvement
\newduneabbrev{hlt}{HLT}{high-level trigger}{A source of triggering at the module level.}

\newduneabbrev{pid}{PID}{Particle ID}{Particle identification}

\newduneword{readout window}{readout window}{A fixed, atomic and
  continuous period of time over which data from a \gls{detmodule}, in
  whole or in part, is recorded. 
  This period may differ based on the trigger than initiated the
  readout}

\newduneabbrev{zs}{ZS}{zero-suppression}{To delete some portion of a data stream which does not significantly deviate from zero or intrinsic noise levels.  It may be applied at different granularity from per-channel to per \dword{detunit}}

% fixme: maybe another sentence
\newduneabbrev{rc}{RC}{run control}{The system for configuring, starting and terminating the DAQ}

\newduneabbrev{snb}{SNB}{supernova neutrino burst}{A prompt and brief increase in the flux of low energy neutrinos.  Can also refer to a trigger command type which may be due to an SNB or detector conditions which can mimic its interaction signature}

\newduneabbrev{snble}{SNB/LE}{supernova neutrino burst and low energy}{Supernova neutrino burst and low-energy physics program}

\newduneabbrev{pps}{1PPS signal}{one-pulse-per-second signal}{An electrical signal with a fast rise time and which arrives in real time with a precise period of one second}

\newduneabbrev{sls}{SLS}{spill location system}{A system residing at the DUNE far detector site which provides information, possibly predictive, indicating periods of time when neutrinos are being produced by the Fermilab Main Injector beam spills}

\newduneabbrev{wib}{WIB}{warm interface board}{Digital electronics situated just outside the SP cryostat which receives digital data from the FEMBs over cold copper connections and sends it to the RCE FE readout hardware}

\newduneabbrev{sipm}{SiPM}{silicon photomultiplier}{A solid-state avalanche photo-diode sensitive to single photo-electron signals}

\newduneabbrev{cisc}{CISC}{cryogenics and slow controls}{A DUNE consortium responsible for the named components}


% entered by Anne, March 2018
%%%%%%%%%%%%%%%%%%%%%%%%% COMMON list for acronyms below %%%%%%%%%%%%%%%
\newduneword{order}{$\mathcal{O}(n)$}{of order $n$}
\newduneword{3d}{3D}{3 dimensional (also 1D, 2D, etc.)} % not phys
\newduneword{beamline}{beamline}{ADD DEF} %%%%%%%%%%%%%%%%%!!!!!!!!!!!!!
\newduneword{cdr}{CDR}{Conceptual Design Report}
\newduneword{cf}{CF}{Conventional Facilities}
\newduneword{cp}{CP}{product of charge and parity transformations}
\newduneword{cpt}{CPT}{product of charge, parity and time-reversal transformations}
\newduneword{cpv}{CPV}{violation of charge and parity symmetry}
\newduneword{doe}{DOE}{U.S. Department of Energy}
\newduneword{dune}{DUNE}{Deep Underground Neutrino Experiment}
\newduneword{esh}{ES\&H}{Environment, Safety and Health}
%\newduneword{eV}{eV}{electron volt, unit of energy (also keV, MeV, GeV, etc.)}
\newduneword{fgt}{FGT}{Fine-Grained Tracker}
\newduneword{fscf}{FSCF}{far site conventional facilities}
\newduneword{nscf}{NSCF}{near site conventional facilities}
\newduneword{gut}{GUT}{grand unified theory}
%\newduneword{exposure}{\ktyr}{exposure (without beam), expressed in metric kilotons times years}
%\newduneword{abc}{\ktMWyr}{exposure, expressed in kilotonnes $\times$ megawatts $\times$ years, based on 56\% beam uptime and efficiency} 
\newduneword{4850}{L}{level, indicates depth in feet underground at the far site, e.g., 4850L}
\newduneword{lar}{LAr}{liquid argon}
\newduneword{lartpc}{LArTPC}{liquid argon time-projection chamber}
\newduneword{lbl}{LBL}{long-baseline (physics)}
\newduneword{lbnf}{LBNF}{Long-Baseline Neutrino Facility}
\newduneword{mh}{MH}{mass hierarchy}
\newduneword{mi}{MI}{Main Injector (at Fermilab)}
%\newduneword{abc}{NDS}{Near Detector Systems; refers to the collection of detector systems at the near site }
%\newduneword{abc}{near detector}{except in Volume 4 Chapter 7, \textit{near detector} refers to the \textit{neutrino} detector system in the NDS}
\newduneword{pot}{POT}{protons on target}
\newduneword{qa}{QA}{quality assurance}
\newduneword{qc}{QC}{quality control}
\newduneword{sm}{SM}{Standard Model of particle physics}
\newduneword{tdr}{TDR}{Technical Design Report}
\newduneword{ton}{t}{metric ton, written \textit{tonne} (also kt)}
%\newduneword{abc}{tonne}{metric ton}

%%%%%%%%%%%%% PROJECT AND PHYSICS VOLUME list for acronyms below %%%%%%%%%%%%
\newduneword{ckm}{CKM}{(CKM matrix) Cabibbo-Kobayashi-Maskawa matrix, also known as
quark mixing matrix} 
\newduneword{cl}{C.L.}{confidence level}
%\newduneword{octant}{octant}{any of the eight parts into which 4$\pi$ is divided by three mutually perpendicular axes; the range of the PMNS angles is $0$ to $\pi/2$, which spans only two of the eight octants}
\newduneword{pmns}{PMNS}{(PMNS matrix) Pontecorvo-Maki-Nakagawa-Sakata matrix, also known as
the lepton or neutrino mixing matrix} 

%%%%%%%%%%%%% PROJECT AND DETECTORS VOLUME list for acronyms below %%%%%%%%%%%%

%\newduneword{apa}{APA}{anode plane assembly} 
\newduneword{blm}{BLM}{(in Volume 4) beamline measurement (system); (in Volume 3) beam loss monitor}
\newduneabbrev{cpa}{CPA}{cathode plane assembly}{}
\newduneabbrev{fc}{FC}{field cage}{}
\newduneabbrev{topfc}{top FC}{top field cage}{}
\newduneabbrev{botfc}{bottom FC}{bottom field cage}{}
\newduneabbrev{ewfc}{endwall FC}{endwall field cage}{}
\newduneabbrev{gp}{GP}{ground plane}{}
\newduneabbrev{alara}{ALARA}{as low as reasonably achievable}{As low as reasonably achievable; usually used with regard to ionizing radiation, but sometimes used more generally. It means making every reasonable effort to maintain e.g., exposures, to as far below the limits as practical, consistent with the purpose for which the activity is undertaken,}


\newduneword{ecal}{ECAL}{electromagnetic calorimeter}
\newduneword{hv}{HV}{high voltage}
\newduneword{spmod}{SP module}{single-phase detector module}
\newduneword{dpmod}{DP module}{dual-phase detector module}
%%%%%%%%%%%%% PHYSICS AND DETECTORS VOLUME list for acronyms below %%%%%%%%%%%%
\newduneword{cc}{CC}{charged current (interaction)}
\newduneword{dis}{DIS}{deep inelastic scattering}
\newduneword{fsi}{FSI}{final-state interactions}
\newduneword{geant4}{GEANT4}{GEometry ANd Tracking, a platform for the simulation of the passage of particles through matter using Monte Carlo methods} 
\newduneword{genie}{GENIE}{Generates Events for Neutrino Interaction Experiments (an object-oriented neutrino Monte Carlo generator)} 
\newduneword{mc}{MC}{Monte Carlo (detector simulation methods)}
\newduneword{qe}{QE}{quasi-elastic (interaction)}

%%%%%%%%%%%%%%%%%%%%%%%%% PROJECT VOLUME list for acronyms below %%%%%%%%%%%%%%%
%\newduneword{abc}{L1, L2, ...}{WBS level within the LBNF and DUNE Projects, where the overall Project is L1}
\newduneword{mou}{MOU}{memorandum of understanding}
\newduneword{pip2}{PIP-II(III)}{Proton Improvement Plan (II or III)}
\newduneword{sdsta}{SDSTA}{South Dakota Science and Technology Authority}
\newduneword{wbs}{WBS}{Work Breakdown Structure}

%%%%%%%%%%%%%%%%%%%%%%%%% PHYSICS VOLUME list for acronyms below %%%%%%%%%%%%%%%
\newduneword{br}{BR}{branching ratio}
\newduneword{dm}{DM}{dark matter}
\newduneword{dsnb}{DSNB}{Diffuse Supernova Neutrino Background}
\newduneword{globes}{GLoBES}{General Long-Baseline Experiment Simulator (software package)}
\newduneword{l/e}{L/E}{length-to-energy ratio}
\newduneword{lri}{LRI}{long-range interactions}
\newduneword{solarmass}{$M_{\odot}$}{solar mass}
\newduneword{nc}{NC}{neutral current (interaction)}
\newduneword{nh}{NH}{normal (mass) hierarchy}
\newduneword{nsi}{NSI}{nonstandard interactions}
\newduneword{msw}{MSW}{Mikheyev-Smirnov-Wolfenstein (effect)}
\newduneword{sme}{SME}{Standard-Model Extension}
\newduneword{susy}{SUSY}{supersymmetry}
\newduneword{wimp}{WIMP}{weakly-interacting massive particle}

%%%%%%%%%%%%%%%%%%%%%%%%% DETECTORS VOLUME list for acronyms below %%%%%%%%%%%%%%%

\newduneword{ce}{CE}{Cold Electronics}

\newduneword{crp}{CRP}{Charge-Readout Planes }
\newduneword{dram}{DRAM}{dynamic random access memory}
\newduneword{fermilab}{Fermilab}{Fermi National Accelerator Laboratory (in Batavia, IL, the Near Site)}
\newduneword{fnal}{FNAL}{see Fermilab}
\newduneword{fs}{FS}{full stream (data volumes)} %?
\newduneword{lem}{LEM}{Large Electron Multiplier}
\newduneword{lng}{LNG}{liquefied natural gas}
%\newduneword{abc}{LNGS}{Laboratori Nazionali (National Laboratory) del Gran Sasso (in L'Aquila, Italy)}
%\newduneword{abc}{MaVaNs}{mass varying neutrinos}
\newduneword{mesh}{mesh screen}{A fine mesh screen, glued directly to the steel frame on both sides of each APA in the single-phase TPC, creates a uniform ground layer beneath the wire planes.}
\newduneword{mip}{MIP}{minimum ionizing particle}
%\newduneword{abc}{MTS}{Materials Test Stand}
\newduneword{muid}{MuID}{muon identifier (detector)}
%\newduneword{abc}{OPERA}{Oscillation Project with Emulsion-Racking Apparatus (experiment at LNGS)}
%\newduneword{abc}{NND}{(used only in Volume 4 Chapter 7) near neutrino detector, same as ND}
%\newduneword{abc}{OD}{outer diameter}
\newduneword{pd}{PD}{photon detection (system)}
\newduneword{pmt}{PMT}{photomultiplier tube}
\newduneword{ppm}{PPM}{parts per million}
\newduneword{ppb}{PPB}{parts per billion}
\newduneword{ppt}{PPT}{parts per trillion}
\newduneword{rio}{RIO}{reconfigurable input output}
\newduneword{rpc}{RPC}{resistive plate chamber}
\newduneword{s/n}{S/N}{signal-to-noise (ratio)}
\newduneword{ssp}{SSP}{SiPM signal processor}
\newduneword{sbn}{SBN}{Short-Baseline Neutrino program (at Fermilab)}
\newduneword{stt}{STT}{straw tube tracker}
%\newduneword{abc}{SURF (also Sanford Lab)}{Sanford Underground Research Facility (in Lead, SD, the Far Site)}
\newduneword{tr}{TR}{transition radiation}
%\newduneword{abc}{W}{Watt (also mW, kW, MW) }
%\newduneword{abc}{WA105}{Single-Phase LArTPC and the Long Baseline Neutrino Observatory Demonstration}
\newduneword{wire board}{wire board}{At the head end of the APA in the single-phase TPCr, stacks of electronics boards referred to as ``wire boards'' are arrayed to anchor the wires.  They also provide the connection between the wires and the cold electronics.}
\newduneword{wls}{WLS}{wavelength shifting}


%%%%% Software and computing %%%%

\newduneword{larsoft}{LArSoft}{Liquid Argon Software (LArSoft),  a shared base of physics software across Liquid Argon (LAr) Time Projection Chamber (TPC) experiments.}
\newduneword{nova}{NOvA}{The NOvA off-axis neutrino oscillation experiment at Fermilab}
\newduneword{minerva}{MINERvA}{The MINERvA neutrino cross sections experiment at Fermilab}
\newduneword{microboone}{MicroBooNE}{The Liquid Argon TPC-based MicroBooNE neutrino oscillation experiment at Fermilab}
\newduneword{wirecell}{wire-cell}{Wire-Cell is a tomographic automated 3D neutrino event reconstruction method for LArTPC's}
\newduneword{ftslite}{F-FTS-lite}{Light weight version of the Fermilab File Transfer system used for rapid data transfers out of the online systems}
\newduneword{fts}{FTS}{File Transfer System developed at Fermilab to catalog and move data to permanent storage}

%%% new ones that I haven't categorized (Anne)
\newduneword{35t}{35 ton prototype}{The 35 ton prototype cryostat and \gls{sp} detector built at Fermilab before the \gls{protodune} detectors.}
\newduneabbrev{dss}{DSS}{detector support system}{}

\title{DUNE Words}
\author{Brett Viren}

\begin{document}
\maketitle
\tableofcontents

\section{Overview}

The DUNE glossary gives a concise definitions of the special
\dwords{dword} and in some cases abbreviations that are part of the
DUNE collaboration's lexicon.
The terms make up a technical vocabulary which DUNE collaborators use
when speaking and writing about their detectors and experiment.

When authoring collaboration documents or papers, particularly those
that span large parts of the DUNE scope, it is best to always refer to
\dwords{dword} through a fixed reference key provided by the DUNE
glossary and avoid ever typing a term literally into the body of the
document (as described below).
This provides the following benefits:


\begin{itemize}
\item Enforces consistent use of terms which helps more easily convey
  meaning to the reader.
\item Reduces repetitive and potentially inconsistent explanation of
  terms.  
\item Synchronizes meaning between collaborators.
\item Optionally allows hyperlinks in a PDF document from a term to
  its definition in the glossary as well as reverse links from the
  definition to locations in the text where the term is used.
\end{itemize}

\noindent Of course, slavish adherence may not always be practical but
before writing and during editing a diligent author should find
themselves continuously checking the glossary for previously defined
terms and adding new terms as needed. 
If a term is used in an unexpected manner it means that there is a
lack of \textbf{conceptual cohesion} between the authors and this
\textbf{requires discussion} before changing the definition in the
glossary. 
Changes may also require reworking surrounding text. 
The glossary is a shared resource expressing our consensus and so must
be cared for by all.

For DUNE, a \LaTeX{} file \texttt{glossary.tex} is provided which
defines some DUNE-specific macros on top of the standard
\texttt{glossaries} \LaTeX{} package. 
The DUNE-specific macros merely provide a simplified view of a subset
of the full \texttt{glossaries} functionality for defining and using
terms.
In most cases, these DUNE-specific macros should be used to define and
refer to DUNE terms instead of directly using macros from the
\texttt{glossaries} package.
However, one is still free to use the underlying \texttt{glossaries}
macros directly, and indeed in some cases this may be needed.
Following the DUNE-specific macros the \texttt{glossary.tex} file
contains the definitions of the various DUNE terms, themselves.


The rest of this document describes how to use the \dwords{dword}
macros.   
First it shows how to use currently defined terms in the document body
text. 
If finishes by showing how to extend the glossary by introducing new
terms.
For documentation on the underlying \texttt{glossaries} package see
the
\href{mirrors.ctan.org/macros/latex/contrib/glossaries/glossaries-user.pdf}{glossaries
  user manual}.


\section{Usage}

This section describes how to use \dwords{dword} in a document. 

\subsection{Integrating into a document}

To use \dwords{dword} in a \LaTeX{} document include the
\texttt{glossary.tex} file:
\begin{verbatim}
%%%%%%     START ADDING WORDS BELOW LINE 50    %%%%%%%%

% see dune-words.tex for explanation.

\usepackage[acronyms,toc]{glossaries}
\makeglossaries


\newcommand{\dshort}[1]{\acrshort{abbrev-#1}}
\newcommand{\dlong}[1]{\acrlong{abbrev-#1}}

% force the "first time" behavior
\newcommand{\dfirst}[1]{\glsfirst{abbrev-#1}}

\newcommand{\dword}[1]{\gls{#1}}
\newcommand{\dwords}[1]{\glspl{#1}}
\newcommand{\Dword}[1]{\Gls{#1}}
\newcommand{\Dwords}[1]{\Glspl{#1}}
%\newcommand{\dwordies}[1]{\glspl{#4}} %Anne -- need to do more research on this!!


% \newduneword{label}{term}{description}
\newcommand{\newduneword}[3]{
    \newglossaryentry{#1}{
        text={#2},
        long={#2},
        name={\glsentrylong{#1}},
        first={\glsentryname{#1}},
        firstplural={\glsentrylong{#1}\glspluralsuffix},
        description={#3}
    }
}

%                 1      2     3       4
% \newduneabbrev{label}{abbrev}{term}{description}
\newcommand{\newduneabbrev}[4]{
  % this makes acronym entries even if they neither term or abbrev is
  % referenced in the text.
  % \newacronym[see={[Glossary:]{#1}}]{abbrev-#1}{#2}{#3}
  \newacronym{abbrev-#1}{#2}{#3}
  \newglossaryentry{#1}{
    text={#2},
    long={#3},
    name={\glsentrylong{#1}{} (\glsentrytext{#1}{})},
    first={\glsentryname{#1}},
    firstplural={\glsentrylong{#1}\glspluralsuffix{} (\glsentrytext{#1}\glspluralsuffix{})},
    description={#4}
  }
}

%  (Anne trying)    1      2     3       4       5
% \newduneabbrev2{label}{abbrev}{term}{termpl}{description}
\newcommand{\newduneabries}[5]{
  % this makes acronym entries even if they neither term or abbrev is
  % referenced in the text.
  % \newacronym[see={[Glossary:]{#1}}]{abbrev-#1}{#2}{#3}
  \newacronym{abbrev-#1}{#2}{#3}{#4}
  \newglossaryentry{#1}{
    text={#2},
    long={#3},
    name={\glsentrylong{#1}{} (\glsentrytext{#1}{})},
    first={\glsentryname{#1}},
    firstplural={\glsentrylong{#1}\glspluralsuffix{} (\glsentrytext{#1}\glspluralsuffix{#4})},
    description={#5}
  }
}


%%%%%%     START ADDING WORDS, IN ALPHABETICAL ORDER IF POSSIBLE!    %%%%%%%%

% Model to copy

%\newduneword{abc}{ABC's real name}{Sentence or two describing ABC. It can use other dunewords defined here, e.g., \gls{fe}.}

%\newduneabries{abc}{ABC}{annes beecy}{annes beecies}{a made-up thing}

\newduneabbrev{nd}{ND}{near detector}{Refers to the detectors or more
  generally the experimental site at Fermilab}

\newduneabbrev{fd}{FD}{far detector}{Refers to the detector or more
  generally the experimental site in or above the Homestake mine in
  Lead, SD}

\newduneabbrev{sp}{SP}{single-phase}{Distinguishes one of the four
  \SI{10}{\kton} \glspl{detmodule} of the DUNE far detector by the
  fact that it operates using argon in just its liquid phase.}


\newduneabbrev{dp}{DP}{dual-phase}{Distinguishes one of the four
  \SI{10}{\kton} \glspl{detmodule} of the DUNE far detector by the
  fact that it operates using argon in both gas and liquid phases.}

\newduneabbrev{pds}{PDS}{photon detection system}{The \gls{submodule}
  system sensitive to light produced in the LAr}

\newduneabbrev{tpc}{TPC}{time projection chamber}{The portion of the
  various DUNE \glspl{submodule} which record ionization electrons
  after they drift away from a cathode, through LAr and potentially
  through GAr as well. 
  The activity is recorded by digitizing the waveforms of current
  induced on anode as the distribution of ionization charge passes by
  or is collected on the electrode}

\newduneabbrev{apa}{APA}{anode plane assembly}{One unit the SP
  detector containing the elements sensitive to activity in the LAr. 
  It contains two faces each of three planes of wires, cold
  electronics and photo detection system.} 

\newduneabbrev{cro}{CRO}{charge readout}{The system for detecting
  ionization charge distributions in the DP detector module}

\newduneabbrev{lro}{LRO}{light readout}{The system for detecting
  scintillation photons in the DP detector module}

% front-end
\newduneabbrev{fe}{FE}{front-end}{The front-end refers a point which is
  ``upstream'' of the data flow for a particular subsystem. 
  For example the front-end electronics is where the cold electronics
  meet the sense wires of the TPC and the front-end DAQ is where the
  DAQ meets the output of the electronics}

% analog digital converter
\newduneabbrev{adc}{ADC}{analog digital converter}{A sampling of a voltage
  resulting in a discrete integer count corresponding in some way to
  the input}

% data aquisition
\newduneabbrev{daq}{DAQ}{data aquisition}{The data acquisition system
  accepts data from the detector FE electronics, buffers
  it, performs a \gls{trigdecision}, builds events from the selected
  data and delivers the result to the offline \gls{diskbuffer}}

% detector module
\newduneword{detmodule}{detector module}{The entire DUNE far detector is
  segmented into four modules, each with a nominal \SI{10}{\kton}
  fiducial mass}

% detector unit
\newduneword{detunit}{detector unit}{A \gls{submodule} may be partitioned
  into a number of similar parts. 
  For example the single-phase TPC \gls{submodule} is made up of APA
  units}

% secondary DAQ buffer
\newduneword{diskbuffer}{secondary DAQ buffer}{A secondary
  \dshort{daq} buffer holds a small subset of the full rate as
  selected by a \gls{trigcommand}. 
  This buffer also marks the interface with the DUNE Offline}

% data quality monitoring
\newduneabbrev{dqm}{DQM}{data quality monitoring}{Analysis of the raw
  data to monitor the integrity of the data and the performance of the
  detectors and their electronics. This type of monitoring may be
  performed in real time, within the \gls{daq} system, or in later
  stages of processing, using disk files as input}

% DAQ dump buffer
\newduneword{dumpbuffer}{DAQ dump buffer}{This \dshort{daq} buffer
  accepts a high-rate data stream, in aggregate, from an associated
  \gls{submodule} sufficient to collect all data likely relevant to
  a potential Supernova Burst.}


% Global Trigger Logic
\newduneabbrev{gtl}{ETL}{External Trigger Logic}{Trigger processing
  which consumes \gls{detmodule} level \gls{trignote} information
  and other global sources of trigger input and emits
  \gls{trigcommand} information back to the \glspl{mtl}}

\newduneword{trignote}{trigger notification}{Information provided by
  \gls{mtl} to \gls{gtl} about \gls{trigdecision} its processing}

% trigger primitive
\newduneword{trigprimitive}{trigger primitive}{Information derived by
  the DAQ \gls{fe} hardware and which describes a region of space (eg,
  one or several neighboring channels) and time (eg, a contiguous set
  of ADC sample ticks) associated with some activity}

\newduneword{externtrigger}{external trigger candidate}{Information
  provided to the \gls{mtl} about events external to a
  \gls{detmodule} so that it may be considered in forming
  \glspl{trigcommand}}

\newduneabbrev{daqoob}{OOB dispatcher}{out-of-band trigger command
  dispatcher}{This component is responsible for dispatching a SNB dump
  command to all \glspl{daqfer} in the \gls{detmodule}.}

\newduneabbrev{mtl}{MTL}{Module Trigger Logic}{Trigger processing
  which consumes \gls{detunit}-level \gls{trigcommand} information
  and emits \glspl{trigcommand}. 
  It provides the \gls{gtl} with \glspl{trignote} and receives back any
  \glspl{externtrigger}}

% octant
\newduneword{octant}{octant}{Any of the eight parts into which 4$\pi$
  is divided by three mutually perpendicular axes. 
  In particular in referencing the value for the mixing angle
  $\theta_{23}$}

% primary DAQ buffer
\newduneword{ringbuffer}{primary DAQ buffer}{A buffer in the
  DAQ with sufficient size to store data long enough for a
  trigger decision to be made and with sufficient endurance and
  throughput to allow constant flow of full-stream data}

% sub-detector ??? %%%%%%%%%%%%%%%%%%%      Why not ``subdet''? (Anne)  %%%%%% ???????
\newduneword{submodule}{sub-detector}{A detector unit of granularity less
  than one \gls{detmodule} such as the TPC of the single-phase
  \gls{detmodule}}

\newduneword{trigcandidate}{trigger candidate}{Summary information derived
  from the full data stream and representing a contribution toward
  forming a \gls{trigdecision}}

% trigger command
\newduneword{trigcommand}{trigger command}{Information derived from
  one or more \glspl{trigcandidate} and which directs elements of the
  \gls{detmodule} to read out a portion of the data stream}

% trigger command message
\newduneabbrev{tcm}{TCM}{trigger command message}{A message flowing
  down the trigger hierarchy global to local context}

% trigger decision
\newduneword{trigdecision}{trigger decision}{The process by which
  \glspl{trigcandidate} are converted into \glspl{trigcommand}}

% trigger primitive message
\newduneabbrev{tpm}{TPM}{trigger primitive message}{A message flowing
  up the trigger hierarchy from local to global context}

\newduneabbrev{eb}{EB}{event builder}{A software agent servicing one
  \gls{detmodule} by executing \glspl{trigcommand} by reading out
  the requested data}


% \fixme{Needs improvement}
\newduneabbrev{cob}{COB}{Cluster On Board}{Four \glspl{rce} together
  with networking an other hardware}

% \fixme{Needs improvement}
\newduneabbrev{rce}{RCE}{Reconfigurable Computing Element}{One of four
  nodes in a \gls{cob} which consists of ARM CPU and FPGA resources}

% \fixme{Needs improvement}
\newduneabbrev{atca}{ATCA}{ATCA}{A computer platform} 

\newduneabbrev{rf}{RF}{radio frequency}{Electromagnetic emissions
  which are within the frequency band of sensitivity of the detector
  electronics.}

% \fixme{Needs improvement}
\newduneabbrev{fpga}{FPGA}{Field Programmable Gate Array}{An
  integrated circuit technology which the hardware to be reconfigured
  to execute different algorithms after manufacture}

\newduneabbrev{felix}{FELIX}{Front-End Link eXchange}{A
  high-throughput interface between front-end and trigger electronics
  and the standard PCIe computer bus}

\newduneabbrev{daqpart}{partition}{DAQ partition}{A cohesive and
 coherent collection of DAQ hardware and software working together to trigger and readout some portion of one detector module consisting of some integral number of\glspl{daqfrag}. 
 Multiple DAQ partitions may operate simultaneously but each instance operates independently.}
 
\newduneabbrev{fec}{FEC}{front-end computer}{The portion of one
  \gls{daqpart} which hosts the \gls{daqdr}, \gls{daqbuf} and
  \gls{daqds}.  It is connected to the \gls{daqfer} via fiber optic. 
Each \gls{detunit} of a  certain granularity, such as two SP APAs, has one front-end computer
  which receives data from the readout hardware, hosts the primary DAQ
  memory buffer for that data, emits trigger candidates derived from
  that data and satisfies requests for producing subsets of that data
  for egress.}

\newduneabbrev{daqfrag}{fragment}{DAQ front-end fragment}{The portion of one
  \gls{daqpart} relating to a single \gls{fec} and corresponding to an
  integral number of \glspl{detunit}.  See also \gls{datafrag}}

\newduneabbrev{datafrag}{fragment}{data fragment}{A block of data read
  out from a single \gls{daqfrag} which covers a contiguous period of time
  as requested by a \gls{trigcommand}}

\newduneabbrev{daqfer}{FER}{DAQ front-end readout}{The portion of a
  \gls{daqfrag} which accepts data from the detector electronics and
  provides it to the \gls{fec}. 
  In the nominal design it is also responsible for generating channel
  level \glspl{trigprimitive}}

\newduneabbrev{daqdr}{DDR}{DAQ data receiver}{The portion of the
  \gls{daqfrag} which accepts data from the \gls{daqfer}, emits
  trigger candidates produced from the input trigger primitives and
  forwards the full data stream to the \gls{daqbuf}}

\newduneabbrev{daqbuf}{primary buffer}{DAQ primary buffer}{The portion
  of the \gls{daqfrag} which accepts full data stream from the
  corresponding \gls{detunit} and retains it sufficiently long for it
  to be available to produce a \gls{datafrag}}

\newduneword{daqds}{data selector}{The portion of the
  \gls{daqfrag} which accepts \glspl{trigcommand} and returns the  corresponding \gls{datafrag}.}
  
\newduneabbrev{femb}{FEMB}{Front-End Mother Board}{Refers a unit of
  the \gls{sp} cold electronics which contains the front-end amplifier
  and ADC ASICs covering 128 channels}

\newduneword{protodune}{ProtoDUNE}{Two prototype detectors operated in
  a CERN beam test. 
  One prototyping \gls{sp} and the other \gls{dp} technology}
\newduneword{pdsp}{ProtoDUNE-SP}{The single-phase ProtoDUNE detector.}
\newduneword{pddp}{ProtoDUNE-DP}{The dual-phase ProtoDUNE detector.}

% oh boy, here's that dirty word: "event"
\newduneword{rawevent}{DAQ event block}{The unit of data output by the
  DAQ. 
  It contains trigger and detector data spanning a unique, contiguous
  time period and a subset of the detector channels}

\newduneabbrev{ssd}{SSD}{solid-state disk}{Any storage device which
  may provide sufficient write throughput to receive, collectively and
  distributed, the sustained full rate of data from a \gls{detmodule}
  for many seconds}

% fixme: this needs improvement
\newduneabbrev{hlt}{HLT}{high-level trigger}{A source of triggering at the module level.}

\newduneabbrev{pid}{PID}{Particle ID}{Particle identification}

\newduneword{readout window}{readout window}{A fixed, atomic and
  continuous period of time over which data from a \gls{detmodule}, in
  whole or in part, is recorded. 
  This period may differ based on the trigger than initiated the
  readout}

\newduneabbrev{zs}{ZS}{zero-suppression}{To delete some portion of a data stream which does not significantly deviate from zero or intrinsic noise levels.  It may be applied at different granularity from per-channel to per \dword{detunit}}

% fixme: maybe another sentence
\newduneabbrev{rc}{RC}{run control}{The system for configuring, starting and terminating the DAQ}

\newduneabbrev{snb}{SNB}{supernova neutrino burst}{A prompt and brief increase in the flux of low energy neutrinos.  Can also refer to a trigger command type which may be due to an SNB or detector conditions which can mimic its interaction signature}

\newduneabbrev{snble}{SNB/LE}{supernova neutrino burst and low energy}{Supernova neutrino burst and low-energy physics program}

\newduneabbrev{pps}{1PPS signal}{one-pulse-per-second signal}{An electrical signal with a fast rise time and which arrives in real time with a precise period of one second}

\newduneabbrev{sls}{SLS}{spill location system}{A system residing at the DUNE far detector site which provides information, possibly predictive, indicating periods of time when neutrinos are being produced by the Fermilab Main Injector beam spills}

\newduneabbrev{wib}{WIB}{warm interface board}{Digital electronics situated just outside the SP cryostat which receives digital data from the FEMBs over cold copper connections and sends it to the RCE FE readout hardware}

\newduneabbrev{sipm}{SiPM}{silicon photomultiplier}{A solid-state avalanche photo-diode sensitive to single photo-electron signals}

\newduneabbrev{cisc}{CISC}{cryogenics and slow controls}{A DUNE consortium responsible for the named components}


% entered by Anne, March 2018
%%%%%%%%%%%%%%%%%%%%%%%%% COMMON list for acronyms below %%%%%%%%%%%%%%%
\newduneword{order}{$\mathcal{O}(n)$}{of order $n$}
\newduneword{3d}{3D}{3 dimensional (also 1D, 2D, etc.)} % not phys
\newduneword{beamline}{beamline}{ADD DEF} %%%%%%%%%%%%%%%%%!!!!!!!!!!!!!
\newduneword{cdr}{CDR}{Conceptual Design Report}
\newduneword{cf}{CF}{Conventional Facilities}
\newduneword{cp}{CP}{product of charge and parity transformations}
\newduneword{cpt}{CPT}{product of charge, parity and time-reversal transformations}
\newduneword{cpv}{CPV}{violation of charge and parity symmetry}
\newduneword{doe}{DOE}{U.S. Department of Energy}
\newduneword{dune}{DUNE}{Deep Underground Neutrino Experiment}
\newduneword{esh}{ES\&H}{Environment, Safety and Health}
%\newduneword{eV}{eV}{electron volt, unit of energy (also keV, MeV, GeV, etc.)}
\newduneword{fgt}{FGT}{Fine-Grained Tracker}
\newduneword{fscf}{FSCF}{far site conventional facilities}
\newduneword{nscf}{NSCF}{near site conventional facilities}
\newduneword{gut}{GUT}{grand unified theory}
%\newduneword{exposure}{\ktyr}{exposure (without beam), expressed in metric kilotons times years}
%\newduneword{abc}{\ktMWyr}{exposure, expressed in kilotonnes $\times$ megawatts $\times$ years, based on 56\% beam uptime and efficiency} 
\newduneword{4850}{L}{level, indicates depth in feet underground at the far site, e.g., 4850L}
\newduneword{lar}{LAr}{liquid argon}
\newduneword{lartpc}{LArTPC}{liquid argon time-projection chamber}
\newduneword{lbl}{LBL}{long-baseline (physics)}
\newduneword{lbnf}{LBNF}{Long-Baseline Neutrino Facility}
\newduneword{mh}{MH}{mass hierarchy}
\newduneword{mi}{MI}{Main Injector (at Fermilab)}
%\newduneword{abc}{NDS}{Near Detector Systems; refers to the collection of detector systems at the near site }
%\newduneword{abc}{near detector}{except in Volume 4 Chapter 7, \textit{near detector} refers to the \textit{neutrino} detector system in the NDS}
\newduneword{pot}{POT}{protons on target}
\newduneword{qa}{QA}{quality assurance}
\newduneword{qc}{QC}{quality control}
\newduneword{sm}{SM}{Standard Model of particle physics}
\newduneword{tdr}{TDR}{Technical Design Report}
\newduneword{ton}{t}{metric ton, written \textit{tonne} (also kt)}
%\newduneword{abc}{tonne}{metric ton}

%%%%%%%%%%%%% PROJECT AND PHYSICS VOLUME list for acronyms below %%%%%%%%%%%%
\newduneword{ckm}{CKM}{(CKM matrix) Cabibbo-Kobayashi-Maskawa matrix, also known as
quark mixing matrix} 
\newduneword{cl}{C.L.}{confidence level}
%\newduneword{octant}{octant}{any of the eight parts into which 4$\pi$ is divided by three mutually perpendicular axes; the range of the PMNS angles is $0$ to $\pi/2$, which spans only two of the eight octants}
\newduneword{pmns}{PMNS}{(PMNS matrix) Pontecorvo-Maki-Nakagawa-Sakata matrix, also known as
the lepton or neutrino mixing matrix} 

%%%%%%%%%%%%% PROJECT AND DETECTORS VOLUME list for acronyms below %%%%%%%%%%%%

%\newduneword{apa}{APA}{anode plane assembly} 
\newduneword{blm}{BLM}{(in Volume 4) beamline measurement (system); (in Volume 3) beam loss monitor}
\newduneabbrev{cpa}{CPA}{cathode plane assembly}{}
\newduneabbrev{fc}{FC}{field cage}{}
\newduneabbrev{topfc}{top FC}{top field cage}{}
\newduneabbrev{botfc}{bottom FC}{bottom field cage}{}
\newduneabbrev{ewfc}{endwall FC}{endwall field cage}{}
\newduneabbrev{gp}{GP}{ground plane}{}
\newduneabbrev{alara}{ALARA}{as low as reasonably achievable}{As low as reasonably achievable; usually used with regard to ionizing radiation, but sometimes used more generally. It means making every reasonable effort to maintain e.g., exposures, to as far below the limits as practical, consistent with the purpose for which the activity is undertaken,}


\newduneword{ecal}{ECAL}{electromagnetic calorimeter}
\newduneword{hv}{HV}{high voltage}
\newduneword{spmod}{SP module}{single-phase detector module}
\newduneword{dpmod}{DP module}{dual-phase detector module}
%%%%%%%%%%%%% PHYSICS AND DETECTORS VOLUME list for acronyms below %%%%%%%%%%%%
\newduneword{cc}{CC}{charged current (interaction)}
\newduneword{dis}{DIS}{deep inelastic scattering}
\newduneword{fsi}{FSI}{final-state interactions}
\newduneword{geant4}{GEANT4}{GEometry ANd Tracking, a platform for the simulation of the passage of particles through matter using Monte Carlo methods} 
\newduneword{genie}{GENIE}{Generates Events for Neutrino Interaction Experiments (an object-oriented neutrino Monte Carlo generator)} 
\newduneword{mc}{MC}{Monte Carlo (detector simulation methods)}
\newduneword{qe}{QE}{quasi-elastic (interaction)}

%%%%%%%%%%%%%%%%%%%%%%%%% PROJECT VOLUME list for acronyms below %%%%%%%%%%%%%%%
%\newduneword{abc}{L1, L2, ...}{WBS level within the LBNF and DUNE Projects, where the overall Project is L1}
\newduneword{mou}{MOU}{memorandum of understanding}
\newduneword{pip2}{PIP-II(III)}{Proton Improvement Plan (II or III)}
\newduneword{sdsta}{SDSTA}{South Dakota Science and Technology Authority}
\newduneword{wbs}{WBS}{Work Breakdown Structure}

%%%%%%%%%%%%%%%%%%%%%%%%% PHYSICS VOLUME list for acronyms below %%%%%%%%%%%%%%%
\newduneword{br}{BR}{branching ratio}
\newduneword{dm}{DM}{dark matter}
\newduneword{dsnb}{DSNB}{Diffuse Supernova Neutrino Background}
\newduneword{globes}{GLoBES}{General Long-Baseline Experiment Simulator (software package)}
\newduneword{l/e}{L/E}{length-to-energy ratio}
\newduneword{lri}{LRI}{long-range interactions}
\newduneword{solarmass}{$M_{\odot}$}{solar mass}
\newduneword{nc}{NC}{neutral current (interaction)}
\newduneword{nh}{NH}{normal (mass) hierarchy}
\newduneword{nsi}{NSI}{nonstandard interactions}
\newduneword{msw}{MSW}{Mikheyev-Smirnov-Wolfenstein (effect)}
\newduneword{sme}{SME}{Standard-Model Extension}
\newduneword{susy}{SUSY}{supersymmetry}
\newduneword{wimp}{WIMP}{weakly-interacting massive particle}

%%%%%%%%%%%%%%%%%%%%%%%%% DETECTORS VOLUME list for acronyms below %%%%%%%%%%%%%%%

\newduneword{ce}{CE}{Cold Electronics}

\newduneword{crp}{CRP}{Charge-Readout Planes }
\newduneword{dram}{DRAM}{dynamic random access memory}
\newduneword{fermilab}{Fermilab}{Fermi National Accelerator Laboratory (in Batavia, IL, the Near Site)}
\newduneword{fnal}{FNAL}{see Fermilab}
\newduneword{fs}{FS}{full stream (data volumes)} %?
\newduneword{lem}{LEM}{Large Electron Multiplier}
\newduneword{lng}{LNG}{liquefied natural gas}
%\newduneword{abc}{LNGS}{Laboratori Nazionali (National Laboratory) del Gran Sasso (in L'Aquila, Italy)}
%\newduneword{abc}{MaVaNs}{mass varying neutrinos}
\newduneword{mesh}{mesh screen}{A fine mesh screen, glued directly to the steel frame on both sides of each APA in the single-phase TPC, creates a uniform ground layer beneath the wire planes.}
\newduneword{mip}{MIP}{minimum ionizing particle}
%\newduneword{abc}{MTS}{Materials Test Stand}
\newduneword{muid}{MuID}{muon identifier (detector)}
%\newduneword{abc}{OPERA}{Oscillation Project with Emulsion-Racking Apparatus (experiment at LNGS)}
%\newduneword{abc}{NND}{(used only in Volume 4 Chapter 7) near neutrino detector, same as ND}
%\newduneword{abc}{OD}{outer diameter}
\newduneword{pd}{PD}{photon detection (system)}
\newduneword{pmt}{PMT}{photomultiplier tube}
\newduneword{ppm}{PPM}{parts per million}
\newduneword{ppb}{PPB}{parts per billion}
\newduneword{ppt}{PPT}{parts per trillion}
\newduneword{rio}{RIO}{reconfigurable input output}
\newduneword{rpc}{RPC}{resistive plate chamber}
\newduneword{s/n}{S/N}{signal-to-noise (ratio)}
\newduneword{ssp}{SSP}{SiPM signal processor}
\newduneword{sbn}{SBN}{Short-Baseline Neutrino program (at Fermilab)}
\newduneword{stt}{STT}{straw tube tracker}
%\newduneword{abc}{SURF (also Sanford Lab)}{Sanford Underground Research Facility (in Lead, SD, the Far Site)}
\newduneword{tr}{TR}{transition radiation}
%\newduneword{abc}{W}{Watt (also mW, kW, MW) }
%\newduneword{abc}{WA105}{Single-Phase LArTPC and the Long Baseline Neutrino Observatory Demonstration}
\newduneword{wire board}{wire board}{At the head end of the APA in the single-phase TPCr, stacks of electronics boards referred to as ``wire boards'' are arrayed to anchor the wires.  They also provide the connection between the wires and the cold electronics.}
\newduneword{wls}{WLS}{wavelength shifting}


%%%%% Software and computing %%%%

\newduneword{larsoft}{LArSoft}{Liquid Argon Software (LArSoft),  a shared base of physics software across Liquid Argon (LAr) Time Projection Chamber (TPC) experiments.}
\newduneword{nova}{NOvA}{The NOvA off-axis neutrino oscillation experiment at Fermilab}
\newduneword{minerva}{MINERvA}{The MINERvA neutrino cross sections experiment at Fermilab}
\newduneword{microboone}{MicroBooNE}{The Liquid Argon TPC-based MicroBooNE neutrino oscillation experiment at Fermilab}
\newduneword{wirecell}{wire-cell}{Wire-Cell is a tomographic automated 3D neutrino event reconstruction method for LArTPC's}
\newduneword{ftslite}{F-FTS-lite}{Light weight version of the Fermilab File Transfer system used for rapid data transfers out of the online systems}
\newduneword{fts}{FTS}{File Transfer System developed at Fermilab to catalog and move data to permanent storage}

%%% new ones that I haven't categorized (Anne)
\newduneword{35t}{35 ton prototype}{The 35 ton prototype cryostat and \gls{sp} detector built at Fermilab before the \gls{protodune} detectors.}
\newduneabbrev{dss}{DSS}{detector support system}{}
\end{verbatim}
It should be included in the preamble \textbf{after} the packages
\texttt{siunitx} and \texttt{hyperref}. 
The latter is not strictly required but will allow the resulting PDF
to have clickable glossary links. 

To generate a list of \dwords{dword} used in the document, add the
standard \verb|\printglossaries| macro provided by the
\texttt{glossaries} package where you want it to appear.

\subsection{Referencing terms}
\label{sec:referencing}

The benefit of the glossary is to use a common vocabulary throughout
the document. 
To do that one should avoid typing a literal DUNE term and instead
reference it through its \textbf{label}. 
Besides assuring consistency it allows editors to easily make sweeping
changes to the ``spelling'' of a term across the document. 
In the case of DUNE terms with abbreviations some automated
conveniences are provided such as including the abbreviation in
parenthesis the first time a term is used and only using the
abbreviation thereafter while in the text always using the same macro.

To reference an item, use one of the following macros. 
The first four cover the case of capitalization and pluralization and,
if an abbreviation exists, will automatically follow the behavior
above. 
To force an abbreviated or long form one the final two may be used.

\begin{itemize}
\item \verb|\dword{label}| nominal term
\item \verb|\dwords{label}| plural term
\item \verb|\Dword{label}| capitalized term
\item \verb|\Dwords{label}| capitalized and plural term
\item \verb|\dshort{label}| force usage of the abbreviated term
\item \verb|\dshorts{label}| force usage of the abbreviated term, plural
\item \verb|\dlong{label}| force usage of the full term
\item \verb|\dlongs{label}| force usage of the full term, plural
\item \verb|\dfirst{label}| force first usage behavior
\item \verb|\dfirsts{label}| force first usage behavior, plural
\end{itemize}


\noindent Here are several usage examples followed what they produce

\begin{description}

\item[first time] \verb|\dword{sp}|: \dword{sp}

\item[second time, nominal] \verb|\dword{sp}|: \dword{sp}.

\item[second time, long] \verb|\dlong{sp}|: \dlong{sp}.

\item[second time, force first time] \verb|\dfirst{sp}|: \dfirst{sp}.

\item[first time, multiple] \verb|\dwords{adc}|: \dwords{adc}.

\item[second time, singular] \verb|\dword{adc}|: \dword{adc}.

\item[long] \verb|\dlong{adc}|: \dlong{adc}.

\item[long, plural] \verb|\dlongs{adc}|: \dlongs{adc}.

\item[short] \verb|\dshort{adc}|: \dshort{adc}, and \verb|\dshorts{adc}|: \dshorts{adc}

\item[first time, singular] \verb|\dword{daq}|: \dword{daq}.

\item[second time, plual] \verb|\dwords{daq}|: \dwords{daq}.

\item[first time, singular] \verb|\dword{detmodule}|: \dword{detmodule}

\item[second time, capitalized, plural] \verb|\Dwords{detmodule}|: \Dwords{detmodule}

\item[first time, special plural] \verb|\dwords{apa}|: \dwords{apa}

\item[second time, special plural] \verb|\dwords{apa}|: \dwords{apa}

\item[second time, long] \verb|\dlong{apa}|: \dlong{apa}

\item[second time, long plural] \verb|\dlongs{apa}|: \dlongs{apa}

\item[second time, first] \verb|\dfirst{apa}|: \dfirst{apa}

\item[second time, first plural] \verb|\dfirsts{apa}|: \dfirsts{apa}.

\item[first time with capitalization] \verb|\Dword{cpa}|: \Dword{cpa}. \verb|\dfirst|: \dfirst{cpa}, \verb|\dlong|: \dlong{cpa}, \verb|\dlongs|: \dlongs{cpa}

\end{description}

\noindent Note the special pluralization for ``\dlongs{apa}''.

\section{Extending}

As of this writing, the \texttt{glossary.tex} file is included with a
larger document (currently the DUNE Technical Proposal). 
For future documents some effort to break out this glossary into an
independently managed file should be pursued. 
Until that happens, it is expected that this file will evolve along
and as part of each major DUNE document.

\subsection{Adding new terms}

In general, new \dwords{dword} may be added to the DUNE glossary using
the macros provided by the standard \texttt{glossaries} package. 
However, to simplify the most common cases a small number of
DUNE-specific macros have been developed. 
Their use should be preferred.

 To define a DUNE term that has no abbreviation use:

\begin{verbatim}
\newduneword{label}{term}{description}
\end{verbatim}

 To define a DUNE term with an abbreviation use:

\begin{verbatim}
\newduneabbrev{label}{abbrev}{term}{description}
\end{verbatim}

 To define a DUNE term with an abbreviation and a special plural form use:

\begin{verbatim}
\newduneabbrevs{label}{abbrev}{term}{terms}{description}
\end{verbatim}

 The macro arguments are as:

\begin{description}
\item[\texttt{label}] a key that is used to connect this definition to
  a reference in the text (referencing described above in
  Section~\ref{sec:referencing}). 
  A good choice for a \texttt{label} for a term with an abbreviation
  is to lowercase that abbreviation. 
  For terms that lack an abbreviation it is suggested to invent a
  \texttt{label} which is concise, memorable and similar to the full
  term.
  The \texttt{label} may contain spaces.
\item[\texttt{abbrev}] the abbreviation or acronym for the term (only used for \verb|\newduneabbrev|).
\item[\texttt{term}] the \dword{dword} term itself written out in
  long-hand words.
\item[\texttt{terms}] the plural form of the \dwords{dword} term.
\item[\texttt{description}] a concise but descriptive explanation of
  what the term means. 
  Avoid over specifying and over generalizing. 
  Shoot for one or two sentences. 
  One quirk is that the description must not end in punctuation. 
\end{description}

The term shown in the examples of Section~\ref{sec:referencing} are
defined like:

\begin{verbatim}
\newduneabbrev{adc}{ADC}{Analog Digital Converter}{A sampling of a voltage
  resulting in a discrete integer count corresponding in some way to
  the input}
\newduneabbrev{daq}{DAQ}{data aquisition}{The Data Acquisition system
  accepts data from the detector \acrshort{fe} electronics, buffers
  it, performs a \gls{trigdecision}, builds events from the selected
  data and delivers the result to the offline \gls{diskbuffer}}
\newduneword{detmodule}{detector module}{The entire DUNE far detector is
  segmented into four modules, each with a nominal \SI{10}{\kton}
  fiducial mass}
\newduneabbrevs{apa}{APA}{anode plane assembly}{anode plane assemblies}{One unit the SP
  detector containing the elements sensitive to activity in the LAr. 
  It contains two faces each of three planes of wires, cold
  electronics and photo detection system.} 
\end{verbatim}

Note, the use of \texttt{siunitx} to handle units and numerals in a
consistent way.
Also note that it is acceptable to use \texttt{glossaries} macros (eg
\verb|\gls{}|) to reference DUNE terms inside the descriptions of
other DUNE terms. 
Because the final glossary below only includes terms that have been
referenced, this is a good way to assure completeness.
 



\cleardoublepage
\printglossaries


\end{document}
