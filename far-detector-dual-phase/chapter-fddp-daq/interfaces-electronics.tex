
%%%%%%%%%%%%%%%%%%%%%%%%%%%%%%%%%%%
\subsection{TPC Electronics}
\label{sec:fd-daq-intfc-elec}

Details about the interfaces between the DAQ and the \dword{dp}
\dlong{cro} TPC electronics are documented in~\cite{docdb6778}.
\fixme{Add references to bib.}

In the case of the \dword{dp} \dword{detmodule}, signals from the
\dlong{cro} electrodes are amplified and then digitzed by \dword{amc}
boards residing in 240 \dword{utca} crates. 
Each crate produces \SI{2.5}{\MHz} samples from 640 channels on a
\SI{10}{\Gbps} optical fiber. 
Data is losslessly compressed and sent via \dword{udp} producing a
stream with an expected average throughput of \SI{2}{\Gbps}.
The DAQ consortium will be responsible for acquiring the fibers while
the DP-Electronic consortium will be responsible for their
installation on the cryostat roof down to their connection to the
\dword{utca} crates.

\fixme{Need some statement about the white rabbit fiber?}

%%%%%%%%%%%%%%%%%%%%%%%%%%%%%%%%%%%
\subsection{PD Electronics}
\label{sec:fd-daq-intfc-photon}

Details about the interfaces between the DAQ the \dword{dp}
\dlong{lro}~\cite{docdb6802} are documented. 

\fixme{Add references to bib.}

For the \dword{dp} \dword{lro}, the signals are digitized in 14 bits
at \SI{65}{\MHz} and then downsampled to \SI{2.5}{\MHz}. 
The data is then handled largely symmetrically with the \dword{cro}
data except that only five \dword{utca} crates are required and their
average, uncompressed output is expected to fill \SI{5}{\Gbps} on
\SI{10}{\Gbps} fiber optical connections. 
The installation of these fibers are as described above for the
\dwords{cro} fibers.

