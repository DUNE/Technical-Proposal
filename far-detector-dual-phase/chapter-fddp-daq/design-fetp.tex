\subsection{Front-end Trigger Primitive Generation}
\label{sec:fd-daq-fetp}

\metainfo{Josh Klein \& J.J. Russel \& Brett Viren.  This section is DP-specific.
This file is \texttt{far-detector-dual-phase/chapter-fddp-daq/design-fero.tex}}

In the nominal design, the bump-on-wire (BOW) computers decompress the
CRO data stream and execute algorithms to search for per-channel
localized activity above some threshold based on a recent measure of
the noise. 
These trigger primitives are then sent out along with the original,
compressed CRO data to the \dword{fec} associated with the BOW.

Initial studies have shown some promise that this type of trigger
primitive pipeline can be implemented on commodity CPUs and keep up
with the data. 
The studies made use of \dword{sp} signal and noise simulation but
given the relatively higher signal-to-noise ratio expected in the DP
detector data, coupled with fewer total channels, these studies should
be applicable and even better performance may be expected. 
With that said, more realistic studies using \dword{dp}-specific
simuilations and with ProtoDUNE-DP data are needed.

Most of the same technical triggering issues apply in the nominal and
the alternative design (generically diagrammed in
Fig~\ref{fig:daq-overview-alt}) as both call for deploying trigger
primitive pipelines on commodity CPUs. 
Additional study is required to understand the multiplicity of trigger
processing computers and how that might scale as one considers
different multiplicities of CRO \dword{utca} crates w.r.t. each
\dword{fec}.

In both designs, the LRO data is nominally not involved in triggering
because the CRO triggering is expected to be more efficient.
Future studies may indicate additional benefit to including LRO
information in triggering and both the nominal and alternative design
can elastically accommodate this increase scope although possibly at
the cost of either more BOW or trigger processor computers.


