\subsection{Front-end Trigger Primitive Generation}
\label{sec:fd-daq-fetp}

In the nominal design, the \dword{bow} computers decompress the
CRO data stream and execute algorithms to search for per-channel
localized activity above some threshold based on a recent measure of
the noise. 
These trigger primitives are then sent out along with the original,
compressed CRO data to the \dword{fec} associated with the \dword{bow}.

Initial studies have shown some promise that this type of trigger
primitive pipeline can be implemented on commodity CPUs and keep up
with the data. 
The studies made use of \dword{sp} signal and noise simulation but
given the relatively higher signal-to-noise ratio expected in the DP
detector data, coupled with fewer total channels, these studies should
be applicable and even better performance may be expected. 
With that said, more realistic studies using \dword{dp}-specific
simuilations and with ProtoDUNE-DP data are needed.

Most of the same technical triggering issues apply in the nominal and
the alternative design (generically diagrammed in
Fig~\ref{fig:daq-overview-alt}) as both call for deploying trigger
primitive pipelines on commodity CPUs. 
The main difference is that the \dwords{bow} hosts become trigger farm
hosts. 
They move from being upstream and inline of the data flow before the
\dword{fec} to being downstream and receiving the data after it has
been buffered. 
Their \dwords{trigcandidate} no longer have to be inserted into the
stream and stripped out by the \dword{fec} and instead are sent
directly to the \dword{mtl}. 
Additional study is required to understand the multiplicity of trigger
processing computers and how that might scale as one considers
different multiplicities of CRO \dword{utca} crates w.r.t. each
\dword{fec}.

In both designs, the \dword{lro} data is currently not considered for
triggering as the \dword{cro} triggering is expected to be more
efficient although using light information for triggering has not yet
been ruled out. 
Future studies may indicate additional benefit to including
\dword{lro} information in triggering and both the nominal and
alternative design can elastically accommodate this increase scope
although possibly at the cost of either more \dword{bow} or trigger
processor computers.
As described elsewhere, the \dword{lro} data flow is handled by
separate \dwords{daqfrag} from those which service the \dwords{cro}. 
Thus, if any \dwords{trigprimitive} and \dwords{trigcandidate} are to
be formed from \dword{lro} data they would come from these separate
fragments and be combined in the \dword{mtl} as peers of the
\dword{cro} and external candidates.

