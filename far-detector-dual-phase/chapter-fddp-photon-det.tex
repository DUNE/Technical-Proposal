\chapter{Photon Detection System}
\label{ch:fddp-pd}

%%%%%%%%%%%%%%%%%%%%%%%%%%%%%%%%%%%%%%%%%%%%%%%%%%%%%%%%%%%%%%%%%%%%%%%%%%%%%%%%%%%%%%%%%%%%%%%%%%%%%%%%%%
\section{Photon Calibration System}
\label{sec:fddp-pd-5}

%%%%%%%%%%%%%%%%%%%%%%%%%%%%%%%%%
\subsection{System Design and Procurement}
\label{sec:fddp-pd-5.1}

A photon calibration system is required to be integrated into the \dword{dpmod} to calibrate the \dwords{pmt} 
installed in the \lar volume. The goal is to determine the \dword{pmt} gain and maintain the \dword{pmt} performance stability. A design similar to the one used in \dword{pddp} will be used although some R\&D measurements are planned to make it more effective, reduce the cost and mitigate issues related to the scaling.

%In \dword{pddp}, an optical fiber will be installed at each \dword{pmt} in order to provide a configurable amount of light (see Figure~\ref{fig:dppd_3_2}). The calibration light will be provided by a blue  \dword{led} of \SI{460}{nm} using a Kapuschinski circuit as  \dword{led} driver which reduces significantly the cost of using a laser. There will be one  \dword{led} connected to one fiber going to one female optical \fdth from Allectra~\cite{allectra}. In total,  there will be six  \dwords{led} placed in a hexagonal geometry. The direct light will go to the fiber, and the stray light to a SiPM used as reference sensor, being a single reference sensor in the center. Fibers of length 22.5-m (from Thorlabs $\phi$ 800-$\mu$m, FT800UMT~\cite{ft800umt}, and stainless-steel jacket) will be used inside the cryostat. Each one of these fibers will be attached to a 1-to-7 fiber bundle (from Thorlabs $\phi$ 200-$\mu$m, FT200UMT~\cite{ft200umt}, stainless-steel jacket common end, and black jacket at split ends), so that one fiber is finally installed at each \dword{pmt}. A diagram of the \dword{pddp} photon calibration system is shown in Figure~\ref{fig:dppd_5_1}. Several tests to quantify the light losses of this design were performed with successful results. 
In \dword{pddp}, an optical fiber is installed at each \dword{pmt} in order to provide a configurable amount of light (see Figure~\ref{fig:dppd_3_2}). The calibration light is provided by a blue  \dword{led} of \SI{460}{nm} using a Kapuschinski circuit as  \dword{led} driver; this is much less expensive than using a laser.
One \dword{led} is connected to one fiber that goes to one female optical \fdth from Allectra~\footnote{Allectra\texttrademark{}, \url{http://www.allectra.com/index.php/en/}.} %\cite{allectra}. 
In total,  six  \dwords{led} are placed in a hexagonal geometry. The direct light goes to the fiber, and the stray light to a \dword{sipm} used as a single reference sensor %, being a single reference sensor 
in the center. Fibers of length \SI{22.5}{m} (from Thorlabs $\phi$ \SI{800}{\micro\meter}, FT800UMT,~\footnote{Thorlabs\texttrademark{}, \url{https://www.thorlabs.com/thorproduct.cfm?partnumber=FT800UMT}.} %~\cite{ft800umt},
 and stainless-steel jacket) are used inside the cryostat. Each of these fibers is attached to a \numrange{1}{7}-fiber bundle (from Thorlabs $\phi$ \SI{200}{\micro\meter}, FT200UMT~~\footnote{Thorlabs\texttrademark{}, \url{https://www.thorlabs.com/thorproduct.cfm?partnumber=FT200UMT}.} %\cite{ft200umt}, 
 stainless-steel jacket common end, and black jacket at split ends), so that one fiber is finally installed at each \dword{pmt}. A diagram of the \dword{pddp} photon calibration system is shown in Figure~\ref{fig:dppd_5_1}. %Several tests to quantify the light losses of this design were performed with successful results. \fixme{citation? Answer: internal only; can rmv last sentence}

%%% anne to here 5/2 4pm
\begin{dunefigure}[Diagram of the photon calibration system to be implemented in \dword{pddp}.]{fig:dppd_5_1}
{Diagram of the photon calibration system to be implemented in \dword{pddp}}
\includegraphics[width=0.4\textwidth]{dppd_5_1}
\end{dunefigure}

Assuming the \dword{pddp} design for the DUNE FD with \num{720} \dwords{pmt}, \num{120} bundles, \num{120} fibers, \num{120} light sources, \num{120} flange \fdth{}s, and \num{20} reference sensors will be needed. The length of the fibers and bundles has to be calculated considering the exact position of the \fdth flanges. The number of flanges required to host \num{120} SMA \fdth{}s will depend on their size. However, alternatives to this design will be pursued with R\&D measurements in order to reduce the amount of fibers, study other options for the reference sensor, and increase the input light if necessary. In order to reduce the number of fibers, light diffusers can be used, so that one fiber can illuminate at least \num{4} \dwords{pmt}. For instance, a diffuser could be placed at the ground grid. 

%%%%%%%%%%%%%%%%%%%%%%%%%%%%%%%%%
\subsection{Validation Tests}
\label{sec:fddp-pd-5.2}

In order to validate the design, the most important result will come from the \dword{pddp} performance. In any case, since the fibers to be used in DUNE FD will be longer, dedicated calculations and measurements to confirm that sufficient light reaches the \dwords{pmt} will be performed. Also, alternative designs, will be validated in different laboratories. The possibility of using a diffuser can be tested in a vessel. The light source will also be validated by studying the different options in the lab. All these measurements will be performed at room temperature and in liquid nitrogen to test the behavior at cryogenic temperatures.

Once the design is fixed, basic characterization measurements will be performed on the fibers upon receiving them from the manufacturer. Those measurements will consist of providing light with a known source and measuring the output with a power meter. Measurements at cryogenic temperatures may not be needed at this point.

Finally, during the photon calibration system installation, each fiber and source will be re-tested to check that the expected light is arriving to each \dword{pmt} using a photodiode. A dedicated procedure will be designed with this purpose, similar to the one used in \dword{pddp}.

%%%%%%%%%%%%%%%%%%%%%%%%%%%%%%%%%%%%%%%%%%%%%%%%%%%%%%%%%%%%%%%%%%%%
\section{Photon Detector Performance}
\label{sec:fddp-pd-6}

%\fixme{the text could use a little polishing and shortening (anne) - response: leave as is}
To define the \dword{pds} performance, a good understanding of the light generation is needed. For this, optical simulations and a good knowledge of the light properties are required. The DUNE experiment expects to record not only accelerator neutrino interactions, but also rare non-beam events such as \dwords{snb} or nucleon decays. In those cases, an internal trigger is required: an optimized light collection system is hence mandatory. This section describes the tools developed in the consortium for the light simulation in large detector volumes for these purposes.

The main feature of a \lartpc detector is to collect electrons produced by the energy loss of charged tracks when crossing the volume. This signal provides a high resolution \threed image of the event. The reconstructed topology and the amount of charge collected gives the characterization of the tracks (identification and energy). Together with the charge, scintillation light is also produced in \lar. There are many advantages to collect and exploit the scintillation signal. As only a fraction of the initial energy deposition is converted into electrons, the rest being emitted as photons, light collection can improve the calorimetry of the detector. The light signal can provide the $t_0$ of the event, which is a necessary observable for a proper reconstruction. The study of the slow component can give insights into the purity of the \lar. 

When energy deposition occurs, either the knocked argon atom gets excited or an electron is ejected. For the latter case, the electron has a probability to be recaptured by an argon ion, which depends on the drift field and on the amount of energy deposited. In this case, an excited argon state is also produced. In order to decay to ground state, the excited argon combines with another argon atom, to form an excited eximer. A photon at \SI{127}{nm} is then emitted to allow the eximer to return to ground state. As the eximer can be formed in a singlet or triplet state, two time constants will be observed: the singlet at \SI{6}{ns}
and the triplet at \SI{1.3}{$\mu$s}. These principles are sketched in Figure~\ref{fig:dppd_6_0}.

\begin{dunefigure}[A sketch depicting the mechanism of light production in argon.]{fig:dppd_6_0}
{A sketch depicting the mechanism of light production in argon.}
\includegraphics[width=0.8\textwidth]{dppd_6_0}
\end{dunefigure}

%In the \dual technology, due to the amplification area, there are two light signals produced. The first one, S1, is made by scintillation processes when a charged particle crosses the \lar volume. The second signal, S2, is produced in the gaseous phase. As the drifting electrons enter in high field regions (such as the extraction field or the amplification field in the \dwords{lem}), their velocities increase and Townsend avalanches occur. The current of electrons will produce electroluminescence light with the same wavelength and similar time structure as for the S1 signal. %The minimum field needed to produce electroluminescence is $\sim$\SI{3.5}{kV/cm} at the gas density at cryogenic temperatures. 
%The S2 light is expected to be an irreducible background for the light studies in \dword{pddp}, as the detector will be on the surface. Indeed, the S2 signal can last as long as the total drift time of the electrons: \SI{0.625}{ms} per meter of drift at a drift field of \SI{500}{V/cm}.
In the \dual technology two light signals produced, one (S1) when a charged particle crosses the \lar volume and the second (S2) once the particle is above the liquid surface in the argon gas. 
As electrons drifting in the gas enter high field regions (such as the extraction field or the amplification field in the \dwords{lem}), their velocities increase and Townsend avalanches occur. This current of electrons produces electroluminescence light with the same wavelength and similar time structure as for the S1 signal. 
The S2 light is expected to be an irreducible background for the light studies in \dword{pddp}, since the detector is on the surface. Indeed, the S2 signal can last as long as the total drift time of the electrons: \SI{0.625}{ms} per meter of drift at a drift field of \SI{500}{V/cm}.

%Table~\ref{tab:dppd_t_6_0} summarizes the default optical parameters chosen for the light simulation methods described in the following subsection. The \lar optical properties are the subject of significant discussions in the community, in particular regarding the \lar absorption length and the Rayleigh scattering length. The former will affect the light yield collected whereas the latter will impact mostly its uniformity and timing resolution. The absorption/reflection of the VUV photons on stainless-steel (constituting the drift cage, cathode, extraction grid and ground grid) and on copper (on the \dword{lem} surfaces) are poorly known. The knowledge of those reflection coefficients is limited by the fact that they depend strongly on the polishing procedure. Hence, one cannot rely on the literature as the tooling will certainly be different. The measurement of the quantum efficiency of the \dwords{pmt} at vacuum ultra-violet (VUV) wavelengths requires a specific setup operating in vacuum as VUV photons are absorbed in air. For the construction of the   \dword{wa105} $3\times1\times1$\,m$^3$ detectorDP demonstrator, the \dword{pmt} quantum efficiencies were measured before and after the \dword{tpb} coating using an  \dword{led} that could emit light in the [\num{200}, \num{800}]\,nm range. Finally, the electroluminescence gain G, defined as the number of S2 photons produced per extracted drifting electron, is also subject to discussion. Experimental measurements of G have been performed in a setup quite similar to the amplification design of the DP technology, although the amount of photons emitted were measured above the \dword{lem}. In our case, the S2 photons are the ones leaving the \dword{lem} from below, which can be significantly lower. 
Table~\ref{tab:dppd_t_6_0} summarizes the default optical parameters chosen for the light simulation methods described in Section~\ref{sec:fddp-pd-6.1}. The \lar optical properties are the subject of significant discussions in the community, in particular regarding the \lar absorption length and the Rayleigh scattering length. The former affects the collected light yield  whereas the latter mostly affects its uniformity and timing resolution. The absorption and reflection of the \dword{vuv} photons on stainless steel (i.e., the drift cage, cathode, extraction grid and ground grid) and on copper (the \dword{lem} surfaces) are poorly known, 
largely because those reflection coefficients depend strongly on the polishing procedure. 

The measurement of the quantum efficiency of the \dwords{pmt} at \dword{vuv} wavelengths requires a specific setup operating in vacuum since \dword{vuv} photons are absorbed in air. The \dword{pmt} quantum efficiencies in the \dword{wa105} \dwords{pmt} were measured before and after the \dword{tpb} coating using a  \dword{led} that could emit light in the \numrange{200}{800}\,{nm} range. 

Finally, the electroluminescence gain $G$, defined as the number of S2 photons produced per extracted drifting electron, is also subject to discussion. Experimental measurements of $G$ have been performed in a setup quite similar to the amplification design of the \dual technology, although the measurements were made above the \dword{lem}~\cite{Monteiro:2012zz}.  In our case, the S2 photons are the ones leaving the \dword{lem} from below, where the number can be significantly lower.  

%\fixme{the S2 photons leave the LEM from below, and this quantity can be lower? I'm not getting this (anne). REsponse: nothing to change here}


\begin{dunetable}
[Default optical parameters chosen for the light simulation methods]
{lcc p{0.8\textwidth}}
{tab:dppd_t_6_0}
{Default optical parameters chosen for the light simulation methods.  Below the thick line are presented some quantities used in our studies although they are not linked to the optical properties of the \lar.}
 & \dwords{vuv} photons & Shifted photons \\ 
 & $\lambda$ = \SI{127}{nm} & $\lambda$ = \SI{435}{nm}\\ \toprowrule
 Absorption length & \multicolumn{2}{c}{$\infty$} \\ \colhline
 Rayleigh scattering length & \SI{55}{cm} & \SI{350}{cm}\\ \colhline
 Absorption coefficients & \num{100}\% & \num{50}\% \\ \colhline
 \lar refractive index & \num{1.38} & \num{1.25}\\ \colhline
 \dword{pmt} quantum efficiency & \multicolumn{2}{c}{0.2 }\\ \colhline
 Electroluminescence gain & \num{300}\\ 
\end{dunetable}

To understand the performance of the \dword{pds}, it is important to take into account the following indicators:
\begin{itemize}
\item Overall detected light yield, in \phel{}s per \si{MeV} of deposited energy in \lar{};
\item Uniformity of the light yield across the entire \lartpc active volume;
\item Event time resolution extracted from the detected photon signal. 
\end{itemize}

In turn, these indicators directly affect the strategy and performance of the \dword{dpmod} trigger system (Section~\ref{sec:fddp-pd-7}), and determine whether the \dword{pd} technical design is sufficient to meet the DUNE physics goals. These higher-level studies will be available on the \dword{tdr} timescale. 

Our current understanding of these performance indicators is largely based on \dword{pddp} simulations and the current status of the simulation work is discussed in detail in Section~\ref{sec:fddp-pd-6.1} Work is focused on \dword{pddp} in a first phase, and will expand to the \dword{dpmod}. For a realistic \dword{pddp} geometry, an average light yield of \SI{2.5}{\phel/MeV} is expected across the entire active volume. This promising yield assumes \num{36} \SI{8}{in} \dwords{pmt} located below the \dword{pddp} cathode plane, averaging to one \dword{pmt} per \si{m$^2$}. On the other hand, spatial non-uniformities in the \dword{pd} response are found to be important and need to be modeled in detail. Variations as large as one order of magnitude both parallel to the drift direction (due to geometrical effects and absorption of light by \lar) as well as perpendicular to it (due to light absorption on detector boundaries) are obtained. %\fixme{ref? Response: none available} 
The event time resolution due solely to light production and light propagation times, i.e., neglecting electronics and \dword{daq} effects for now, is expected to be of order $\mathcal{O}$(\SI{100}{ns}) and hence largely sufficient for our purposes. These initial low-level performance estimates will be refined with more realistic simulations and with \dword{pddp} data (Section~\ref{sec:fddp-pd-6.2}) in the future. They will also be extended to the full \dword{dpmod} geometry on the \dword{tdr} timescale.
%In turn, these indicators will directly impact the strategy and performance of the DUNE trigger system (Sec.~\ref{sec:fddp-pd-7}), and will determine whether the \dword{pd} technical design is sufficient to meet the DUNE physics goals. These higher-level studies will be available on the TDR timescale. Our current understanding of the performance indicators listed above is largely based on \dword{pddp} simulations. The current status of the simulation work is discussed in detail in Sec.~\ref{sec:fddp-pd-6.1}, work is focused on \dword{pddp} in a first phase, and then will be expanded to DUNE DP FD. For a realistic \dword{pddp} geometry, an average light yield of \SI{2.5}{\phel{}s/MeV} is expected across the entire active volume. This promising yield is obtained  by assuming thirty-six 8-inch \dwords{pmt} located below the \dword{pddp} cathode plane, averaging to one \dword{pmt} per m$^2$. On the other hand, spatial non-uniformities in the \dword{pd} response are found to be important and need to be modeled in detail. Variations as large as one order of magnitude both parallel to the drift direction (due to geometrical effects and absorption of light by \lar) as well as perpendicular to it (due to light absorption on detector boundaries) are obtained. The event time resolution due to light production and light propagation times, hence neglecting electronics and \dword{daq} effects for now, is expected to be of order $\mathcal{O}$(\SI{100}{ns}) and hence largely sufficient for our purposes. These initial low-level performance estimates will be refined with more realistic simulations and with \dword{pddp} data (Sec.~\ref{sec:fddp-pd-6.2}) in the future. They will also be extended to the full FD geometry on the TDR timescale.
%%%%%%%%%%%%%%%%%%%%%%%%%%%%%%%%%
\subsection{Simulations}
\label{sec:fddp-pd-6.1}

At zero drift field, when the electron recombination is maximum, roughly \SI{40000}{$\gamma$/MeV} are produced. At the nominal drift field of \SI{500}{V/cm}, then \num{24000}{$\gamma$/MeV} are generated. For reference, the energy deposited by a \dword{mip} track is \SI{2.12}{MeV/cm}. Given the size of the \dword{pddp} ($6\times6\times6$\,m$^3$) and the fact that it is located on surface, roughly \num{100} muons are expected to cross the fiducial volume during the \SI{4}{ms} time window of the \dword{daq}. With a full \dword{geant4}~\cite{geant4} simulation, it takes more than three hours to propagate all the photons emitted by a single \dword{mip} track crossing the \dword{pddp} detector. A full optical simulation is hence computationally prohibitive. Three simulation approaches are being explored to provide the light simulation needed for the design optimization of the \dword{dpmod}. %, described in the following.


\subsubsection{Generation of light maps}
\label{subsec:fddp-pd-6.1.1}

In this method, the photons are propagated in a full dedicated \dword{geant4} simulation only once. The main light characteristics needed for light studies (photon detection probability, called \textit{visibility} hereafter, and time profile) are stored in a map in a ROOT \cite{root} file format which can then be read by any other simulation program. This work was done first using LightSim, a dedicated software developed at LAPP, France. These maps have been adapted to be readable by \larsoft, where light maps are known as \textit{photon libraries}. Work to generate them directly in \larsoft is in progress, in particular for S2 light, which has not yet been simulated for \dual technology.

In the dedicated \dword{geant4}  code, special care has been taken to precisely describe all subdetector components that might affect the light propagation: \dword{lem} plates, extraction grid, \dword{fc} rings, the cathode and its supporting structure, and the ground grid above the \dwords{pmt}. The \lar fiducial volume is then divided into voxels of \SI{25}{cm^3} and \num{e8} photons are isotropically generated at the center of each voxel. The number of photons reaching each \dword{pmt}, and their arrival times are stored. The light map can then be built from these results. For each voxel and for each \dword{pmt}, the visibility is computed as: $w=N\gamma^{\textrm{collected}}/N\gamma^{\textrm{generated}}$. In order to be able to reproduce the time profile, each distribution is fit to a Landau function. From the fits, three parameters are extracted: the minimum time for photons to arrive to the \dword{pmt}, $t_0$; the peak of the distribution, $t_{\textrm{peak}}$ from the Landau most probable value (MPV); and the distribution spread, the $\sigma$ of the Landau function. These three parameters are stored in the light maps for each \dword{pd}. The same procedure is done for the gaseous phase, although the voxel size is smaller in height (only \SI{5}{mm}). In Figure~\ref{fig:dppd_6_1_1_ab}, two fitted time distributions are presented. As is visible, the shapes of the time distributions depend strongly on the source-to-\dword{pmt} distance. For close sources, the distributions are very sharp and the Landau description may not be the optimal function to use. On the other hand, for longer distances, the distribution is broader and the Landau fit reproduces the simulations quite well. %In order to minimize the amount of parameters to be stored in the map, the Landau descriptions for all cases were kept as only a small fraction of the fits could be considered problematic.
For practical purposes, the Landau parametrization was used for all cases.

\begin{dunefigure}[Landau fits of the travel time distributions for a sources close to and far from the \dword{pmt}.]{fig:dppd_6_1_1_ab}
{Landau fits (red line) of the travel time distributions (black histogram) for a source close to (left) and far from (right) the \dword{pmt}.}
\includegraphics[width=0.45\textwidth]{dppd_6_1_1_a}
\includegraphics[width=0.45\textwidth]{dppd_6_1_1_b}
\end{dunefigure}

As the map has been computed with discrete entries, an interpolation of the four light parameters ($w$, $t_0$, $t_{\textrm{peak}}$, $\sigma$) between the actual source position and the closest voxel centers is performed. An example of the distribution of the visibility 
and its \threed interpolation is presented in Figure~\ref{fig:dppd_6_1_1_cd}. The loss of photons due to the cathode and ground grid are visible. For the \dword{pddp} cathode and supporting structure design, and using the default optical parameters presented in Table~\ref{tab:dppd_t_6_0}, it has been shown that up to $\sim$\num{70}\% of the photons generated in the active volume are absorbed by those structures before reaching the \dword{pmt} array.

%During the generation of the light maps, the light propagation parameters are the ones presented in Table~\ref{tab:dppd_t_6_0}. 
Table~\ref{tab:dppd_t_6_0} presents the light propagation parameters used during the generation of the light maps. 
%One can study afterwards the loss of photons due to the \lar absorption length using an approximation of the probability of the photon to be absorbed by the medium as: $p_{\textrm{abs}} = \exp(-\frac{D_{\textrm{travel}}}{\lambda_{\textrm{abs}}})$. For the study of other light propagation parameters (Rayleigh scattering and absorption on the stainless-steel and copper) new maps have to be generated.
After generation, it is possible to study the loss of photons due to absorption using an approximation of the probability that the medium absorbs the photon %to be absorbed by the medium 
as: $p_{\textrm{abs}} = \exp(-\frac{D_{\textrm{travel}}}{\lambda_{\textrm{abs}}})$. For the study of other light propagation parameters (Rayleigh scattering and absorption on the stainless-steel and copper) new maps have to be generated.

It takes roughly three days of computing to generate the light maps for \dword{pddp}, even though only %1/8$^{th}$ 
an eighth of the voxels need to be simulated, since the detector and the \dword{pmt} positioning are symmetric. Generating maps for larger volumes such as the \dword{dpmod}, where the maximum source-to-\dword{pmt} distance is around \SI{60}{m}, could be too time-consuming. Moreover, the light simulation for the \dword{dpmod} is foreseen to drive the optimization of the positioning of the \dwords{pmt} and will guide the studies of possible implementation of light reflectors. As most of the light propagation parameters in \lar are still subject to large uncertainties, these studies will have to %be performed 
consider various absorption and diffusion values. Therefore, it is crucial to find %be able to have 
a faster way to %get a quite reliable light simulation, 
simulate the light reliably, even at the cost of losing some precision.

\begin{dunefigure}[Evolution of the visibility seen by a central \dword{pmt}]{fig:dppd_6_1_1_cd}
{Evolution of the visibility seen by a central \dword{pmt} (see arrow) in \dword{pddp} as a function of different source positions in $x$ and $z$ ($y$ is set at \SI{0}{mm}). The position of the cathode and the ground grid are highlighted. Results are limited by the number of photons generated (\num{e7} photons per voxel), and voxels with less than \num{50} photons arriving to the \dword{pmt} are not taken into account. Left: discrete values from the maps, right: after \threed interpolation.}
\includegraphics[width=0.45\textwidth]{dppd_6_1_1_c}
\includegraphics[width=0.45\textwidth]{dppd_6_1_1_d}
\end{dunefigure}

\subsubsection{Parametrization from the Light Maps}
\label{subsec:fddp-pd-6.1.2}

Without considering the border effects where the photons are mostly absorbed,
the visibility and the time profile depend only on the source-to-\dword{pmt} distance.

%This approach has been followed for the SBND~\cite{sbnd} light simulation and is under consideration for the \dword{dpmod} as well. In Figure~\ref{fig:dppd_6_1_2}, the evolution of the visibility and the peak time as a function of the source-to-\dword{pmt} distance are shown. As these plots have been generated from the light maps, where the borders are taken into account, the same evolutions are also presented only for voxels at least \SI{1}{m} away from the active volume boundaries. For the visibility, the structure is quite complicated when taking all the voxels highlighting the complexity of the light simulation in a closed space. When looking at voxels away from the boundaries, one can see a clear correlation between distance and visibility. As for the time distribution (here for the peak time, but same goes for $t_0$ and $\sigma$ parameters), one can notice two different regimes for short and large distance (the transition being at around \SI{2}{m}).
This approach has been followed for the SBND~\cite{sbnd} light simulation and is under consideration for the \dword{dpmod} as well. In Figure~\ref{fig:dppd_6_1_2} the evolution of the visibility and the peak time as a function of the source-to-\dword{pmt} distance are shown. On the left, the borders are taken into account, and the visibility structure is quite complicated due to the complexity of the light simulation in a closed space. However, on the right, the same evolutions are presented only for voxels at least \SI{1}{m} from the active volume boundaries, and a clear correlation between distance and visibility is observed. As for the time distribution (here for the peak time, but the same goes for $t_0$ and $\sigma$ parameters), one can notice two different regimes when looking at all voxels, with a transition at a propagation distance of around  \SI{2}{m}. When considering only the central voxels, the evolution of the peak time is fully correlated with the propagation distance. The parametrization of the light propagation parameters as a function of the propagation distance is a promising option for very large volumes such as the \dword{dpmod}, at least for light sources far from the fiducial volume boundaries. 
%\fixme{previous pgraph needs some work (anne) - they provided new text, above}

This preliminary study is quite encouraging for the light simulation in the \dword{dpmod}, at least for light sources far from the fiducial volume boundaries. Since it is complicated to disentangle the effects due to the propagation and absorption parameters from the light maps, a careful dedicated study should be performed to get parametrization of the visibility and time distribution parameters as a function of the photon traveling distance. %\fixme{sounds like you're parameterizing parameters...? Response: Not yet, but yes this is the purpose.}

\begin{dunefigure}[Evolution of the visibility and peak time as a function of source-\dword{pmt} distance (preliminary)]{fig:dppd_6_1_2}
{Evolution of the visibility (top) and peak time (bottom) as a function of the source-\dword{pmt} distance as simulated in the \dword{pddp} geometry (Preliminary study). On the left, all voxels are considered, on the right only the voxels at least 1\,m away from the fiducial border are considered.}
\includegraphics[width=0.75\textwidth]{dppd_6_1_2}
\end{dunefigure}

\subsubsection{Analytical approach}
\label{subsec:fddp-pd-6.1.3}

The propagation of light in a uniform material such as \lar can be described by the Fokker-Planck diffusion equation:

$$\frac{\partial}{\partial t}p(x,y,z,t) = D\left[\frac{\partial^2}{\partial x^2}p(x,y,z,t) + \frac{\partial^2}{\partial y^2}p(x,y,z,t) + \frac{\partial^2}{\partial z^2}p(x,y,z,t)\right]$$ 

where $D$ is the diffusion coefficient. In an unbound medium, the Fokker-Planck equation is solved by the Green function:

\begin{eqnarray*}
G(\textbf{r}, t; \textbf{r}_0, t_0) &=& \frac{1}{[4\pi D c (t-t_0)^{3/2}]}\exp\left(-\frac{|\textbf{r}-\textbf{r}_0|^2}{4Dc(t-t0)}\right) \\
D &=& \frac{1}{3(\mu_A + (1-g)\mu_S)}
\end{eqnarray*}

where $\mu_A$ and $\mu_S$ are the absorption and scattering coefficients, respectively (both in units of \si{m$^{-1}$}), $g$ is the average scattering cosine ($g$ = \num{0.025}). In \lar with the default optical properties in Table~\ref{tab:dppd_t_6_0}, $D$ = \SI{18.8}{\cm}. In a bound medium, with full absorption of the photons by the \dword{fc} and \dwords{lem}, a few additional techniques have to be used to obtain a solution. With this method, it takes only a few \si{ms} to generate the photon density at a given \dword{pd} from a specific point source. From preliminary studies, relatively good agreement between analytical approach and full simulation has been found. In particular, the arrival time distributions of photons on the \dwords{pmt} are well reproduced. The only drawback is that one cannot easily implement or study a complicated geometry including regions that are semi-transparent to light. Hence, the visibilities generated by the two methods are not in agreement. 
in the overall light yield, but have a very similar trend in terms of spatial dependences. Studies to improve the analytical method results are in progress since this approach could be extremely powerful for physics studies in the \dword{spmod}.
%%%%%%% anne to here Wed night  - 

\subsubsection{Simulation of light yield}
\label{subsec:fddp-pd-6.1.4}
The light collected per \dword{pmt} can be simulated together with the charge for crossing tracks in a standard simulation code where a detailed description of the detector is not needed. At each step of the track propagation, the energy deposited is computed by \dword{geant4}. This energy is converted into number of electrons and photons produced. As for the light simulation, the number of photons reaching each \dword{pmt} and their time of arrival is now obtained from the light maps. As an example, the light yield from a uniform generation of \SI{10}{MeV} electrons in the active volume is shown in Figure~\ref{fig:dppd_6_1_4}. The number of \phel{}s/\si{MeV} shown is summed over all \dwords{pmt} and average over the $y$ axis ($z$ being the drift direction). One can notice the large spread in terms of light yield.

\begin{dunefigure}[Light yield in terms of \phel/\si{MeV} summed over all \dwords{pmt}]{fig:dppd_6_1_4}
{Light yield in terms of \phel/MeV summed over all \dwords{pmt} and averaged along the y-axis. The mean of all voxels gives a light yield of \SI{2.5}{\phel/MeV}, although the distribution is not uniform, in particular along the $z$ (drift) axis.}
\includegraphics[width=0.6\textwidth]{dppd_6_1_4}
\end{dunefigure}

For larger volumes such as the \dword{dpmod}, the light maps might be too big and the time spent accessing the four parameters might strongly reduce the speed of the simulation. Either the parametrization method or the analytical approach are foreseen to replace the current light map usage, the exact strategy is yet to be defined.

%%%%%%%%%%%%%%%%%%%%%%%%%%%%%%%%%
\subsection{Light Data in \dual Prototypes}
\label{sec:fddp-pd-6.2}

The  \dword{wa105} was operated from June to November 2017 with cosmic data. About \num{5} million light events were taken with various configurations. The study of the S1 light as a function of the drift field was performed. An example of an averaged waveform fitted to a fast and a slow scintillation components is shown in Figure~\ref{fig:dppd_6_2}. The amount of S2 light can be monitored as a function of the extraction and \dword{lem} amplification fields.

\begin{dunefigure}[Averaged waveform of the S1 light signal taken with one \dword{pmt} from the  \dword{wa105}]{fig:dppd_6_2}
{Averaged waveform of the S1 light signal taken with one \dword{pmt} from the  \dword{wa105}, fitted with a function (red line) that is the sum of a Gaussian, parametrized by $t_0$ and $\sigma$, and two exponential functions, with decay time constants $\tau_{fast}$ and $\tau_{slow}$, and normalization factors $A_{fast}$ and $A_{slow}$}
\includegraphics[width=0.6\textwidth]{dppd_6_2}
\end{dunefigure}

Light maps have also been generated with the demonstrator geometry, and data-\dword{mc} comparisons are ongoing. The preliminary results look promising, although the statistics in each setting and the relatively small size of the detector still constitute a challenge to extract the entire optical properties of the \lar.


%%%%%%%%%%%%%%%%%%%%%%%%%%%%%%%%%
\subsection{Simulation of Physics Events}
\label{sec:fddp-pd-6.3}

A preliminary study to understand whether the \dual \dword{pd} technical design meets the experiment's physics requirements has been performed. In this study, event topologies of interest for DUNE physics have been simulated using \larsoft fast optical simulation tools.

The simulation framework used represents the current state of the art. It includes realistic models for the primary scintillation production yields in \lar, for Rayleigh scattering in \lar, for detector optical properties (such as \dword{fc} reflectivity and cathode transparency), for the density of \dwords{pmt} underneath the cathode, and for the quantum efficiency of the \dword{tpb}-coated \dwords{pmt} (taken to be \num{20}\%). On the other hand, these simulations do not yet include the full \dword{dpmod} geometry, but are rather performed in a \dword{pddp} geometry with the same average \dword{pmt} density as the one proposed here for the \dword{dpmod} (one \dword{pmt} per \si{m$^2$}). Relevant aspects such as secondary scintillation light emission in gaseous argon (a nuisance for event $t_0$ determination), light absorption in \lar, electronics effects, reconstruction effects, and background contributions coming from $^{39}$Ar decays are not accounted for either in this study. While more realistic simulation results including the above effects will be produced on the DUNE \dword{tdr} timescale, this preliminary study already provides a sense of the capabilities of the planned \dword{pd} design.

Figure~\ref{fig:dppd_6_3_1} shows the expected light yield for \dword{snb} neutrino CC interactions. As a representative \nue flux from a \dword{snb}, we assume a Fermi-Dirac distribution with $T$=\SI{3.5}{\MeV} temperature and no neutrino oscillation effects, yielding an average neutrino energy of about \SI{11}{MeV}. Low-energy \nue CC interactions throughout the entire \lartpc active volume are generated with the \larsoft-based Marley package. For the assumed \dword{snb} neutrino flux and for a single interacting neutrino (hence, after convoluting flux and cross-section effects), Marley expects about \SI{19}{\MeV} of energy deposited in the \lar active volume, primarily from the final state electron and from nuclear de-excitation gamma rays. The left panel of Figure~\ref{fig:dppd_6_3_1} shows a broad light yield distribution, averaging at about \num{50} detected \phel{}s per interaction and after summing all \dwords{pmt}. This is as expected from the light yield distributions per deposited energy shown in Figure~\ref{fig:dppd_6_1_4}. The right panel shows the fraction of \dword{snb} \nue CC interactions within the \lartpc active volume above a given \phel detection threshold, as a function of the \phel threshold. From the figure, we conclude that about a \SI{70}{\%} fraction of \dword{snb} \nue CC interactions would be seen by the \dword{pd}, if the detector threshold was set at \num{10}~\phel{}s on the sum of the \dword{pmt} charges.

\begin{dunefigure}[Response for simulated \dword{snb} neutrino interactions  (\dword{pddp} geometry)]{fig:dppd_6_3_1}
{Photon detector response for simulated \dword{snb} neutrino interactions in the \dword{pddp} geometry. Left panel: distribution of detected \phel{}s per neutrino interaction, for \dword{snb} \nue CC interactions throughout the active volume. Right panel: fraction of \dword{snb} \nue CC interactions above \phel threshold, as a function of the \phel threshold.}
\includegraphics[width=0.44\textwidth]{dppd_6_3_1_1} \hfill 
\includegraphics[width=0.44\textwidth]{dppd_6_3_1_2} 
\end{dunefigure}

Figure~\ref{fig:dppd_6_3_2} shows the corresponding plots for a representative nucleon decay final state in DUNE, namely $p\to\bar{\nu}K^+$. Nucleon decay events are generated using \dword{genie}, accounting for both initial and final state nuclear effects in argon nuclei. Particles exiting the nucleus are then propagated in \lar using all relevant, \dword{geant4}-based, physics processes. The deposited energy per nucleon decay, of order \SI{300}{\MeV}, is much higher than the one per \dword{snb}  neutrino interaction. As a result, the expected light yield for $p\to\bar{\nu}K^+$ events throughout the active volume, shown in the left panel of Figure~\ref{fig:dppd_6_3_2}, averages to about 800~\phel{}s in this case. The right panel of Figure~\ref{fig:dppd_6_3_2} shows that about a 98\% fraction of %$p\to\bar{\nu}K^+$ 
\ptoknubar decays in the TPC active volume are expected to be seen by the \dword{pd}, for a \dword{pd} threshold of 10~\phel{}s on the \dword{pmt} charge sum.

\begin{dunefigure}[Response for simulated nucleon decays (\dword{pddp} geometry)]{fig:dppd_6_3_2}
{Photon detector response for simulated nucleon decays in the \dword{pddp} geometry. Left panel: distribution of detected \phel{}s per nucleon decay, for $p\to\bar{\nu}K^+$ decays throughout the active volume. Right panel: fraction of %$p\to\bar{\nu}K^+$ 
\ptoknubar decays above \phel threshold, as a function of the \phel threshold.}
\includegraphics[width=0.44\textwidth]{dppd_6_3_2_1} \hfill 
\includegraphics[width=0.45\textwidth]{dppd_6_3_2_2} 
\end{dunefigure}

%%%%%%%%%%%%%%%%%%%%%%%%%%%%%%%%%%%%%%%%%%%%%%%%%%%%%%%%%%%%%%%%%%%%
\section{Photon Detector Operations}
\label{sec:fddp-pd-7}

%%%%%%%%%%%%%%%%%%%%%%%%%%%%%%%%%
\subsection{Trigger Strategy}
\label{sec:fddp-pd-7.2}

As explained in Section~\ref{sec:fddp-pd-1.5}, the \dword{pds} will operate in different acquisition modes. These modes include the external trigger, which is the case of the beam events; the trigger for non-beam events such as  \dwords{snb}; and the calibration mode. 

In the \lartpc there are different uses of the light signal: cosmic ray and track timing for the reconstruction; non-beam events trigger such as \dword{snb}, atmospheric neutrinos, and proton decay; and calorimetry, as the light and charge signal are anti-correlated. These physics studies imply different requirements in terms of dynamics of the electronics and data sampling, from a few \phel to a much higher level.

For the non-beam event trigger strategies, the requirements can be very different. In the event of a nearby (\SI{10}{kpc})  \dword{snb}, it is expected that a few thousands of neutrinos will homogeneously interact in the \dword{detmodule} for a period as long as $\sim$\SI{100}{s}. Hence, the  \dword{snb} trigger strategy is mostly driven by the energy threshold set for $\nu$ detection and its efficiency: \SI{30}{MeV} is sufficient for a galactic SN, \SI{5}{MeV} is needed for a burst in Andromeda. A high-efficiency trigger for proton decay events has to be designed considering the worst case scenario, e.g., the event happening at the top of the \dword{detmodule}, \SI{12}{m} away from the closest \dword{pmt}. In order to minimize the amount of spurious triggers, one can think of signal thresholds for a cluster of close-by \dwords{pmt}.

All these important studies will be further investigated once a reliable light simulation of the \dword{dpmod} is available. For the \dual  technology, the main light trigger concerns are the amount of light collectable for a photon traveling distance of \SI{12}{m} and the S1-S2 separation. The data that is collected in the \dword{pddp} will provide crucial inputs for the optimization of the \dword{dpmod} light-collection system and for the design of an efficient trigger strategy for rare non-beam events. 

The \dword{pds} trigger design is flexible so as to fulfill the different physics requirements explained before. The light readout \dword{fe} board will be in charge of the \dword{pds} trigger generation. The trigger will be decided based on the coincidence of several \dword{pmt} signals over a threshold during a time window. The number of \dwords{pmt} that contribute to the trigger, the signal threshold and the length of the coincidence time window will be programmable on-line to be able to adapt to different physics cases.


%%%%%%%%%%%%%%%%%%%%%%%%%%%%%%%%%
\subsection{Data Quality Monitoring}
\label{sec:fddp-pd-7.3}

The \dwords{pmt} installed at the bottom of the tank will be operated for \numrange{10}{20} years with no possibility to access them. Monitoring tools to ensure data quality of the \dword{pds} will have to be developed to catch any malfunctioning detector before data analysis. For instance, the amount of dark noise and the stability of the \dword{pmt} response will have to be monitored over time. For the gain evolution, either studies of standard candles, e.g., from Michel electrons or average collected light produced by cosmic tracks, or with the dedicated calibration system are considered.

Monitoring tasks were performed during the six months of operation of the \dword{wa105} with no dedicated light calibration system. This and the forthcoming operation of the \dword{pddp}, will again provide crucial input towards the \dword{pds} monitoring system in the \dword{dpmod}.

%%%%%%%%%%%%%%%%%%%%%%%%%%%%%%%%%%%%%%%%%%%%%%%%%%%%%%%%%%%%%%%%%%%%
\section{Interfaces}
\label{sec:fddp-pd-8}

The \dword{pds} will have several interfaces with other subsystems and the global DUNE systems. The interface documents related to \dword{dpmod} \dword{pds} are given in Table~\ref{tab:dppd_t_8}. Only part of the basic interfaces are summarized below. 

\begin{dunetable}
[\dual \dword{pd} interface documents]
{|l|c| p{0.8\textwidth}}
{tab:dppd_t_8}
{\dual \dword{pd} interface documents}

\dual \dword{pd} Interface Document & DUNE docdb number \\ \toprowrule
\dword{dpmod} electronics & 6772 \\
\dword{dpmod} \dword{hv} & 6799 \\
\dword{daq} & 6802 \\
\dword{cisc} & 6781 \\
DUNE Physics & 7087 \\
Software and Computing & 7114 \\
Calibration & 7060 \\
\dword{itf} & 7033 \\
Detector and Facilities (LBNF) Infrastructure & 6979 \\
Installation & 7006 \\
\end{dunetable}


\begin{itemize}

\item \dword{dpmod} electronics: The \dword{pds} shares the same \dword{fe} electronics standard as the charge readout, which is $\mu$TCA based \cite{utca}. Specifications of both \dword{pds} and \dword{fe} electronics will be determined by the simulations and \dword{pddp} data.

\item \dword{hv}: This interface includes the consideration of the distance between the cathode and the \dword{pmt} planes, power dissipation from the \dwords{pmt} and the combined impact on the simulations.

\item \dword{daq}: The hardware interface is mainly through optical fibers. \dword{dpmod} \dword{pds} provides trigger and data in continuous streaming;  the interface also includes the \dword{daq} software.

\item \dword{cisc}: The main interface points are the layout of the cryogenic instrumentation (e.g., purity monitors and light emitting system for the cameras) and the \dword{pmt} support structures and cabling; and the slow control and the \dword{pds} power supplies and calibration system.

\item DUNE physics: \dual \dword{pd} has interfaces with the overall physics requirements on energy and time together with classification of events, decay modes and neutrino flavors.

\item Software and Computing: This interface is on the development of simulation, reconstruction and analysis tools.

\item Calibration: The \dword{pds} is participating in the Global Calibration Task Force and will provide handles to allow global monitoring of the \dword{pmt} performance.

\item \dword{itf}: The operations at the \dword{itf} are described in Section~\ref{sec:fddp-pd-9.2}. The interface items can be summarized as shipping and receiving of the \dword{pds} components and basic testing and repairing at the facility. The interface also includes recycling and returning the packaging materials.

\item Detector and Facilities (LBNF) Infrastructure: The \dword{pds} %will have 
\dword{pmt} mounting structures stand on the cryostat membrane; cold cables are routed in cable trays to the ceiling \fdth flanges and racks and to cable trays on top of the cryostat. Other interfaces with the facility include access to conventional facilities and participation in the detector safety systems. \fixme{check this one; small edits}

\item Installation: This interface %will be through 
includes the transportation of the \dword{pds} components to and between underground areas, clean room activities and storage, and installation coordination with the other teams. 

\end{itemize}

%%%%%%%%%%%%%%%%%%%%%%%%%%%%%%%%%%%%%%%%%%%%%%%%%%%%%%%%%%%%%%%%%%%%
\section{Installation, Integration and Commissioning}
\label{sec:fddp-pd-9}

%%%%%%%%%%%%%%%%%%%%%%%%%%%%%%%%%
\subsection{Transport and Handling}
\label{sec:fddp-pd-9.1}

The \dpnumpmtch \dwords{pmt} of the \dword{pds} are shipped from various locations following base and cable assembly for the \dword{tpb} coating at the \dword{itf}. %The shipping boxes will contain \num{24} \dwords{pmt} resulting in \num{30} deliveries to total \num{720} \dwords{pmt}. 
They are shipped in boxes of \num{24}, for a total of \num{30} deliveries.
The \dwords{pmt} are individually wrapped with special wrapping materials. 

The \dwords{pmt} are placed in modular shock-absorbing assemblies inside the boxes. %The assemblies will also allow a limited amount of safe inclination. 
The boxes have integrated pallets for easy handling and short distance transportation, and a limited amount of safe inclination for the assemblies is allowed. The \dwords{pmt} will reach the \dword{itf} by means of air and ground transportation. Each box has a dedicated bar-code which visible on each side. This bar-code is also associated with the shipping documents. 

%%%%%%%%%%%%%%%%%%%%%%%%%%%%%%%%%
\subsection{Integration and Testing Facility Operations}
\label{sec:fddp-pd-9.2}

The \dword{itf} receives the \dword{pmt} boxes and also manages a shipping and delivery database. The received status of the boxes is available to the \dual \dword{pd} consortium as the boxes arrive at the \dword{itf}. The \dword{pds} characteristics database managed by the \dual \dword{pd} consortium is updated accordingly to reflect the received status of the contents of the boxes. Each \dword{pmt} assembly gets an identifying bar code that is directly connected to the \dword{pds} characteristics database. This database stores the \dword{pmt} serial number, the base board serial number, special information about \dword{tpb} coating and assembly if any, and performance and calibration characteristics. This database  forms the basis of the operations database, providing the initial calibration values. It also stores information about the \dword{itf} tests and underground installation and commissioning tests.

The \dword{tpb} coating is performed at the \dword{itf} in the coating stations, after which  the \dwords{pmt} are placed back in their boxes and dedicated testing electronics are connected to the \dword{pmt} cables soldered to the \dword{pmt} bases. The test electronics enables connecting several \dwords{pmt} at a time. The tests include basic functionality checks of both the \dwords{pmt} and the base boards to assess the performance after transportation. No detailed performance characteristics are measured at the \dword{itf}. The tests are performed in a dedicated room with light and climate control. Once the performance of the \dwords{pmt} in a box is validated, the boxes are closed with the original covers. Before closing, additional quality checks on the shock absorbing assemblies are made.

The preparation of the \dword{pmt} boxes for underground transportation includes installing holding and lifting fixtures to the top and sides %. The fixtures will 
that allow crane operation. The boxes are delivered to the surface station by ground transportation with %relevant modification
appropriate updates to the shipping database.

%%%%%%%%%%%%%%%%%%%%%%%%%%%%%%%%%
\subsection{Underground Installation and Integration}
\label{sec:fddp-pd-9.3}

Once the \dword{pmt} boxes are underground, the same top and side covers are opened as at the \dword{itf}. The \dwords{pmt} are carried to the underground storage area in sub-units of the modular shock-absorbing assemblies. The underground storage area for the \dword{pds} is expected to be sufficiently large to store at least \num{30} \dwords{pmt}, enabling continuous installation operations.

The removal of the individual \dword{pmt} wrappings is done in the clean room. \Dwords{pmt} together with their base boards undergo visual inspection by the \dword{pds} installation supervisor. Once signed-off, the installation can proceed with multiple \dwords{pmt} at a time by multiple teams. Cabling is carried out in parallel and relevant database updates are made in situ. The installation time management is done in coordination with the cathode and \dword{fc} installation groups.

The bundles of cables are routed through the cable trays along the cryostat walls from the \dword{pds} flanges. Following the mechanical mounting of the \dwords{pmt} to the cryostat floor, the \dword{pmt} cables are 
connected to the cables coming from the flanges. In parallel, the calibration fibers are installed and routed through cable trays.

%%%%%%%%%%%%%%%%%%%%%%%%%%%%%%%%%
\subsection{Commissioning}
\label{sec:fddp-pd-9.4}

The commissioning of the \dword{pds} is performed in partitions. The size of a single partition will mainly be determined by the \dword{daq} and the \dword{hv} systems. The \dword{daq} and \dword{hv} partitions are commissioned, including the relevant control systems, prior to the connection of the \dwords{pmt} to these systems. Once the physical sector corresponding to a partition is installed, the \dwords{pmt} are powered up, and basic functionality and performance checks are performed. These include pedestal data taking, i.e., recording event data with external periodic triggering, and tests with the calibration system where the data taking is triggered in synchronization with a light source, as described in Section \ref{sec:fddp-pd-5}.

As a result of the commissioning tests, the basic performance characteristics of the \dwords{pmt}, e.g., the dark count rate and gain, are measured in their final places. Installation-related issues are identified and eliminated at this stage. A commissioned sector %will be 
becomes a part of the overall detector and can join the global calibration data taking and commissioning.

%%%%%%%%%%%%%%%%%%%%%%%%%%%%%%%%%%%%%%%%%%%%%%%%%%%%%%%%%%%%%%%%%%%%
\section{Quality Control}
\label{sec:fddp-pd-10}


%%%%%%%%%%%%%%%%%%%%%%%%%%%%%%%%%
 \subsection{Production and Assembly}
 \label{sec:fddp-pd-10.1}
 
The \dword{qc} performed at the different institutions labs includes reception of \dwords{pmt} from the manufacturer and execution of the \dword{qc} tests to accept or return the \dwords{pmt} according to the acceptance and rejection criteria.

\begin{itemize}
\item The \dword{pmt} support structure design is validated by immersing its mounted \dword{pmt} in cryogenic temperatures and at an over-pressure equivalent to %pressure of the 
a depth of \SI{12}{m} in \lar{}. % of the detector.
\item Design validation tests are carried out to confirm that the \dword{pmt} base design fulfills the specifications at room and cryogenic temperatures. A cable with SHV connector is soldered to each \dword{pmt} base to facilitate the different base and \dword{pmt} tests and the final \dword{pmt} connection during the installation. The \dword{pmt} bases are labeled (on the cable) in order to keep track of them. After production of the \dword{pmt} base boards they are individually tested before mounting to the \dword{pmt} to verify that components are correctly mounted. Later they are cleaned and tested at maximum voltage in an argon gas environment to confirm that there are no sparks on these (worst case conditions).
After mounting the bases on the \dwords{pmt} they are tested again in argon gas at maximum voltage to confirm that there are no sparks due to bad soldering.
\item All the light readout units (\dword{pmt} + base + support) are tested and characterized in liquid nitrogen in order to check their performance at cryogenic temperature and to obtain a database with the most important parameters from each \dword{pmt} (gain versus voltage, dark counts, etc.). The \dword{pmt} base number attached to each \dword{pmt} is also included in the database. 
\item The wrapping materials and techniques are studied with one fully assembled light readout unit. The handling, transportation and installation scenarios are carefully studied and the transportation box design is validated. The transport box and \dword{pmt} wrapping must  ensure complete darkness. 
\item The light output of the \dwords{led} and the fibers' light transmission from the photon calibration system are measured with a power meter.
\end{itemize}

%%%%%%%%%%%%%%%%%%%%%%%%%%%%%%%%%
 \subsection{Post-Factory Installation}
 \label{sec:fddp-pd-10.2}
 
Upon receipt at \dword{itf}, the \dwords{pmt} go through verification measurements in order to identify any items damaged during transport.  Gain versus voltage and dark current values are compared with those obtained before transportation.

The \dword{tpb} coating is also performed at the \dword{itf}. The first few samples undergo microscopic examination and surface uniformity tests, and the coating procedure is validated. The production \dwords{pmt} are randomly sampled for basic coating \dword{qc}.

After  transport from the \dword{itf} to \surf the \dwords{pmt} are tested again before installation to confirm that no damage occurred during the last stage of transportation. During the installation, the \dwords{pmt} database is updated with the position in the \dword{detmodule} of each \dword{pmt} (identified by its serial number and base number). After installation, the full connection from the \dword{fe} to the \dwords{pmt} is checked. The \dword{fe} channel and splitter number connected to each \dword{pmt} are also included in the \dword{pmt} database. %At the moment that it could be possible to make darkness in the detector the \dwords{pmt} will be tested applying voltage and checking the signal with a scope or with the FE electronics if they are already available.
Once it is possible to fully darken the \dword{detmodule}, voltage can be applied to the \dwords{pmt} to test the signal with a scope or, if available, with the \dword{fe} electronics.

%%%%%%%%%%%%%%%%%%%%%%%%%%%%%%%%%%%%%%%%%%%%%%%%%%%%%%%%%%%%%%%%%%%%
\section{Safety}
\label{sec:fddp-pd-11}

Safety is the highest priority at all stages of \dual \dword{pd} operations. Since DUNE is an international project, the international safety regulations will be followed closely during the course of preparation of safety documents.

The main risks at the production and testing sites are electrocution, exposure to excessive heat, chemicals and cryogenics, and heavy lifting. Detailed procedures will be developed by the relevant institutes and approved by the \dual \dword{pd} consortium. Contents of the electrical safety rules will range from utilizing regular power equipment to handling \dwords{pmt} for testing. The chemical and heat exposure hazards only concern the sites where the \dword{tpb} coating is performed. The heavy-lifting risks concern mainly %objects that carry safety risks will mainly be 
the \dword{pmt} delivery boxes.

The \dword{itf} \dual \dword{pd} safety regulations will be developed the same way. Main hazards at this site are electrocution and heavy lifting. Also, due to the quantity and frequency of shipments from all other subsystems, tripping and operating in a limited space will also be considered.

The underground operation and installation safety rules will %mainly 
follow the general facility rules on e.g., working in confined spaces, oxygen deficiency hazard and emergency procedures. The \dual \dword{pd}-specific safety rules are particularly related to lifting the boxes and their contents %of heavy objects 
for installation and working at heights for cabling.

%%%%%%%%%%%%%%%%%%%%%%%%%%%%%%%%%%%%%%%%%%%%%%%%%%%%%%%%%%%%%%%%%%%%
\section{Management and Organization}
\label{sec:fddp-pd-12}

The \dual \dword{pd} consortium was formed in 2017 and it is composed of eleven institutes from France, Peru, Spain, UK and USA. The charge of the \dual \dword{pd} consortium is to plan and execute the construction, installation and commissioning of the \dword{dpmod} \dword{pds}.


%%%%%%%%%%%%%%%%%%%%%%%%%%%%%%%%%
\subsection{Consortium Organization}
\label{sec:fddp-pd-12.1}

The \dual \dword{pd} consortium Leader (CL) is In\'{e}s Gil-Botella from CIEMAT (Spain) and the Technical Lead (TL) is Dominique Duchesneau from LAPP (France). They are members of the DUNE Technical Board and they represent the consortium to the overall DUNE collaboration. The CL is responsible for the subsystem deliverables and for the effective management of the consortium. The TL acts as the overall project manager and is the interface to the International Project Office (IPO); he is responsible for monitoring and reporting on progress with respect to the agreed schedule and for issues related to interface documentation.

The institutions participating in the consortium are responsible for the design or construction of a particular subsystem. It is hoped that the national groups within the consortia will be able to approach relevant funding agencies with a specific construction-phase proposal, such that a likely funding line can be established in or before 2019. The \dual \dword{pd} consortium is open to any new institution willing to join the current effort.

The current institutions participating in the \dual \dword{pd} consortium are summarized in Table \ref{tab:dppd_t_12_1}.

\begin{dunetable}
[\dual \dword{pd} consortium institutions]
{|l|l|l| p{0.8\textwidth}}
{tab:dppd_t_12_1}
{\dual \dword{pd} consortium institutions}

Country & Institution & Contact \\ \toprowrule
France & LAPP & Dominique Duchesneau \\ \colhline
Peru & PUCP & Alberto Gago \\ \colhline
Spain & IFAE & Thorsten Lux \\ \colhline
Spain & CIEMAT & In\'{e}s Gil-Botella\\ \colhline
Spain & IFIC & Michel Sorel \\ \colhline
United Kingdom & University College London & Anna Holin \\ \colhline
USA & Argonne National Lab & Zelimir Djurcic \\ \colhline
USA & Duke University & Kate Scholberg \\ \colhline
USA & University of Iowa & Jane Nachtman \\ \colhline
USA & South Dakota School of Mines and Technology & Juergen Reichenbacher\\ \colhline
USA & University of Texas at Austin & Karol Lang \\
% \colhline
\end{dunetable}

The \dual \dword{pd} consortium is divided into four working groups: photosensors and electronics, calibration system, mechanics and integration, and simulation and physics. The corresponding WG conveners are:
\begin{itemize}

\item WG1: Photosensors and Electronics - A. Verdugo (CIEMAT)
\item WG2: Calibration System - C. Cuesta (CIEMAT)
\item WG3: Mechanics and Integration - B. Bilki (Iowa)
\item WG4: Sim. \& Phys. - K. Scholberg (Duke), M. Sorel (IFIC), L. Zambelli (LAPP)

\end{itemize}

%The \dual \dword{pd} consortium holds regular bi-weekly meetings on Thursdays (4pm CET, 9am CST). Agendas and presentations can be found at: \url{https://indico.fnal.gov/category/699/}.

%%%%%%%%%%%%%%%%%%%%%%%%%%%%%%%%%
\subsection{Planning Assumptions}
\label{sec:fddp-pd-12.2}

The optimization and final design of the \dual \dword{pd} system will be driven by:
\begin{enumerate}
\item \dword{pddp} data (expected by beginning of 2019)
\item Simulation studies (in progress)
\end{enumerate}

\dword{pddp} operation and data analysis are fundamental steps to understanding whether the current \dword{pds} design considered as baseline, based on cryogenic \dwords{pmt} with \dword{tpb} coating, is able to provide $t_0$ for non-beam events, background rejection and triggering on non-beam events. These data will be used to tune the \dword{mc} simulations and extrapolate the performance of the system to the \dword{dpmod}. 

Simulations are needed to determine and optimize the \dual \dword{pd} system to meet the physics requirements in terms of:
\begin{itemize}
\item Light collection efficiency,
\item Number of channels,
\item Photosensor requirements,
\item Dynamic range of readout electronics and timing resolution,
\item Trigger strategy on non-beam events.
\end{itemize}

The DUNE physics requirements in terms of expected performance of the \dword{pds} should be provided by the DUNE Physics WG. Alternative design aspects of the proposed \dword{pds} considered as baseline for the \dword{dpmod} (see CDR document arXiv:1601.02984) will be developed based on the compatibility of \dword{pddp} data and \dword{mc} light simulation results with the DUNE physics requirements.

%%%%%%%%%%%%%%%%%%%%%%%%%%%%%%%%%
\subsection{WBS and Responsibilities}
\label{sec:fddp-pd-12.3}

The \dual \dword{pd} consortium has developed a detailed breakdown of deliverables and responsibilities included in the overall DUNE collaboration \dword{wbs}, DUNE-doc-5594, coordinated by the IPO. The main deliverables are based on the \dword{pddp} \dword{pds}  and are divided into seven topics: 

\dword{wbs} Element \textit{- Institutions} \\
%\dword{pddp} \dword{pds} 
\begin{enumerate}
\item Management \dual \dword{pds} (includes milestones and review dates) \textit{- LAPP, CIEMAT }
\item Physics and Simulations \textit{- Duke, LAPP, IFIC, SDSMT, CIEMAT, PUCP, UCL, Texas-Austin}
\item Design, Engineering, R\&D and validation tests \textit{- Iowa, CIEMAT, IFIC, UCL, Texas-Austin, IFAE, SDSMT}
\item Production Setup (includes tooling) \textit{- UCL}
\item Production (includes component production, assembly, testing, and \dword{qc}) \textit{- Iowa, CIEMAT, IFAE, IFIC, UCL, Texas-Austin, Duke, SDSMT, LAPP}
\item Integration (contributions to activities at global integration facility) \textit{- SDSMT}
\item Installation (contributions to activities at \surf) \textit{- CIEMAT, IFIC, SDSMT, Iowa}
\end{enumerate}


%%%%%%%%%%%%%%%%%%%%%%%%%%%%%%%%%
\subsection{High-Level Cost and Schedule}
\label{sec:fddp-pd-12.4}

The cost of the baseline \dual  \dword{pds} will be defined in a separate document. \fixme{need a ref to it?}

The \dual \dword{pds} consortium's main activities during the next \num{16} months are focused on developing the \dword{tdr}. The main high-level milestones are detailed in Table~\ref{tab:dppd_t_12_5} for this period. The plan for the activities in the post-\dword{tdr} period is summarized in Table~\ref{tab:dppd_t_12_6}.

\begin{dunetable}
[Pre-\dword{tdr} Key Milestones]
{|l|l| p{0.8\textwidth}}
{tab:dppd_t_12_5}
{Pre-\dword{tdr} Key Milestones}

Milestone & End date \\ \toprowrule
Simulations and physics: %Finalize the 
Implementation of \dual optical & \\
simulation in \larsoft for \dword{pddp} & 08/2018 \\ \colhline
Simulations and physics: Optimization of the & \\
\dword{dpmod} performance to fulfill the physics requirements and & \\
definition of a trigger strategy & 05/2019 \\ \colhline
Photosensors: Components selection and final design & 03/2019 \\ \colhline
\dword{pmt} calibration system design and selection of components & 03/2019 \\ \colhline
Cabling definition and design of flange & 03/2019 \\ \colhline
Design review in light of \dword{pddp} calibration data & 03/2019 \\ \colhline
\dword{qc} plan & 06/2018 \\ \colhline
Identification of Interfaces & 06/2018 \\ \colhline
Integration, installation and commissioning plans & 12/2018 \\ \colhline
\dword{dpmod} \dword{tdr} & 06/2019 \\ 
\end{dunetable}

\begin{dunetable}
[Post-\dword{tdr} Key Milestones]
{|l|l|l| p{0.8\textwidth}}
{tab:dppd_t_12_6}
{Post-\dword{tdr} Key Milestones}

Milestone & Start date & End date \\ \toprowrule
\textbf{\dword{pmt} preparation and installation} (can be done in batches) & & \\ \colhline
\dword{pmt} procurement procedure and production & 01/2021 & 12/2022 \\ \colhline
\dword{pmt} base design and manufacturing & 01/2022 & 12/2022 \\ \colhline
\dword{pmt} support structure production and assembly & 08/2022 & 01/2023 \\ \colhline
\dword{pmt} characterization - \num{10} \dwords{pmt}/week (two facilities) & 02/2023 & 12/2023 \\ \colhline
\dword{tpb} coating (two facilities similar to that for CERN ICARUS) & 01/2024 & 12/2024 \\ \colhline
Splitter production and tests & 05/2024 & 12/2024 \\ \colhline
\textbf{Installation at \surf} & & \\ \colhline
\dword{pmt} cable and fiber routing in cryostat from flange to bottom & & \\
                  (depends on \dword{fc} and flange installation) & 9/2024 & 9/2024 \\ \colhline
\dword{pmt} testing, installation in cryostat and cabling (\num{72} \dwords{pmt}/month) & 10/2024 & 07/2025 \\ \colhline
\dword{pmt} support installation on the membrane & & \\
                  (in parallel by sector with \dword{pmt} installation) & 10/2024 & 07/2025 \\ \colhline
Splitter installation & & \\
                  (in parallel with \dword{pmt} installation to test cabling and connections) & 10/2024 & 07/2025 \\ \colhline
\textbf{Light calibration system} & & \\ \colhline
Fibers, light source tests and procurement & 06/2023 & 05/2024 \\ \colhline
Fiber calibration system installation & & \\
                  (in parallel with \dword{pmt} installation with validation test) & 9/2024 & 07/2025 \\ 
\end{dunetable}

