\chapter{Photon Detection System}
\label{ch:fddp-pd}

%%%%%%%%%%%%%%%%%%%%%%%%%%%%%%%%%%%%%%%%%%%%%%%%%%%%%%%%%%%%%%%%%%%%
\section{Overview}
\label{sec:fddp-pd-1}

%%%%%%%%%%%%%%%%%%%%%%%%%%%%%%%%%
\subsection{Introduction}
\label{sec:fddp-pd-1.1}

This chapter describes the Photon Detection System (PDS) for the DUNE Dual-Phase (DP) Far Detector (FD). The PDS is used primarily as decision to acquire the event (trigger) for non-beam events, for determination of the event absolute time for beam and non-beam events, and for calorimetric measurements. It is essential to ensure that the DP FD PDS is optimized for the full DUNE physics program. In particular, low energy signals like supernova (SN) neutrinos and multi-messenger astronomy, other low-energy signals, and proton decay, will have more stringent requirements on PDS performance than the primarily higher energy, beam-synchronous, neutrino oscillation physics. The final specifications of the system will be determined in order to achieve these physics requirements. This Technical Proposal chapter will concentrate on direct projection of the ProtoDUNE-DP design to the DUNE scale. The optimization and final design of the Dual-Phase Photon Detector (DPPD) system will be driven by the ProtoDUNE-DP \cite{protoDUNDP-tdr} data and simulation studies.

The chapter begins with an overview of the system in section~\ref{sec:fddp-pd-1}. Section~\ref{sec:fddp-pd-2} describes the photo-sensors, namely photomultiplier tubes (PMTs) and the related high-voltage system, wavelength shifters and light collectors. The mechanics associated with the PMTs is described in Section~\ref{sec:fddp-pd-3}, and the readout electronics in ~\ref{sec:fddp-pd-4}. Section~\ref{sec:fddp-pd-5} details the photon calibration system to monitor the PMT gain and stability. Then, the photon detector performance is described in Section~\ref{sec:fddp-pd-6}, and the operations in Section~\ref{sec:fddp-pd-7}. Interfaces with other subsystems are described in Section~\ref{sec:fddp-pd-8}. Section~\ref{sec:fddp-pd-9} includes the installation, integration and commissioning plans. Then, the quality control procedures are outlined in Section~\ref{sec:fddp-pd-10}. The main safety issues to consider are specified in Section~\ref{sec:fddp-pd-11}. To finish, the management and organization is described in Section~\ref{sec:fddp-pd-12}.

%%%%%%%%%%%%%%%%%%%%%%%%%%%%%%%%%
\subsection{Physics and the Role of Photodetection}
\label{sec:fddp-pd-1.2}

The main physics goals of the DUNE DP liquid argon (LAr) Time Projection Chamber (TPC) is to register beam events from LBNF at Fermilab, and operate outside of the beam spill as an efficient observatory for supernovae explosions and proton decays. DUNE will also collect atmospheric neutrino and muon events, and will conduct searches for a number of exotic phenomena postulated by extensions of the Standard Model.  Expected or searched for signals can range in energy from a few MeV to many GeV and have characteristic time duration and topological features that challenge the performance of large noble liquid TPCs. An essential and critical part of the LAr TPC is the PDS, sensitive to light produced by interactions in argon \cite{Cuesta:2017nrs}. In DP TPCs, the timing of prompt scintillation light (usually referred as S1 signal) in LAr is needed for time stamping of events and propagation of tracks in the detector. The extraction and amplification of drift electrons in the gas phase (usually referred to as S2 signal) yields information on the drift time and amount of ionization charge, thus supplementing information from the charge readout on the anode plane. The interplay between the charge and light from an event allows to achieve the pattern recognition and the measurement of energy of interactions.

Ionizing radiation in liquid noble gases leads to the formation of excimers in either singlet or triplet states, which decay radiatively to the dissociative ground state with characteristic S1 fast and slow lifetimes (fast is about \SI{6}{ns}, slow is about \SI{1.3}{$\mu$s} in LAr with the so-called second continuum emission spectrum peaked at the wavelength of approximately \SI{127}{nm}, \SI{126.8}{nm} with a full width at half maximum of \SI{7.8}{nm} \cite{Heindl}). This prompt and relatively high-yield (about \num{40000} photons per MeV) of \SI{127}{nm} scintillation light is exploited in LAr TPC to provide the absolute time ($t_0$) of the ionization signal collected at the anode, thereby providing the absolute value of the drift coordinate of fully contained events, as well as a prompt signal used for triggering purposes.

The secondary scintillation light S2 is produced in the gas phase of the detector when electrons, extracted form the liquid, are accelerated in the electric field between the liquid phase and the anode. The secondary scintillation in the argon gas (i.e. a vapor phase) is the luminescence in gas caused by accelerated electrons in the electric field and in the Large Electron Multiplier (LEM) anode through Townsend amplification. For a given argon gas density, the number of photons of this S2 signal is proportional to the number of electrons, the electric field, and the length of the drift path in gas covered by the electrons. In an extraction field of \SI{2.5}{kV/cm} in gas, one electron generates about \num{75} photons. The time stretch of S2 reflects the extraction time of original ionization in the liquid phase into the gas phase, thus for about \SI{1}{kV/cm} electric field, the time scale of S2 is of the order of hundreds of microseconds. The time between the occurrence of the primary scintillation light and the secondary scintillation light is equivalent to the drift time of the electrons from the ionization coordinate to the LAr surface.

The baseline design of the light collection system calls for 8-inch diameter cryogenic PMTs distributed uniformly on the floor of the cryostat and electrically shielded from the bottom cathode plane. The proposed density of PMTs and their arrangement follows the design of the ProtoDUNE-DP detector. On the other hand, modeling and simulations of light collection both for ProtoDUNE and the DUNE detectors are still ongoing. Therefore, even critical system parameters and their impact on the physics reach are still tentative. Results from the ProtoDUNE will provide the critical validation of simulations and will guide optimizations for the large DUNE detector.

%%%%%%%%%%%%%%%%%%%%%%%%%%%%%%%%%
\subsection{Technical Requirements}
\label{sec:fddp-pd-1.3}

Photomultipliers provide the sharpest time information of events in the LAr TPC and in the gas phase of the extraction stage. Due to necessary wavelength-shifting of photons from the argon luminescence and shadowing by the cathode plane, the efficiency of light detection is challenging and requires careful mechanical, electrical, and optical designs. DPPD Consortium is presently in contact with PMT manufacturers, including Hamamatsu Photonics in Japan \cite{hamamatsu}, Electron Tubes Limited in the US and UK \cite{electrontubeslim}, and HZC in China \cite{hzc}, to define optimal choice and configuration of PMTs satisfying our requirements, tentatively summarized in Table~\ref{tab:dppd_t_1_3}. These requirements will be reviewed based on the physics needs. For this, simulations and ProtoDUNE-DP results will be key.

\begin{dunetable}
[Summary of tentative requirements for the photon detection system of the DP LAr TPC.]
{|l|l|l| p{0.8\textwidth}}
{tab:dppd_t_1_3}
{Summary of tentative requirements for the PDS of the DP LAr TPC. The table assumes the baseline choice of the R5912-MOD20 photomultiplier manufactured by Hamamatsu Photonics \cite{hamamatsu-5912}.}
\footnotesize
%  & & \\
{\bf Feature}	& {\bf Goal}  	& {\bf Comment	}	\\
%  & & \\ 
 \toprowrule
{\bf Optical} & & \\ \colhline
spectral response & \SI{127}{nm} 	& wavelength shifters are required				\\
light yield & 2.5 PE/MeV & for ProtoDUNE-DP, final value depending on \\

 	 & 	 & simulation results \\
\colhline
{\bf Electronic}	&			&										\\ \colhline
minimum light signal & SPE & required to perform the PMT gain measurement \\
gain			& $\sim$\num{10}$^6$-\num{10}$^9$ & given by PMT specifications			\\
noise (or signal/noise) &  <\SI{1}{mV} & to distinguish SPE from noise, depends on PMT gain
\\
timing resolution & few ns & to distinguish S1 from S2 component \\
	power		& < \SI{0.2}{W/PMT}& used successfully in the WA105 $3\times1\times1$\,m$^3$ detector			\\
ADC dynamic range & TBD & depending on simulation results, see section \ref{sec:fddp-pd-4.3}  \\ 
\colhline		
 	{\bf Electrical}	&			&									\\ \colhline
 	HV range		& \si{0-2500}{V}	& individual cable per each PMT			\\
	HV resolution	& \SI{1}{V}		&	fine tuning of PMT gain			\\
	HV noise		& <\SI{100}{mV}		& extra filtering will be required \\
	HV grounding	& isolated		& HV outputs shall be floating, crate ground is 	\\
				&			& independent of the return of the HV channels. 			\\ 
PMT placement		& isolated		& PMT's electrically shielded from the TPC cage	\\
\colhline			
      {\bf Mechanical} 	&			&									\\ \hline
      	temperature 	& cryogenic		& PMTs operated in LAr and tested in liquid nitrogen 	\\
	pressure 		& \SI{+2}{bar}		& the argon column will be about \SI{10}{m} high	\\
\colhline			
\end{dunetable}

PMTs will be installed with the baseline density of \num{1} per \SI{1}{m^2}. The choice of the R5912-MOD20 photomultiplier manufactured by Hamamatsu Photonics \cite{hamamatsu-5912} is assumed as baseline plan. In order to extend the PMT light sensitivity region to the LAr light emission wavelength of \SI{127}{nm}, a wavelength shifter has to be used. Therefore, in the baseline plan, the hemispherical windows of the PMTs will be evaporated with a thin layer of Tetra-Phenyl-Butadiene (TPB) \cite{tpb} for wavelength-shifting into the range suitable for R5912 PMT photocathode sensitivity \cite{hamamatsu-5912}. PMTs have to be rigidly anchored to the bottom of the cryostat. Different PMT densities and placements along the walls are also being considered in simulations. High voltage (HV)/signal cables will be routed along the cryostat walls to feedthroughs installed in the roof of the cryostat. Each PMT will be controlled individually so that its gain can be individually adjusted to match the front-end dynamic range and signal-to-noise ratio. An average light yield of 2.5 PE/MeV is required based on ProtoDUNE-DP simulations (see Section \ref{sec:fddp-pd-6}). The precision of the light yield requirement will be established after the completion of the full DUNE simulations.

The cathode plane, described in the HV DP chapter,  is placed at a height of about \SI{2}{m} above the bottom of the cryostat, and the PMT plane will be distant enough from the cathode plane, taking into account the high electrical rigidity of the LAr phase. In order to protect the PMTs, the ground grid will be installed and placed at an identical potential as the PMT photocathode (\SI{0}{V}). The PMTs will be powered to about \num{1.5}-\SI{2.0}{kV} such that the PMT gain is $\sim$\num{10}$^7$-\num{10}$^9$. Future developments on the quantization of possible TPB dissolution in LAr will be encouraged and followed \cite{TPBdiss}.

%%%%%%%%%%%%%%%%%%%%%%%%%%%%%%%%%
\subsection{Detector Layout}
\label{sec:fddp-pd-1.4}

The PMT plane will be placed below the cathode plane far enough to be sufficiently electrically shielded. According to the baseline plan, the PMTs will be uniformly distributed across this plane with a density of \SI{1}{PMT/m^2}, with a total of \num{720} PMTs installed. Other PMT configurations as determined by the simulations are also being considered. The PMTs will be individually mounted to the cryostat floor. The exact location of the PMTs will be determined by the location of the other floor structures like the cryogenic piping. The outline of the DUNE-DP is shown in Fig. \ref{fig:dppd_3_1}.

\begin{dunefigure}[The DUNE DP detector (partially open) with cathode, PMTs, field cage and anode plane with chimneys.]{fig:dppd_3_1}
{The DUNE DP detector (partially open) with cathode, PMTs, field cage and anode plane with chimneys.}
\includegraphics[width=0.95\textwidth]{dppd_3_1}
\end{dunefigure}

Since few light sensors are directly sensitive to \SI{127}{nm}, a wavelength shifter will be required. TPB coating directly on the PMT is the default plan. Light collectors to increase the photons detected are under study. A single cable will be used per PMT to carry power and signal, and splitters will be placed out of the cryostat. A photon calibration system will be formed by external light sources and internal optical fibers.  

The cable trays from the side walls of the cryostat to the PMTs will carry the cables and calibration fibers. The cables and fibers will be routed from the feedthrough flanges at the top of the cryostat and  combined at the side wall trays. These side trays will carry the HV/signal cables in blocks of \num{24} PMTs and four calibration fibers. Therefore, each block of \num{24} PMTs in a \num{6}$\times$\SI{4}{m^2} area will form a sector of underground installation totaling in \num{30} sectors.

% %%%%%%%%%%%%%%%%%%%%%%%%%%%%%%%%%
\subsection{Operation Principles}
\label{sec:fddp-pd-1.5}

The physics program defines the operation principles of the DUNE DP Far Detector: the measurement of the neutrino oscillation parameters requires to record events based on an external trigger coming from the beam, while non-beam physics such as SN bursts, proton decay, or other exotic transitions events will require special trigger conditions including the PDS. Another operation mode will be the PMT calibration which has to be performed regularly. In this case the data recording will be started by a hardware trigger provided by the calibration system. \\    

Thus, the modes will be:
\begin{itemize}
\item External trigger: this is mainly the case of a hardware trigger generated by the beam, but also test data with random trigger generated by software can be taken. 
\item Non-beam physics trigger: the electronics based on the PDS signals provides the trigger for SN burst, proton decay events, etc.
\item Calibration: during PDS calibrations, the trigger will be provided by the light calibration system.
\end{itemize}

The modes of the external trigger and the non-beam physics trigger will not be excluding each other but will be running in parallel to ensure that rare events such as SN bursts are recorded effectively.

%%%%%%%%%%%%%%%%%%%%%%%%%%%%%%%%%%%%%%%%%%%%%%%%%%%%%%%%%%%%%%%%%%%%
\section{Photosensor System}
\label{sec:fddp-pd-2}

%%%%%%%%%%%%%%%%%%%%%%%%%%%%%%%%%
\subsection{Photodetector Selection and Procurement}
\label{sec:fddp-pd-2.1}

The photodetector selected as baseline for the light-readout system is the Hamamatsu R5912-MOD20 PMT as used in ProtoDUNE-DP. The Hamamatsu R5912-MOD20, see Figure \ref{fig:dppd_2_1}, is an 8-inch, 14-stage, high gain PMT (nominal gain of \num{10}$^9$). In addition, this PMT was designed to work at cryogenic temperature adding a thin platinum layer between the photocathode and the borosilicate glass envelope to preserve the conductance of the photocathode at low temperature. This particular PMT has proven reliability on other cryogenic detectors. The same or similar PMTs have been successfully operated in other LAr experiments like MicroBooNE \cite{microboone}, MiniCLEAN \cite{miniclean}, ArDM, ICARUS T600 \cite{icarus}, as well as in ProtoDUNE-DP \cite{protoDUNDP-tdr}. Contacts with other manufacturers such as Electron Tubes Limited (UK) \cite{electrontubeslim} and HZC (China) \cite{hzc} are on-going to engage them in the program.

\begin{dunefigure}[Picture of the Hamamatsu R5912-MOD20 PMT.]{fig:dppd_2_1}
{Picture of the Hamamatsu R5912-MOD20 PMT \cite{hamamatsu-5912}.}
\includegraphics[width=0.2\textwidth]{dppd_2_1}
\end{dunefigure}

As the baseline number of PMTs, \num{720} + \num{80} spares, is high and several operations and tests have to be performed with them before the installation, the PMTs have to be ordered with some time in advance. The envisioned operations are: assembly of the voltage divider circuit, mounting on the support structure, test at room and cryogenic temperatures, application of TPB coating, packing and shipment. And finally, they have to be re-tested on-site before installation (see Sections \ref{sec:fddp-pd-9} and \ref{sec:fddp-pd-10}). Considering the large number of PMTs required by DPPD, the purchase order has to be sent, at least, two years in advance of installation. An staged order to achieve a steady supply of PMTs would be convenient to execute the plan mentioned before. 

%%%%%%%%%%%%%%%%%%%%%%%%%%%%%%%%%
\subsection{Photodetector Characterization}
\label{sec:fddp-pd-2.2}

Before the installation, the most important parameters of the PMT response have to be measured with two aims: first, to reject under-performing PMTs and second, to store the characterization information in a database for later use during the detector commissioning and operation.

Basic and most important parameters to characterize are the dark counts rate vs voltage and the gain vs voltage. Both parameters must be measured at room and at cryogenic temperatures. Prepulsing and afterpulsing are not expected to be an issue, but will be measured, too. 

From the mechanical point of view, the test setup will require a light tight vessel that could be filled with a cryogenic liquid (argon or nitrogen) plus the infrastructure for filling and operating the vessel with temperature and liquid level controls. For ProtoDUNE-DP, \num{10} PMTs were tested at a time during a week, as the tests of the PMTs on cryogenics require several days for the PMT thermalization. Figure \ref{fig:dppd_2_2a} shows the PMTs being installed in the testing vessel used for the ProtoDUNE-DP PMTs. Increasing the capacity of the vessel, and thus the number of PMTs tested at a time, will reduce the characterization test duration.

\begin{dunefigure}[Picture of the PMTs being installed in the testing vessel used for the ProtoDUNE-DP PMTs]{fig:dppd_2_2a}
{Picture of the PMTs being installed in the testing vessel used for the ProtoDUNE-DP PMTs.}
\includegraphics[width=0.3\textwidth]{dppd_2_2a}
\end{dunefigure}

Figure \ref{fig:dppd_2_2b} shows the sketch of the envisaged setup for PMT characterization tests. From the electronics point of view, the test setup will require a HV power supply, a discriminator, a counter for the dark rate measurements, a pulsed light source, and a charge-to-digital or analog-to-digital converter for the PMT gain vs voltage measurements. All those instruments must allow computer control to automate the data acquisition.

\begin{dunefigure}[Sketch of the setup for PMT characterization tests.]{fig:dppd_2_2b}
{Sketch of the setup for PMT characterization tests.}
\includegraphics[width=0.9\textwidth]{dppd_2_2b}
\end{dunefigure}

%%%%%%%%%%%%%%%%%%%%%%%%%%%%%%%%%
\subsection{High Voltage System}
\label{sec:fddp-pd-2.3}

Based on the experience with the WA105 $3\times1\times1$\,m$^3$ DP prototype, for the PMT HV system, the A7030 power supply modules from CAEN \cite{caen-a7030} were chosen as baseline design. These modules provide up to \SI{3}{kV} with a maximal output current of \SI{1}{mA} and a common floating ground to minimize the noise. Module versions with \num{12}, \num{24}, \num{36}, or \num{48} HV channels are available. The HV polarity can be chosen for each module. According to the baseline PMT powering scheme, modules with positive HV polarity will be acquired for the experiment. Modules with \num{48} HV channels and Radiall \num{52} connector are considered. The corresponding HV cable will connect the modules with the HV splitters which are described in Section \ref{sec:fddp-pd-4.2}. This choice will allow the design of a compact and most cost-effective system occupying between \num{1} and \num{2} racks only. For \num{720} PMTs, \num{15} A7030 modules (+ \num{2} spares) will be needed. These \num{15} HV modules will be installed in mainframes from CAEN.

Each PMT will be powered individually thus allowing the gain of all PMTs to be equalized by adjusting the operating voltage. A control software for this task will be provided taking into account the development of an interface to the PMT calibration system (Section \ref{sec:fddp-pd-5}) which will provide the calibration factors needed for the gain equalization.

%%%%%%%%%%%%%%%%%%%%%%%%%%%%%%%%%
\subsection{Wavelength Shifters}
\label{sec:fddp-pd-2.4}

The detector approach foresees to convert the \si{127}{nm} photons by the use of suitable wavelength shifting material into visible photons. The baseline plan is the already validated concept of coating the PMT windows with a thin film of TPB \cite{tpb}. TPB is a wavelength shifter with high efficiency for conversion of LAr scintillation VUV photons into visible light, where PMT cathode is sensitive. The TPB is deposited on the PMT by means of a thermal evaporator which consists of a vacuum chamber with two copper crucibles (Knudsen cells) placed at the bottom of the chamber, following the sanding of the PMT window. A PMT is fixed at the top of the evaporator, with its window pointing downwards, on a rotating support in order to ensure a uniform coating. The crucibles, filled with the TPB, are heated up to \si{220}{$^\circ$C}. At this temperature, the TPB evaporates through a split in the crucible lid into the vacuum chamber, eventually reaching the PMT window.

Several tests were performed in order to tune some parameters like the coating thickness (TPB surface density) and the deposition rate. For the tests, a PMT mock up covered with mylar foils has been used. A TPB density of \SI{0.2}{mg/cm^2} was chosen for ProtoDUNE-DP as this is the value where the PMT efficiency is stable as a function of the density. Efficiency measurements were performed using a VUV monochromator by comparing the cathode current of a coated PMT with the current value of a calibrated photodiode. As a result of the efficiency tests, about \SI{0.8}{g} of TPB must be placed in the crucibles at each evaporation, in order to achieve the desired PMT coating density. %The best deposition rate was fixed to about 6.5\,\AA/s. 
This value optimizes the quantity of TPB used per evaporation keeping, at the same, the coating density fluctuations below \num{5}$\%$. With these specifications, two to four PMTs can be coated per day at a single coating station. Then, a multiple coating stations will be required for timely operations.

%%%%%%%%%%%%%%%%%%%%%%%%%%%%%%%%%
\subsection{Light Collectors}
\label{sec:fddp-pd-2.5}

Although we are still lacking detailed physics simulations of photon collection in the full DUNE far detector modules, it can be generally argued that further optimization, or cost-effectiveness per physics reach, of light collection is desirable. In addition to  maximizing the overall light yield, another crucial figure of merit is the uniformity of the light collection efficiency within the full TPC active volume. Geometrical acceptance effects, as well as light absorption processes at the detector boundaries and within the LAr itself, can greatly degrade the uniformity in response. Detector active regions close to the field cage and further away from the cathode are the most penalized. As a result, up to one order of magnitude differences in response throughout the TPC active volume are not uncommon in a LAr-TPC.

In the case of a LAr-TPC, there are at least four main parameters for optimizing the light yield and the uniformity in response: i) the number of PMTs per unit area, ii) the placement of PMTs, iii) the augmentation of PMTs with additional light collectors, and iv) the choice of where and how the original \SI{127}{nm} photons can be wavelength-shifted. The most obvious direction for optimizing cost effectiveness are the latter two options. Detector components that are not strictly part of the photon detector system may also play a role in this optimization process, one relevant example being the transparency of the cathode plane. The options to use shifter-reflectors (Winston cones) to increase the effective area of individual PMT windows, or to move shifting of light closer to the cathode and attaching wave-guides coupled to the PMTs, are under study.

Another promising and cost-effective option to increase both light yield and response uniformity is the use of TPB-coated reflector foils covering the detector inner walls. This option is routinely used in dual-phase LAr-TPCs searching for dark matter, such as the ArDM and DarkSide experiments. This is also under investigation for the DUNE single-phase far detector concept, building on the experience already accumulated with the LArIAT experiment, and the one to be gained with SBND. In the DP case, up to four of the six inner faces of the TPC could be covered with dielectric foils, the ones corresponding to the field cage structure. The same WLS used to coat the PMT windows, TPB, would be vacuum-evaporated on the foils. The shifted blue light emitted by the foils would then have a greater chance to reach the PMT windows compared to \SI{127}{nm} light, owing to the better reflective properties given by the combination of foils and blue light. To be adopted, a light collection involving reflective foils would first need to demonstrate satisfactory stability over time during the entire duration of the experiment.

%%%%%%%%%%%%%%%%%%%%%%%%%%%%%%%%%%%%%%%%%%%%%%%%%%%%%%%%%%%%%%%%%%%%
\section{Mechanics}
\label{sec:fddp-pd-3}

% %%%%%%%%%%%%%%%%%%%%%%%%%%%%%%%%%
% \subsection{Mechanical Structure of the Photosensor}
% %\label{sec:fddp-pd-3.1}

An individual PMT mount has been designed and tested in the WA105 $3\times1\times1$\,m$^3$ detector \cite{Zambelli:2017dkg}. The same design will be used for ProtoDUNE-DP and is foreseen for DUNE DP FD. A PMT with the mechanical structure is shown in Figure \ref{fig:dppd_2_1}. The support frame structure is mainly composed of \num{304}L stainless steel with some small Teflon (PTFE) pieces assembled by A4 stainless steel screws that minimize the mass while ensuring the PMT support to the cryostat membrane. The design was done taking into account the shrinking of the different materials during the cooling process to avoid the break of the PMT glass.

%%%%%%%%%%%%%%%%%%%%%%%%%%%%%%%%%
%\subsection{Photosensor Fixation to the Membrane Floor}
%\label{sec:fddp-pd-3.2}

A uniform array of \num{720} cryogenic Hamamatsu R5912-MOD20 PMTs, below the transparent cathode structure, will be fixed on the membrane floor in the areas between the membrane corrugations. The arrangement of the PMTs will need to be optimized in order to be compatible with the presence of the cryogenic piping on the membrane floor, or any other element found in this area. Over-pressure tests were carried out for ProtoDUNE-DP, and further tests to ensure the correct performance under pressure will be carried out.

The mechanics for the attachment of the PMTs has been carefully studied for ProtoDUNE-DP. It must counteract the PMT buoyancy while avoiding stress to the PMT glass due to differentials in the thermal contraction between the support and the PMT itself. The fixation is done via a stainless steel supporting base, that could be point glued to the membrane. The weight of the support and photomultiplier exceeds the buoyancy force of the system. Given the large standing surface of the stainless steel plate support basis, these supports will ensure as well stability against lateral forces possibly acting on the PMTs due to the liquid flow. Figure \ref{fig:dppd_3_2} depicts the PMT together with its support base attached to the bottom of the cryostat.

\begin{dunefigure}[Cryogenic Hamamatsu R5912-MOD20 photomultiplier fixed on the membrane floor.]{fig:dppd_3_2}
{Cryogenic Hamamatsu R5912-MOD20 photomultiplier fixed on the membrane floor, with the optical fiber of the calibration system.}
\includegraphics[width=0.42\textwidth]{dppd_3_2}
\end{dunefigure}

%%%%%%%%%%%%%%%%%%%%%%%%%%%%%%%%%%%%%%%%%%%%%%%%%%%%%%%%%%%%%%%%%%%%
\section{Readout Electronics}
\label{sec:fddp-pd-4}

%%%%%%%%%%%%%%%%%%%%%%%%%%%%%%%%%
\subsection{PMT High Voltage Dividers}
\label{sec:fddp-pd-4.1}

For the PMT power supply, the cathode grounding and positive high voltage applied to the anode was chosen for ProtoDUNE-DP. A single cable is required for each PMT to carry power and signal. This configuration requires half of the cables and feedthroughs on the detector than the negative voltage configuration, which is a clear advantage since the number of PMTs in the detector is large. In addition, the cathode grounding shows less dark counts than the anode grounding scheme. The drawback is that a coupling capacitor must be used to separate the high voltage from the PMT signal, but, this signal and power splitting can be done externally from the detector. Figure~\ref{fig:dppd_4_1} shows the positive power supply and cathode grounding scheme.

\begin{dunefigure}[Positive power supply and cathode grounding scheme.]{fig:dppd_4_1}
{Positive power supply and cathode grounding scheme.}
\includegraphics[width=0.5\textwidth]{dppd_4_1}
\end{dunefigure}

The PMT base circuit will be based only on resistors and capacitors as semiconductors do not work well in cryogenic temperatures. Nevertheless, the components will be carefully selected and tested to minimize the variations in their characteristics with temperature. The polarization current of the voltage divider (total circuit resistance) will be chosen to meet the PMT light linearity range and maximum power requirements. The dynodes voltage ratio will follow the manufacturer recommendations for increased linearity range on the space-charge effect area (tapered divider). In addition, capacitors will be added to the last stages to increase the PMT linearity in pulsed mode. The precise values for the components have not been decided yet as they depend on concrete requirements and also on the results from ProtoDUNE-DP. The ProtoDUNE-DP base design is considered as the baseline solution.

For the connection between the PMT base and the feed-through, the RG-303/U cable was selected by its low attenuation and its proven reliability on cryogenic environments. On one side, this cable will be directly soldered to the PMT base, and, on the other side, it will end with an SHV connector to be attached to the flange. 

%%%%%%%%%%%%%%%%%%%%%%%%%%%%%%%%%
\subsection{High Voltage/Signal Splitters}
\label{sec:fddp-pd-4.2}

HV/signal splitters will separate fast PMT response signal from the positive high voltage with capacitive decoupling. In addition, they will include a low pass filter between the HV supply and the PMT to reduce the noise.

Radiated electromagnetic interference (EMI) picked-up by the cables and conducted noise from the HV power supply can be synchronous across many PMT channels (coherent noise) that could be added-up producing false detector triggers. As the PMT signal can be as low as few mV, another important issue is the control of the EMI over the circuit. The EMI induced and conducted by the power supply cables will be reduced by the splitter HV input filter. To reduce the EMI directly received in the splitter circuit as well as the cross-talk between different splitter channels, each splitter channel will be enclosed into an individual metallic grounded box.

Figure \ref{fig:dppd_4_2} shows a generic splitter circuit where R1 and C1 form the HV input low pass filter (cut off frequency bellow 60 Hz). The resistor R7 and the LED have safety purpose only, warning when high voltage is applied to the splitter. C4 capacitor splits the signal coming from the PMT from the HV, and R2 prevents that the PMT signal goes to ground through the C1 capacitor. R4 and R5 are zero ohm optional resistors that allow some flexibility on the grounding configuration. Finally, R3 ensures the discharging of C4 if the splitter is not connected to the \SI{50}{$\Omega$} input at the DAQ system. The RC constant of the capacitor C4 and the load (\SI{50}{$\Omega$}) must be as big as possible to minimize baseline oscillations due to the charge-discharge of the capacitor. Values of C4 between \SI{150}{nF} and \SI{300}{nF} have already been tested on the WA105 $3\times1\times1$\,m$^3$ detector.

\begin{dunefigure}[Generic splitter circuit diagram.]{fig:dppd_4_2}
{Generic splitter circuit diagram.}
\includegraphics[width=0.75\textwidth]{dppd_4_2}
\end{dunefigure}

For the connections between the HV power supply and the splitters and, also, between the splitters and the cryostat feedthroughs, the HTC 50-3-2 cable have been chosen as baseline. The HTC 50-3-2 has a similar attenuation compared to the RG-303/U (used inside the cryostat), but, with a factor of 8 to 10 lower cost. Both cables will be attached on one side directly to the HV splitter and will have an SHV connector on the other end. For the connection between the splitter and the front-end an RG-58 cable ended on the connector required by the front-end card will be used.

%%%%%%%%%%%%%%%%%%%%%%%%%%%%%%%%%
\subsection{Signal Readout Requirements}
\label{sec:fddp-pd-4.3}

In order to meet the physics requirements, the information that needs to be extracted from the PMT signals is the following:

\begin{itemize}
\item S1 fast component shape, charge and timing
\item S1 slow component shape
\item S2 shape, charge and timing (distance from S1 and duration)
\item Single photoelectron (SPE) charge spectrum for gain calculation during PMT calibration
\item Trigger signal generation by the coincidence of several PMT signals
\end{itemize}

At this moment, we do not have an estimate of the \textbf{dynamic range }of the light that could reach the PMTs on the DUNE-DP far detector. Our calculations are based on the signals detected by the PMTs on the WA105 $3\times1\times1$\,m$^3$ detector. Although this prototype has a different dimension from the DUNE-DP detector, it is the only reference that we have for these estimates, until the ProtoDUNE-DP detector and simulations are operational.

In general, the PMT signal dynamic range goes from the mV level to several volts (over \SI{50}{$\Omega$} load). During the operation of the WA105 $3\times1\times1$\,m$^3$ detector, PMT signals larger than \SI{2}{V} were observed with PMT gains around \num{10}$^6$. Figure~\ref{fig:dppd_4_3_ab} shows the SPE waveforms (left, normalized) and amplitudes (right) for the WA105 $3\times1\times1$\,m$^3$ detector at different voltages. The light levels in the DUNE-DP detector will have a larger dynamic range due to its large volume, so, higher gains will be required to see the far light signals. However, higher gains will make closer light signals to produce larger outputs, so, it is also essential that the front-end electronics can cover a large range of input voltages. To cover a dynamic range of \SI{10}{V} with a resolution below the mV level, \num{14} bits will be necessary (least significant bit (LSV) $\sim$\SI{0.6}{mV}). For \SI{2}{V} of dynamic range \num{12} bits would be sufficient (LSB $\sim$\SI{0.5}{mV}). To finalize the required dynamic range, results from ProtoDUNE-DP and relevant simulations are needed.

\begin{dunefigure}[SPE waveforms and amplitudes from WA105 $3\times1\times1$\,m$^3$ detector at different voltages.]{fig:dppd_4_3_ab}
{SPE waveforms (left) (normalized for comparison) and amplitudes (right) from WA105 $3\times1\times1$\,m$^3$ detector at different voltages.}
\includegraphics[width=0.47\textwidth]{dppd_4_3_a}
\includegraphics[width=0.47\textwidth]{dppd_4_3_b}
\end{dunefigure}

To calculate the PMT gains, the SPE charge measurement will be performed. Depending on the PMT gain, the SPE amplitude varies from the mV level to hundreds of mV, as it is shown in Figure \ref{fig:dppd_4_3_ab}. Due to the very long cables from the PMTs to the front-end electronics, the noise into the cables could be high. If one considers a noise level around \SI{1}{mV},  the PMT gain must be set to \num{10}$^6$ or higher in order to distinguish the SPE from noise. The average SPE pulse width is around \SI{3.5}{ns} full width at half maximum (FWHM). In order to digitize this signal to reconstruct it with fidelity, a sampling period on the order of \SI{}{ns} is required.

The \textbf{sampling frequency} also affects the time tagging precision. The time uncertainty due to the PMT alone is around \SI{3}{ns} (transit time spread). There will be other factors, e.g. Rayleigh scattering, that will increase this uncertainty. The sampling period will also increase this uncertainty, so, the lower sampling frequency, the better. At the WA105 $3\times1\times1$\,m$^3$ detector \SI{4}{ns} sampling was used to digitize waveforms. 

The rate of the events observed on the WA105 $3\times1\times1$\,m$^3$ detector was around \SI{300}{kHz} with the threshold at the SPE level. The rate at the DUNE FD, much bigger although placed underground is not know yet, but the time tagging system should be able to process events at high rates to ensure that no events are lost. Figure~\ref{fig:dppd_4_3_c} shows the event rates for different trigger thresholds observed with the WA105 $3\times1\times1$\,m$^3$ detector.

\begin{dunefigure}[Event rates for different trigger thresholds observed on the WA105 $3\times1\times1$\,m$^3$ detector.]{fig:dppd_4_3_c}
{Event rates for different trigger thresholds observed on the WA105 $3\times1\times1$\,m$^3$ detector.}
\includegraphics[width=0.6\textwidth]{dppd_4_3_c}
\end{dunefigure}

The light signal has to be \textbf{synchronized} with the DAQ. All the DAQ electronics will be in sync using White Rabbit protocol. A dedicated White Rabbit $\mu$TCA \cite{utca} slave node will be on the light read-out front-end electronics as sync receiver, distributing clocks to the different front-end cards.

%%%%%%%%%%%%%%%%%%%%%%%%%%%%%%%%%%%%%%%%%%%%%%%%%%%%%%%%%%%%%%%%%%%%
\section{Photon Calibration System}
\label{sec:fddp-pd-5}

%%%%%%%%%%%%%%%%%%%%%%%%%%%%%%%%%
\subsection{System Design and Procurement}
\label{sec:fddp-pd-5.1}

A photon calibration system integrated in DUNE DP far detector is required to monitor the calibration of the PMTs installed inside the LAr volume. The goal is to determine the PMT gain and maintain the PMT performance stability. A similar design as the one used in ProtoDUNE-DP will be used although some R\&D measurements are planned to make it more effective, reduce the cost and mitigate issues related to the scaling.

In ProtoDUNE-DP, an optical fiber will be installed at each PMT in order to provide a configurable amount of light (see Fig.~\ref{fig:dppd_3_2}). The calibration light will be provided by a blue LED of \SI{460}{nm} using a Kapuschinski circuit as LED driver which reduces significantly the cost of using a laser. There will be one LED connected to one fiber going to one female optical feedthrough from Allectra~\cite{allectra}. In total,  there will be six LEDs placed in a hexagonal geometry. The direct light will go to the fiber, and the stray light to a SiPM used as reference sensor, being a single reference sensor in the center. Fibers of length 22.5-m (from Thorlabs $\phi$ 800-$\mu$m, FT800UMT~\cite{ft800umt}, and stainless-steel jacket) will be used inside the cryostat. Each one of these fibers will be attached to a 1-to-7 fiber bundle (from Thorlabs $\phi$ 200-$\mu$m, FT200UMT~\cite{ft200umt}, stainless-steel jacket common end, and black jacket at split ends), so that one fiber is finally installed at each PMT. A diagram of the ProtoDUNE-DP photon calibration system is shown in Figure~\ref{fig:dppd_5_1}. Several tests to quantify the light losses of this design were performed with successful results. 

\begin{dunefigure}[Diagram of the photon calibration system to be implemented in ProtoDUNE-DP.]{fig:dppd_5_1}
{Diagram of the photon calibration system to be implemented in ProtoDUNE-DP}
\includegraphics[width=0.4\textwidth]{dppd_5_1}
\end{dunefigure}

Assuming the ProtoDUNE-DP design for the DUNE FD with \num{720} PMTs, \num{120} bundles, \num{120} fibers, \num{120} light sources, \num{120} flange feedthroughs, and \num{20} reference sensors will be needed. The length of the fibers and bundles has to be calculated considering the exact position of the feedthrough flanges. The number of flanges required to host \num{120} SMA feedthroughs will depend on their size. However, alternatives to this design will be pursued with R\&D measurements in order to reduce the amount of fibers, study other options for the reference sensor, and increase the input light if necessary. In order to reduce the number of fibers, light diffusers can be used, so that one fiber can illuminate at least \num{4} PMTs. For instance, a diffuser could be placed at the ground grid. 

%%%%%%%%%%%%%%%%%%%%%%%%%%%%%%%%%
\subsection{Validation Tests}
\label{sec:fddp-pd-5.2}

In order to validate the design, the most important result will come from the ProtoDUNE-DP performance. In any case, since the fibers to be used in DUNE FD will be longer, dedicated calculations and measurements to confirm that sufficient light reaches the PMTs will be performed. Also, alternative designs, will be validated in different laboratories. The possibility of using a diffuser can be tested in a vessel. The light source will also be validated by studying the different options in the lab. All these measurements will be performed at room temperature and in liquid nitrogen to test the behavior at cryogenic temperatures.

Once the design is fixed, basic characterization measurements will be performed on the fibers upon receiving them from the manufacturer. Those measurements will consist of providing light with a known source and measuring the output with a power meter. Measurements at cryogenic temperatures may not be needed at this point.

Finally, during the photon calibration system installation, each fiber and source will be re-tested to check that the expected light is arriving to each PMT using a photodiode. A dedicated procedure will be designed with this purpose, similar to the one used in ProtoDUNE-DP.

%%%%%%%%%%%%%%%%%%%%%%%%%%%%%%%%%%%%%%%%%%%%%%%%%%%%%%%%%%%%%%%%%%%%
\section{Photon Detector Performance}
\label{sec:fddp-pd-6}

To define the PDS performance, a good understanding of the light generation is needed. For this, optical simulations and a good knowledge of the light properties are required. The DUNE experiment expects to record not only accelerator neutrino interactions, but also rare non-beam events such as supernova neutrino bursts or nucleon decays. In those cases, an internal trigger is required: an optimized light collection system is hence mandatory. This section will describe the tools developed in the consortium for the light simulation in large detector volumes for these purposes.

The main feature of a LAr TPC detector is to collect electrons produced by the energy loss of charged tracks when crossing the volume. This signal provides a high resolution 3D image of the event. The reconstructed topology and the amount of charge collected gives the characterization of the tracks (identification and energy). Together with the charge, scintillation light is also produced in LAr. There are many advantages to collect and exploit the scintillation signal. As only a fraction of the initial energy deposition is converted into electrons, the rest being emitted as photons, light collection can improve the calorimetry of the detector. The light signal can provide the $t_0$ of the event, which is a necessary observable for a proper reconstruction. The study of the slow component can give insights into the purity of the LAr. 

When energy deposition occurs, either the knocked argon atom gets excited or an electron is ejected. For the latter case, the electron has a probability to be recaptured by an argon ion, which depends on the drift field and on the amount of energy deposited. In this case, an excited argon state is also produced. In order to decay to ground state, the excited argon will combine with another argon atom, to form an excited eximer. A photon at \SI{127}{nm} will then be emitted to allow the eximer to return to ground state. As the eximer can be formed in a singlet or triplet state, two time constants will be observed: the singlet at \SI{6}{ns}
and the triplet at \SI{1.3}{$\mu$s}. These principles are sketched in Fig.~\ref{fig:dppd_6_0}.

\begin{dunefigure}[A sketch depicting the mechanism of light production in argon.]{fig:dppd_6_0}
{A sketch depicting the mechanism of light production in argon.}
\includegraphics[width=0.8\textwidth]{dppd_6_0}
\end{dunefigure}

% \begin{dunefigure}[A sketch depicting the mechanism of light production in argon.]{fig:dppd_6_0}
% {A sketch depicting the mechanism of light production in argon.}
% \begin{tikzpicture}
% \node[circle, very thick,draw=black, fill=white, inner sep=0pt,minimum size=.9cm, text=black] (ini) at (0,0) {Ar};
% \draw[very thick,red,->](-.95, -.5) -- (0,.8) node [pos=.5, color=red, above, sloped] {track$^\pm$};
% \node[circle, very thick,draw=black, fill=white, inner sep=0pt,minimum size=.9cm, text=black] (exc) at (4,1.5) {Ar$^*$};
% \draw[-latex, bend left=10] (ini) to node[pos=.5, sloped, above] {excitation} (exc);
% \node[circle, very thick,draw=black, fill=white, inner sep=0pt,minimum size=.9cm, text=black] (ion) at (2.5,-2) {Ar$^+$};
% \node[circle, very thick,draw=black, fill=white, inner sep=0pt,minimum size=.6cm, text=black] (ele) at (3.5,-2) {e$^-$};
% \draw[-latex, bend right=10] (ini) to node[pos=.5, sloped, below] {ionization} (ion);
% \draw[-latex] (ele) to (4.5, -2);
% \begin{scope}[xshift=5.2cm, yshift=-2.8cm, scale=0.3]
% \begin{axis}[
% 		axis line style = thick,
%         domain=0.001:2.1,
%         xmin=0., xmax=2.1,
%         ymin=0, ymax=1.05,
%         samples=400,
%         grid=major,
%         xlabel={Drift Field [kV/cm]},
%         ylabel={e$^-$ rec. factor},
%         label style={font=\Huge},
%         tick label style={font=\huge} 
%         ]
%         \addplot+[mark=none, line width=4pt, teal] {0.8/(1+0.0741/x)};
%  \end{axis}
%  \end{scope}
% \node (rec) at (6.1, -0.8) {Recombination};
% \node[circle, very thick,draw=black, fill=white, inner sep=0pt,minimum size=.9cm, text=black] (ion2) at (8.2,0) {Ar$^+$};
% \node[circle, very thick,draw=black, fill=white, inner sep=0pt,minimum size=.6cm, text=black] (ele2) at (9,-0.05) {e$^-$};
% \node[circle, very thick,draw=black, fill=white, inner sep=0pt,minimum size=.9cm, text=black] (ato) at (8.7,-0.8) {Ar};
% \draw[-latex, bend left = 10] (rec) to (ion2);
% \node[ellipse, very thick, draw=black, fill=white, minimum height=2.7cm, minimum width=1.4cm] (ell) at (11,1) {};
% \node[text=black] at (11.5, 2.2) {*};
% \node[circle, very thick,draw=black, fill=white, inner sep=0pt,minimum size=.9cm, text=black] (exc1) at (11,1.5) {Ar};
% \node[circle, very thick,draw=black, fill=white, inner sep=0pt,minimum size=.9cm, text=black] (exc2) at (11,0.5) {Ar};
% \draw[-latex, bend left = 10] (exc) to (ell);
% \draw[-latex, bend right = 10] (ele2) to (ell);
% \node (light) at (14, 1) [draw=orange,very thick,fill=white, minimum width=2cm,minimum height=1cm] {Light Signal};
% \draw[-latex, thick](ell) to (light);
% \node (charge) at (10, -2) [draw=blue,very thick,fill=white, minimum width=2cm,minimum height=1cm] {Charge Signal};
% \draw[-latex, thick](7.4, -2) to (charge);
% \end{tikzpicture}
% \end{dunefigure}

In the DP technology, due to the amplification area, there are two light signals produced. The first one, S1, is made by scintillation processes when a charged particle crosses the LAr volume. The second signal, S2, is produced in the gaseous phase. As the drifting electrons enter in high field regions (such as the extraction field or the amplification field in the LEMs), their velocities increase and Townsend avalanches occur. The current of electrons will produce electroluminescence light with the same wavelength and similar time structure as for the S1 signal. %The minimum field needed to produce electroluminescence is $\sim$\SI{3.5}{kV/cm} at the gas density at cryogenic temperatures. 
The S2 light is expected to be an irreducible background for the light studies in ProtoDUNE-DP, as the detector will be on the surface. Indeed, the S2 signal can last as long as the total drift time of the electrons: \SI{0.625}{ms} per meter of drift at a drift field of \SI{500}{V/cm}.

Table~\ref{tab:dppd_t_6_0} summarizes the default optical parameters chosen for the light simulation methods described in the following subsection. The LAr optical properties are the subject of significant discussions in the community, in particular regarding the LAr absorption length and the Rayleigh scattering length. The former will affect the light yield collected whereas the latter will impact mostly its uniformity and timing resolution. The absorption/reflection of the VUV photons on stainless-steel (constituting the drift cage, cathode, extraction grid and ground grid) and on copper (on the LEM surfaces) are poorly known. The knowledge of those reflection coefficients is limited by the fact that they depend strongly on the polishing procedure. Hence, one cannot rely on the literature as the tooling will certainly be different. The measurement of the quantum efficiency of the PMTs at vacuum ultra-violet (VUV) wavelengths requires a specific setup operating in vacuum as VUV photons are absorbed in air. For the construction of the  WA105 $3\times1\times1$\,m$^3$ detectorDP demonstrator, the PMT quantum efficiencies were measured before and after the TPB coating using an LED that could emit light in the [\num{200}, \num{800}]\,nm range. Finally, the electroluminescence gain G, defined as the number of S2 photons produced per extracted drifting electron, is also subject to discussion. Experimental measurements of G have been performed in a setup quite similar to the amplification design of the DP technology, although the amount of photons emitted were measured above the LEM. In our case, the S2 photons are the ones leaving the LEM from below, which can be significantly lower. 

\begin{dunetable}
[Default optical properties. Below the thick line are presented some quantities used in our studies although they are not linked to the optical properties of the LAr.]
{lcc p{0.8\textwidth}}
{tab:dppd_t_6_0}
{Default optical properties. Below the thick line are presented some quantities used in our studies although they are not linked to the optical properties of the LAr.}
 & VUV photons & Shifted photons \\ 
 & $\lambda$ = \SI{127}{nm} & $\lambda$ = \SI{435}{nm}\\ \toprowrule
 Absorption length & \multicolumn{2}{c}{$\infty$} \\ \colhline
 Rayleigh scattering length & \SI{55}{cm} & \SI{350}{cm}\\ \colhline
 Absorption coefficients & \num{100}\% & \num{50}\% \\ \colhline
 LAr refractive index & \num{1.38} & \num{1.25}\\ \colhline
 PMT quantum efficiency & \multicolumn{2}{c}{0.2 }\\ \colhline
 Electroluminescence gain & \num{300}\\ \colhline
\end{dunetable}

To understand the performance of the PDS, we need to take into account the following indicators:
\begin{itemize}
\item Overall detected light yield, in PEs per MeV of deposited energy in LAr
\item Uniformity of the light yield across the entire LAr TPC active volume
\item Event time resolution extracted from the detected photon signal 
\end{itemize}

In turn, these indicators will directly impact the strategy and performance of the DUNE trigger system (Sec.~\ref{sec:fddp-pd-7}), and will determine whether the photon detector technical design is sufficient to meet the DUNE physics goals. These higher-level studies will be available on the TDR timescale. Our current understanding of the performance indicators listed above is largely based on ProtoDUNE-DP simulations. The current status of the simulation work is discussed in detail in Sec.~\ref{sec:fddp-pd-6.1}, work is focused on ProtoDUNE-DP in a first phase, and then will be expanded to DUNE DP FD. For a realistic ProtoDUNE-DP geometry, an average light yield of \SI{2.5}{PEs/MeV} is expected across the entire active volume. This promising yield is obtained  by assuming thirty-six 8-inch PMTs located below the ProtoDUNE-DP cathode plane, averaging to one PMT per m$^2$. On the other hand, spatial non-uniformities in the photon detector response are found to be important and need to be modeled in detail. Variations as large as one order of magnitude both parallel to the drift direction (due to geometrical effects and absorption of light by LAr) as well as perpendicular to it (due to light absorption on detector boundaries) are obtained. The event time resolution due to light production and light propagation times, hence neglecting electronics and DAQ effects for now, is expected to be of order $\mathcal{O}$(\SI{100}{ns}) and hence largely sufficient for our purposes. These initial low-level performance estimates will be refined with more realistic simulations and with ProtoDUNE-DP data (Sec.~\ref{sec:fddp-pd-6.2}) in the future. They will also be extended to the full FD geometry on the TDR timescale.

%%%%%%%%%%%%%%%%%%%%%%%%%%%%%%%%%
\subsection{Simulations}
\label{sec:fddp-pd-6.1}

At zero drift field, when the electron recombination is maximum, roughly \SI{40000}{$\gamma$/MeV} are produced. At the nominal drift field of \SI{500}{V/cm}, then \num{24000}{$\gamma$/MeV} are generated. For reference, the energy deposited by a minimum ionizing particle (MIP) track is \SI{2.12}{MeV/cm}. Given the size of the ProtoDUNE-DP ($6\times6\times6$\,m$^3$) and the fact that it is located on surface, roughly \num{100} muons are expected to cross the fiducial volume during the \SI{4}{ms} time window of the data acquisition. With a full Geant4 \cite{geant4} simulation, it takes more than three hours to propagate all the photons emitted by a single MIP track crossing the ProtoDUNE-DP detector. A full optical simulation is hence computationally prohibitive. Three simulation approaches are being explored to provide a light simulation needed for the design optimization of the DUNE FD module, described in the following.


\subsubsection{Generation of light maps}
\label{subsec:fddp-pd-6.1.1}

In this method, the photons are propagated in a full dedicated Geant4 simulation only once. The main light characteristics (photon detection probability called visibility hereafter, and time profile) needed for light studies are stored in a map in a ROOT \cite{root} file format which can then be read by any other simulation program. This work was done first using LightSim, a dedicated software developed at LAPP. These maps have been adapted to be read by LArSoft and work is on-going to directly generate them in LArSoft where light maps are known as photon libraries. In particular, S2 light needs to be simulated in LArSoft for the first time, as no previous effort on DP technology was done.

In the dedicated Geant4 code, special care has been taken to precisely describe all sub-detector components that might affect the light propagation: LEM plates, extraction grid, field cage rings, the cathode and its supporting structure and the ground grid above the PMTs. The LAr fiducial volume is then divided into voxels of \SI{25}{cm^3} and \num{10}$^8$ photons are isotropically generated at the center of each voxel. The number of photons reaching each PMT, and their arrival times are stored. The light map can then be built from these results. For each voxel and for each PMT, the visibility is computed as: $w=N\gamma^{\textrm{collected}}/N\gamma^{\textrm{generated}}$. In order to be able to reproduce the time profile, each distribution is fit to a Landau function. From the fits, three parameters are extracted: the minimum time for photons to arrive to the PMT, $t_0$; the peak of the distribution, $t_{\textrm{peak}}$ from the Landau most probable value (MPV); and the distribution spread, the $\sigma$ of the Landau function. These three parameters are stored in the light maps for each photon detector. The same procedure is done for the gaseous phase, although the voxel size is smaller in height (only \SI{5}{mm}). In Fig.~\ref{fig:dppd_6_1_1_ab}, two fitted time distributions are presented. As one can see, the shapes of the time distributions depend strongly on the source to PMT distance. For close sources, the distributions are very sharp and the Landau description may not be the optimal function to use. On the other hand, when the distance is larger, the distribution is broader and the Landau fit reproduces the simulations quite well. In order to minimize the amount of parameters to be stored in the map, the Landau descriptions for all cases were kept as only a small fraction of the fits could be considered problematic.

\begin{dunefigure}[Landau fits (red line) of the travel time distributions (black histogram) for a source close (left) and far (right) to the PMT.]{fig:dppd_6_1_1_ab}
{Landau fits (red line) of the travel time distributions (black histogram) for a source close (left) and far (right) to the PMT.}
\includegraphics[width=0.45\textwidth]{dppd_6_1_1_a}
\includegraphics[width=0.45\textwidth]{dppd_6_1_1_b}
\end{dunefigure}

As the map has been computed with discrete entries, an interpolation of the four light parameters ($w$, $t_0$, $t_{\textrm{peak}}$, $\sigma$) between the actual source position and the closest voxel centers is performed. An example of the evolution of the visibility and its 3D interpolation is presented in Fig.~\ref{fig:dppd_6_1_1_cd}. The loss of photons due to the cathode and ground grid are visible. Given the ProtoDUNE-DP cathode and supporting structure design, and considering the default optical parameters presented in Table~\ref{tab:dppd_t_6_0}, it has been shown that up to $\sim$\num{70}\% of the photons generated in the active volume are absorbed by those structures before reaching the PMT array.

During the generation of the light maps, the light propagation parameters are the ones presented in Table~\ref{tab:dppd_t_6_0}. One can study afterwards the loss of photons due to the LAr absorption length using an approximation of the probability of the photon to be absorbed by the medium as: $p_{\textrm{abs}} = \exp(-\frac{D_{\textrm{travel}}}{\lambda_{\textrm{abs}}})$. For the study of other light propagation parameters (Rayleigh scattering and absorption on the stainless-steel and copper) new maps have to be generated.

The light maps generation is a long process. It takes roughly three days of computing time to generate the maps for ProtoDUNE-DP, even though only 1/8$^{th}$ of the voxels were simulated as the detector and the PMT positioning is symmetric. Generating maps for larger volumes such as the DUNE FD module, where the maximum source to PMT distance will be around \SI{60}{m}, could be too much time consuming. Moreover, the light simulation in the FD is foreseen to drive the optimization of the positioning of the PMTs and will guide the studies of possible implementation of light reflectors. As most of the light propagation parameters in LAr are still subject to large uncertainties, these studies will have to be performed considering various absorption and diffusion values. Therefore, it is crucial to be able to have a faster way to get a quite reliable light simulation, at the cost of losing some precision.

\begin{dunefigure}[Evolution of the visibility seen by a central PMT (pointed by the arrow) in ProtoDUNE-DP as a function of different source positions in $x$ and $z$ ($y$ is set at \SI{0}{mm}). The position of the cathode and the ground grid are highlighted.]{fig:dppd_6_1_1_cd}
{Evolution of the visibility seen by a central PMT (pointed by the arrow) in ProtoDUNE-DP as a function of different source positions in $x$ and $z$ ($y$ is set at \SI{0}{mm}). The position of the cathode and the ground grid are highlighted. Results are limited by the number of photons generated (\num{10}$^7$ photons per voxel), and voxels with less than \num{50} photons arriving to the PMT are not taken into account. Left: discrete values from the maps, right: after 3D interpolation.}
\includegraphics[width=0.45\textwidth]{dppd_6_1_1_c}
\includegraphics[width=0.45\textwidth]{dppd_6_1_1_d}
\end{dunefigure}

\subsubsection{Parametrization from the light maps}
\label{subsec:fddp-pd-6.1.2}

Without considering the border effects, where the photons are mostly absorbed, it is intuitive that the visibility and the time profile depend only on the source to PMT distance.

This approach has been followed for the SBND~\cite{sbnd} light simulation and is considered for the DUNE-DP module as well. In Fig.~\ref{fig:dppd_6_1_2}, the evolution of the visibility and the peak time as a function of the source to PMT distance are shown. As these plots have been generated from the light maps, where the borders are taken into account, the same evolutions are also presented only for voxels at least \SI{1}{m} away from the active volume boundaries. For the visibility, the structure is quite complicated when taking all the voxels highlighting the complexity of the light simulation in a closed space. When looking at voxels away from the boundaries, one can see a clear correlation between distance and visibility. As for the time distribution (here for the peak time, but same goes for $t_0$ and $\sigma$ parameters), one can notice two different regimes for short and large distance (the transition being at around \SI{2}{m}).

This preliminary study is quite encouraging for the light simulation in the FD module, at least for light sources being far away from the fiducial volume boundaries. As it is complicated to disentangle the effects due to the propagation and absorption parameters from the light maps, a careful dedicated study should be performed in order to get parametrization of the visibility and time distribution parameters as a function of the photon traveling distance.

\begin{dunefigure}[Evolution of the visibility and peak time as a function of the source-PMT distance as simulated in the ProtoDUNE-DP geometry (Preliminary study).]{fig:dppd_6_1_2}
{Evolution of the visibility (top) and peak time (bottom) as a function of the source-PMT distance as simulated in the ProtoDUNE-DP geometry (Preliminary study). On the left, all voxels are considered, on the right only the voxels at least 1\,m away from the fiducial border are considered.}
\includegraphics[width=0.75\textwidth]{dppd_6_1_2}
\end{dunefigure}

\subsubsection{Analytical approach}
\label{subsec:fddp-pd-6.1.3}

The propagation of light in a uniform material such as LAr can be described by the Fokker-Planck diffusion equation:

$$\frac{\partial}{\partial t}p(x,y,z,t) = D\left[\frac{\partial^2}{\partial x^2}p(x,y,z,t) + \frac{\partial^2}{\partial y^2}p(x,y,z,t) + \frac{\partial^2}{\partial z^2}p(x,y,z,t)\right]$$ 

where $D$ is the diffusion coefficient. In an unbound medium, the Fokker-Planck equation is solved by the Green function:

\begin{eqnarray*}
G(\textbf{r}, t; \textbf{r}_0, t_0) &=& \frac{1}{[4\pi D c (t-t_0)^{3/2}]}\exp\left(-\frac{|\textbf{r}-\textbf{r}_0|^2}{4Dc(t-t0)}\right) \\
D &=& \frac{1}{3(\mu_A + (1-g)\mu_S)}
\end{eqnarray*}

where $\mu_A$ and $\mu_S$ are the absorption and scattering coefficients respectively (both in units of m$^{-1}$), $g$ is the average scattering cosine ($g$ = \num{0.025}). In LAr with the default optical properties of Table \ref{tab:dppd_t_6_0}, $D$ = \SI{18.8}{\cm}. In a bound medium, with full absorption of the photons by the field cage and LEMs, a few additional techniques have to be used to obtain a solution. With this method, it takes only a few ms to have the photon density at a given photon detector from a specific point source. From preliminary studies, a relatively good agreement between analytical approach and full simulation has been found. In particular, the arrival time distributions of photons on the PMTs are well reproduced. The only drawback is that one cannot easily implement/study a complicated geometry including regions that are semi-transparent to light. Hence, the comparisons of the visibilities that one gets from the two methods are not in agreement in the overall light yield, but have a very similar trend in terms of spatial dependences. Some studies are ongoing in order to improve the analytical method results as this approach could be extremely powerful for physics studies in the FD module.

\subsubsection{Simulation of light yield}
\label{subsec:fddp-pd-6.1.4}
The light collected per PMT can be simulated together with the charge for crossing tracks in a standard simulation code where a detailed description of the detector is not needed. At each step of the track propagation, the energy deposited is computed by Geant4. This energy is converted into number of electrons and photons produced. As for the light simulation, the number of photons reaching each PMT and their time of arrival is now obtained from the light maps. As an example, the light yield from a uniform generation of \SI{10}{MeV} electrons in the active volume is shown in Fig. \ref{fig:dppd_6_1_4}. The number of PEs/MeV shown is summed over all PMTs and average over the $y$ axis ($z$ being the drift direction). One can notice the large spread in terms of light yield.

\begin{dunefigure}[Light yield in terms of PE/MeV summed over all PMTs and averaged along the y-axis.]{fig:dppd_6_1_4}
{Light yield in terms of PE/MeV summed over all PMTs and averaged along the y-axis. The mean of all voxels gives a light yield of 2.5 PE/MeV, although the distribution is not uniform, in particular along the z (drift) axis.}
\includegraphics[width=0.6\textwidth]{dppd_6_1_4}
\end{dunefigure}

For larger volumes such as the DUNE FD, the light maps might be too big and the time spent accessing the four parameters might strongly reduce the speed of the simulation. Either the parametrization method or the analytical approach are foreseen to replace the current light map usage, the exact strategy is yet to be defined.
%%%%%%%%%%%%%%%%%%%%%%%%%%%%%%%%%
\subsection{Light data in DP prototypes}
\label{sec:fddp-pd-6.2}

The DP demonstrator (WA105 $3\times1\times1$\,m$^3$) was operated from June to November 2017 with cosmic data. About \num{5} million light events were taken with various configurations. The study of the S1 light as a function of the drift field was performed. An example of an averaged waveform fitted to a fast and a slow scintillation components is shown in Figure~\ref{fig:dppd_6_2}. The amount of S2 light can be monitored as a function of the extraction and LEM amplification fields.

\begin{dunefigure}[Averaged waveform of the S1 light signal taken with one PMT from the WA105 $3\times1\times1$\,m$^3$ LAr DP TPC]{fig:dppd_6_2}
{Averaged waveform of the S1 light signal taken with one PMT from the WA105 $3\times1\times1$\,m$^3$ LAr DP TPC, fitted with a function (red line) that is the sum of a Gaussian, parametrized by t$_0$ and $\sigma$, and two exponential functions, with decay time constants $\tau_{fast}$ and $\tau_{slow}$, and normalization factors $A_{fast}$ and $A_{slow}$}
\includegraphics[width=0.6\textwidth]{dppd_6_2}
\end{dunefigure}

Light maps have also been generated with the demonstrator geometry, and data/MC comparisons are ongoing. The preliminary results look promising, although the statistics in each setting and the relatively small size of the detector still constitute a challenge to extract the entire optical properties of the LAr.


%%%%%%%%%%%%%%%%%%%%%%%%%%%%%%%%%
\subsection{Simulation of physics events}
\label{sec:fddp-pd-6.3}

A preliminary study to understand whether the dual-phase photon detector technical design meets the experiment's physics requirements has been performed. In this study, event topologies of interest for DUNE physics have been simulated using LArSoft fast optical simulation tools.

The simulation framework used represents our current state-of-the-art. It includes realistic models for the primary scintillation production yields in LAr, for Rayleigh scattering in LAr, for detector optical properties (such as field cage reflectivity and cathode transparency), for the density of PMTs underneath the cathode, and for the quantum efficiency of the TPB-coated PMTs (taken to be \num{20}\%). On the other hand, these simulations do not yet include the full dual-phase far detector geometry, but are rather performed in a ProtoDUNE-DP geometry with the same average PMT density as the one proposed here for the far detector (one PMT per m$^2$). Relevant aspects such as secondary scintillation light emission in gaseous argon (a nuisance for event $t_0$ determination), light absorption in LAr, electronics effects, reconstruction effects, and background contributions coming from $^{39}$Ar decays are not accounted for either in this study. While more realistic simulation results including the above effects will be produced on the DUNE TDR timescale, this preliminary study already provides a sense of the capabilities of the planned photon detector design.

Figure~\ref{fig:dppd_6_3_1} shows the expected light yield for SN neutrino CC interactions. As a representative $\nu_e$ flux from a supernova neutrino burst, we assume a Fermi-Dirac distribution with $T$=\SI{3.5}{\MeV} temperature and no neutrino oscillation effects, yielding an average neutrino energy of about 11~MeV. Low-energy $\nu_e$ CC interactions throughout the entire LAr TPC active volume are generated with the LArSoft-based Marley package. For the assumed SN neutrino flux and for a single interacting neutrino (hence, after convoluting flux and cross-section effects), Marley expects about \SI{19}{\MeV} of energy deposited in the LAr active volume, primarily from the final state electron and from nuclear de-excitation gamma rays. The left panel of Fig.~\ref{fig:dppd_6_3_1} shows a broad light yield distribution, averaging at about \num{50} detected PEs per interaction and after summing all PMTs. This is as expected from the light yield distributions per deposited energy shown in Fig.~\ref{fig:dppd_6_1_4}. The right panel shows the fraction of SN $\nu_e$ CC interactions within the LAr TPC active volume above a given PE detection threshold, as a function of the PE threshold. From the figure, we conclude that about a 70\% fraction of SN $\nu_e$ CC interactions would be seen by the photon detector, if the detector threshold was set at \num{10}~PEs on the sum of the PMT charges.

\begin{dunefigure}[Photon detector response for simulated SN neutrino interactions in the ProtoDUNE-DP geometry.]{fig:dppd_6_3_1}
{Photon detector response for simulated SN neutrino interactions in the ProtoDUNE-DP geometry. Left panel: distribution of detected PEs per neutrino interaction, for SN $\nu_e$ CC interactions throughout the active volume. Right panel: fraction of SN $\nu_e$ CC interactions above PE threshold, as a function of the PE threshold.}
\includegraphics[width=0.44\textwidth]{dppd_6_3_1_1} \hfill 
\includegraphics[width=0.44\textwidth]{dppd_6_3_1_2} 
\end{dunefigure}

Figure~\ref{fig:dppd_6_3_2} shows the corresponding plots for a representative nucleon decay final state in DUNE, namely $p\to\bar{\nu}K^+$. Nucleon decay events are generated using GENIE, accounting for both initial and final state nuclear effects in argon nuclei. Particles exiting the nucleus are then propagated in LAr using all relevant, Geant4-based, physics processes. The deposited energy per nucleon decay, of order \SI{300}{\MeV}, is much higher than the one per SN neutrino interaction. As a result, the expected light yield for $p\to\bar{\nu}K^+$ events throughout the active volume, shown in the left panel of Fig.~\ref{fig:dppd_6_3_2}, averages to about 800~PEs in this case. The right panel of Fig.~\ref{fig:dppd_6_3_2} shows that about a 98\% fraction of $p\to\bar{\nu}K^+$ decays in the TPC active volume are expected to be seen by the photon detector, for a photon detector threshold of 10~PEs on the PMT charge sum.

\begin{dunefigure}[Photon detector response for simulated nucleon decays in the ProtoDUNE-DP geometry.]{fig:dppd_6_3_2}
{Photon detector response for simulated nucleon decays in the ProtoDUNE-DP geometry. Left panel: distribution of detected PEs per nucleon decay, for $p\to\bar{\nu}K^+$ decays throughout the active volume. Right panel: fraction of $p\to\bar{\nu}K^+$ decays above PE threshold, as a function of the PE threshold.}
\includegraphics[width=0.44\textwidth]{dppd_6_3_2_1} \hfill 
\includegraphics[width=0.45\textwidth]{dppd_6_3_2_2} 
\end{dunefigure}

%%%%%%%%%%%%%%%%%%%%%%%%%%%%%%%%%%%%%%%%%%%%%%%%%%%%%%%%%%%%%%%%%%%%
\section{Photon Detector Operations}
\label{sec:fddp-pd-7}

%%%%%%%%%%%%%%%%%%%%%%%%%%%%%%%%%
\subsection{Trigger Strategy}
\label{sec:fddp-pd-7.2}

As explained in section \ref{sec:fddp-pd-1.5}, the PDS will operate in different acquisition modes. These modes include the external trigger, which is the case of the beam events; the trigger for non-beam events such as SN bursts; and the calibration mode. 

In the LAr TPC there are different uses of the light signal: cosmic ray/track timing for the reconstruction; non-beam events trigger such as SN neutrino burst, atmospheric neutrinos, and proton decay; and calorimetry, as the light and charge signal are anti-correlated. These physics studies imply different requirements in terms of dynamics of the electronics and data sampling, from a few PE to a much higher level.

For the non-beam event trigger strategies, the requirements can be very different. In the event of a nearby (10\,kpc) SN burst, it is expected that a few thousands of neutrinos will homogeneously interact in the detector for a period as long as $\sim$\SI{100}{s}. Hence, the SN burst trigger strategy is mostly driven by the energy threshold set for $\nu$ detection and its efficiency: \SI{30}{MeV} is sufficient for a galactic SN, \SI{5}{MeV} is needed for a burst in Andromeda. A high-efficiency trigger for proton decay events has to be designed considering the worst case scenario, e.g. the event happening at the top of the detector, \SI{12}{m} away from the closest PMT. In order to minimize the amount of spurious triggers, one can think of signal thresholds for a cluster of close-by PMTs.

All these important studies will be further investigated once a reliable light simulation of the DP far detector module is available. For the DP technology, the main light trigger concerns are the amount of light collectable for a photon traveling distance of \SI{12}{m} and the S1/S2 separation. The data that will be collected in the ProtoDUNE-DP will provide crucial inputs for the optimization of the DP far detector light collection system and for the design of an efficient trigger strategy for rare non-beam events. 

The PDS trigger will be flexible to fulfill the different physics requirements explained before. The light readout front-end board will be in charge of the PDS trigger generation. The trigger will be decided based on the coincidence of several PMT signals over a threshold during a time window. The number of PMTs that contribute to the trigger, the signal threshold and the length of the coincidence time window will be programmable on-line to be able to adapt to different physics cases.


%%%%%%%%%%%%%%%%%%%%%%%%%%%%%%%%%
\subsection{Data Quality Monitoring}
\label{sec:fddp-pd-7.3}

The PMTs installed at the bottom of the tank will be operated for 10-20 years with no possibility to access them. Monitoring tools to ensure data quality of the PDS will have to be developed to catch any malfunctioning detector before data analysis. For instance, the amount of dark noise and the stability of the PMT response will have to be monitored over time. For the gain evolution, either studies of standard candles, e.g. from Michel electrons or average collected light produced by cosmic tracks, or with the dedicated calibration system are considered.

Monitoring tasks were performed during the 6 months of operation of the WA105 $3\times1\times1$\,m$^3$ with no dedicated light calibration system. This and the forthcoming operation of the ProtoDUNE-DP, will again provide crucial input towards the PDS monitoring system in the FD module.

%%%%%%%%%%%%%%%%%%%%%%%%%%%%%%%%%%%%%%%%%%%%%%%%%%%%%%%%%%%%%%%%%%%%
\section{Interfaces}
\label{sec:fddp-pd-8}

The PDS will have several interfaces with other subsystems and the global DUNE systems. The interface documents related to DP PDS are given in Table~\ref{tab:dppd_t_8}. Only part of the basic interfaces are summarized below. 

\begin{dunetable}
[DPPD interface documents]
{|l|c| p{0.8\textwidth}}
{tab:dppd_t_8}
{DPPD interface documents}

DPPD Interface Document & DUNE docdb number \\ \toprowrule
DP Electronics & 6772 \\
DP HV & 6799 \\
DAQ & 6802 \\
Cryogenic Instrumentation and Slow Control & 6781 \\
DUNE Physics & 7087 \\
Software and Computing & 7114 \\
Calibration & 7060 \\
Integration and Testing Facility & 7033 \\
Detector and Facilities (LBNF) Infrastructure & 6979 \\
Installation & 7006 \\
\end{dunetable}


\begin{itemize}

\item DP Electronics: The PDS will share the same front-end electronics standard as the charge readout, which is $\mu$TCA based \cite{utca}. Specifications of both PDS and front-end electronics will be determined by the simulations and ProtoDUNE-DP data.

\item HV: This interface includes the consideration of the distance between the cathode and the PMT planes, power dissipation from the PMTs and the combined impact on the simulations.

\item DAQ: The hardware interface will be mainly through optical fibers. DP PDS will be providing trigger and data in continuous streaming and the interface will also include the DAQ software.

\item Cryogenic Instrumentation and Slow Control: The main interface points are the layout of the cryogenic instrumentation (e.g. purity monitors and light emitting system for the cameras) and the PMT support structures and cabling; and the slow control and the PDS power supplies and calibration system.

\item DUNE Physics: DPPD will have interfaces with the overall physics requirements on energy and time together with classification of events, decay modes and neutrino flavors.

\item Software and Computing: This interface will be on the development of simulation, reconstruction and analysis tools.

\item Calibration: The DP PDS will participate in the Global Calibration Task Force and will provide handles to allow global monitoring of the PMT performance.

\item Integration and Testing Facility: The operations at the Integration and Testing Facility are described in Section~\ref{sec:fddp-pd-9.2}. The interface items can be summarized as shipping and receiving of the DP PDS components and basic testing and repairing at the facility. The interface also includes recycling/returning the packaging materials.

\item Detector and Facilities (LBNF) Infrastructure: DP PDS will have PMT mounting standing on the membrane, cold cables routed in cable trays to the ceiling feedthrough flanges and racks, and cable trays on top of the cryostat. Other interfaces with the facility include access to conventional facilities and participation in the detector safety systems. 

\item Installation: This interface will be through the transportation of the DP PDS components to and between underground areas, clean room activities and storage, and installation coordination with the other teams. 

\end{itemize}

%%%%%%%%%%%%%%%%%%%%%%%%%%%%%%%%%%%%%%%%%%%%%%%%%%%%%%%%%%%%%%%%%%%%
\section{Installation, Integration and Commissioning}
\label{sec:fddp-pd-9}

%%%%%%%%%%%%%%%%%%%%%%%%%%%%%%%%%
\subsection{Transport/Handling}
\label{sec:fddp-pd-9.1}

The PMTs of the PDS will be shipped from various locations following base and cable assembly for the TPB coating at the Integration and Testing Facility. The shipping boxes will contain \num{24} PMTs resulting in \num{30} total deliveries of \num{720} PMTs. The PMTs will be individually wrapped with different wrapping materials. The wrappings will have special openings to enable the basic electronics tests at the Integration and Testing Facility.

The PMTs will be placed in modular shock absorbing assemblies inside the boxes. The assemblies will also allow a limited amount of safe inclination. The boxes will have integrated pellets for easy handling and short distance transportation. The PMTs will reach the Integration and Testing Facility by means of air and ground transportation. Each box will have a dedicated bar-code which will be visible on each side. This bar-code will also be associated with the shipping documents. 

%%%%%%%%%%%%%%%%%%%%%%%%%%%%%%%%%
\subsection{Integration and Testing Facility Operations}
\label{sec:fddp-pd-9.2}

The PMT boxes will be received by the Integration and Testing Facility (ITF). A shipping and delivery database will be managed by the ITF. The received status of the boxes will be available to the DPPD Consortium as the boxes arrive at the ITF. The PDS characteristics database managed by the DPPD Consortium will be updated accordingly to reflect the received status of the contents of the boxes. Each PMT assembly will have identifying bar-codes that will be directly connected to the PDS characteristics database. This database will store the PMT serial number, the base board serial number, special information about TPB coating and assembly if any, and performance and calibration characteristics. This database will form the basis of the operations database providing the initial calibration values and it will also store information about the ITF tests and underground installation and commissioning tests.

The TPB coating will be performed at the ITF in the coating stations. Then, the PMTs will be placed back in their boxes, and dedicated testing electronics will be connected to the PMT cables soldered to the PMT bases. The test electronics will enable connecting several PMTs at a time. The tests will include basic functionality checks of both the PMTs and the base boards to assess the performance after transportation. No detailed performance characteristics will be measured at the ITF. The tests will be performed in a dedicated room with light and climate control. Once the performance of the PMTs in a box is validated, the boxes will be closed with the original covers. Before closing, additional quality checks on the shock absorbing assemblies will be made.

The preparation of the PMT boxes for underground transportation includes installing holding/lifting fixtures to the top and sides. The fixtures will allow crane operation. The boxes will be delivered to the surface station by ground transportation with relevant modification in the shipping database.

%%%%%%%%%%%%%%%%%%%%%%%%%%%%%%%%%
\subsection{Underground Installation and Integration}
\label{sec:fddp-pd-9.3}

Once the PMT boxes are underground, the same top and side covers will be opened as at the ITF. The PMTs will be carried to the underground storage area in sub-units of the shock absorbing assemblies which will be modular. The underground storage area for the PDS will be sufficiently large to store at least \num{30} PMTs for continuous installation operations.

The removal of the individual PMT wrappings will be done in the clean room. PMTs together with their base boards will go through visual inspection by the PDS installation supervisor. Once signed-off, the installation can proceed with multiple PMTs at a time by multiple teams. Cabling will be carried out in parallel and relevant database modifications will be made in-situ. The installation time management will be done in coordination with the cathode and field cage installation groups.

The bundles of cables will be routed through the cable trays along the cryostat walls from the PDS flanges. Following the mechanical mounting of the PMTs to the cryostat floor, the PMT cables will be 
connected to the cables coming from the flanges. In parallel, the calibration fibers will be installed and routed through cable trays.

%%%%%%%%%%%%%%%%%%%%%%%%%%%%%%%%%
\subsection{Commissioning}
\label{sec:fddp-pd-9.4}

The commissioning of the PDS will be performed in partitions. The size of the single partition will mainly be determined by the data acquisition system and the high voltage system. The data acquisition system and high voltage partitions will be commissioned, including the relevant control systems, prior to the connection of the PMTs to these systems. Once the physical sector corresponding to a partition is installed, the PMTs will be powered up and basic functionality and performance checks will be performed. These include pedestal data taking which consists of recording event data with external periodic triggering, and tests with the calibration system where the data taking is triggered in synchronization with a light source as described in Section \ref{sec:fddp-pd-5}.

As a result of the commissioning tests, the basic performance characteristics of the PMTs, e.g. the dark count rate and gain, will be measured in their final places. Installation-related issues will be identified and eliminated at this stage. A commissioned sector will be the part of the overall detector and can join the global calibration data taking and commissioning.

%%%%%%%%%%%%%%%%%%%%%%%%%%%%%%%%%%%%%%%%%%%%%%%%%%%%%%%%%%%%%%%%%%%%
\section{Quality Control}
\label{sec:fddp-pd-10}

%%%%%%%%%%%%%%%%%%%%%%%%%%%%%%%%%
 \subsection{Production and Assembly}
 \label{sec:fddp-pd-10.1}
 
The quality control performed at the different institutions labs will include reception of PMTs from the manufacturer and performing of the quality control tests to accept or return the PMTs according to the acceptance/rejection criteria.

\begin{itemize}
\item The PMT support structure design will be validated by immersing the PMT mounted on it at cryogenic temperatures and at an over-pressure equivalent pressure of the 12\,m depth of LAr of the detector.
\item Design validation tests will be carried out in order to confirm that the PMT base design fulfills the specifications at room and cryogenic temperatures. A cable with SHV connector will be soldered to each PMT base to make easier the different base and PMT tests and the final PMT connection during the installation. The PMT bases will be labeled (on the cable) in order to keep track of them. After production of the PMT base boards they will be individually tested before mounting to the PMT to verify that components are correctly mounted. Latter they will be cleaned and tested at maximum voltage on argon gas environment to confirm that there are no sparks on these (worst case) conditions.
After mounting the bases on the PMTs they will be tested again in argon gas at maximum voltage to confirm that there are no sparks due to bad soldering.
\item All the light readout units (PMT + base + support) will be tested and characterized in liquid nitrogen in order to check their performance at cryogenic temperature and to obtain a database with the most important parameters from each PMT (gain vs voltage, dark counts, etc.). The PMT base number attached to each PMT will also be included on the database. 
\item The wrapping materials and techniques will be studied with one fully assembled light readout unit. The handling, transportation and installation scenarios will be carefully studied and the transportation box design will be validated. The transport box and PMT wrapping must warrant darkness.
\item The light output of the LEDs and fibers light transmission from the photon calibration system will be measured with a power meter.
\end{itemize}

%%%%%%%%%%%%%%%%%%%%%%%%%%%%%%%%%
 \subsection{Post-Factory Installation}
 \label{sec:fddp-pd-10.2}
 
At the reception of the PMTs at ITF (Integration \& Test Facility), they will go through verification measurements in order to discard possible damages during transportation. Gain vs voltage and dark current values will be compared with the ones obtained before transportation.

The TPB coating will also be performed at the ITF. The first few samples will go through microscopic examination and surface uniformity tests, and the coating procedure will be validated. The production PMTs will be randomly sampled for basic coating quality assurance.

After the transport from the ITF to the laboratory the PMTs will be tested again before installation to confirm that there has not been any damage during the last transportation. During the installation, the PMTs database will be updated with the position in the detector of each PMT (identified by its serial number and base number). After installation, the full connection from the FE to the PMTs will be checked. The FE channel and splitter number connected to each PMT will, also, be included on the PMT database. At the moment that it could be possible to make darkness in the detector the PMTs will be tested applying voltage and checking the signal with a scope or with the FE electronics if they are already available.

%%%%%%%%%%%%%%%%%%%%%%%%%%%%%%%%%%%%%%%%%%%%%%%%%%%%%%%%%%%%%%%%%%%%
\section{Safety}
\label{sec:fddp-pd-11}

Safety is the highest priority at all stages of DPPD operations. Since DUNE is an international project, the international safety regulations will be followed closely during the course of preparation of safety documents.

Main risks at the production and testing sites are electrocution, exposure to excessive heat, chemicals and cryogenics, and heavy lifting. Detailed procedures will be developed by the relevant institutes and approved by the DPPD Consortium. Contents of the electrical safety rules will range from utilizing regular power equipment to handling PMTs for testing. The chemical and heat exposure hazards only concern the sites where the TPB coating is going to be performed. The heavy objects that will carry safety risks will mainly be the PMT delivery boxes.

The ITF DPPD safety regulations will be developed the same way. Main hazards on this site are electrocution and heavy lifting. Also, due to the density of shipments from all other subsystems, tripping and operations in limited space should also be considered.

The underground operation and installation safety rules will mainly follow the general facility rules on e.g. working in confined spaces, oxygen deficiency hazard and emergency procedures. DPPD specific safety rules will particularly be related to lifting of heavy objects for installation and working at heights for cabling.

%%%%%%%%%%%%%%%%%%%%%%%%%%%%%%%%%%%%%%%%%%%%%%%%%%%%%%%%%%%%%%%%%%%%
\section{Management and Organization}
\label{sec:fddp-pd-12}

The DPPD Consortium was formed in 2017 and it is composed by eleven institutes from France, Peru, Spain, UK and USA. The charge of the DPPD Consortium is to plan and execute the construction, installation and commissioning of the DUNE DP FD PDS.

%%%%%%%%%%%%%%%%%%%%%%%%%%%%%%%%%
\subsection{Consortium Organization}
\label{sec:fddp-pd-12.1}

The DPPD Consortium Leader (CL) is In\'{e}s Gil-Botella from CIEMAT (Spain) and the Technical Lead (TL) is Dominique Duchesneau from LAPP (France). They are members of the DUNE Technical Board and they represent the consortium to the overall DUNE collaboration. The CL is responsible for the subsystem deliverables and for the effective management of the consortium. The TL acts as the overall project manager and he is the interface to the International Project Office (IPO), and is responsible for monitoring/reporting progress against the agreed schedule and issues related to interface documentation.

The institutions participating in the consortium are responsible for the design or construction of a particular sub-system. It is hoped that the national groups within the consortia will be able to approach relevant funding agencies with a specific construction-phase proposal, such that a likely funding line can be established in or before 2019. The DPPD Consortium is open to any new institution willing to join the current effort.

The current institutions participating in the DPPD Consortium are summarized in Table \ref{tab:dppd_t_12_1}.

\begin{dunetable}
[DPPD Consortium institutions]
{|l|l|l| p{0.8\textwidth}}
{tab:dppd_t_12_1}
{DPPD Consortium institutions}

Country & Institution & Contact \\ \toprowrule
France & LAPP & Dominique Duchesneau \\
Peru & PUCP & Alberto Gago \\
Spain & IFAE & Thorsten Lux \\
Spain & CIEMAT & In\'{e}s Gil-Botella\\
Spain & IFIC & Michel Sorel \\
United Kingdom & University College London & Anna Holin \\
USA & Argonne National Lab & Zelimir Djurcic \\
USA & Duke University & Kate Scholberg \\
USA & University of Iowa & Jane Nachtman \\
USA & South Dakota School of Mines and Technology & Juergen Reichenbacher\\
USA & University of Texas at Austin & Karol Lang \\
% \colhline
\end{dunetable}

The DPPD Consortium is divided in four working groups: photosensors and electronics, calibration system, mechanics and integration, and simulation and physics. The corresponding WG conveners are:
\begin{itemize}

\item WG1: Photosensors and Electronics - A. Verdugo (CIEMAT)
\item WG2: Calibration System - C. Cuesta (CIEMAT)
\item WG3: Mechanics and Integration - B. Bilki (Iowa)
\item WG4: Sim. \& Phys. - K. Scholberg (Duke), M. Sorel (IFIC), L. Zambelli (LAPP)

\end{itemize}

The DPPD Consortium has regular bi-weekly meetings on Thursdays (4pm CET, 9am CST). Agendas and presentations can be found at: https://indico.fnal.gov/category/699/

%%%%%%%%%%%%%%%%%%%%%%%%%%%%%%%%%
\subsection{Planning Assumptions}
\label{sec:fddp-pd-12.2}

The optimization and final design of the DPPD system will be driven by:
\begin{enumerate}
\item ProtoDUNE-DP data (expected by beginning of 2019)
\item Simulation studies (in progress)
\end{enumerate}

ProtoDUNE-DP operation and data analysis are fundamental steps to understand if the current photon detection system considered as baseline, based on cryogenic PMTs with TPB coating, is able to provide $t_0$ for non-beam events, background rejection and triggering on non-beam events. These data will be used to tune the MC simulations and extrapolate the performance of the system to the DUNE Far Detector. 

Simulations are needed to determine and optimize the DPPD system to meet the physics requirements in terms of:
\begin{itemize}
\item Light collection efficiency
\item Number of channels
\item Photosensor requirements
\item Dynamic range of readout electronics and timing resolution
\item Trigger strategy on non-beam events
\end{itemize}

The DUNE physics requirements in terms of expected performance of the PDS should be provided by the DUNE Physics WG. Alternative design aspects of the proposed PDS considered as baseline for the DP FD (see CDR document arXiv:1601.02984) will be developed based on the compatibility of ProtoDUNE-DP data and MC light simulation results with the DUNE physics requirements.

%%%%%%%%%%%%%%%%%%%%%%%%%%%%%%%%%
\subsection{WBS and Responsibilities}
\label{sec:fddp-pd-12.3}

The DPPD Consortium has developed a detailed breakdown of deliverables/responsibilities included in the overall DUNE collaboration Work Breakdown Structure (WBS), DUNE-doc-5594, coordinated by the IPO. The main deliverables are based on the ProtoDUNE-DP photon detection system and are divided in seven topics: 

\textbf{WBS Element} \textit{- Institutions} \\
DP Photon Detection  System (DP-PDS)
\begin{enumerate}
\item Management DP-PDS (includes milestones \& review dates) \textit{- LAPP, CIEMAT }
\item Physics \& Simulations \textit{- Duke, LAPP, IFIC, SDSMT, CIEMAT, PUCP, UCL, Texas-Austin}
\item Design, Engineering, R\&D and validation tests \textit{- Iowa, CIEMAT, IFIC, UCL, Texas-Austin, IFAE, SDSMT}
\item Production Setup (includes tooling) \textit{- UCL}
\item Production (includes component production, assembly, testing, \& QC) \textit{- Iowa, CIEMAT, IFAE, IFIC, UCL, Texas-Austin, Duke, SDSMT, LAPP}
\item Integration (contributions to activities at global integration facility) \textit{- SDSMT}
\item Installation (contributions to activities at SURF) \textit{- CIEMAT, IFIC, SDSMT, Iowa}
\end{enumerate}

%%%%%%%%%%%%%%%%%%%%%%%%%%%%%%%%%
\subsection{High-Level Cost and Schedule}
\label{sec:fddp-pd-12.4}

The cost of the baseline DP photon detection system will be defined in a separated document.

The main activities to be developed by the DPPD Consortium during the next 16 months are focused to complete the Technical Design Report of the DP PDS. The main high-level milestones are detailed in Table \ref{tab:dppd_t_12_4}

% Anne added newstyle table along with updates to dune.cls that are required to go with it
\begin{table}[htpb] \label{tab:dppd_t_12_4}
\scriptsize
\begin{center}
\caption{DPPD schedule of activities and milestones.}
\begin{tabular}{|l|c|c|c|c|c|c|}
\hline
\rowtitlestyle &  \multicolumn{4}{c|}{2018} & \multicolumn{2}{|c|}{2019} \\ % \hline
\rowtitlestyle {\bf Simulation \& Physics} & Q1 & Q2 & Q3 & Q4 & Q1 & Q2\\
\hline

Understanding the DUNE physics requirements affecting the DPPD system & & \cellcolor{gray} & & & & \\ \hline
Finalize the implementation of DP optical simulation in LArSoft for ProtoDUNE-DP & &  \cellcolor{gray} & & & & \\ \hline
Propose a solution for a full DP Far Detector optical simulation in LArSoft & & &  \cellcolor{gray} & & & \\ \hline
Include electronics response simulation & & &  \cellcolor{gray} & & & \\ \hline
Study the physics reach with the current DPPD performance and identify possible issues & & & &  \cellcolor{gray} & & \\ \hline
Tuning light simulation using ProtoDUNE-DP light data & & & & &  \cellcolor{gray} & \\ \hline
Optimization of the DPPD performance to fulfil the physics requirements & & & & & &  \cellcolor{gray} \\ \hline
Definition of a trigger strategy & & & & & &  \cellcolor{gray} \\ \hline
\rowcolor{dunetablecolor}  \multicolumn{7}{|l|}{\bf Photosensors} \\
\hline
Review PMT specifications \& readout electronics based on ProtoDUNE-DP design & &  \cellcolor{gray} & & & & \\ \hline
Characterization \& certification plans \& test facility design & & &  \cellcolor{gray} & & & \\ \hline
Validation of PMTs \& readout electronics performance with ProtoDUNE-DP data & & & & &  \cellcolor{gray} & \\ \hline
Selection of PMT \& wavelength-shifting & & & & &  \cellcolor{gray} & \\ \hline
Final design of voltage divider and HV/signal splitters & & & & &  \cellcolor{gray} & \\ \hline
Final definition of readout electronics requirements & & & & &  \cellcolor{gray} & \\ \hline
\rowcolor{dunetablecolor} \multicolumn{7}{|l|}{\bf PMT calibration system} \\ \hline
Initial design of the system for Technical Proposal & &  \cellcolor{gray} & & & & \\ \hline
Definition of calibration requirements & & &  \cellcolor{gray} & & & \\ \hline
Review the proposed design in light of ProtoDUNE-DP calibration data & & & &  \cellcolor{gray} & & \\ \hline
Selection of components, production and testing plan & & & &  \cellcolor{gray} & & \\ \hline
\rowcolor{dunetablecolor} \multicolumn{7}{|l|}{\bf Mechanics} \\ \hline
Design of PMT mechanical support \& production plans & &  \cellcolor{gray} & & & & \\ \hline
PMT layout definition & & &  \cellcolor{gray} & & & \\ \hline
Review the proposed design in light of ProtoDUNE-DP operation & & & & &  \cellcolor{gray} & \\ \hline
Definition of the mechanical integration with cryostat & & & & & &  \cellcolor{gray} \\ \hline
\rowcolor{dunetablecolor} \multicolumn{7}{|l|}{\bf Cabling \& flanges} \\ \hline
Definition of warm and cold cables & &  \cellcolor{gray} & & & & \\ \hline
Routing plan & & & & &  \cellcolor{gray} & \\ \hline
Flanges design & & & & &  \cellcolor{gray} & \\ \hline
\rowcolor{dunetablecolor} \multicolumn{7}{|l|}{\bf Quality Control} \\ \hline
QC plan and database definition & &  \cellcolor{gray} & & & & \\ \hline
\rowcolor{dunetablecolor} \multicolumn{7}{|l|}{\bf Interfaces} \\ \hline
Identification of hardware interfaces &  \cellcolor{gray} & & & & & \\ \hline
Identification of software interfaces & &  \cellcolor{gray} & & & & \\ \hline
\rowcolor{dunetablecolor} \multicolumn{7}{|l|}{\bf Integration, installation \& commissioning} \\ \hline
Transportation plan & &  \cellcolor{gray} & & & & \\ \hline
Safety requirements & & &  \cellcolor{gray} & & & \\ \hline
Integration facility design and definition of tests & & &  \cellcolor{gray} & & & \\ \hline
Underground installation plan & & & &  \cellcolor{gray} & & \\ \hline
Detector operation definitions and commissioning plan  & & & &  \cellcolor{gray} & & \\ \hline
\rowcolor{dunetablecolor} \multicolumn{7}{|l|}{\bf Management \& Organization} \\ \hline
Definition of milestones \& activities &  \cellcolor{gray} & & & & & \\ \hline
Initial schedule \& risks evaluation & &  \cellcolor{gray} & & & & \\ \hline
DPPD Technical proposal & &  \cellcolor{gray} & & & & \\ \hline
Initial WBS \& high-level cost estimations & &  \cellcolor{gray} & & & & \\ \hline
Identification of risks & & & & &  \cellcolor{gray} & \\ \hline
DPPD Technical Design Report & & & & & &  \cellcolor{gray} \\ \hline
WBS and institutional responsibilities \& cost & & & & & &  \cellcolor{gray} \\ \hline
\end{tabular}
\label{tab:schedule}
\end{center}
\end{table}
% end Anne's addition

% \begin{table}[htpb] \label{tab:dppd_t_12_4}
% \scriptsize
% \begin{center}
% \caption{DPPD schedule of activities and milestones.}
% \begin{tabular}{|l|c|c|c|c|c|c|}
% \hline
%  &  \multicolumn{4}{c|}{2018} & \multicolumn{2}{|c|}{2019} \\ \hline
% {\bf Simulation \& Physics} & Q1 & Q2 & Q3 & Q4 & Q1 & Q2\\
% \hline
% Understanding the DUNE physics requirements affecting the DPPD system & & \cellcolor{gray} & & & & \\ \hline
% Finalize the implementation of DP optical simulation in LArSoft for ProtoDUNE-DP & &  \cellcolor{gray} & & & & \\ \hline
% Propose a solution for a full DP Far Detector optical simulation in LArSoft & & &  \cellcolor{gray} & & & \\ \hline
% Include electronics response simulation & & &  \cellcolor{gray} & & & \\ \hline
% Study the physics reach with the current DPPD performance and identify possible issues & & & &  \cellcolor{gray} & & \\ \hline
% Tuning light simulation using ProtoDUNE-DP light data & & & & &  \cellcolor{gray} & \\ \hline
% Optimization of the DPPD performance to fulfil the physics requirements & & & & & &  \cellcolor{gray} \\ \hline
% Definition of a trigger strategy & & & & & &  \cellcolor{gray} \\ \hline
%  \multicolumn{7}{|l|}{\bf Photosensors} \\
% \hline
% Review PMT specifications \& readout electronics based on ProtoDUNE-DP design & &  \cellcolor{gray} & & & & \\ \hline
% Characterization \& certification plans \& test facility design & & &  \cellcolor{gray} & & & \\ \hline
% Validation of PMTs \& readout electronics performance with ProtoDUNE-DP data & & & & &  \cellcolor{gray} & \\ \hline
% Selection of PMT \& wavelength-shifting & & & & &  \cellcolor{gray} & \\ \hline
% Final design of voltage divider and HV/signal splitters & & & & &  \cellcolor{gray} & \\ \hline
% Final definition of readout electronics requirements & & & & &  \cellcolor{gray} & \\ \hline
% \multicolumn{7}{|l|}{\bf PMT calibration system} \\ \hline
% Initial design of the system for Technical Proposal & &  \cellcolor{gray} & & & & \\ \hline
% Definition of calibration requirements & & &  \cellcolor{gray} & & & \\ \hline
% Review the proposed design in light of ProtoDUNE-DP calibration data & & & &  \cellcolor{gray} & & \\ \hline
% Selection of components, production and testing plan & & & &  \cellcolor{gray} & & \\ \hline
% \multicolumn{7}{|l|}{\bf Mechanics} \\ \hline
% Design of PMT mechanical support \& production plans & &  \cellcolor{gray} & & & & \\ \hline
% PMT layout definition & & &  \cellcolor{gray} & & & \\ \hline
% Review the proposed design in light of ProtoDUNE-DP operation & & & & &  \cellcolor{gray} & \\ \hline
% Definition of the mechanical integration with cryostat & & & & & &  \cellcolor{gray} \\ \hline
% \multicolumn{7}{|l|}{\bf Cabling \& flanges} \\ \hline
% Definition of warm and cold cables & &  \cellcolor{gray} & & & & \\ \hline
% Routing plan & & & & &  \cellcolor{gray} & \\ \hline
% Flanges design & & & & &  \cellcolor{gray} & \\ \hline
% \multicolumn{7}{|l|}{\bf Quality Control} \\ \hline
% QC plan and database definition & &  \cellcolor{gray} & & & & \\ \hline
% \multicolumn{7}{|l|}{\bf Interfaces} \\ \hline
% Identification of hardware interfaces &  \cellcolor{gray} & & & & & \\ \hline
% Identification of software interfaces & &  \cellcolor{gray} & & & & \\ \hline
% \multicolumn{7}{|l|}{\bf Integration, installation \& commissioning} \\ \hline
% Transportation plan & &  \cellcolor{gray} & & & & \\ \hline
% Safety requirements & & &  \cellcolor{gray} & & & \\ \hline
% Integration facility design and definition of tests & & &  \cellcolor{gray} & & & \\ \hline
% Underground installation plan & & & &  \cellcolor{gray} & & \\ \hline
% Detector operation definitions and commissioning plan  & & & &  \cellcolor{gray} & & \\ \hline
% \multicolumn{7}{|l|}{\bf Management \& Organization} \\ \hline
% Definition of milestones \& activities &  \cellcolor{gray} & & & & & \\ \hline
% Initial schedule \& risks evaluation & &  \cellcolor{gray} & & & & \\ \hline
% DPPD Technical proposal & &  \cellcolor{gray} & & & & \\ \hline
% Initial WBS \& high-level cost estimations & &  \cellcolor{gray} & & & & \\ \hline
% Identification of risks & & & & &  \cellcolor{gray} & \\ \hline
% DPPD Technical Design Report & & & & & &  \cellcolor{gray} \\ \hline
% WBS and institutional responsibilities \& cost & & & & & &  \cellcolor{gray} \\ \hline
% \end{tabular}
% \label{tab:schedule}
% \end{center}
% \end{table}
