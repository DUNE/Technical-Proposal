%%%%%%%%%%%%%%%%%%%%%%%%%%%%%%%%%%%
\subsection{Liquid Level Monitoring}
\label{sec:fdgen-slow-cryo-liq-lev}
% john L, anselmo
% DP

The goals for the level monitoring system are basic level sensing when filling, and precise level sensing during static operations. 

For filling the \dword{detmodule} the differential pressure between the top of
the detector and known points below it can be converted to depth using
the known density of \lar.  The temperatures of \dwords{rtd} at known
heights may also be used to determine when the cold liquid reaches 
each \dword{rtd}.

During operation, the purpose of liquid level monitoring is twofold:
the cryogenics system uses it to tune the \lar flow, and 
the \dword{spmod} uses it to guarantee that the top \dwords{gp} are always
submerged (otherwise the risk of dielectric breakdown is high).
Two differential pressure level meters are installed as part of
the cryogenics system, one on each side of the \dword{detmodule}.  They 
have a precision of \SI{0.1}{\%}, which corresponds to \SI{14}{mm} at the
nominal \lar surface.  This precision is sufficient for the \dword{spmod}, since the plan is to keep the \lar surface at least \SI{20}{cm} above the \dwords{gp} (this is the value used for the \dword{hv}
interlock in \dword{pdsp}); thus, no additional level meters are
required for the \single. 
However, in the \dual \lar system the surface level should be controlled at the millimeter level, which can be accomplished with capacitive monitors. Using the same capacitive monitor system in each \dword{detmodule} reduces design differences and provides a redundant system for the \single.  Either system could be used for the \dword{hv} interlock.

Table \ref{tab:fdgen-liq-lev-req} summarizes the
requirements for the liquid level monitor system.

\begin{dunetable}
[Liquid level monitor requirements]
{p{0.45\linewidth}p{0.50\linewidth}}
{tab:fdgen-liq-lev-req}
{Liquid level monitor requirements}   
Requirement & Physics Requirement Driver \\ \toprowrule
 Measurement accuracy (filling) \(\sim \SI{14}{mm}\) & Understand status of detector during filling \\ \colhline
 Measurement accuracy (operation, \dual) \(\sim \SI{1}{mm}\) & Maintain correct depth of gas phase. (Exceeds \single requirements) \\ \colhline
 Provide interlock with \dword{hv} & Prevent damage to \dword{detmodule} from \dword{hv} discharge in gas \\
\end{dunetable}


%\subsubsection{Production and Assembly}
Cryogenic pressure sensors will be purchased from commercial sources.
Installation methods and positions will be determined as part of the
cryogenics internal piping plan.  Sufficient redundancy will be designed in
to ensure that no single point of failure compromises the level measurement.

Multiple capacitive level sensors are deployed along the top of
the fluid to be used during stable operation and checked against each
other.

%"Challenges associated with the DP liquid level in the 3X1X1 and any lessons learned for DUNE."
During operations of the \dword{wa105}, the cryogenic programmable logic controller (PLC) continuously checked the measurements from one level meter on the charge readout plane (\dword{crp}) in order to regulate the flow from the liquid recirculation to maintain a constant liquid level inside the cryostat. Continuous measurements from the level meters around the drift cage and the \dword{crp} demonstrated the stability of the liquid level within the \SI{100}{\micro\meter} intrinsic precision of the instruments. The observation of the level was complemented by live feeds from the custom built cryogenic cameras, thereby providing qualitative feedback on the position and flatness of the surface.

 In the \dword{dpmod} each \dword{crp} is suspended with three ropes actuated by step motors and can be individually adjusted with respect to the liquid level in terms of parallelism and distance. The \dword{crp} has a gap of 1 cm in between the bottom of the surface of the \dwords{lem} and the extraction grid. The \dword{crp} positioning requirements include its parallelism to the liquid surface and the liquid level situated across this gap, namely the extraction grid is required to be immersed and the \dwords{lem} are required to not be wet.  The distance of the \dword{crp} with respect to the liquid level is required to be measured with the accuracy of 1 mm. The slow control system takes care of the control of the step motors, which control the \dword{crp} suspensions. The positions of the motors (and of the corresponding suspension points) can be surveyed from the cryostat roof with a typical 0.2 mm accuracy. By reading the controls of the motors and by knowing the results of this survey, the position of the \dwords{crp} with respect to the liquid can be predicted in absolute terms, provided that the liquid level is measured also with respect to the same reference system by the cryostat level meters. 
 
 The horizontal alignment of the \dwords{crp} can be measured as well thanks to the survey of the reference points on the suspension feedthroughs. The relative position of a \dword{crp} with respect to the liquid level can be measured with high-accuracy level meters mounted on the \dword{crp} itself. These level meters are multi-parallel plate capacitors with the capacitance changing as a function of the liquid level between the plates. These level meters can be attached to the \dword{crp} borders and they guarantee a measurement accuracy at the level of 0.1 mm. It is possible to install these level meters only on the borders of the \dwords{crp}, which are at the periphery of the detection surface and not on the borders where the \dwords{crp} are side by side. In addition to the installed level meters, the liquid height in the extraction region of all \dwords{crp} can be inferred by measuring the capacitance between the grid and the bottom electrode of each \dword{lem}. Averaging over all \num{12} \dwords{lem}, the measured values of this capacitance typically ranged  from 150 pF with the liquid below the grid to around 350 pF when the \dwords{lem} are submerged. This  method offers the potential advantage of monitoring the liquid level in the \dword{crp} extraction region with a 50$\times$50 \si{cm$^2$}  granularity and can be used for the \dword{crp} level adjustment in the future large-scale detectors where, due to the space constraints, placement of the level meters along the \dword{crp} perimeter is not possible.