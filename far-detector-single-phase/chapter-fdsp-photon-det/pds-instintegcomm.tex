%%%%%%%%%%%%%%%%%%%%%%%%%%%%%%%%%%%%%%%%%%%%%%%%%%%%%%%%%%%%%%%
\section{Installation, Integration and Commissioning}
\label{sec:fdsp-pd-install}
%\todo{\color{blue} Content: Kemp}
%\metainfo{(Length: TDR=30 pages, TP=6 pages)}


%>> Revision: Ernesto Kemp & Norm Buchanan Mar/15/2018 >>>>>>>>>>>>>
%>> Start: Ernesto Kemp Feb/10/2018 >>>>>>>>>>>>>  

%=================================================
\subsection{Transport and Handling}
\label{sec:fdsp-pd-install-transport}

%Following assembly and testing of the PD modules they need to be carefully packaged and shipped to the far detector site for checkout and any final testing prior to installation into the cryostat. Handling and shipping procedures will depend on the environmental requirements determined for the photon detectors.

%\begin{itemize}

%\item Development of a testing plan to determine environmental requirements for photon detector handling and shipping. The environmental conditions apply for both surface and underground transport, storage and handling. Requirements for light (UV filtered areas), temperature, and humidity exposure should be developed.

%\item Handling procedures need to be developed to ensure environmental requirements are met. This should include handling at all stages of component and system production and assembly, testing, shipping, and storage. It is likely that PD modules and components will be stored for periods of time during production and prior to installation into the FD cryostats. Appropriate storage facilities need to be constructed at locations where storage will take place.

%\item Shipping and storage containers need to be designed and produced. Given the large number of photon detector modules to be installed in the FD, it will be advantageous to develop shipping containers that can be reused.

%\item Documentation and tracking of all components and PD modules will be required during the full production and installation schedule. Well-defined procedures need to be in place to ensure that all components/modules are tested and examined prior to, and after, shipping. Information coming from such testing and examinations will be stored in a hardware database.

%\item It is expected that photon detector will be assembled at more than one location and that components will come from a number of other locations. Oversight of the shipping and handling procedures will be critical. This responsibility should fall to a single institution and manager.

%\item Integration and Test Facility (ITF) Operations:

%	\begin{itemize}
%	\item Transportation to and from ITF should be planned. The PDS units will be shipped from the production area in quantities compatible with the APA transport rates.   
%	\item Operations: The PDS deliveries will be stored in temperature and humidity controlled storage area. Their mechanical status will be inspected.
%	\end{itemize}

%\item Transportation to SURF: The delivery to SURF will be such that the storage time before integration will be at most two weeks.

%\end{itemize}


Following assembly and testing of the PD modules they will be carefully packaged and shipped to the far detector site for checkout and any final testing prior to installation into the cryostat. Handling and shipping procedures will depend on the environmental requirements determined for the photon detectors. 

A testing plan will be developed to determine environmental requirements for photon detector handling and shipping. The environmental conditions apply for both surface and underground transport, storage and handling. Requirements for light (UV filtered areas), temperature, and humidity exposure will also be developed.

Handling procedures that ensure environmental requirements are met will be developed. This will include handling at all stages of component and system production and assembly, testing, shipping, and storage. It is likely that PD modules and components will be stored for periods of time during production and prior to installation into the FD cryostats. Appropriate storage facilities need to be constructed at locations where storage will take place. Shipping and storage containers need to be designed and produced. Given the large number of photon detector modules to be installed in the FD, it will be cost effective to take advantage of reusable shipping containers.

Documentation and tracking of all components and PD modules will be required during the full production and installation schedule. Well defined procedures need to be in place to ensure that all components/modules are tested and examined prior to, and after, shipping. Information coming from such testing and examinations will be stored in a hardware database.

An Integration and Test Facility (ITF) will be constructed at a location to be decided by the collaboration/project for the integration of the PDs into APAs. Transportation to and from ITF should be carefully planned. The PDS units will be shipped from the production area in quantities compatible with the APA transport rates.
    
Operations: The PDS deliveries will be stored in temperature and humidity controlled storage area. Their mechanical status will be inspected.

Transportation to SURF: The delivery to SURF will be such that the storage time before integration will be at most two weeks.


%%%%%%%%%%%%%%%%%%%%%%%%%%%%%%%%%%%
\subsection{Integration with APA and Installation}
\label{sec:fdsp-pd-install-pd-apa}

%\begin{itemize}

%\subsubsection{Integration}
%If the PDS detectors can be installed after the wire winding is completed, as in the present baseline design, their integration on the APA frame will happen at the ITF. Experts from both groups will work with the installation team. An electrical test with APA/PDS/CE will be performed at the ITF in a cold box, after the integration of PDS and CE on the APA frame has been completed.

%\subsubsection{Installation}
%The APA consortium will be responsible for the transportation of the integrated APA frames from the integration facility to the LBNF/SURF facility. The UIT team, under supervision of the APA group, will be responsible to move the equipment into the clean room. Work on the 2-APA connection and inspection in the toaster is performed by the APA group. Work on cabling in the toaster is performed by PDS and CE groups under supervision of the APA group. Once the APAs will be moved inside the cryostat, the PDS and CE consortia will be responsible for the routing of the cables in the trays hanging from the top of the cryostat. 

%\end{itemize}

PD modules integration into the APA frame will happen at the Integration facility. Experts from both groups will work with the installation team. 
An electrical test with APA/PDS/CE will be performed at the integration facility in a cold box, after the integration of PDS and CE on the APA frame has been completed.

The APA consortium will be responsible for the transportation of the integrated APA frames from the integration facility to the LBNF/SURF facility. 
The UIT team, under supervision of the APA group, will be responsible to move the equipment into the clean room. 
Work on the 2-APA connection and inspection in the toaster is performed by the APA group.
Work on cabling in the toaster is performed by PDS and CE groups under supervision of the APA group.
Once the APAs will be moved inside the cryostat, the PDS and CE consortia will be responsible for the routing of the cables in the trays hanging from the top of the cryostat. 

\fixme{Need to replace the term "toaster" -- suggestions please}


%%%%%%%%%%%%%%%%%%%%%%%%%%%%%%%%%%%
%\subsection{Installation into Cryostat and Cabling - Kemp}
\subsection{Installation into the Cryostat and Cabling}
\label{sec:fdsp-pd-install-pd-cryo}

%The APA will be integrated with the PDS. There are several PD modules per APA, inserted into alternating sides of the APA frame, several from each direction. Once a PD module is inserted, it is attached mechanically to the APA frame with fasteners, a single electronics cable is attached. Once the PD module installation is complete several cold electronics (CE) units are installed at the top of the APA frame.

%Each CE unit consists of electronics enclosure that contains the TPC readout. Each unit also includes a bundle of cables that connect the electronics to the outside of the cryostat via the flange on the feed through port. 

%After the APA has been integrated with the PDS and CE, it will be moved via the rails in the clean room to the integrated cold test stand. After the tests APA will be moved into the cryostat. The two anode planes of the TPC will be assembled inside the cryostat, each of the fully tested APAs mechanically linked together. Signal cables from the TPX readout and the PD modules are routed up to the feedthrough flanges on the cryostat top side. The cables from each of the CE and PD modules on the APA are then routed and connected to the final flanges on the cryostat.

The PD modules are installed into the APAs. There are ten PD's per APA, inserted into alternating sides of the APA frame, five from each direction. Once a PD is inserted, it is attached mechanically to the APA frame  and cabled up with a single power/readout cable. Following PD installation cold electronics (CE) units are installed at the top of the APA frame.

After the APA has been integrated with the PDS and CE, it will be moved via the rails in the clean room to the integrated cold test stand for testing and be moved into the cryostat. The two anode planes of the TPC will be assembled inside the cryostat, each of the fully tested APAs mechanically linked together. Signal cables from the TPC readout and the PD modules are routed up to the feedthrough flanges on the cryostat top side. The cables from each of the CE and PD's on the APA are then routed and connected to the final flanges on the cryostat.

%%%%%%%%%%%%%%%%%%%%%%%%%%%%%%%%%%%

\subsection{Commissioning Calibration and Monitoring}
\label{sec:fdsp-pd-install-calib}

%The calibration and monitoring systems for the photon detector system will interface with several groups within the DUNE FD project, including the calibration group, slow controls group, and data acquisition group.

%\begin{itemize}
%\item The number of penetrations required for the calibration and monitoring systems need to be determined so that the cryostat design, including feedthroughs can be finalized.
%\item Mounting requirements for the calibration system inside the cryostat need to be determined. Any light delivery hardware to be mounted on the cathode plane assemblies will need to be developed in coordination with the CPA group. A plan for fiber routing will need to be made. Cable routing and power distribution plans for both the calibration and monitoring
%\item Readout of both the calibration and monitoring systems will need to be developed with the DAQ and slow controls groups. Mappings of the systems will need to be reflected in the PD hardware database.
%\item Any calibration systems utilizing radioactive sources will need to be developed in coordination with the radiopurity and physics groups. It is important to ensure in addition, that any components of such a system do not contaminate the LAr or compromise the electron lifetime.
%\end{itemize}

\fixme{I am not sure of the purpose of section 6.6.4 Calibration and Monitoring ? it is intended to be the Commissioning part of this section? It is similar to the interface section, it describes things that need to be in place, but not how they will be used for commissioning.  Perhaps some introductory text explaining that Calibration and Monitoring will be the initial primary commissioning tools? Follow the current text by a subsection on how it will be used? E.g. use of cosmics as a tool?}


The calibration and monitoring systems for the photon detector system will interface with several groups within the DUNE FD project, including the calibration group, slow controls group, and data acquisition group.

The number of penetrations required for the calibration and monitoring systems need to be determined so that the cryostat design, including feedthroughs can be finalized. Mounting requirements for the calibration system inside the cryostat also needs to be determined. Any light delivery hardware to be mounted on the cathode plane assemblies will need to be designed and fabricated in coordination with the CPA group. A plan for fiber routing will need to be made. 

Readout of both the calibration and monitoring systems will need to be developed with the DAQ and slow controls groups. Mappings of the systems will need to be reflected in the PD hardware database.

Any calibration systems utilizing radioactive sources will need to be developed in coordination with the radiopurity and physics groups. It is important to ensure in addition, that any components of such a system do not contaminate the LAr or compromise the electron lifetime.

%>> Start: Ernesto Kemp Feb/10/2018 <<<<<<<<<<<<<
%>> Revision: Ernesto Kemp & Norm Buchanan Mar/15/2018 <<<<<<<<<<<<<
 

