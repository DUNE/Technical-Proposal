%%%%%%%%%%%%%%%%%%%%%%%%%%%%%%%%%%%%%%%%%%%%%%%%%%%%%%%%%%%%%%%%
\section{Quality Assurance and Quality Control}
\label{sec:fdsp-pd-qaqc}
%\todo{\color{blue} Content: Warner}
%\metainfo{(Length: TDR=10 pages, TP=2 pages)}


%%%%%%%%%%%%%%%%%%%%%%%%%%%%%%%%%%%%
%\subsection{Production and Assembly (Local)}
%\label{sec:fdsp-pd-qc-local}

%%%%%%%%%%%%%%%%%%%%%%%%%%%%%%%%%%%
%\subsection{Post-factory Installation (Remote)}
%\label{sec:fdsp-pd-qc-remote}

\subsection{Design Quality Assurance}
\label{sec:fdsp-pd-designqa}

Photon Detector design quality assurance focuses on ensuring that the detector modules meet the following goals:
\begin{itemize}
\item Physics goals as specified in the Photon Detector requirements document
\item Interfaces with other detector sub-systems as specified by the sub-system interface documents
\item Materials selection and testing to ensure non-contamination of the LAr volume
\end{itemize}

The PDS consortium will perform the design and fabrication of the components in accordance with the applicable requirements of the LBNF/DUNE Quality Assurance Plan. If the institute (working under the supervision of the consortium) performing the work has a documented Quality Assurance Program the work may be performed in accordance with their own program.

Upon completion of the PDS design and QA/QC plan there will be a preliminary design review process, with the reviewers charged to ensure that the design demonstrates compliance with the goals above.

\subsection{Production and Assembly Quality Assurance}
\label{sec:fdsp-pd-prodqa}

The photon detector system will undergo a careful QA review for all components prior to completion of the design and development phase of the project.  The ProtoDUNE-SP test will represent the most significant test of near-final PD components in a near-DUNE configuration, but additional tests will also be performed.  The Quality assurance plan will include, but not be limited to, the following areas:

\begin{itemize}
\item Materials certification (in the FNAL materials test stand and other facilities) to ensure materials compliance with cleanliness requirements
\item Cryogenic testing of all materials to be immersed in LAr, to ensure satisfactory performance through repeated and long-term exposure to LAr.  Special attention will be paid to cryogenic behavior of plastic materials (such as radiators and light guides), SiPMs, cables and connectors.  Testing will be conducted both on small-scale test assemblies (such as the small test cryostat at CSU) and full-scale prototypes (such as the full-scale CDDF cryostat at CSU) 
%(Add pictures?)
\item Mechanical interface testing, beginning with simple mechanical go-nogo gauge tests, followed by installation into the ProtoDUNE-SP system, and finally full-scale interface testing of the PD system into the final pre-production TPC system models
\item Full-system readout tests of the PD readout electronics, including trigger generation and timing, including tests for electrical interference between the TPC and PD signals.
\end{itemize}

Prior to the release of the TDR the PDS will undergo a final design review, where these and other QA tests will be reviewed and the system declared ready to move to the pre-production phase.


\subsection{Production and Assembly Quality Control}
\label{sec:fdsp-pd-prodqc}

Prior to the start of fabrication, a Manufacturing/QC Plan will be developed detailing the key manufacturing, inspection and test steps.  The fabrication, inspection and testing of the components will be performed in accordance with documented procedures. The work will be documented on travelers and applicable test or inspection reports. Records of the fabrication, inspection and testing will be maintained. When a component has been identified as being in noncompliance to the design, the nonconforming condition shall be documented, evaluated and dispositioned as use-as-is (does not meet design but can meet functionality as is), rework (bring into compliance with design), repair (will be brought into meeting functionality but will not meet design) and scrap. For products with a disposition of accept as is or repair, the nonconformance documentation shall be submitted to the design authority for approval.   

All QC data  (from assembly and pre and post installation into the APA) will be directly stored to the DUNE database for ready access of all QC data.  Monthly summaries of key performance metrics (TBD) will be generated and inspected to check for quality trends.

Based on the ProtoDUNE-SP model, we expect to conduct the following production testing:
\begin{itemize}
\item Dimensional checks of critical components and completed assemblies to insure satisfactory system interfaces
\item Post-assembly cryogenic checkouts of SiPM mounting PCBs (prior to assembly into PD modules)
\item Cryogenic testing of completed modules (in CSU CDDF or similar facility) to provide a final pre-shipping module test
\item Warm scan of complete module using motor-driven LED scanner (Or UV LED array)
\item Complete visual inspection of module against a standard set of inspection points, with photographic records kept for each module.
\item End-to-end cable continuity and short circuit tests of assembled cables
\item Front-end electronics functionality check
\end{itemize}

\subsection{Installation Quality Control}
\label{sec:fdsp-pd-prodqc}

PDS pre-installation testing will follow the model established for ProtoDUNE-SP.  Prior to installation in the APA, the PD modules will undergo a warm scan in a scanner identical to the one at the PD module assembly facility and the results compared.  In addition, the module will undergo a complete visual inspection for defects and a set of photographs of all optical surfaces taken and entered into the QC record database.  Following installation into the APA and cabling an immediate check for electrical continuity to the SiPMs will be conducted.

It is expected that following the mounting of the TPC cold electronics and the photon detectors the entire APA will undergo a cold system test in a GAr cold box, similar to that performed during ProtoDUNE-SP.  During this test, the PDS system will undergo a final integrated system check prior to installation, checking dark and LED-stimulated SiPM performance for all channels, checking for electrical interference with the cold electronics, and confirming compliance with the detector grounding scheme.
