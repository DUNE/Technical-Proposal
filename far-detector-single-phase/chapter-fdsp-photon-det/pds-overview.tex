\section{Photon Detection System (PD) Overview}
\label{sec:fdsp-pd-ov}
%\fixme{(Length: TDR=10 pages, TP=3 pages)}
\metainfo{(Length: TDR=10 pages, TP=3 pages)}

%%%%%%%%%%%%%%%%%%%%%%%%%%%%%%%%%
\subsection{Introduction}
\label{sec:fdsp-pd-intro}
\metainfo{\color{blue} Content: Segreto}
Liquid Argon is known to be an abundant scintillator and emits about 40 
photons/keV when excited  by minimum ionizing particles, in absence of 
electric fields. The passage of ionizing radiation in LAr produces excitations 
and ionizations of the argon atoms which ends up with the formation of the 
excited dimer Ar$^2_*$. Photons' emission proceeds through the de-excitation 
of the lowest lying singlet and triplet excited states, $^{1}\Sigma$ and 
$^{3}\Sigma$ to the dissociative ground state. The de-excitation from the 
$^{1}\Sigma$ state is very fast and has a characteristic time of the order of 
$\tau_{fast}$ $\simeq$ 6 ns. The de-excitation from the $^{3}\Sigma$ state is 
much slower with a characteristic time of $\tau_{slow}$ $\simeq$ 1.3 $\mu$sec, 
since it is forbidden by the selection rules. The relative abundance of the 
fast and slow components is related to the ionization density of LAr and 
depends on the ionizing particle, being 0.3 for electron, 1.3 for alpha 
particles and 3 for neutrons. This circumstance is at the base of the  
particle discrimination capabilities of LAr, which is exploited by many 
experiments.\\ 

In both decays, photons are emitted in a 10 nm band centered around 127 nm, in 
the Vacuum Ultra Violet (VUV) region of the electromagnetic spectrum.\\ 

The paradigm for the detection of LAr scintillation light foresees the use of 
wavelength shifters since the common (cryogenic) photosensors are not 
sensitive to VUV radiation due to the lack of transparency of fused silica and 
glass optical windows. The most widely used wavelength shifter used in 
combination with LAr is Tetra-PhenylButadiene (TPB), which absorbs VUV photons 
and re-emits them with a spectrum centered around 430 nm, where most of the 
photosensors have their maximum Quantum Efficiency. TPB conversion efficincy 
is known to be high, when compared to other wavelength shifters (....), and 
it is often assumed to be 100\%, even if a reliable direct measurement of this 
relevant quantity is not available in the literature. 
In large LAr TPC, it is common to use photon collector systems which allow to 
collect light from large areas and drive it in an efficient way towards the 
active sensors.\\

Scintillation light detection is important in a LAr TPC: it is often 
used for triggering purposes and $T_0$ time measurements, for the 
determination of the absolute position of the event inside the active volume, 
but can also give accurate calorimetric measurements of the deposited energy 
and is a powerful instrument for particle identification.\\

At the moment, the photon detection system of  the DUNE Single Phase far 
detector is required to measure the $T_0$ of non-beam events with deposited 
energy above 200 MeV (proton decay and atmospheric neutrinos) with high 
efficiency to enable 3D spatial localization of candidate events. The photon
system will provide the $T_0$ timing of events relative to TPC timing with a 
resolution better than 1 $\mu$sec. The minimum light yield requirement is of 
0.1 phel/MeV for events near the cathode, which is the farthest region from 
the Photon Detection System, installed behind the anodic plane. A revision 
of these requirements is being considered in order to better exploit the 
characteristics of LAr scintillation light and to improve the performances 
of the detector for low energy (Supernova) events.

\fixme{Include an image of the overall system, indicating its parts. Show how the system fits into the overall detector.}

The operating principle is illustrated in Figure~\ref{figure-label}... (add figure)

%\begin{dunefigure}[optional caption for LoF]{fig:figure-label}
%{required full caption (Credit: xyz)}
%\includegraphics[width=0.8\textwidth]{image-filename}
%\end{dunefigure}

\fixme{Include summary of simulation status? {\color{red}  Szelc/Himmel}}

%%%%%%%%%%%%%%%%%%%%%%%%%%%%%%%%%%%%%
\subsection{Design Considerations}
\label{sec:fdsp-pd-des-consid}
\metainfo{\color{blue} Content: Segreto/Warner/Mualem}
The photon detection system of the Single Phase DUNE far detector is 
constrained to be structured into modules shaped in form of thin bars (less 
then 1 cm thick) with approximate dimensions of 8 cm $\times$ 200 cm by the need of 
being installed on the mechanical frame of the APA and slided between the wire
planes on its two sides.

Three different designs of PD modules have been developed and are being 
considered by the SP PD COnsortium. Two of them are based on the concept of 
light guides coupled to solid state silicon photomultipliers (SiPM) while the 
third one, namely the ARAPUCA, is a light trap, which captures the photons inside
boxes with 
higlhly reflective internal surfaces where they are eventually detected by 
SiPM.

The dip-coated light guides are pre-treated commercially-cut acrilic slabs dip 
coated with a solution of TPB, acrylic and toluene. When toluene evaporates it 
leaves a thin film of TPB embedded in  acrylic matrix which acts as wavelenght 
shifter. A fraction of the light is captured inside the acrylic bar by total 
internal reflection and is detected at one end of the bar by an array of SiPM.

The double-shift light guides are slightly different, since the conversion and 
guiding processes of the photons are decoupled. They are constituted by a 
radiator plate coated by pure TPB (through a spraying process) which is 
optically ocoupled to a commercial shifting/guiding bar which absorbs the TPB blue
light 
and re-emits it in the green. A fraction of the double-shifted light is 
captured 
inside the bar by total internal reflections and again it is detected by an 
array of SiPM installed at one of its ends.

The ARAPUCA module is composed of an array of smaller boxes (of the order of 
20) each one acting as a smaller detector. Each box has an optical window made 
by a dichroic filter deposited with two different wavelength shifters, one on 
the external side and another on the internal one, which allows the light to get
inside the box, but not to exit. The internal surface of each box is lined 
with a highly reflective material so that the trapped photon can bunch several 
times. An array of SiPM is installed inside the box and will eventually detect 
the trapped light.

The two guiding bars designs have been developed in the course of the last few 
years and have reached a reasonable level of maturity and reliability. Further 
improvements are possible and could come from a double ended read-out, that is 
installing SiPM at both ends of the bars and coating the smaller sides of the 
bars with reflective foils. Given the maturity of the technology, the effects 
of these improvements can be easily predicted through Monte Carlo simulations, 
eventually combined with the outcomes of the protoDUNE experiment where a 
large number of bars of both flavors are installed (29 of each type).

Both designs have been demonstrated to have attenuation lengths for the 
trapped light comparable to the length of the bars themselves, which ensures a 
reasonable uniformity along the beam direction. Their absolute efficiency has 
been measured in a quite reliable way only recently and ranges between 0.1\% and
0.2\%.\\
 
The ARAPUCA concept, on the other hand, is quite new since it was proposed for 
the first time in 2015 and was accepted for the installation in protoDUNE in 
mid 2016. A series of LAr tests have been performed with different 
realizations of ARAPUCA devices, which resulted in efficiencies ranging from 
0.4\% to 1.8\%. Monte Carlo simulations show that efficiencies at the level of 
few per cent could be reasonably reached with few modifications to the design. 
While the results of the experimental tests are encouraging, they also show 
that a deeper understanding of the optical phenomena involved is needed.\\ 


Having the photon detection modules installed only on the anodic plane does 
not allow to have an uniform light collection over the entire active volume. A
possible solution 
for making the light collection more uniform is to install a 
reflective foil coated with wavelength shifter on the cathode.. It would be very
beneficial to remove the $^{39}Ar$ 
background which otherwise could cause a huge counting rate for events near 
the anodic plane. It would also increase the light yield of the detector and 
could allow to perform calorimetric measurements based on light.  This 
possibility is under study through Monte Carlo simulations and its mechanical
feasibility is being discussed with the HV Consortium.\\

The actual minimal requirements for the light yield of the PD system is of 0.1 
phel/keV near the cathode, which would allow to efficiently  detect ($>$ 90\%) 
proton decay events (visible energy $>$ 200 MeV). All the three designs 
satisfies the minimum requirements according to preliminary Monte Carlo 
studies, but there is a wide consensus inside the Collaboration that an higher 
light yield would be very beneficial for the detection of low energy Supernova 
neutrinos and a critical revision of the requirements is taking place.

These considerations are driving the actual strategy for the design of the SP 
FD DUNE far detector design and in particular for the R\&D program which will 
be carried on before the Technical Design Report (mid 2019). A strong effort 
should be put in exploring the possibilty of increasing the light yield of the 
detector with the ARAPUCA concept by a factor ten with respect to bars' 
read-out, compatibily with the resources available inside the Consortium, in 
therms of manpower and fundings. This R\&D program will lead the Consortium to 
define a baseline design for the photon collector and one alternate design for risk
mitigation.  

Cocerning the active photosensors few options are under evaluation.
SensL SiPM, which have been heavily used in protoDUNE did not demonstrated to 
be reliable enugh for cryogenic operations after a modification to their 
packaging and new vendors are being considered. 
The plan is to use only devices which are certified for cryogenic operations 
by the vendor. There are actually ongoing tests of cryogenic SiPM produced by
 Hamamatsu and by FBK (Fondazione Bruno Kessler, Italy).

The electronic design will be strongly influenced by the outcomes of Monte 
Carlo simulations, whcich will tell about the relevance of having pulse 
shape capabilities and how they improve Supernova neutrino detection. These
consideration will have a role in defining the read-out scheme and the digitization
frequency of the signals.  

\fixme{Anne suggests: Within this section add ref to requirements document  when it's ready, and maybe list the most important half dozen in a table here). E.g.,}  


\begin{dunetable}
[Important requirements on the PD system design]
{p{0.8\textwidth}}
{pdphysicsparams}
{Important requirements on the PD system design}   
Requirement  \\ \toprowrule
  \\ \colhline
   \\ \colhline
 ...\\ 
\end{dunetable}

\fixme{By the end of the volume, for every requirement listed in this section, there should exist an explanation of how it will be satisfied.}

\subsection{Criteria for photon collection options performance evaluation}?\fixme{\color{blue} Content: Segreto}

The performances of the different photon collection options will be 
evaluated in 
the facilities available to the Consortium which will allow to perform 
relative and absolute measurements at room and cryogenic temperature.

An important piece of information will come from the protoDUNE experiment, 
which is being constructed at CERN and will be operated starting from the 
second half of 2018.\\
All the three different designs are present in protoDUNE, in particular 29 
double-shift guides, 29 dip-coated guides and 2 ARAPUCA arrays. The 
presence of the Time Projection Chamber will allow to precisely reconstruct 
in 3D the track of any ionizing event inside the active volume. The 
matching of the track with the associated light signal will enable to 
accurately estimate the relative detection efficiencies of the installed PD 
modules. 
Absolute measurements will be possible depending on the accuracy of the
Monte Carlo simulations of the optical properties of the detector. 
Unfortunately the fact that  some of the optical parameters which 
regulate VUV light propagation in LAr are poorly known, such as Ryleigh 
scattering length, will influence the precision of the absolute 
measurements.\\
The protoDUNE test will also give the opportunity of performing long 
term test of the PD modules for the first time. It will be possible to 
verify if there is a decay in their performances and eventually to 
quantify it. This is a relevant information, which needs to be taken into 
account also in the final design of the modules to ensure that the 
minimal requirements are satisfied along the whole life of the DUNE 
experiment.\\

The R\&D program will go on in parallel with the protoDUNE operation and 
more comparative measurements will be needed before the Technical Design 
Report, when a baseline design will be singled out.
The TallBo facility at Fermilab will be extremely important to this scope. 
It is constituted by a 450 liters cryostat with 56 cm inner diameter and up 
to a 183 cm liquid depth and allows to host up to three different PD 
modules with dimensions close to the real ones.\\
Other facilities are accessible to the Consortium which will allow to test 
smaller scale prototypes of the modules (or a piece of them), like the 
SCENE set-up at Fermilab, the cryogenic facilities at Colorado State 
UNiversity and UNICAMP (Brazil) and the optical facilities at Fermilab, 
Indiana University and UNICAMP. These facilities will be useful for the 
single modules optimization process.
         


\section{Scope}
The scope of the photon detector (PD) system for the DUNE far detector 
reference design includes design, procurement, fabrication, testing,
 delivery and installation of the following components:
\begin{itemize}
        \item light collection system;
        \item silicon photo-multipliers (SiPMs);
        \item readout electronics;
        \item calibration system (???)
        \item realted infrastructures
\end{itemize}

%%%%%%%%%%%%%%%%%%%%%%%%%%%%%%%%
\subsection{Scope}
\label{sec:fdsp-pd-scope}
\fixme{\color{blue} Content: Segreto/Warner/Mualem}

The scope of the Photon Detection system includes the continued procurement of materials for, and the fabrication, testing, delivery and installation of the following systems: 

\fixme{Whatever the items are...}

\begin{itemize}
\item Photon Collectors 
\item SiPMs
\item ...
\end{itemize}
