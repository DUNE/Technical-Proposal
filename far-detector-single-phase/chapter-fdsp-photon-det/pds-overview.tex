\section{Photon Detection System (PD) Overview}
\label{sec:fdsp-pd-ov}
%\metainfo{(Length: TDR=10 pages, TP=3 pages)}

%%%%%%%%%%%%%%%%%%%%%%%%%%%%%%%%%
\subsection{Introduction}
\label{sec:fdsp-pd-intro}
%\todo{\color{blue} Content: Segreto}

The photon detection system is an essential subsystem of the DUNE Single-Phase TPC. The detection of the prompt scintillation light signal, emitted in coincidence with an ionizing event inside the active volume, allows the determination of the time of occurrence of an event of interest a thousand times sooner than charge collected from ionization track and with much higher precision. For physics that is uncorrelated with the accelerator, such as proton decay and neutrinos from supernova burst  this information allows determination of the drift time of the ionizing particles.
Knowledge of the drift time provides localization of the event inside the active volume and provides the ability to correct the measured charge for  effects that depend on the drift path length or for specific locations in the detector if there are non-uniformities.  This correction is important for the reconstruction of the energy deposited by the ionizing event, which otherwise would depend on the drift time and on the purity of the liquid argon. In addition to allowing optimum track reconstruction, scintillation light measured by the system could also be used as trigger and for improved calorimetric measurement in combination with charge measurement.

Liquid argon (LAr) is known to be an abundant scintillator and emits about \SI{40}{photons/keV} when excited  by minimum ionizing particles\cite{ref:lar-scint}, in absence of external electric fields. The passage of ionizing radiation in LAr produces excitations 
and ionizations of the argon atoms that ultimately results in the formation of the 
excited dimer Ar$^*_2$.  Photon emission proceeds through the de-excitation 
of the lowest lying singlet and triplet excited states, $^{1}\Sigma$ and 
$^{3}\Sigma$ to the dissociative ground state. The de-excitation from the 
$^{1}\Sigma$ state is very fast and has a characteristic time of the order of 
$\tau_{fast}$ $\simeq$ 6 ns. The de-excitation from the $^{3}\Sigma$ state is 
much slower with a characteristic time of $\tau_{slow}$ $\simeq$ \SI{1.3}{$\mu$sec}, 
since it is forbidden by the selection rules. 
In both decays, photons are emitted in a \SI{10}{nm} band centered around \SI{127}{nm}, in 
the Vacuum Ultra Violet (VUV) region of the electromagnetic spectrum.
The relative abundance of the  fast and slow components is related to the ionization density of LAr and 
depends on the ionizing particle: \num{0.3} for electrons, \num{1.3} for alpha 
particles and \num{3} for neutrons. This phenomena is the basis for the  
particle discrimination capabilities of LAr exploited by many 
experiments.

The paradigm for the detection of LAr scintillation light depends on the use of 
chemical wavelength shifters since the currently available commercial (cryogenic) photosensors are not 
directly sensitive to VUV radiation due to the lack of transparency of fused silica and 
glass optical windows. The most widely used wavelength shifter used in 
combination with LAr is Tetra-Phenyl Butadiene (TPB), which absorbs VUV photons 
and re-emits them with a spectrum centered around \SI{430}{nm}, where most of the 
photosensors have their maximum quantum efficiency for photoconversion. TPB conversion efficiency 
is known to be high, when compared to other wavelength shifters, and 
it is often taken to be \num{100}\%, even though a reliable direct measurement of this 
relevant quantity is not available in the literature. 
In large LAr TPC, it is common to use so-called photon collector systems that attempt to 
collect light from large areas and channel it in an efficient way towards  
photosensors that produce an electrical signal.

%Scintillation light detection plays several important roles in a LAr TPC such as
%a prompt trigger for potential events of interest and to precisely tag the time of the event, $T_0$, which is needed to 
%determine the absolute position of the event inside the active volume.  It may also allow accurate calorimetric measurement of 
%the deposited energy and so contributes to capability of the TPC as a powerful instrument for particle identification.

Table~\ref{tab:pds-req} summarizes the key performance requirements for the photon detection system.
For example, the photon system will provide the timing of events relative to TPC timing, $t_0$, with a 
resolution better than \SI{1}{$\mu$sec}.  The capability to measure the $t_0$ of non-beam events with deposited 
energy above \SI{200}{MeV} will allow observation of proton decay and atmospheric neutrinos with high 
efficiency by enabling 3D spatial localization of candidate events. 
There is a minimum light yield requirement is of  \num{0.1}\si{photoelectrons(pe)/MeV} for events that occur near the cathode, which is the farthest region from photon collectors that are embedded in the Anode Plane Assembly (APA) described in Chapter\ref{ch:fdsp-apa}. 
A revision of these requirements is being considered in order to better exploit the 
characteristics of LAr scintillation light and to improve the performance of the detector for low energy events such neutrinos from core collapse supernovae. Maximizing the number of detected photons provides two important benefits: the efficiency to identify the interaction time 
(and subsequently improve the calorimetric energy resolution of the interaction through drift-distance correction) is substantially improved at all energies; and, perhaps more importantly, the supernova neutrino tagging efficiency is more robust against uncertainty in the final module performance and manufacturing variations between modules once the light collector efficiency. 

Low-energy neutrinos from a galactic core-collapse supernova represent one of the most challenging physics channels for the photon detection system and figure~\ref{fig:pds-sn-eff-simulation} shows the impact of improved PD system performance on neutrino detection efficiency as a function of the energy of the electron produced in the neutrino interaction (this study was done for the double-shift light guide design).  The curve labeled ``standard'' corresponds to the current level of performance and results in an average efficiency for the photon detection system to unambiguously reconstruct the neutrino interaction time to be \num{30}\% at \SI{5}{MeV} and \num{70}\% at \SI{15}{MeV} of visible energy. A factor of two increase in PD light yield increases the efficiency at \SI{5}{MeV} by almost \num{70}\%. 
Since the technology is by now quite mature, the effects of these improvements can be predicted quite well through Monte Carlo simulations; validation of the simulations will come from extensive data from ProtoDUNE-SP, which will have \num{29} bars of each type.

\begin{dunefigure}[Neutrino detection efficiency dependence on PD system performance.]
{fig:pds-sn-eff-simulation}
{Preliminary neutrino detection efficiency estimate as a function of the energy of the electron produced in the neutrino interaction for a variety of relative efficiencies of the double-shift light guide detector. The curve marked ``standard'' represents performance of recent prototypes.} 
\includegraphics[width=0.5\columnwidth]{pds-sn-eff-simulation.png}
\end{dunefigure}

%\fixme{Needs: Performance requirements for the PD system table to expand?}

\begin{dunetable}
[Key performance requirements for the PD system (Note: these are under review).]
%{p{0.8\textwidth}}
{cc}
{tab:pds-req}
{Highest-level PD performance requirements to achieve the detection efficiency of $90$\% for energy deposit of \SI{> 200}{MeV}} 
Requirement  & Value \\ \toprowrule
Light Yield  & \SI{0.1}{pe/MeV} for events near the cathode plane  \\ \colhline
Timing Resolution & \SI{1}{$\mu$s}   \\ \colhline
\end{dunetable}


%\fixme{By the end of the volume, for every requirement listed in this section, there should exist an explanation of how it will be satisfied.}

%\fixme{?? Image of the overall system, indicating its parts. Show how the system fits into the overall detector.}

%\begin{dunefigure}[optional caption for LoF]{fig:figure-label}
%{required full caption (Credit: xyz)}
%\includegraphics[width=0.8\textwidth]{image-filename}
%\end{dunefigure}

%\fixme{Include summary of simulation status? {\color{red}  Szelc/Himmel}}

%%%%%%%%%%%%%%%%%%%%%%%%%%%%%%%%%%%%%
\subsection{Design Considerations}
\label{sec:fdsp-pd-des-consid}
%\todo{\color{blue} Content: Segreto/Warner/Mualem}

The physical dimension of the photon detection (PD) system is constrained by the need to fit within the innermost wire planes of the Anode Plane Assembly (APA) and to be installed into slots in the APA mechanical frame after it is wound (see Section~\ref{sec:fdsp-apa-design}). 
Individual photon detector modules are shaped in the form of thin bars (less than \SI{1}{cm} thick) with approximate dimensions of \SI{8}{cm} $\times$ \SI{200}{cm}. It is currently anticipated that there will be ten PD modules per APA, for a total of \num{1500} modules. 

%%% rjw 15mar18
%Three different designs of PD modules have been developed and are being 
%considered by the Single-Phase Photon Detector Consortium. Two are based on the concept of 
%light guides coupled to solid state silicon photomultipliers (SiPM) while the 
%third one, namely the ARAPUCA, is functionally a light trap that captures wavelength-shifted photons inside
%boxes with highly reflective internal surfaces where they are eventually detected by an
%SiPM. Figure~\ref{fig:3dtpc-pd} shows a 3-D model of the Single-Phase TPC with a zoom in to the anode plane where the three candidates %photon collector technologies are visible for illustration - in the final detector there will for a single type.
%%%%%%%%%%

%%% rjw 15mar18
Three different designs of PD modules have been developed and are being considered by the Single-Phase Photon Detector Consortium. The baseline design, ARAPUCA\footnote{{\it Arapuca} is the name of a simple trap for catching birds originally used by the Guarani people of Brazil.}, is a relatively new concept that is scalable and has the potential for the best performance by a significant factor. It is functionally a light trap that captures wavelength-shifted photons inside boxes with highly reflective internal surfaces where they are eventually detected by an SiPM.  There are currently two alternative designs based on the use of wavelength-shifters and light guides coupled to solid state silicon photomultipliers (SiPM). Both have undergone more development than ARAPUCA but have performance that meet the physics requirements with only a small safety margin and are not easily scalable within the geometric constraints of the Single-Phase detector.
Figure~\ref{fig:3dtpc-pd} shows a 3-D model of the Single-Phase TPC with a zoom in to the anode plane where the three candidates photon collector technologies are visible for illustration - in the final detector there will for a single type.
%%%%%%%%%%

\begin{dunefigure}[3-D model of photon detectors in the APA.]{fig:3dtpc-pd}
{3-D model of photon detectors in the APA. The model on the left shows the full width of the TPC with the configuration APA-CPA-APA-CPA-APA. The right figure shows a zoom in to the top far side of the TPC where three candidates photon collector technologies are visible for illustration - in the final detector there will be a single type.}
\includegraphics[height=5cm]{pds-dune-sp-tpc-3d.jpg}
\includegraphics[height=5cm]{pds-dune-sp-tpc-3d-zoom.jpg}
\end{dunefigure}

%DWW 15mar18 start %%%%%%

%The ARAPUCA module is composed of an array of smaller boxes (about \num{20} per module) 
% each one acting as a smaller detector. Each box has an optical window formed  
%by a dichroic filter deposited with two different wavelength shifters, one on 
%the external side and another on the internal one, which allows the light to get
%inside the box, but not to exit. The internal surface of each box is lined 
%with a highly reflective material so that the trapped photon can reflect several 
%times before absorption. An array of SiPMs is installed inside the box to detect the light either 
%directly from the window or after some number of bounces.

The ARAPUCA module is composed of an array of sixteen approximately 8.6$\times$8.6 cm$^2$ boxes, 
 each one acting as an individual detector element, allowing for additional segmentation along the detector bar. 
In the case of the original ARAPUCA, which only collects light from one side, each box has an optical window formed  
by a dichroic filter deposited with a layer of pTP\footnote{p-TerPhenyl,  supplier: Sigma-Aldrich\textregistered},
% https://www.sigmaaldrich.com/catalog/substance/pterphenyl230309294411.} 
wavelength shifter on the external surface, which shifts the incident VUV light to a frequency able to pass through the filter plate to the interior
of the box.  
%The internal surface of each box is lined 
In the current version of the device, the inner surface of the box opposite the window is lined 
with a highly reflective material coated with a second wavelength shifter, TPB\footnote{1,1,4,4-Tetraphenyl-1,3-butadiene,
supplier: Sigma-Aldrich\textregistered},
%https://www.sigmaaldrich.com/catalog/substance/1144tetraphenyl13butadiene35847145063111}, 
which converts the light passing through the filter a second time to a wavelength which will be reflected by the filter, so that the trapped photon can reflect several times before absorption. An array of SiPMs is installed inside the box facing the window to detect the light either directly or after some number of bounces.

In the case of ARAPUCAs intended for mounting in the central APA frame, which are required to collect light from both directions, filter plates can be mounted on both sides of the box and the SiPMs moved to the walls.  In this case, the filters are coated with TPB on the outer surface, and with pTP on the inner surface, but otherwise the double-sided concept functions the same as the traditional ARAPUCA. 
Another variation under investigation, described in the next section, uses a wavelength-shifter doped plate in between the windows.

%fixme>  Do we want to add the X-ARAPUCA concept here?

%DWW 15mar18 end %%%%%%


The dip-coated light guides are pre-treated commercially-cut acrylic slabs dip-coated with a solution of TPB, acrylic and toluene. 
When the toluene evaporates it leaves a thin film of TPB embedded in the acrylic matrix that acts as a wavelength 
shifter. A fraction of the light is captured inside the acrylic bar by total 
internal reflection and is detected at one or both ends of the bar by an array of SiPM.
%DWW 15mar18 start %%%%%%
In the double-shift light guides the conversion and guiding processes of the photons are decoupled. The first conversion is
performed by an Ultraviolet transmitting UVT acrylic (radiator) plate coated with pure TPB (through a spraying process) that is positioned \SI{1}{mm} above a commercial doped bar that absorbs the blue
light produced by TPB and re-emits it in the green.
%DWW 15mar18 end %%%%%%

 A fraction of the green light propagates down the bar by total-internal-reflection until it is incident on an
array of SiPM installed at one or both ends.

The two solid bars designs have been developed over several years and have reached a reasonable level of maturity and reliability. 
Both designs have demonstrated attenuation lengths along the long dimension of the bar of the 
trapped light comparable to the length of the bars themselves, which ensures a reasonable uniformity along the beam direction. Their absolute efficiency has been measured to be in a range between \num{0.1}\% and \num{0.2}\%.
%DWW 15mar18 start %%%%%%
\fixme{Does IU report efficiency closer to .3 or .4 \%  for the double-dip --email sent to Stu \& Denver}
%DWW 15mar18 end %%%%%%
Further  improvements could come from simple extensions of the design such as installing SiPM at both ends of the bars and coating the smaller sides of the  bars with reflective foils. 

The ARAPUCA concept is relatively recent -- it was first proposed in 2015 and accepted for installation in ProtoDUNE-SP in mid-2016. A series of tests in LAr have been performed with an evolving prototype design that resulted in efficiency measurements ranging from 
\num{0.4}\% to \num{1.8}\%, demonstrating the potential for substantially higher performance than the light-guide designs. Monte Carlo simulations show that efficiencies at the level of few per cent could be reasonably reached with minor modifications to the basic design. 
While the results of the experimental tests are encouraging, a deeper understanding of the optical phenomena utilized by the ARAPUCA concept is needed.

Since the photon detection modules are installed only on the anode plane light collection is not uniform over the entire active volume of the TPC. A possible solution to improve this is to install a reflective foil coated with wavelength shifter on the cathode.
This would increase the light yield of the detector and could enable calorimetric measurements based on light emitted by the ionizing particles. It may also be possible to remove the \Ar39 background through PD-supplied timing cuts, a background that may otherwise cause a huge counting rate for events near  the anode plane. This possibility is under study through Monte Carlo simulations and the mechanical feasibility is being discussed with the HV Consortium.

The minimal requirement for the light yield of the PD system is of \SI{0.1}{pe/MeV} 
near the cathode, which would allow the detector to efficiently  detect ($>$ \num{90}\%) 
proton decay events (visible energy $>$ \SI{200}{MeV}). All three designs satisfy
the minimum requirements according to preliminary Monte Carlo 
studies, but only barely, and there is a consensus inside the Collaboration that a higher 
light yield would be very beneficial for the detection of low energy supernova 
neutrinos.  Consequently, a critical review of the requirements is underway.

The need for an improved understanding of the potential ARAPUCA performance drives the strategy for the R\&D program that will be carried out before the Technical Design Report (mid-2019). An intense effort 
is underway to explore the possibility that an implementation of the ARAPUCA concept could increase the light yield of the 
detector by a factor ten with respect to the bars; resources (personnel and funding) are being sort by the Consortium to achieve this.  
It is anticipated that by the time of the TDR, the Consortium will present ARAPUCA as a baseline design for the photon collector and one alternative design for risk mitigation.  

In each photon collector concept, the final stage of converting a visible wavelength photon into an electrical signal will be performed by a silicon photomultipler (SiPM). Our experience with a promising early candidate which failed in later batches due to an unadvertised change in SiPM fabrication has emphasized the importance of a multi-source approach where we are actively engaged with potential vendors to develop a device expressly for cryogenic operation. There are ongoing investigations of a cryogenic SiPM produced by Hamamatsu and by FBK (Fondazione Bruno Kessler, Italy), which developed a device for use in the DarkSide cryogenic experiment.

For prototype development and for ProtoDUNE-SP, a very capable waveform digitizer has been developed which enables a thorough investigation of the photosensor signals, particularly as we investigate the impact of electrically ganging multiple sensors. The design of the readout electronics for the final system will be strongly influenced by the outcomes of Monte Carlo simulations that are in progress. Of particular interest is the extent to which pulse  shape capabilities are important to maximizing sensitivity to low energy neutrino interactions from supernovae. 
%These consideration will have a role in defining the read-out scheme and the digitization frequency of the signals.

%DWW 15mar18 start %%%%%% Edited by rjw

Initial Monte Carlo simulations suggest that it may not be necessary to fully digitize the SiPM waveforms in order to achieve the PD performance requirements.  Charge integration electronic readout systems, which offer the promise of significantly lower cost and possibly simpler and smaller cabling harnesses, are under investigation, and is expected to be the baseline option.
A lower-cost waveform digitization based on lower sampling rate commercial electronics will continue to be investigated as a potential backup option in case our evolving understanding of the requirements necessitates collecting waveform data from the SiPMs.

%DWW 15mar18 end %%%%%%

\subsection{Photon Collection Options Evaluation}
%\todo{\color{blue} Content: Segreto}

The performance of the different photon collection options will be 
evaluated in the facilities available to the Consortium that will allow
relative and absolute measurements to be performed at both room and cryogenic temperatures.
The most comprehensive set of data will come from the fully instrumented modules in the ProtoDUNE-SP experiment currently 
under construction at CERN, that will start operations in the last third of \num{2018}.
All  three different photon collector designs are present in ProtoDUNE-SP: \num{29} 
double-shift guides, \num{29} dip-coated guides, and two ARAPUCA arrays. The 
presence of the Time Projection Chamber will allow  precise reconstruction
in 3D the track of any ionizing event inside the active volume. The 
matching of the track with the associated light signal will enable an
accurate comparison of the relative detection efficiencies of the different PD 
modules. 
Absolute measurements will be possible depending on the accuracy of the
Monte Carlo simulations of the optical properties of the detector. 
Unfortunately, since some of the optical parameters that 
regulate VUV light propagation in LAr are poorly known, such as Rayleigh 
scattering length, this will influence the precision of the absolute 
measurements. 
%\todo{sounds like we need a plan to measure this somewhere...}

ProtoDUNE-SP will also give the opportunity of performing a longer-term test of full-scale PD modules for the first time. It will be possible to quantify any deterioration in their performance, which will be important input to determining the necessary initial performance level, such as light yield, that will ensure that the minimal requirements are satisfied for the whole life of the DUNE experiment.

An R\&D program will be executed in parallel with the ProtoDUNE-SP operation since 
additional comparative measurements will be needed, particularly for the newer ARAPUCA concept, prior to establishing the baseline design described in the Technical Design Report.

%DWW 15mar18 start %%%%%%

%The TallBo facility at Fermilab will be extremely important for this program. 
%It provides a \SI{450}{liter} capacity cryostat with \SI{56}{cm} inner diameter and up 
%to a \SI{183}{cm} liquid depth that accommodates  up to three different PD 
%modules with dimensions \todo{how different?} close to the real ones.

% I think we should cut this

%DWW 15mar18 end %%%%%% rjw edits

Several facilities are accessible to the Consortium that will allow testing of smaller scale prototypes of the modules (or secitons of them). These include: the 
ScENE set-up at Fermilab; the cryogenic facilities at Colorado State University and UNICAMP (Brazil); and the optical facilities at Fermilab, 
Indiana University and UNICAMP.

%DWW 15mar18 start %%%%%%

In addition, PD modules and interfaces with the APA system and cold electronics will be conducted in cryogenic gaseous nitrogen in cold box studies at CERN, utilizing a test stand developed for testing of ProtoDUNE-SP components prior to installation into the detector.  A full-scale ProtoDUNE-SP APA is currently being fabricated, and will be instrumented with cold electronics and photon detectors, allowing the interfaces to be carefully studied.

A small-scale TPC is also planned for cold electronics testing at FNAL, and will be instrumented with as many as three 1/2-length PD modules to provide triggering information for the TPC and continuing interface studies with the APA and cold electronics.  It is envisioned that as many as three test cycles will be performed prior to the TDR, allowing testing and continued development of the ARAPUCA concept.
%DWW 15mar18 end %%%%%%
These facilities will be valuable for the single modules optimization process.

A decision on the light collector technology will be made in February 2019.

\subsection{Scope}
\label{sec:fdsp-pd-scope}
%\todo{\color{blue} Content: Segreto/Warner/Mualem}
The scope of the photon detector (PD) system for the DUNE far detector 
reference design includes design, procurement, fabrication, testing,
 delivery and installation of the following components:
\begin{itemize}
        \item light collection system;
        \item photosensors (silicon photomultipliers);
        \item readout electronics and cabling;
        \item calibration system (tbd);
        \item (TPB-coated foils for the cathode plane (if implemented));
        \item related infrastructures.
\end{itemize}