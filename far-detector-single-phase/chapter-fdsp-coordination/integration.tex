\subsection{Integration and Systems Engineering }
\label{sec:fdsp-coord-integ-sysengr}
%				7 pages

The major aspects of detector integration focus on the mechanical and
electrical connections between each of the detector systems. A second
major area is in the support of the detector and its interfaces to the
cryostat. A third major area is in assuring that the detector can be
installed --- that the integrated components can be moved into their
final configuration. A fourth major area is in the integration of the
detector with the necessary services provided by the conventional
facilities.

%%%%%%%%%%%%%%%%%%%%%%%%%%%%%%%%
\subsubsection{Configuration Management}
\label{sec:fdsp-coord-integ-config}

The DUNE Technical Coordination project engineering team will maintain
full 3-D CAD models of the detectors, and the consortia will be
responsible for providing the team with CAD models of their detector
components for integration into the overall models.  The project
engineering team will work with the LBNF project team to integrate the
full detector models into a global LBNF CAD model that includes
cryostats, cryogenic systems, and the conventional facilities.  The
DUNE project engineering team will work directly with the Consortia
Technical Leads and their supporting engineering teams to resolve any
detector component interference and/or connection issues with other
detector systems, detector infrastructure, and facility
infrastructure.

For mechanical design aspects, the DUNE Technical Coordination project
engineering team will maintain full 3-D CAD models of the detectors.
Each consortium will be responsible for providing the project
engineering team with CAD models of their detector components to be
integrated into overall models.  The project engineering team will
work with the LBNF project team to integrate the full detector models
into a global LBNF CAD model that will include cryostats, cryogenic
systems, and the conventional facilities.  The DUNE project
engineering team will work directly with the Consortia Technical Leads
and their supporting engineering teams to resolve any detector
component interference and/or connection issues with other detector
systems, detector infrastructure, and facility infrastructure.

For electrical design aspects, the DUNE Technical Coordination project
engineering team will maintain high level interface documents which
describe all aspects of required electrical infrastructure and
electrical connections.  All consortia must document power
requirements and rack space requirements.  Consortia are responsible
for defining any cabling which bridges the design efforts of two or
more consortia.  This agreed upon and signed-off interface
documentation should include cable specification, connector
specification, connector pinout and any data format, signal levels and
timing.  All cables, connectors, printed circuit board components,
physical layout and construction will be subject to project review.
This is especially true of elements which will be inaccessible during
the project lifetime.  Consortia should provide details on liquid
argon temperature acceptance testing and lifetime of components,
boards, cables and connectors.


At the time of the release of the Technical Design Reports, the
project engineering team will work with the consortia to produce
formal engineering drawings for all detector components.  These
drawings are expected to be signed by the consortia Technical Leads,
project engineers, and Technical Coordinator.  Starting from that
point, the detector models and drawings will sit under formal change
control.  It is anticipated that designs will undergo further
revisions prior to the start of detector construction, but any changes
made after the release of the Technical Design Reports will need to be
agreed to by all of the drawing signers and an updated, signed drawing
produced.

The major areas of configuration management include:
\begin{enumerate}
  \item 3-D Model
  \item Interface Definitions
  \item Envelope Drawings for installation
  \item Drawing management
\end{enumerate}

%%%%%%%%%%%%%%%%%%%%%%%%%%%%%%%%
\subsubsubsection{Configuration Management Processes}
\label{ssec:fdsp-coord-integ-cnfg-mgmt}

Technical Coordination will put into place processes for
configuration management.  Configuration management will provide
technical coordination and engineering staff the ability to define and
control the actual build of the detector at any point in time and to
track any changes which may occur over duration of the build as well
as the lifetime of the project.

For detector elements within the cryostat, configuration management
will be frozen once the elements are permanently sealed within the
cryostat.  However, during the integration and installation process of
building the detector within the cryostat, changes may need to occur.
For detector elements outside the cryostat and accessible, all
repairs, replacements, hardware upgrades, system grounding changes,
firmware and software changes must be tracked.  It is vital that the
Technical Coordination staff be able to define the exact state of the
detector at any given time during the lifetime of data taking.

Any change will require revision control, configuration
identification, change management and release management.

{\bf Revision Control}\\
Consortia will be responsible for providing accurate and well
documented revision control.  Revision control should provide a method
of tracking and maintaining the history of changes to a detector
element.  Each detector element must be clearly identified with a
document which includes a revision number and revision history.  For
mechanical elements, this can be reflected by a drawing number with
revision information.  For electrical elements, schematics can be used
to track revisions.  Consortia will be responsible for identifying the
revision status of each installed detector elements.

{\bf Configuration Identification}\\
Consortia are responsible for providing unique identifiers or part
numbers for each detector element.  Plans should be developed on how
inventories will be maintained and tracked during the build.  Plans
should clearly identify any dynamic configuration modes which may be
unique to a specific detector element.  For example, a printed circuit
board may have firmware which effects its performance.

{\bf Change Management}\\
Technical Coordination will provide guidelines
for formal change management.  During the beginning phase of the
project, drawings and interface documents are expected to be signed by
the consortia Technical Leads, project engineers, and Technical
Coordinator.  Once this initial design phase is complete, the detector
models, drawings, schematics and interface documents will sit under
formal change control.  It is anticipated that designs will undergo
further revisions prior to the start of detector construction, but any
changes made after the release of the Technical Design Reports will
need to be agreed to by all drawing signers and an updated signed
drawing produced.

{\bf Release Management}\\
Release management focuses on the delivery of the more dynamic aspects
of the project such as firmware and software.  Consortia with
deliverables that have the ability to effect performance of the
detector by changing firmware or software must provide plans on how
these revisions will be tracked, tested and released.  The
modification of any software or firmware after the initial release,
must be formally controlled, agreed upon and tracked.


%%%%%%%%%%%%%%%%%%%%%%%%%%%%%%%%
\subsubsection{Engineering process and support}
\label{sec:fdsp-coord-integ-engr-proc}
 

The DUNE Technical Coordination organization will work with the
consortia through its project engineering team to ensure the proper
integration of all detector components.  The project engineering team
will document requirements on engineering standards and documentation
that the consortia will need to adhere to in the design process for
the detector components under their responsibility.  Similarly, the
project QA and ES\&H managers will document quality control and safety
criteria that the consortia will be required to follow during the
construction, installation, and commissioning of their detector
components.


Consortia interfaces with the conventional facilities, cryostats, and
cryogenics are managed through the DUNE Technical Coordination
organization.  The project engineering team will work with the
consortia to understand their interfaces to the facilities and then
communicate these globally to the LBNF project team.  For conventional
facilities the types of interfaces to be considered are requirements
for bringing needed detector components down the shaft and through the
underground tunnels to the detector cavern, overall requirements for
power and cooling in the detector caverns, and the requirements on
cable connections from the underground area to the surface.
Interfaces to the cryostat include the number and sizes of the
penetrations on top of the cryostat, required mechanical structures
attaching to the cryostat walls for supporting cables and
instrumentation, and requirements on the global positioning of the
detector within the cryostat.  Cryogenic system interfaces include
requirements on the location of inlet/output ports, requirements on
the monitoring of the liquid argon both inside and outside the
cryostat, and grounding/shielding requirements on piping attached to
the cryostat.

LBNF will be responsible for the design and construction of the
cryostats used to house the detectors.  The consortia are required to
provide input on the location and sizes of the needed penetrations at
the top of the cryostats.  The consortia also need to specify any
mechanical structures to be attached to the cryostat walls for
supporting cables or instrumentation.  The DUNE project engineering
team will work with the LBNF cryostat engineering team to understand
what attached fixturing is possible and iterate with the consortia as
necessary.  The consortia will also work with the project engineering
team through the development of the 3-D CAD model to understand the
overall position of the detector within the cryostat and any issues
associated with the resulting locations of their detector components.

LBNF will be responsible for the cryogenics systems used to purge,
cool, and fill the cryostats.  It will also be responsible for the
system that continually re-circulates liquid argon through filtering
systems to remove impurities.  Any detector requirements on the flow
of liquid within the cryostat should be developed by the consortia and
transmitted to LBNF through the project engineering team.  Similarly,
any requirements on the rate of cool-down or maximum temperature
differential across the cryostat during the cool-down process should
be specified by the consortia and transmitted to the LBNF team.



%%%%%%%%%%%%%%%%%%%%%%%%%%%%%%%%
\subsubsubsection{Engineering Processes}
\label{ssec:fdsp-coord-integ-eng-processes}

The engineering design process is defined by a set of steps taken to
fulfill the requirements of the design.  By the time of the Technical
Design Report, all design requirements must be fully defined and
proposed designs must be shown meet these requirements.  Based on
prior work, some detector elements may be quite advanced in the
engineering process, while others may be in earlier stages.  Each
design process shall closely follow the engineering steps described
below.


{\bf Development of specifications}\\
Each consortium is responsible for the technical review and approval
of the engineering specifications.  The documented specifications for
all major design elements should include the scope of work, project
milestones, relevant codes and standards to be followed, acceptance
criteria and specify appropriate internal or external design reviews.
Specifications shall be treated as controlled documents and cannot be
altered without approval of the DUNE Technical Coordination team.  The
project engineering team will participate in and help facilitate all
major reviews.  Special Technical Board reviews will be scheduled for
major detector elements.

{\bf Engineering Risk Analysis}\\
Each consortium is responsible for identifying and defining the level
of risk associated with their deliverables.  The DUNE Technical
Coordination organization will work with the consortia, through its
project engineering team, to document these risks in a Risk Database
and follow them throughout the project until they are realized or can
be retired.

{\bf Specification Review}\\
The DUNE Technical Coordination organization and project engineers
shall review consortia specifications for overall compliance with the
project requirements.  Consortia must document all internal reviews
and provide that documentation to the Technical Coordination staff.
Additional higher level reviews may be scheduled by the Technical
Coordination staff.

{\bf System Design}\\
The system design process includes the production of mechanical
drawings, electrical schematics, calculations which verify compliance
to engineering standards, firmware, printed circuit board layout,
cabling and connector specification, software plans, and any other
aspects that lead to a fully documented functional design.  All
relevant documentation shall be recorded, with appropriate document
number, into the chosen engineering data management system and be
available for the review process.

{\bf Engineering Design Review}\\
The design review process is determined by the complexity and risk
associated with the design element.  For a simple design element, the
consortia may do an internal review.  For a more complex or high risk
element a formal review will be scheduled.  The DUNE Technical
Coordination staff will facilitate the review, bringing in outside
experts are needed.  In all cases, the result of any reviews should be
well documented and captured in the engineering data management
system.  If recommendations are made, those recommendations will be
tracked in a database and the consortia will be expected to provide a
response.

{\bf Procurement}\\ The procurement process should include the
documented technical specifications for all procured materials and
parts.  All procurement technical documents are reviewed for
compliance to engineering standards, safety and environmental
concerns.  The DUNE Technical Coordination staff will assist the
consortia in working with procurement staff as needed.

{\bf Implementation}\\ During the implementation phase of the project,
the consortia shall provide the Technical Coordinator with updates on
schedule.  A test plan should be fully developed which will allow for
verification that the initial requirements have been met. In addition,
a quality assurance plan should be documented and followed.

{\bf Testing and Validation\\}
The testing plan documented in the above step should be followed and
results should be well documented.  The Technical Coordinator and
engineering team should be informed as to the results and whether the
design meets the specifications.  If not, a plan should be formulated
to address any shortcomings and presented to the Technical
Coordinator. [QA Manager should be mentioned here]

{\bf Operations\\}
The release to operations allows equipment to be put into service.
Prior to this happening, the equipment must be shown to have met the
specifications, met engineering standards, passed a safety review, and
be completely documented. [How does this apply to DUNE?]

{\bf Final Documentation\\}
A final report should be generated which describes the as built
equipment, any lessons learned, safety reports, installation procedures
and checks, and operations manual.
