\section{Project Support}
\label{sec:fdsp-coord-supp}

The DUNE Project is coordinated by Technical Coordination (TC). The DUNE
Project consists of a Far Detector (FD) and a Near Detector (ND). The Near
Detector is at a pre-conceptual state; as the Conceptual Design and
organization emerges, it will become part of the DUNE
Project. Currently the DUNE Project consists of the DUNE Far Detector
Consortia and Technical Coordination.

As defined in the DUNE Management Plan (DMP), the DUNE Technical Board (TB) is
the technical decision making body for the collaboration. It consists
of all consortia scientific and technical leads. It reports through
the Executive Board (EB) to Collaboration Management. The DUNE Technical
Board is chaired by the Technical Coordinator.

[Need and Org chart]

		%				3 pages

%%%%%%%%%%%%%%%%%%%%%%%%%%%%%%%%
\subsection{Project Controls}
\label{sec:fdsp-coord-controls}

Technical Coordination Project Controls (PC) maintains a web page
(currently located at
{https://web.fnal.gov/collaboration/DUNE/DUNE\%20Project/\_layouts/15/start.aspx\#/})
with links to project documents. Technical Coordination maintains
repositories of project documents and drawings. These include the WBS,
Schedule, risk register, requirements, milestones, strategy, detector
models and drawings that define the DUNE detector.

In order to insure that the DUNE detector remains on schedule, TC
project controls will monitor schedule statusing from each Consortium,
will organize reviews of the schedules and risks as appropriate. PC
will maintain a master schedule that links all Consortia schedules and
contains appropriate milestones to monitor progress.

%%%%%%%%%%%%%%%%%%%%%%%%%%%%%%%%
\subsection{Reviews}
\label{sec:fdsp-coord-reviews}

Technical Coordination is responsible to review all stages of detector
development and works with each Consortium to arrange reviews of the
design, production readiness, production progress and operational
readiness of their system.  These reviews provide input for the TB to
make technical decisions.  Review reports are tracked by TC and
provide guidance as to key issues that will require engineering
oversight by TC.

%%%%%%%%%%%%%%%%%%%%%%%%%%%%%%%%
\subsection{Quality Assurance}
\label{sec:fdsp-coord-qa}


The LBNF/DUNE Quality Assurance Plan outlines the QA requirements for
all DUNE Consortia and describes how the requirements shall be
met. The Consortia will be responsible for implementing a quality plan
that meet the requirements of the LBNF/DUNE Quality Assurance Plan.
The Consortia implement the plan through the development of individual
quality plans, procedures, guides, QC inspection and test requirements
and travelers/test reports.  The Consortia will be responsible for
implementing a quality plan that meet the requirements of the
LBNF/DUNE Quality Assurance Plan. In lieu of a Consortia Specific
Quality Plan, the Consortia may work under the LBNF/DUNE Quality
Assurance Plan and develop Manufacturing/QC Plans, procedures and
documentation specific to their work scope.  The DUNE Technical
Coordinator and Consortia Leaders are responsible for providing the
resources needed to conduct the Project successfully, including those
required to manage, perform and verify work that affects quality.  The
DUNE Consortia Leaders are responsible for identifying adequate
resources to complete the work scope and to ensure that their team
members are adequately trained and qualified to perform their assigned
work.

The Consortia work will be documented on travelers and applicable test
or inspection reports. Records of the fabrication, inspection and
testing will be maintained. When a component has been identified as
being in noncompliance to the design, the nonconforming condition
shall be documented, evaluated and dispositioned as use-as-is (does
not meet design but can meet functionality as is), rework (bring into
compliance with design), repair (will be brought into meeting
functionality but will not meet design) and scrap. For products with a
disposition of accept as.

The LBNF/DUNE Quality Assurance Manager reports to the LBNF Project
Manager and DUNE Technical Coordinator and provides oversight and
support to the Consortia Leaders to ensure a consistent quality
program. The LBNF/DUNE Quality Assurance Manager will plan reviews as
independent assessments to assist the DUNE Technical Coordinator in
identifying opportunities for quality/performance-based improvement
and to ensure compliance with specified requirements. The LBNF/DUNE
Quality Assurance Manager is responsible to work with the Consortia in
developing their QA/QC Plans. The LBNF/DUNE QA Manager will review
Consortia QA/QC activity, including production site visits.  The
LBNF/DUNE Quality Assurance Manager will participate in Consortia
Design Reviews, conduct Production Readiness Reviews prior to the
start of production, conduct Production Progress Reviews on a regular
basis, and perform follow-up visits to Consortia facilities prior to
shipment of components to ensure all components and documentation is
satisfactory.


%%%%%%%%%%%%%%%%%%%%%%%%%%%%%%%%
\subsection{ES\&H}
\label{sec:fdsp-coord-esh}

The DUNE Environmental, Safety and Health (ESH) program is described
in the LBNF/DUNE Integrated Environmental, Safety and Health
Plan. This plan is maintained by the LBNF/DUNE ESH Manager, who
reports to the TC. The ESH is responsible to work with the Consortia
in reviewing their hazards and their ESH plans.  The ESH Manager is
responsible to review ESH at production sites, integration sites and
at SURF.



