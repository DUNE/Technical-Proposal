Construction of the DUNE Far Detector is a massively complex
activity. While the technology for massive noble liquid detectors has
developed over the last 45 years, the first large liquid Argon Time
Projection Chamber (LArTPC) was completed in 2010. While multiple
LArTPCs have operated worldwide, the technology is still relatively
new and the scale up to DUNE is significant.

DUNE consists of a large number of institutions distributed throughout
the world. They are supported by a large number of funding sources and
collaborate with a large number of commercial partners. DUNE has
empowered several Consortia (currently nine) with the responsibility
to secure funding for, design, fabricate, assemble, install,
commission and operate the key components of the DUNE far detector.

DUNE Technical Coordination, under the direction of the DUNE Technical
Coordinator, has responsibility to monitor the technical aspects of
the detector construction of the detector subsystems, to integrate the
detector and for a number of common projects. Groups of institutes
within DUNE form Consortia that take complete responsibility for
construction of their system.

Given the horizontal nature of the Consortia structure and the
extensive interdependencies between the systems, a significant
engineering organization is required to deliver DUNE on schedule and
within specifications and funding constraints.



\section{Project Support}
\label{sec:fdsp-coord-supp}

The DUNE Project is coordinated by Technical Coordination (TC). The DUNE
Project consists of a Far Detector (FD) and a Near Detector (ND). The Near
Detector is at a pre-conceptual state; as the Conceptual Design and
organization emerges, it will become part of the DUNE
Project. Currently the DUNE Project consists of the DUNE Far Detector
Consortia and Technical Coordination.

As defined in the DUNE Management Plan (DMP), the DUNE Technical Board (TB) is
the technical decision making body for the collaboration. It consists
of all consortia scientific and technical leads. It reports through
the Executive Board (EB) to Collaboration Management. The DUNE Technical
Board is chaired by the Technical Coordinator.

[Need and Org chart]

		%				3 pages

%%%%%%%%%%%%%%%%%%%%%%%%%%%%%%%%
\subsection{Project Controls}
\label{sec:fdsp-coord-controls}

Technical Coordination Project Controls (PC) maintains a web page
(currently located at
{https://web.fnal.gov/collaboration/DUNE/DUNE\%20Project/\_layouts/15/start.aspx\#/})
with links to project documents. Technical Coordination maintains
repositories of project documents and drawings. These include the WBS,
Schedule, risk register, requirements, milestones, strategy, detector
models and drawings that define the DUNE detector.

In order to insure that the DUNE detector remains on schedule, TC
project controls will monitor schedule statusing from each Consortium,
will organize reviews of the schedules and risks as appropriate. PC
will maintain a master schedule that links all Consortia schedules and
contains appropriate milestones to monitor progress.

%%%%%%%%%%%%%%%%%%%%%%%%%%%%%%%%
\subsection{Reviews}
\label{sec:fdsp-coord-reviews}

Technical Coordination is responsible to review all stages of detector
development and works with each Consortium to arrange reviews of the
design, production readiness, production progress and operational
readiness of their system.  These reviews provide input for the TB to
make technical decisions.  Review reports are tracked by TC and
provide guidance as to key issues that will require engineering
oversight by TC.

%%%%%%%%%%%%%%%%%%%%%%%%%%%%%%%%
\subsection{Quality Assurance}
\label{sec:fdsp-coord-qa}

The DUNE Quality Assurance program is described in the LBNF/DUNE
Quality Assurance Plan. This plan is maintained by the LBNF/DUNE
Quality Assurance Manager (QAM), who reports to the TC. The QAM is
responsible to work with the Consortia in developing their QA
plans. The QAM is responsible to review QC activity as appropriate,
including production site visits.

%%%%%%%%%%%%%%%%%%%%%%%%%%%%%%%%
\subsection{ES\&H}
\label{sec:fdsp-coord-esh}

The DUNE Environmental, Safety and Health (ESH) program is described
in the LBNF/DUNE Integrated Environmental, Safety and Health
Plan. This plan is maintained by the LBNF/DUNE ESH Manager, who
reports to the TC. The ESH is responsible to work with the Consortia
in reviewing their hazards and their ESH plans.  The ESH Manager is
responsible to review ESH at production sites, integration sites and
at SURF.



