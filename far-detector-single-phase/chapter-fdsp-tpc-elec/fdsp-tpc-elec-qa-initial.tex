The \dword{qa} program for the DUNE \dword{spmod} electronics has been ongoing for several years.  The \dword{qa} program started with the appropriate design choices for operation
in \lar, including the measurement of transistor properties, and later
continued with tests of all of the cold components, cables, \fdth and flange mechanicals, and warm electronics for suitability for the \dword{spmod} requirements.  The current focus of the \dword{qa} program is the instrumenting of the \dword{pdsp} detector with \num{120} \dword{femb} (\num{960} of each of the current \dword{fe} and \dword{adc} \dword{asic} designs) and six full \dword{apa} readout chains as described in Section~\ref{sec:fdsp-tpc-elec-design}. There are aspects of the DUNE design that are not going to be fully validated in the 2018 \dword{pdsp} data taking and that will require additional confirmation. Some aspects of the detector, like the use of stacked \dwords{apa} with long cables routed possibly through the \dword{apa} frames, and the integration and installation procedure, will have to be demonstrated in independent tests. The \dwords{asic} and \dwords{femb} used for the \dword{spmod} will be an evolution of the current \dword{pdsp} ones. Below we focus mostly on the electronics and on system tests to demonstrate the final design meets the \dword{spmod} specifications, but plans are being made to test all the detector components and their assembly and installation prior to the submission of the \dword{tdr}.

The existing \dword{larasic} design will be revised from the \dword{pdsp} version. Several issues, 
including pedestal uniformity and baseline restoration in the existing \dword{larasic}, have been 
addressed by BNL in a spring 2018 submission. It will be tested at BNL and other sites to 
verify that the issues have been resolved. All new \dword{asic} designs, including the new cold \dword{adc}, \dword{coldata}, and CRYO will be tested first at the component level, both at room temperature and at \lar (or LN$_2$) temperature. The next step will be to modify the \dword{femb} to accommodate the new \dwords{asic}.  Tests of the new \dwords{femb} (at least two versions) will be followed by small-scale system tests as described below, and plans are being made to test three full-scale \dwords{apa} with the final \dword{asic}(s) and \dwords{femb} in a second run in the \dword{pdsp} cryostat.  The cold data and \dword{lv} cables used in \dword{pdsp} have been selected to be candidates for the \dword{spmod} baseline design, and tests are in progress to demonstrate that they can successfully transmit the high-speed data over the longest possible cable length in the \dword{spmod}. It should be noted that the current schedule for the the \dword{spmod} construction, which is discussed in Section~\ref{sec:fdsp-tpc-elec-org-timeline}, foresees the possibility of two more iterations in the design of the \dwords{asic} and the \dwords{femb}, including the time for performing system tests.

The updates to the the \dword{spmod} cryostat penetrations and spool pieces, as well as the warm interface electronics, are expected to be small iterations on the already existing system for \dword{pdsp}.  Prototypes will be ordered in small batches and tested at the responsible institutions for the different components.  After individual testing, integrated system tests including other \dword{spmod} components will be critical to validate that the performance of the \dword{ce} meets the DUNE \dword{fd} requirements.  Issues identified in the integrated system tests will be fed back into the design requirements of the individual components.
