The quality assurance (QA) program for the DUNE single-phase far detector electronics has been ongoing for several years.  The QA program started with the appropriate design choices for operation
in LAr, including the measurement of transistor properties, and later
continued with tests of all of the cold components, cables, feedthrough and flange mechanicals, and warm electronics for suitability for the SP detector requirements.  The current focus of the QA program is the instrumenting of the protoDUNE-SP detector with 120 FEMB (960 of each of the current FE and ADC ASIC designs) and 6 full APA readout chains as described in Section~\ref{sec:fdsp-tpc-elec-design}. There are aspects of the DUNE design that are not going to be fully validated in the 2018 protoDUNE data taking and that will require additional confirmation. Some aspects of the detector, like the use of stacked APAs with long cables routed possibly through the APA frames, and the integration and installation procedure, will have to be demonstrated in independent tests. The ASICs and FEMBs used for DUNE will be an evolution of the current protoDUNE ones. Below we focus mostly on the electronics and on system tests to demonstrate the final design meets the DUNE FD specifications, but plans are being made to test all the detector components and their assembly and installation prior to the submission of the Technical Design Report.

The existing LArASIC design will be revised from the protoDUNE-SP version. Several issues, 
including pedestal uniformity and baseline restoration, in the existing LArASIC have been 
addressed by BNL in a spring 2018 submission. It will be tested at BNL and other sites to 
verify that the issues have been resolved. 

All new ASIC designs, including the new Cold ADC, COLDATA, and CRYO will be tested first at the component level, both at room temperature and at liquid argon (or liquid nitrogen) temperature. The next step will be to modify the FEMB to accommodate the new ASICs.  Tests of the new FEMBs (at least two versions) will be followed by small scale system tests as described below, and plans are being made to test three full scale DUNE APAs with the final ASIC(s) and FEMBs in a second run in the protoDUNE-SP cryostat.  The cold data and LV cables used in protoDUNE-SP have been selected to be candidates for the DUNE baseline design, and tests are in progress to demonstrate that they can successfully transmit the high-speed data over the longest possible cable length in DUNE.

The updates to the DUNE cryostat penetrations and spool pieces, as well as the warm interface electronics, are expected to be small iterations on the already existing system for protoDUNE-SP.  Prototypes will be ordered in small batches and tested at the responsible institutions for the different components.  After individual testing, integrated system tests including other FD etector components will be critical to validate that the performance of the CE meets the DUNE FD requirements.  Issues identified in the integrated system tests will be fed back into the design requirements of the individual components.

