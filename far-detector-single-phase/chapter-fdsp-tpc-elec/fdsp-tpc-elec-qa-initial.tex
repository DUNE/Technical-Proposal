The QA for the DUNE single-phase far detector electronics has been ongoing for several years. It includes testing all of the cold components, cables, feedthrough and flange mechanicals, and warm electronics for suitability for the SP detector requirements. The current status of the QA program is the instrumenting of the ProtoDUNE-SP detector with 120 FEMB (960 of each of the current FE and ADC ASIC designs) and 6 full APA readout chains as described in Section~\ref{sec:fdsp-tpc-elec-design}.

%%% THIS IS MORE SUITABLE FOR QC DISCUSSION (NEXT ROUND)
%The components for the cold electronics system will be validated individually as far as possible to ensure they meet the SP detector requirements for both lifetime and performance.  Individual FE ASICs will be tested in small batches at room temperature and in liquid nitrogen at BNL after the upcoming ``P3'' revision.

The FE ASIC~\ref{sec:fdsp-tpc-elec-design-femb-fe} design is fairly mature, and has been tested extensively on the bench and in integrated system tests. The ADC ASIC will be selected from several design options as described in Section~\ref{sec:fdsp-tpc-elec-design-femb-adc} and Section~\ref{sec:fdsp-tpc-elec-design-alt}. The ADC candidates will be tested at the institutions responsible for developing the designs. The COLDATA ASIC is a new design for DUNE~\ref{sec:fdsp-tpc-elec-design-femb-coldata}, and will be tested individually at Fermilab both standalone and communicating to FE ASICs currently on FEMB analog motherboards. All tests will be done at both room temperature and in liquid nitrogen.

The next step will be a redesign of the FEMB to accommodate the new ADC and COLDATA. The current design of the FEMB for ProtoDUNE-SP uses a 16-channel, 2~MHz ADC to digitize the output of the FE ASICs. Correspondingly, the modification to the baseline ADC option (if used) for DUNE will be fairly minor. Prototypes of the DUNE FEMB will be tested individually at several institutions at both room temperature and in liquid nitrogen. The cold data and LV cables used in ProtoDUNE-SP have been selected to be candidates for DUNE and already verified to successfully transmit the high-speed data over $\sim35$~meters, the longest possible cable length in the far detector.

The updates to the DUNE mechanical components and warm interface electronics are expected to be small iterations on the already existing system for ProtoDUNE-SP. Prototypes will be ordered in small batches and tested at the responsible institutions for the different components.

After individual testing, integrated system tests will be critical to validate that the performance of the CE meets the DUNE FD requirements. Issues identified in the integrated system tests will be fed back into the design of the individual components.
