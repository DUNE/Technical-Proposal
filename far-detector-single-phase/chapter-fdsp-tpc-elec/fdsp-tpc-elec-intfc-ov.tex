Some of the components designed and built by the \dword{ce} consortium are
mounted on or need to work together with detector components provided by other DUNE
consortia. Interface documents have been developed to ensure that the boundaries
between systems are fully understood and that no detector components are missed or
not properly defined when organizing the detector construction project into
consortia. These interface documents are a work-in-progress. With time they will
evolve into a very detailed definition of mechanical and electrical interfaces,
including in some cases the description of data transmission protocols. These
interface documents include a list of the responsibilities of each consortium during the
R\&D, design, and prototyping phases, and discuss all of the procedures to be
followed during the integration of detector components and the following testing
and commissioning process. In some cases multiple consortia (\dword{apa}, \dword{pds}, \dword{hv}, \dword{cisc},
and \dword{daq}) %; CISC refers to ``cryogenic instrumentation and slow controls'') 
can be involved, depending on the complexity and maturity of the test setup.

The most important interfaces for the \dword{ce} consortium are with the \dword{apa} and \dword{daq}
consortia, followed by those with \dword{pds}, \dword{hv}, and \dword{cisc}. One of the most
important aspects of all these interfaces is the enforcement of appropriate grounding rules
and of the physical separation between electrical circuits to minimize the noise and
crosstalk between different detector components. Different \dwords{apa} should be electrically
insulated, and the same should apply for the \dword{pds} and the
\dword{ce} readout of the \dwords{apa}. The flanges on the chimneys are used to provide,
separately for the two \dwords{apa} and for the \dword{pds}, the reference
voltage for all the detector elements. The flanges are electrically connected
to the cryostat structure that acts as the detector ground. The same approach
has to be implemented for the \dword{cisc} instrumentation in the \lar.

The two most complex interfaces to the TPC electronics subsystem, the
interfaces to the \dwords{apa} and \dword{daq}, are discussed below.

%%% INTERFA\dword{ce} DOCUMENTS:
%\bibitem{interface:did_ce_apa}  DocDB 6670
%\bibitem{interface:did_ce_daq}  DocDB 6742
%\bibitem{interface:did_pds_ce}  DocDB 6718
%\bibitem{interface:did_hv_ce}   DocDB 6739
%\bibitem{interface:did_cisc_ce} DocDB 6745
