The installation and commissioning of the detector components built or
purchased by the \dword{ce} consortium takes place both prior to
and after the insertion of the \dwords{apa} in the
detector cryostat, with testing performed after each step, to avoid the
need for rework that could cause significant delays. The installation and
initial commissioning of the \dword{ce} electronics is likely to be on the critical
path for the completion of the \dword{spmod} and the amount of
time available for testing and possibly repairing or replacing components
after their installation is going to be very limited. Cold tests of the complete
\dwords{apa} for the entire detector require the
availability of at least two independent cold boxes at the integration facility.
To reduce the number of failures in these tests, all \dword{ce} components will
be qualified for operation in \lar prior to their installation.

After the completion of the cryostat, the spool pieces that are to house all of
the cables for the \dword{ce}, \dword{pds}, and \dword{hv} consortia (with the exception of the
high-voltage \fdth for the cathode planes) are installed and leak tested.
This also includes the installation of the crossing tube cable support and of
the flanges that provide the cold-to-warm interface for all of the cables.
In parallel, the racks that house the \dword{ce} and \dword{pds} electronics components on
the top of the cryostat and the corresponding \dword{cisc} and detector safety
system monitoring, controlling, and interlock hardware can be installed.
Cable trays between different spool pieces belonging to a set of six \dwords{apa}
and two \dwords{cpa} can also be put in place. Readout fiber bundles between the
spool pieces and the central utility cavern used for the readout of the wire
information from the \dwords{apa} can also be put in place. In the meantime, inside
the cryostat, all of the cable trays used to support the \dword{ce} and \dword{pds} cables, and
to accommodate the slack of the cables, can be put in place.

In parallel, the \num{20} %\dword{fe} motherboards (
\dwords{femb} required to read out the
wires from one \dword{apa} are installed with their shielding box onto the \dword{apa} and
connected to the CR boards. This work will be performed at the integration
facility (or facilities), because there will not be enough space to perform
this type of work in parallel on multiple \dwords{apa} in the detector cavern. Once
the \dwords{femb} are installed, a temporary set of cables will be
used to connect each \dword{femb} to a temporary power, control, and readout system
to ensure that all the wires can be properly read out. All \dwords{apa}
will then be inserted into a cold box similar
to the one used at CERN for \dword{pdsp} and tested at a temperature of
$\sim$\SI{150}{K}. Once these tests
are completed, the temporary cables are disconnected and the \dword{apa} is prepared
for shipment to \surf. It is not considered feasible to transport the
top \dword{apa} from the integration facility to the detector cryostat with the
\dword{ce} cables already installed. For the bottom \dword{apa} the final cables can be
installed only after the two \dwords{apa} are joined together in the \textit{toaster} area
just outside of the detector cryostat.

Further work on the \dword{ce} components installed on the \dwords{apa} is performed after
the \dwords{apa} are transported to the clean area outside the cryostat at
\surf. All of the cables that provide power and control and are used
to read out the \num{20} \dwords{femb} associated with the top and bottom \dwords{apa} need to be
installed; the cables that connect to the SHV boards that are used to
distribute the bias voltage to the \dword{apa} wires, electron diverters, and \dword{fc}
termination electrodes need to be installed as well. The cables for the bottom
\dword{apa} need to be routed through the frames of the \dwords{apa}, an operation that
can be performed only after the two \dwords{apa} are mechanically coupled inside
the toaster area. Quick tests are performed after the installation of the
final cables to ensure that the detector has not been damaged in the transport
and that all cable connections have been performed correctly. Only then 
can the \dwords{apa} be moved to their final positions inside
the detector cryostat. At that point, the \dword{ce} and \dword{pds} cables can be routed
through the spool piece and connected to the respective flanges and strain
reliefs. The flanges are then moved to the final position on the spool pieces
and leak tests and electrical connectivity tests can be performed. Then the
bottom plate of the crossing tubes, along with its additional strain relief for
the \dword{ce} and \dword{pds} cables, can be put in place; the final slack of the cables
are also arranged in the cable trays attached to the supports inside the cryostat.
After the cabling work is completed, the \dwords{wiec} for a pair of \dwords{apa} can be
installed on the spool piece and more testing of the entire power, control,
and readout chain can be performed: first with local control, and later
after connection of the readout and timing and control fibers to the \dwords{wib}
using the final \dword{daq} system. The installation of the next row of \dwords{apa} will
begin only when all of these tests have been completed, with a requirement
that all \dwords{femb} are properly read out from the \dword{daq}, allowing at most for
a \num{0.1}\,\% fraction of non-working channels.

A total of seven months is available in the schedule for the installation
of the \num{25} rows of \dwords{apa} and \dwords{cpa}. The current plans foresee that six \dwords{apa}
and two \dwords{cpa} are installed and
tested in one week before moving on to the next set. As soon as another
set of \dwords{apa} and \dwords{cpa} is in place, any replacement of components or rework
of connections inside the cryostat becomes very difficult or impossible.
This requires that all readout tests are performed very quickly after
each step in the installation. The schedule foresees four work days each
week for the installation work (in two eight-hour shifts). The testing
activities may require that work is performed in parallel, i.e., that
the \dword{ce} components of a pair of \dwords{apa} are tested while another pair is
being installed and/or connected to the powering, control, and readout
system. It is also likely that the testing work may require a more
extended working schedule (six or seven working days per week). More
complex tests, involving the reading out of multiple rows of \dwords{apa}, will continue
throughout the entire period (eight months) during which all the detector
components are installed inside the cryostat. Final tests should be
performed reading out the entire detector through the \dword{daq} system
prior to the closure of the \dword{tco} and before
starting filling the cryostat with argon and cooling down. There will be a hiatus
in the commissioning activities during the filling of the cryostat, during
which the conditions of the detector will continue to be monitored. The
commissioning at \lar temperature most likely will be possible only
after the cryostat is completely full of \lar.
