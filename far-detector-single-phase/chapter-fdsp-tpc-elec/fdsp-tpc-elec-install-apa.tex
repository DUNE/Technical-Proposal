The installation and commissioning of the detector components built or
purchased by the Cold Electronics consortium takes place both prior to
and after the insertion of the Anode Plane Assemblies (APAs) in the
detector cryostat, with testing performed after each step, to avoid the
need for rework that could cause significant delays. The installation and
initial commissioning of the CE electronics is likely to be on the critical
path for the completion of the single phase detector and the amount of
time available for testing and possibly repairing / replacing components
after their installation is going to be very limited. Cold tests of the complete
APAs for the entire detector require the
availability of at least two independent cold boxes at the integration facility.
To reduce the number of failures in these tests, all CE components will
be qualified for operation in LAr prior to their installation.

After the completion of the cryostat, the spool pieces that are house all the
cables for the CE, PDS, and HV consortia (with the exception of the high
voltage feedthrough for the cathode planes) are installed and leak tested.
This also includes the installation of the crossing tube cable support and of
the flanges that provide the cold to warm interface for all the cables.
In parallel the racks that house the CE and PDS electronics components on
the top of the cryostat, and the corresponding CISC and detector safety
system monitoring, controlling, and interlock hardware can be installed.
Cable trays between different spool pieces belonging to a same set of 6 APAs
and 2 CPAs can also be put in place. Readout fibre bundles between the
spool pieces and the central utility cavern used for the readout of the wire
information from the APAs can also be put in place. In the meantime inside
the cryostat all the cable trays used to support the CE and PDS cables and to
accommodate the slack of the cables can be put in place.

In parallel the 20 front-end motherboards (FEMBs) required to read out the
wires from one APA are installed with their shielding box onto the APA and
connected to the CR boards. This work will be performed at the integration
facility (or facilities), because there will not be enough space to perform
this type of work in parallel on multiple APAs in the detector cavern. Once
the FEMBs are installed a temporary set of cables will be
used to connect each FEMB to a temporary power, control, and readout system
to ensure that all the wires can be properly read-out. All APAs
will then be inserted in a cold box similar
to the one used at CERN for protoDUNE and tested at a temperature of
$\sim$150~K. Once these tests
are completed the temporary cables are disconnected and the APA is prepared
for shipment to Sanford Lab. It is not considered feasible to transport the
top APA from the integration facility to the detector cryostat with the
CE cables already installed. For the bottom APA the final cables can be
installed only after the two APAs are mated together in the ``toaster" area
just outside of the detector cryostat.

Further work on the CE components installed on the APAs is performed after
the APAs are transported to the clean area outside the cryostat inside
Sanford Lab. All the cables that provide power and control and that are used
to read out the 20 FEMBs of the top and bottom APAs need to be installed,
as well as the cables that connect to the SHV boards that are used to
distribute the bias voltage to the APA wires, to the electron diverters,
and to the field cage termination electrodes. The cables for the bottom
APA need to be routed through the frames of the APAs, an operation that
can be performed only after the two APAs are mechanically coupled inside
the ``toaster" area. Quick tests are performed after the installation of the
final cables to ensure that the detector hasn't been damaged in the transport
and that all cable connections have been performed correctly. Only then 
the APAs can be moved to their final position inside
the detector cryostat. At that point the CE and PDS cables can be routed
through the spool piece and connected to the respective flanges and their strain
reliefs. The flanges are then moved to the final position on the spool pieces
and leak tests and electrical connectivity tests can be performed. Then the
bottom plate of the crossing tubes with its additional strain relief for
the CE and PDS cables can be put in place, and the final slack of the cables
can be arranged in the cable trays attached to the supports inside the cryostat.
After the cabling work is completed, the WIECs for a pair of APAs can be
installed on the spool piece and more testing of the entire power, control,
and readout chain can be performed, first with local control, and later
after connection of the readout and timing and control fibres to the WIBs
using the final DAQ system. The installation of the next row of APAs will
begin only when all these tests have been completed, with a requirement
that all FEMBs are properly read-out from the DAQ, and allowing at most for
a fraction of non-working channels well below 0.1\%. 

A total of seven months is available in the schedule for the installation
of the 25 rows of APAs and CPAs. The current plans foresee that six APAs and two CPAs are installed and
tested in one week before moving on to the next set. As soon as another
set of APAs and CPAs is in place any replacement of components or rework
of connections inside the cryostat becomes very difficult or impossible.
This requires that all readout tests are performed very quickly after
each step in the installation. The schedule foresees four work days each
week for the installation work (in two eight hours shifts). The testing
activities may require that work is performed in parallel, i.e. that
the CE components of a pair of APAs are tested while another pair is
being installed and/or connected to the powering, control, and readout
system. It is also likely that the testing work may require a more
extended working schedule (six or seven working days per week). More
complex tests, involving readout out multiple rows of APAs, will continue
throughout the entire period (8 months) during which all the detector
components are installed inside the cryostat. Final tests should be
performed reading out the entire detector through the DAQ system
prior to the closure of the TCO and before starting filling the cryostat
with Ar and cooling down. There will be a hiatus in
the commissioning activities during the filling of the cryostat, during
which the conditions of the detector will continue to be monitored. The
commissioning at LAr temperature will be most likely possible only
after the cryostat is completely full.
