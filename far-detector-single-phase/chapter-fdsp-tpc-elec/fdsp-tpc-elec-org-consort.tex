For the moment the CE consortium does not have an official substructure
with coordinators appointed to oversee specific areas. This is in part due
to the current focus on ASIC development: informal subgroups exist that
are following the design of the various ASIC. Hucheng Chen (BNL) is leading the
design of the new version of LArASIC, and in parallel following the studies
of commercial ADCs for SBND. Carl Grace (LBNL) is leading the effort of
LBNL, BNL, and FNAL, on the design of a new ADC in 65 nm technology. Terri
Shaw (FNAL) is overseeing the development of the new COLDATA ASIC, and
Angelo Dragone (SLAC) is overseeing the adaptation of the nEXO CRYO ASIC
for use in DUNE. A new working group is tasked with studying reliability
issues in the CE components and preparing recommendations for the choice
of ASICs, the design of printed circuit boards, and testing. This working group
will consider past experience from cryogenic detectors operated for a long
time (ATLAS Liquid Argon calorimeters, NA48 Liquid Krypton calorimeter, HELIOS),
from space based experiments (FERMI/GLAST), and the lessons learnt from
the ProtoDUNE construction and commissioning. Input from other fields will
also be sought. Later this working group will develop the QC program for the CE detector
components, starting from the protoDUNE experience. It is planned to
reassess the structure of the group in a few months, with a likely split
between components inside and outside the cryostat, a new group responsible
for testing, and various contact people for calibration, physics, software and
computing, and integration and installation.
 
The main decision that the consortium has to face in the next 12-18 months
is the choice of ASIC(s) to be used in the DUNE FEMBs. A first decision will
be taken early in Summer 2018, when a decision will be taken on whether system
tests, beyond those planned by the SBND Collaboration, should be performed for 
commercial ADC chips and for the ATLAS ADC. In February 2019, following tests
performed with a ProtoDUNE APA in the cold box at CERN and with a small TPC in
LAr at Fermilab, a list of options to be prepared for presentation in the TDR.
Only ASICs that satisfy the DUNE performance and reliability requirements
will be included in this list of options. A final choice for the ASIC(s) to
be used in DUNE should be taken in Summer 2019, prior to the DOE CD-2/CD-3b
review. Physics performance, reliability, and power constraints considerations
will be taken into account when formulating this choice, which will go through
the Executve Board approval procedure.
