For the moment the CE consortium does not have an official substructure
with coordinators appointed to oversee specific areas. This is in part due
to the current focus on ASIC development: informal subgroups exist that
are following the design of the various ASIC. Hucheng Chen (BNL) is leading the
design of the new version of LArASIC, and in parallel following the studies
of commercial ADCs for SBND. Carl Grace (LBNL) is leading the effort of
LBNL, BNL, and FNAL, on the design of a new ADC in 65 nm technology. Terri
Shaw (FNAL) is overseeing the development of the new COLDATA ASIC, and
Angelo Dragone (SLAC) is overseeing the adaptation of the nEXO CRYO ASIC
for use in DUNE. A new working group tasked with studying reliability
issues in the CE components and preparing recommendations for the choice
of ASICs, the design of printed circuit boards, and testing. This working
group could form the basis for the group tasked with developing the QA/QC
program for the CE detector components and then monitoring it. We plan to
reassess the structure of the group in a few months, with a likely split
between components inside and outside the cryostat, a group responsible
for testing, and contact people for calibration, physics, software and
computing, and integration and installation.
