For the moment the \dword{ce} consortium does not have a formal substructure
with coordinators appointed to oversee specific areas. This is in part due
to the current focus on \dword{asic} development; informal subgroups exist that
are following the design of the various \dwords{asic}. 
%Hucheng Chen (BNL) 
A BNL collaborator is currently leading the
design of the new version of \dword{larasic}, and in parallel following the studies
of commercial \dwords{adc} for SBND, which is using the same \dword{asic}.  
%Carl Grace (LBNL) is leading the effort of
Collaborators from LBNL, BNL, and \fnal are working on the design of a new \dword{adc} \dword{asic} with 65 nm technology. 
Collaborators from \fnal and SLAC, respectively, %Terri Shaw (\fnal) is 
are overseeing the development of the new \dword{coldata} \dword{asic} and 
%Angelo Dragone (SLAC) is overseeing 
the adaptation of the nEXO CRYO \dword{asic} 
for use in DUNE. 
A new working group is tasked with studying reliability 
issues in the \dword{ce} components and preparing recommendations for the choice
of \dwords{asic}, the design of printed circuit boards, and testing. This working group
will consider past experience from cryogenic detectors operated for a long
time (ATLAS \lar calorimeters, NA48 liquid krypton calorimeter, HELIOS),
from space-based experiments (FERMI/GLAST), and the lessons learned from
\dword{pdsp} construction and commissioning. Input from other fields will
also be sought. Later this working group will develop the \dword{qc} program for the \dword{ce} detector
components, starting from the \dword{pdsp} experience. It is planned to
reassess the structure of the group in a few months, with a likely split
between components inside and outside the cryostat, a new group responsible
for testing, and various contact people for calibration, physics, software and
computing, and integration and installation.
 
The main decision that the consortium has to face in the next \numrange{12}{18} months
is the choice of \dwords{asic} to be used in the DUNE \dwords{femb}. A first decision will
be taken early in Summer 2018, when it will be determined whether or not system
tests, beyond those planned by the SBND collaboration, should be performed for 
commercial \dword{adc} chips and for the ATLAS \dword{adc}. In February 2019, following tests
performed with a \dword{pdsp} \dword{apa} in the cold box at CERN and with a small TPC in
\lar at \fnal, a list of options will be prepared for presentation in the \dword{tdr}.
Only \dwords{asic} that satisfy the DUNE performance and reliability requirements
will be included in this list of options. A final choice for the \dwords{asic} to
be used in DUNE should be taken in summer 2019, prior to the DOE CD-2/CD-3b
review. Physics performance, reliability, and power constraint considerations
will be taken into account when making this choice, which will go through
the Executve Board approval procedure.
