The COLDATA ASIC is responsible for all communication between the cold TPC electronics on FEMBs and electronics located outside the cryostat.  Each FEMB contains two COLDATA ASICs. COLDATA receives command and control information.  It provides clocks to the Cold ADC ASICs and relays commands to the LArASIC front-end and to the Cold ADC ASICs to set operating modes and initiate calibration procedures.  COLDATA receives data from the ADC ASICs, reformats these data, merges data streams, formats data packets, and sends these data packets to the warm electronics using 1.28 Gbps links.  These links are designed for use with 30 m long cables and include line drivers with pulse preemphasis.  A block diagram of COLDATA is shown in Figure~\ref{fig:coldata}.  

\begin{dunefigure}
[Block diagram of COLDATA ASIC design.]
{fig:coldata}
{Block diagram of COLDATA ASIC design.}
\includegraphics[width=0.9\linewidth]{tpcelec-COLDATABlockDiagram.png}
\end{dunefigure}

COLDATA is designed in 65 nm CMOS using ``cold'' transistor models based on data collected by members of the FNAL, BNL, and SMU ASIC groups.  A special library of standard cells, based on these models and using a minimum channel length of 90 nm, was developed by members of the U. Pennsylvania and FNAL groups.  This library was designed to eliminate the risk posed by the hot carrier effect.  The digital sections of COLDATA use these standard cells and were synthesized from RTL using automatic place and route tools.  Key circuit elements of COLDATA, including the control interface and the PLL and serializer, were prototyped successfully in 2017.  Submission of a full chip is expected in June 2018. 
