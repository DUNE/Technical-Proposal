The DUNE single-phase TPC readout electronics are often referred to as the ``Cold Electronics'' (CE) given that they reside in the liquid argon, mounted directly on the APA.  The charge carrier mobility in silicon is higher and thermal fluctuations are lower at liquid argon temperature than at room temperature.  For CMOS electronics, this results in substantially higher gain and lower noise at liquid argon temperature than at room temperature~\cite{LArCMOS}.  Mounting the front-end electronics on the APA frames also minimizes the input capacitance.  Furthermore, placing the digitizing and multiplexing (MUX) electronics inside of the cryostat allows for a reduction in the total number of penetrations into the cryostat and minimizes the number of cables coming out of the cryostat, reducing the expense and complexity of the experiment.  As the full TPC electronics chain for the single-phase detector includes many components on the warm side of the cryostat as well, the DUNE consortium designated to organize CE development is referred to as the DUNE ``Single-Phase TPC Electronics'' consortium.

The lower noise levels (by about a factor of two) enabled by having the front-end electronics in the cold greatly extends the reach of the DUNE physics program~\cite{LArCMOS}.  The CP violation and neutrino mass ordering measurements depend on a precise characterization of the reconstructed neutrino energy spectrum; improving the charge resolution as much as possible (by lowering noise levels in the wire readout) allows for one to better resolve features in reconstructed neutrino energy spectrum that are relevant for these physics measurements.  Decreasing the noise level also allows for smaller charge deposits to be measurable, which acts as a source of risk mitigation in the case that the electron lifetime in the detector is lower than desired (due to the prevelance of electronegative impurities in the detector), and also increases the reach of low-energy physics measurements such as those associated with stellar core-collapse supernova burst neutrinos.  Finally, the low noise levels allows the experiment to utilize low-energy $\mathrm{{}^{39}Ar}$ beta decays for the purpose of precision calibration in the DUNE far detector.

An extensive discussion of the TPC electronics system design is given in Section~\ref{sec:fdsp-tpc-elec-design}.  What immediately follows is a brief overview of the major design considerations for the cold electronics to be used in the DUNE single-phase far detector.
