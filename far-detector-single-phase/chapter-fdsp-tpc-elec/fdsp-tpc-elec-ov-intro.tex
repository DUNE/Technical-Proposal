The DUNE single-phase TPC readout electronics are often referred to as the ``Cold Electronics'' (CE) given that they reside in the liquid argon, mounted directly on the APA.  The charge carrier mobility in silicon is higher and thermal fluctuations are lower at liquid argon temperature than at room temperature.  For CMOS electronics, this results in substantially higher gain and lower noise at liquid argon temperature than at room temperature~\cite{LArCMOS}.  Mounting the front-end electronics on the APA frames also minimizes the input capacitance.  Furthermore, placing the digitizing and multiplexing (MUX) electronics inside of the cryostat allows for a reduction in the total number of penetrations into the cryostat and minimizes the number of cables coming out of the cryostat, reducing costs.  As the full TPC electronics chain for the single-phase detector includes many components on the warm side of the cryostat as well, the DUNE consortium designated to organize CE development is referred to as the DUNE ``Single-Phase TPC Electronics'' consortium.

One requirement drives the choice of architecture of the TPC electronics and that is the noise requirement. This requirement is difficult to establish exactly, but it is clear that the lower the electronic noise is, the greater the physics reach of the DUNE experiment will be.  Given the TPC design, a noise of less than approximately 1000$e^-$ is required for satisfactory reconstruction of accelerator neutrino interactions, but significantly lower noise than this will yield significantly better two-track separation and primary vertex resolution, and thus higher efficiency and/or lower background for identifying electron neutrino interactions.

The lower noise levels enabled by having the front-end electronics in the cold (roughly half as much noise at liquid argon temperature than at room temperature) greatly extends the reach of the DUNE physics program.  Decreasing the noise level allows for smaller charge deposits to be measurable, which acts as a source of risk mitigation in the case that the electron lifetime in the detector is lower than desired (due to the prevelance of electronegative impurities in the detector), and also increases the reach of low-energy physics measurements such as those associated with stellar core-collapse supernova burst neutrinos.  Finally, the low noise levels allows the experiment to utilize low-energy $\mathrm{{}^{39}Ar}$ beta decays for the purpose of precision calibration in the DUNE far detector.

In order to retain maximum flexibility to optimize reconstruction algorithms after the DUNE data is collected, the DUNE TPC electronics are designed to produce a digital record that is a representation of the waveform of the current induced in the anode wires.  Each anode wire signal is input to a charge sensitive amplifier, followed by a pulse shaping circuit and an analog to digital convertor.  The pulse shaping time and the sampling frequency of the ADC are chosen to retain two-track information contained in the induced current signals.  In order to minimize the number of cables and cryostat penetrations, the ADCs as well as the amplifier/shapers are located in the liquid argon, and digitized data from many wires are merged onto a much smaller set of high speed serial links.  %[Figure xx illustrates the front-end electronics architecture.  128-channel “Front-End Mother Boards” produce digitized waveforms and transmit those waveforms on four serial links through a feedthrough in the top of the cryostat to “Warm Interface Boards” located in crates mounted directly on spool pieces on the outside of the cryostat. The Warm Interface Boards provide clean power and timing signals to the cold electronics.  All connections to the DAQ and slow control systems are made using optical fibers.] – might better cover later.

%INCLUDE AN OVERVIEW FIGURE HERE WITH THIS CAPTION:  ``Each Anode Plane Assembly (APA) has 2560 instrumented wires.  These are read out by 20 Front-End MotherBoards (FEMBs).  All cables between FEMBs on a given APA are routed through a single cryostat feedthrough.  A printed circuit board connects those cables to the backplane of a Warm Electronics Interface Crate.  Warm Interface Boards mounted in this crate receive data from the FEMBs and transmit it to the data acquisition system.''
