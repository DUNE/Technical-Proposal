DUNE \single TPC electronics hardware signal processing takes place inside the \lar, in boards that are directly mounted on the \dword{apa}; accordingly, the TPC readout electronics are referred to as the \dword{ce}.  The electronics are mounted inside the \lar to exploit the fact that charge carrier mobility in silicon is higher and that thermal fluctuations are lower at \lar temperature than at room temperature.  For \dword{cmos} (complementary metal-oxide-semiconductor) electronics, this results in substantially higher gain and lower noise at \lar temperature than at room temperature~\cite{larCMOS}.  Mounting the front-end electronics on the \dword{apa} frames also minimizes the input capacitance.  Furthermore, placing the digitizing and multiplexing (MUX) electronics inside of the cryostat reduces the total number of penetrations into the cryostat and minimizes the number of cables coming out of the cryostat.  As the full TPC electronics chain for the \dword{spmod} includes many components on the warm side of the cryostat as well, the DUNE consortium designated to organize \dword{ce} development is referred to as the DUNE \textit{Single-Phase TPC Electronics} consortium.
\fixme{can we add (CE) here? Anne}
The overall noise requirement drives the choice of architecture of the TPC electronics. This requirement is difficult to establish precisely, but it is clear that the lower the electronic noise is, the greater the physics reach of the DUNE experiment will be.  An equivalent noise charge (\dword{enc}) of less than approximately 1000$e^-$ is required for satisfactory reconstruction of accelerator neutrino interactions, but a lower noise level will yield significantly better two-track separation and primary vertex resolution, and thus higher efficiency and/or lower background for identifying electron neutrino interactions.  Setting the noise level requirement for the DUNE \dword{spmod} more precisely is an ongoing effort.

The noise level enabled by having the front-end electronics in the cold (roughly half as much noise at \lar temperature than at room temperature) greatly extends the reach of the DUNE physics program.  Decreasing the noise level allows for smaller charge deposits to be measurable, which acts as a source of risk mitigation in the case that the desired drift field can not be reached or the electron lifetime in the detector is lower than desired (due to the electronegative impurities in the detector), and also increases the reach of low-energy physics measurements such as those associated with stellar core-collapse supernova burst neutrinos.  Finally, the low noise level allows the experiment to utilize low-energy $\mathrm{{}^{39}Ar}$ beta decays for the purpose of calibration in the DUNE \dword{spmod}.  The noise level requirement of \dword{enc}\,$<\num{1000}\,e^-$ will allow for the use of $\mathrm{{}^{39}Ar}$ beta decays in calibrations at the DUNE \dword{spmod}.

In order to retain maximum flexibility to optimize reconstruction algorithms after the DUNE data is collected, the \dword{spmod} electronics are designed to produce a digital record that is a representation of the waveform of the current produced by charge collection/induction on the anode wires.  Each anode wire signal is input to a charge sensitive amplifier, followed by a pulse shaping circuit and an analog-to-digital converter (\dword{adc}).  In order to minimize the number of cables and cryostat penetrations, the \dwords{adc} as well as the amplifier/shapers are located in the \lar, and digitized data from many wires are merged onto a much smaller set of high speed serial links.  %[Figure xx illustrates the front-end electronics architecture.  128-channel “Front-End Mother Boards” produce digitized waveforms and transmit those waveforms on four serial links through a \fdth in the top of the cryostat to “Warm Interface Boards” located in crates mounted directly on spool pieces on the outside of the cryostat. The Warm Interface Boards provide clean power and timing signals to the \dword{ce}.  All connections to the DAQ and slow control systems are made using optical fibers.] – might better cover later.

%INCLUDE AN OVERVIEW FIGURE HERE WITH THIS CAPTION:  ``Each Anode Plane Assembly (\dword{apa}) has 2560 instrumented wires.  These are read out by 20 Front-End MotherBoards (\dwords{femb}).  All cables between \dwords{femb} on a given \dword{apa} are routed through a single cryostat \fdth.  A printed circuit board connects those cables to the backplane of a Warm Electronics Interface Crate.  Warm Interface Boards mounted in this crate receive data from the \dwords{femb} and transmit it to the data acquisition system.''
