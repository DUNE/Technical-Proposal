The CE signal processing is implemented in application-specific integrated circuit (ASIC) chips
using CMOS technology, which has been
demonstrated to perform well at cryogenic temperatures, and includes amplification, shaping,
digitization, buffering, and multiplexing (MUX) of the signals.  The CE is continuously read out,
resulting in a digitized ADC sample from each APA channel (wire) up to every 500 ns (2~MHz sampling
rate).

Each individual APA has 2,560 channels that are read out by 20 Front-End Motherboards (FEMBs), with
each FEMB providing digitized wire readout from 128 channels.  One cable bundle connects each FEMB to
the outside of the cryostat via a feedthrough (a CE feedthrough) in the signal cable flange at the
top of the cryostat, where a single flange services each APA, as shown in Figure~\ref{fig:connections}.
Each cable bundle contains wires for low-voltage (LV) power, high-speed data readout, and
clock/digital-control signal distribution.  Eight separate cables carry the TPC wire-bias voltages
from the signal flange to the APA wire-bias boards.

\begin{dunefigure}
[Connections between the signal flange and APA at ProtoDUNE-SP.  The interface between the electronics chain, the APAs, and the cryostat will be very similar at the DUNE far detector.]
{fig:connections}
{Connections between the signal flange and APA at ProtoDUNE-SP.  The interface between the electronics chain, the APAs, and the cryostat will be very similar at the DUNE far detector.}
\includegraphics[width=0.35\textwidth]{tpcelec-PDSP_Connections.png}
\end{dunefigure}

The components of the CE system are the following:
\begin{itemize}
\item{front-end mother boards, which house the cold ASICs and are installed on the
APAs;}
\item{cables for the data, clock/control signals, LV power, and wire-bias voltages between the APA and the signal flanges (cold cables);}
\item{signal flanges with a CE feedthrough to pass the data, clock/control signals, LV power, and APA wire-bias voltages between the inside and outside of the cryostat;}
\item{warm interface electronics crates (WIECs) that are mounted on the signal flanges and contain
the warm interface boards (WIBs) and power and timing cards (PTCs) for further processing
and distribution of the signals entering/exiting the cryostat;}
\item{fiber cables for transmitting data and clock/control signals between the WIECs and the
data acquisition (DAQ) and slow control systems;}
\item{cables for LV power and wire-bias voltages between the signal flange and external power
supplies (warm cables); and}
\item{LV power supplies for the CE and bias-voltage power supplies for the APAs.}
\end{itemize}

The electrical cables for each APA enter the cryostat through a single signal flange, creating
an integrated unit that provides local diagnostics for noise and validation testing, and follows
the grounding guidelines in Section~\ref{sec:fdsp-tpc-elec-design-ground}. The components, the quantity of each required for the DUNE far detector, and the number of channels that each component has, are listed in Table~\ref{tab:elecNums}.

\begin{dunetable}
[TPC electronics components and quantities for the DUNE single-phase far detector.]
{llr}
{tab:elecNums}
{TPC electronics components and quantities for the DUNE single-phase far detector.}
\textbf{Element} &\textbf{Quantity} & \textbf{Channels per element}\\ \toprowrule
APA & 150 & 2,560 \\ \colhline
Front-end mother board (FEMB) & 20 per APA & 128 \\ \colhline
FE ASIC chip & 8 per FEMB & 16 \\ \colhline
ADC ASIC chip & 8 per FEMB & 16 \\ \colhline
COLDATA ASIC chip & 2 per FEMB & 64 \\ \colhline
Cold cable bundle & 1 per FEMB & 128 \\ \colhline
Signal flange & 1 per APA & 2,560 \\ \colhline
CE feedthrough & 1 per APA & 2,560 \\ \colhline
Warm interface board (WIB) & 5 per APA & 512 \\ \colhline
Warm interface electronics crate (WIEC) & 1 per APA & 2,560 \\ \colhline
Power and timing card (PTC) & 1 per APA & 2,560 \\ \colhline
Passive Backplane (PTB) & 1 per APA & 2,560 \\
%LV power mainframe & ? & ? \\ \colhline
%LV supply module & ? & ? \\ \colhline
%Wire-bias mini-crate & ? & ? \\ \colhline
%Wire-bias supply module & ? & ? \\
\end{dunetable}

The baseline design for the DUNE far detector single-phase TPC electronics calls for three types of ASICs to be located inside of the liquid argon:
\begin{itemize}
\item{a 16-channel front-end (FE) ASIC including amplification and pulse shaping;}
\item{a 16-channel 12-bit ADC ASIC operating at 2~MHz; and}
\item{a 64-channel control and communications ASIC.}
\end{itemize}
The front-end ASIC has been prototyped and is close to meeting requirements (discussed in Section~\ref{sec:fdsp-tpc-elec-ov-req}).  Key portions of the control and communications ASIC (also referred to as the COLDATA ASIC) have been prototyped and meet requirements.  However, it has been determined that the ADC ASIC now being used in ProtoDUNE-SP does not meet requirements, and accordingly, its development has been terminated.  A new cold ADC ASIC is being developed by an LBL-FNAL-BNL collaboration and first prototypes are expected by the end of summer 2018.  The first full prototype of the controls and communication ASIC is also expected to be available for testing by the end of the summer 2018.  In order to maximize the probability of developing a complete design for cold TPC front-end electronics in a timely fashion, an alternative solution is also being investigated, a single 64-channel ASIC that will include all three functions described above.  This design is being done at SLAC and first prototypes are expected late in spring or early summer 2018.

A series of tests are planned to demonstrate that at least one of these designs will meet DUNE requirements. These include two system tests: one using the ProtoDUNE ``cold box'' at CERN, and one using a new small liquid argon TPC at Fermilab. The latter will also accommodate one half-length DUNE photodetector, and will provide a low noise environment that will allow one to make detailed comparisons of the performance of the new ASICs. It will also enable the study of interactions between the TPC readout and other systems, including the photodetector readout and the HV distribution.  These test facilities are discussed in more detail in Section~\ref{sec:fdsp-tpc-elec-qa-facilities}.
