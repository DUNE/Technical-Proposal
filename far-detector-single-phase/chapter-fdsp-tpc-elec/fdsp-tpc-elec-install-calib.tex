Directly following the installation of the instrumented \dwords{apa} into the cryostat, commissioning of the
TPC electronics will commence, and will be carried out both before and after the \lar fill;
the former establishes whether or not the installation procedure led to impairment of the
electronics, and the latter checks that the cryogenic electronics do not experience failures
in the cold \lar.

As described in Section~\ref{sec:fdsp-tpc-elec-install-apa}, the electronics checkout prior to
\lar fill will begin as soon as the first \dword{apa} is installed; %, and before all \dwords{apa} are installed; 
that is,
installation and testing will proceed in parallel.  This has the advantage of informing the 
 installation of subsequent \dwords{apa}, and thus minimizing the total number of electronics channels lost
during the installation process.  Items to be checked during the commissioning process include
the noise level on every channel, dead or noisy (high \rms) channels,
cross-talk across neighboring wires or neighboring channels in the electronics,
pick-up noise (including spatial dependence in the detector) if present, and
noise coherent across channels sharing common electronics (e.g.,~the same \dword{femb}).
These measurements will be performed first during the tests in the cold box.  Repeating these
measurements prior to filling the cryostat with \lar will allow for final repairs or replacements
for all the \dword{ce} components prior to the closure of the \dword{tco}. This will also give an opportunity
to intervene on noise or electronics issues that would become evident when reading out
the entire detector at once, before proceeding proceeding with the \lar fill.

Once every \dword{apa} is installed and tested as described above, the cool-down process and \lar fill
will be carried out.  Monitoring of noise levels and dead/noisy channel count will be
done continuously during this process.  After the \lar fill, another electronics checkout
similar to the one described above will be carried out.  Any electronics issues identified
during this procedure will be fixed before continuing, if possible.  The wire bias \dword{hv} and
cathode \dword{hv} will then be brought up, with noise studies repeated after each subsystem is
turned on.  With the wire bias \dword{hv} and cathode \dword{hv} up, one can then utilize ionization signals
(e.g.,~from $\mathrm{{}^{39}Ar}$ beta decays) to distinguish between different possible issues
that might impair the readout of a given channel, such as a short between the wire planes
(which would alter the wire field response on nearby wires) and a problem with the electronics.

Both during the commissioning phase of the experiment and during normal operations, it may
be desired to perform an in situ calibration of one or more parts of the TPC electronics chain.
These calibrations will utilize a combination of noise data and data collected while a
calibration pulser (internal DAC on the FE \dword{asic}) is periodically injecting charge into the TPC
electronics channels.  Calibrations of interest include determining the gain and shaping time of
every electronics channel in the TPC (as the FE \dwords{asic} may experience changes in these quantities
in the cold) and characterizing the linearity of the \dword{adc} \dwords{asic} in the cold.  With the updated \dword{adc} \dword{asic}
design, it is not expected that nonlinearity of the \dwords{adc} will be a significant issue (see
Section~\ref{sec:fdsp-tpc-elec-design-femb-adc} for a discussion of the on-chip calibration
that is used), but it is important to verify this with data collected in the experiment.
Specific in situ calibration algorithms are being studied at \dword{pdsp}, and will be further
developed there and in the other electronics test facilities described in
Section~\ref{sec:fdsp-tpc-elec-qa-facilities}.  Experience from \microboone and other running
\lartpc experiments will be very useful in informing TPC electronics calibration procedures for
the DUNE \dwords{spmod}.
