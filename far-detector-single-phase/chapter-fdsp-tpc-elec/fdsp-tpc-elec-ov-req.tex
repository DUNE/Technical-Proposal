In addition to the noise requirement (less than 1000$e^{-}$), several additional requirements determine most of the other important TPC electronics specifications.  These are:

\begin{itemize}
\item{The front-end peaking time should be 1-3~\si{\micro\second}.  This requirement is derived primarily from the time required for drifting charges to travel from one plane of anode wires to the next.}
\item{The front-end should have an adjustable baseline.  This requirement reflects the fact that the signal from induction wires is bipolar while the signal from the collection wires is mostly unipolar.}
\item{The ADC sampling frequency should be 2~MHz.  This value is chosen to match a front-end shaping time of 1~\si{\micro\second} (approximate Nyquist condition) while minimizing the data rate.}
\item{The system must have a linear response up to an impulse input of at least 530,000$e^{-}$.  This roughly corresponds to the charge collected on a single wire from one stopping proton and two more highly ionizing protons, all assumed to have trajectories at 45$^{\circ}$ with respect to the beam axis.  This number was chosen so that saturation will occur in less than 5-10\% of beam related events.  Studies are ongoing (including an evaluation of LArIAT data and simulation studies) to better understand this requirement.}
\item{The dynamic range of the system must be at least 3000:1. This number is given by the ratio between (the maximum signal for no saturation) and $\sim$50\% of the lowest possible noise level.  It implies a 12-bit ADC.}
\item{The effective noise that the ADC adds to the front-end noise must be insignificant.  This requirement is dependent on the gain of the front-end, but for each gain setting translates into requirements on ADC parameters including non-linearity and noise.}
\item{The power dissipated by the electronics located in the liquid argon must be less than 50 mW/channel.  Lower power dissipation is desirable because the mass of the power cables scales with the power. Studies are ongoing to understand if the amount of power dissipated by the electronics should be minimized further due to potential complications from argon boiling; in principle this should not be a problem because the CE boxes housing the FEMBs are designed to channel bubbles to the APA frames.}
\end{itemize}

Finally, all electronics located in the liquid argon must be highly reliable because it will not be possible to access the cold electronics for repair once the cryostat is filled with liquid argon. Studies are ongoing to quantify the impact of failures in the TPC and electronics, including single wire failures, and failures of groups of 16, 64, or 128 channels.

% A number of risks associated with the cold electronics (or their integration with another subsystem) are detailed below:

%\begin{itemize}
%\item{failures in the TPC and electronics, including single wire failures and failures of groups of 16, 64, or 128 channels;}
%\item{electrical failures associated with having two APAs vertically chained together;}
%\item{problems in the final ADC ASIC design that are not able to be resolved in time for installation at the far detector;}
%\item{issues associated with having cables longer than those that will be tested at ProtoDUNE-SP;}
%\item{potential ground loops in the far detector;}
%\item{difficulty keeping the electronics power level low enough as to avoid significant boiling of the argon; and}
%\item{bubbles associated with any boiling of the argon leading to electrical breakdown within the detector, which may damage the electronics.}
%\end{itemize}

%Studies associated with better understanding these risks and means to minimize them are ongoing and will be discussed in a future document.

%%%PREVIOUS
%
%The core components of the TPC electronics subsystem are the FEMB ASICs that perform shaping, digitization, and multiplexing of channel data in the cold.  Two basic requirements for the ASIC designs are:
%\begin{itemize}
%\item{negligible risk of failure due to the hot carrier effect (less than 0.7\% channel failure per APA plane in 30 years of operation); and}
%\item{a total power consumption of less than 50~mW/channel.}
%\end{itemize}
%If multiple ASIC designs meet these two requirements, the down-select decision will be based on performance, power consumption, estimated reliability, and estimated cost.  Of these factors, performance will be given the largest weight.
%
%For the FE ASICs, more specific requirements include:
%\begin{itemize}
%\item{less than 1000e$^-$ equivalent noise charge (ENC) in LAr for the induction channels;}
%\item{baseline recovery for large pulses;}
%\item{crosstalk between neighboring channels below the 2\% level;}
%\item{crosstalk between non-neighboring channels well below the 1\% level;}
%\item{a nonlinearity below the 0.5\% level or precisely characterized; and}
%\item{channel-to-channel gain uniformity that is either negligible or precisely characterized.}
%\end{itemize}
%A secondary performance measurement is dynamic range.  If noise or linearity depends on the gain setting, then most weight (90\%) will be given to the setting with best performance for which an impulse of 85~fC at the input does not saturate the output.
%
%Additional ADC ASIC requirements are:
%\begin{itemize}
%\item{an associated noise level that is less than 0.5 least significant bits (LSB);}
%\item{a usable dynamic range of at least 3000:1;}
%\item{differential nolinearity (DNL) that is less than 1 LSB; and}
%\item{integrated nonlinearity (INL) that is either less than 1 LSB or consistent from channel-to-channel and chip-to-chip.}
%\end{itemize}
%
%Finally, specific requirements of the COLDATA ASIC design include:
%\begin{itemize}
%\item{fully functional at both room temperature and liquid argon temperature; and}
%\item{both control and data links must operate with negligible error rate over cables up to 21~m in length.}
%\end{itemize}
%
%In addition to these requirements on the ASICs, the TPC electronics subsystem must also:
%\begin{itemize}
%\item{provide the means to read out the TPC wires and transmit their data in a useful format to the DAQ;}
%\item{operate for the life of the facility without significant loss of function;}
%\item{record the channel waveforms continuously without dead time;}
%\item{be constructed only from materials that are compatible with high-purity LAr;}
%\item{provide sufficient precision and range in the digitization to discriminate electrons from photon conversions, optimize the reconstruction of high-energy and low-energy tracks from accelerator-neutrino interactions, distinguish a minimum-ionizing particle (MIP) from noise with a signal-to-noise ratio of 9:1 or greater, and measure ionization up to 15 times that of a MIP so that stopping kaons from proton decay can be identified;}
%\item{ensure that all poewr supplies have local monitoring and control, remote monitoring and control through the DAQ, and over-current and over-voltage protection circuits; and}
%\item{ensure that the CE feedthroughs are able to withstand twice their nominal operating voltages with a maximum specified leakage current in 1-atm argon gas.}
%\end{itemize}
%
%The quality assurance (QA) process, described in Section~\ref{sec:fdsp-tpc-elec-qa}, will make use of both initial validation procedures using electronics test stands as well as integrated test facilities in order to ensure that final component designs satisfy the above requirements.  Once component designs have been decided upon, the quality control (QC) process will employ similar requirements to make sure that every component to be installed in the DUNE single-phase far detector is up to standard.  This is described in more detail in Section~\ref{sec:fdsp-tpc-elec-qc}.
