ProtoDUNE-SP was intended to be a full slice of the DUNE far detector as near as possible to the final DUNE Single Phase design. It contains 6 full-size APAs instrumented with 20 FEMBs each for a total readout channel count of 15,360 digitized sense wires. Critically, the CE on each APA is read out via a full CE read out system, via a CE flange and WIEC with 5 WIBs and 1 PTC. Each APA also has a full Photon Detector readout system installed. Five of the 6 ProtoDUNE APAs have been validated in the cold box at CERN, and then installed in the ProtoDUNE-SP cryostat, while the last APA was installed after passing only room temperature tests. Any issues that are discovered either during the cold box tests or the ProtoDUNE-SP commissioning and data-taking will be incorporated into the next iteration of the system design for DUNE.

In addition to the tests described in Section~\ref{sec:fdsp-tpc-elec-qa-facilities-coldbox}, tests have also been done on the ProtoDUNE-SP APA to check for any additional noise introduced on the TPC wire readout by operating the PD system or enabling the wire-bias HV system. So far, no significant increase in the noise on the APA wire readout has been observed when operating these other systems.

The DUNE APAs and the readout electronics will be different from the ones used in ProtoDUNE and for this reason plans are being made for re-opening the ProtoDUNE-SP cryostat, replacing three of the six APAs with final DUNE prototypes that will also include final version of the ASICs and of the FEMBs. A second period of data taking with this new configuration of ProtoDUNE is being planned for 2021/2022. This will also allow to check again that there will not be any interference between the readout of the APA wires and the photon detection system.
