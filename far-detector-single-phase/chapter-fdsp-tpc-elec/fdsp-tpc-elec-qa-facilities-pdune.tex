\dword{pdsp} was intended to be a full slice of the \dword{spmod} as close as possible to the final DUNE \single design. It contains six full-size \dwords{apa} instrumented with \num{20} \dwords{femb} each for a total readout channel count of \num{15360} digitized sense wires. Critically, the \dword{ce} on each \dword{apa} is read out via a full \dword{ce} readout system, including a \dword{ce} flange and \dword{wiec} with five \dwords{wib} and one \dword{ptc}. Each \dword{apa} also has a full \dword{pd} readout system installed. Five of the six \dword{pdsp} \dwords{apa} have been validated in the cold box at CERN, and then installed in the \dword{pdsp} cryostat, while the last \dword{apa} was installed after passing only room temperature tests. Any issues that are discovered either during the cold box tests or the \dword{pdsp} commissioning and data-taking will be incorporated into the next iteration of the system design for the \dword{spmod}.

In addition to the tests described in Section~\ref{sec:fdsp-tpc-elec-qa-facilities-coldbox}, tests have also been done on the \dword{pdsp} \dword{apa} to check for any additional noise introduced on the TPC wire readout by operating the \dword{pds} or enabling the wire bias \dword{hv} system. So far, no significant increase in the noise on the \dword{apa} wire readout has been observed when operating these other systems.

The \dwords{apa} and the readout electronics will be different from the ones used in \dword{pdsp}; for this reason, plans are being made for re-opening the \dword{pdsp} cryostat and replacing three of the six \dwords{apa} with final DUNE prototypes that will also include the final versions of the \dwords{asic} and \dwords{femb}. A second period of data-taking with this new configuration of \dword{pdsp} is being planned for 2021-2022. This will also allow for another opportunity to check for interference between the readout of the \dword{apa} wires and the \dword{pds}.
