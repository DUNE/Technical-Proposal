The TPC electronics will be built and handled in such a way as to ensure the safety of both personnel and equipment.  The team will work closely with the project Technical Coordination organization to make sure that all applicable safety procedures are followed and documented.

The instrumentation of the TPC electronics will include multiple printed circuit boards and cabling.  The cabling includes the high-voltage wire bias distribution, low-voltage power and signals.  Each of these elements will require attention to relevant safety standards and solutions will be subject to the review of the project Technical Coordination organization.

All printed circuit boards will be designed such that the connectors and copper-carrying traces are rated to sustain the maximum current load.  In the case of the low-voltage warm electronics, all boards will be fused following prescribed safety standards.  The cold electronic low-voltage boards inside the cryostat will not be fused because these boards will be inaccessible during operations and fire is not a danger once the cryostat is filled with liquid argon.  Special precautions should be taken during installation and commissioning of the cold electronics prior to the cryostat being filled with liquid argon.  The TPC electronics group will work with the project Technical Coordination to implement this.

All cabling and connectors will be selected such that they meet or exceed the possible current ampacity and voltage ratings of the connected power supplies.  In the case of high-voltage wire bias distribution, all accessible warm connectors will be SHV type connectors which will limit the possibility of a touch potential which could shock a person.  In the cold, many of the high-voltage connections will be open soldered connections. Care will be taken that ensures personnel safety should these connections need to be energized while the APA is exposed.  However, it is not anticipated that APA wire bias will be powered by more than 50~V unless the APA is enclosed within a Faraday shield.

Finally, the safety of the equipment must be taken into account during production, initial checkout, installation, and cabling.  Proper electrostatic discharge (ESD) procedures will be followed at all stages, including use of ESD safe bags for storage and ESD wrist straps used by personnel when handling the cards.  The TPC electronics will also make use of shorting connectors on all cables which are attached to the printed circuit cards, but not attached at the far end.  During installation, multiple long cables will be attached to front-end boards, but will not be attached to the connectors on the flanges for some period of time.  Detailed procedures for the use of cable-shorting connectors will be written and used.

Finally, the handling of the APA and attached front-end electronics must follow ESD safe handling procedures whenever the APA is moved from one ground reference to another, or after it has been left in a ``floating'' state for any period of time.  Whenever the APA is moved and could encounter a step potential, a connection must be made through a slow discharge path which will equalize the APA frame potential to the new environment.  Again, ESD safe-handling rules will be documented and followed.
