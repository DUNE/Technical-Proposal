There are two components in the definition of the interface between the CE
and DAQ consortia. The first one is a decision on whether to implement any
firmware and buffering related to the trigger decision inside the WIB, or
instead transmit the data as they are produced from the FEMBs, possibly with
some serialization taking place in the WIB. For the single phase TPC we have
chosen to adopt the latter option, which minimizes the requirements on the
FPGA inside the WIB, and also reduces the power and cooling requirements for the WIEC.
Based on this decision, the interface between the CE and DAQ consortia is
defined by the fibre plant
used to transmit the data from the WIBs to the DAQ components housed in the
Central Utility Cavern (CUC), and to broadcast the clock and controls in the
opposite direction. Only optical links are used in the connection with the DAQ,
which guarantees that the DAQ electronics will not induce any noise on the
APA wires.

The interface is fully defined with the selection of the number
and type of optical fibre links, their speed, and the type of connectors.
The FPGA inside the WIB can be used to reformat the data with changes to
the headers and trailers that include time stamps and geographical addresses
of the FEMBs. It can also be used to serialize the data from multiple
COLDATA ASICs into a single stream. In the simplest scheme each electrical
link from the FEMB is routed to a single optical fibre, transmitting data
at 1.28~Gbps. Depending on the availability and cost of transmitters
capable of sending data at higher speeds data from multiple electrical
links (4, 8 or more) could be serialized onto a single link. Using higher transmission
speeds reduces the number of links that are needed, possibly reducing the
cost of the DAQ part of the detector. The use of links with speeds
of 5~Gbps or larger may present the
drawback that the data has to be deserialized on the CUC side, depending
on the availability of resources for the extraction of trigger primitives
on the DAQ boards. The final choice of the number of fibres, link speed and
matching between FEBMs and DAQ processing units in the CUC should be
the result of a cost optimization process that can be delayed until
the Technical Design Report.

%The use of links at speeds of 5 Gbit/s
%or larger may also present another drawback, in addition to the increases
%in the cost of fibres, transmitters and receivers. We desire to keep the
%data of a single FPGA into a single set of DAQ boards, without deserializing
%the data on the CUC side. Given the granularity of the FEMBs there may be
%combinations of the number of DQ boards and the number of links being
%used, that cannot be operated for this reason. The final choice of the
%number of fibres, link speed and matching between FEMBs and DAQ processing
%units in the CUC can be delayed until the Technical Design Report.

Another aspect of the interface between the DAQ and the CE consortia is the
transmission of clock and commands. In protoDUNE-SP a single fibre
carries this information to the PTC card at each CE flange, which then re-broadcasts
the information to the five WIBs, using the WIEC backplane. For DUNE we
foresee the possibility of transmitting the information directly to each
WIB, the functionality for which is already implemented in the protoDUNE-SP WIB, as
shown in Figure~\ref{fig:tpcelec-wib_timing}.
%This would only require the addition of a receiver on each WIB, and
%no changes to the FPGA firmware, given that already in the protoDUNE scheme
%the clock and control information is already decoded inside the FPGA mounted
%in the WIB.
The final aspect of the interface between the DAQ and the CE is
the definition of the format of the data transmitted by the cold electronics
to the DAQ. This format has been defined for protoDUNE-SP in~(cite: $proto:data_format$).
Some changes are needed for DUNE, to accommodate a larger number of APAs.

%\bibitem{proto:data_format)     DocDB 1701
