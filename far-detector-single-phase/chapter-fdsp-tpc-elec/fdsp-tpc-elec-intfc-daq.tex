There are two components in the definition of the interface between the \dword{ce}
and \dword{daq} consortia. The first one is a decision on whether to implement any
firmware and buffering related to the trigger decision inside the \dword{wib}, or
instead transmit the data as they are produced from the \dwords{femb}, possibly with
some serialization taking place in the \dword{wib}. For the \dword{spmod} %single phase TPC 
we have
chosen to adopt the latter option, which minimizes the requirements on the
\dword{fpga} inside the \dword{wib}, and also reduces the power and cooling requirements for the \dword{wiec}.
Based on this decision, the interface between the \dword{ce} and \dword{daq} consortia is
defined by the fiber plant
used to transmit the data from the \dwords{wib} to the \dword{daq} components housed in the
\dword{cuc}, and to broadcast the clock and controls in the
opposite direction. Only optical links are used in the connection with the \dword{daq},
which guarantees that the \dword{daq} electronics will not induce any noise on the
\dword{apa} wires.

The interface is fully defined with the selection of the number
and type of optical fiber links, their speed, and the type of connectors.
The \dword{fpga} inside the \dword{wib} can be used to reformat the data with changes to
the headers and trailers that include time stamps and geographical addresses
of the \dwords{femb}. It can also be used to serialize the data from multiple
\dword{coldata} \dwords{asic} into a single stream. In the simplest scheme each electrical
link from the \dword{femb} is routed to a single optical fiber, transmitting data
at \SI{1.28}{Gbps}. Depending on the availability and cost of transmitters
capable of sending data at higher speeds data from multiple electrical
links (four, eight, or more) could be serialized onto a single link. Using higher transmission
speeds reduces the number of links that are needed, possibly reducing the
cost of the \dword{daq} part of the detector. The use of links with speeds
of \SI{5}{Gbps} or larger may present the
drawback that the data has to be deserialized on the \dword{cuc} side, depending
on the availability of resources for the extraction of trigger primitives
on the \dword{daq} boards. The final choice of the number of fibers, link speed, and
matching between \dwords{femb} and \dword{daq} processing units in the \dword{cuc} should be
the result of a cost optimization process that can be delayed until
the \dword{tdr}.

Another aspect of the interface between the \dword{daq} and the \dword{ce} consortia is the
transmission of clock and command signals. In \dword{pdsp}, a single fiber
carries this information to the \dword{ptc} card at each \dword{ce} flange, which then re-broadcasts
the information to the five \dwords{wib}, using the \dword{wiec} backplane. For the \dword{spmod}, we
foresee the possibility of transmitting the information directly to each
\dword{wib}, the functionality for which is already implemented in the \dword{pdsp} \dword{wib}, as
shown in Figure~\ref{fig:tpcelec-wib_timing}.

The final aspect of the interface between the \dword{daq} and the \dword{ce} is
the definition of the format of the data transmitted by the \dword{ce}
to the \dword{daq}.  This format has been defined previously for \dword{pdsp}.
Some changes are needed for DUNE to accommodate a larger number of \dwords{apa}.

%%% PROTODUNE DATA FORMAT:
%\bibitem{proto:data_format)     DocDB 1701
