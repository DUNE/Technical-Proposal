DUNE will plan for system tests of the baseline and alternative option in both the CERN cold box and a small test TPC at Fermilab.  The cold box tests establish performance of the electronics coupled to a full-scale APA in a correctly grounded environment, but in a gaseous environment no colder than 150K, and without TPC drift or high voltage.  The small test TPC will provide tests in an operational LArTPC but at much smaller scale, with quick turn around (2-4 weeks) for changing components and refilling.  The 40\% APA at BNL employs liquid nitrogen instead of liquid argon, does not integrate the CE with the PDS, and has no TPC drift.  However, it allows for quick turn-around and is located very close to where the development is happening (BNL and Fermilab), and is thus invaluable for initial board and component testing.

Generic board and component testing often includes (a) the measurement of the baseline noise level and frequency spectrum, (b) the measurement of the response to the calibration input signal provided on the FEMB, and (c) the measurement of the response to cosmic rays in a TPC.  Measurements are often done at both room temperature and liquid nitrogen or argon temperature in three configurations: nothing attached to the inputs, dummy capacitive load attached to the inputs, and an APA attached to the inputs as the capacitive coupling of long wires is subtly different than a dummy capacitor with the equivalent capacitance of a single wire.  Measurements specific to the characterization of the ADC performance, such as INL and DNL determination, are done before the ADC is mounted on the FEMB.
