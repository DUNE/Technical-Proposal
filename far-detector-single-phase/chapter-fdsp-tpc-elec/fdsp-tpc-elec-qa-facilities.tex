DUNE will plan for system tests of the baseline and alternative option in both the CERN cold box and a small test TPC at \fnal.  The cold box tests establish performance of the electronics coupled to a full-scale \dword{apa} in a correctly grounded environment, but in a gaseous environment no colder than \SI{150}{K}, and without TPC drift or \dword{hv}.  The small test TPC will provide tests in an operational \lartpc but at much smaller scale, with quick turn around (two to four weeks) for changing components and refilling.  The \num{40}\,\% \dword{apa} at BNL employs liquid nitrogen instead of \lar, does not integrate the \dword{ce} with the \dword{pds}, and has no TPC drift.  However, it allows for quick turn-around and is located very close to where the development is happening (BNL and \fnal), and is thus invaluable for initial board and component testing.

Generic board and component testing often includes measurement of (1) the baseline noise level and frequency spectrum, (2) the response to the calibration input signal provided on the \dword{femb}, and (3) the response to cosmic rays in a TPC.  Measurements are often done at both room temperature and liquid nitrogen or argon temperature in three configurations: nothing attached to the inputs, dummy capacitive load attached to the inputs, and an \dword{apa} attached to the inputs as the capacitive coupling of long wires is subtly different than a dummy capacitor with the equivalent capacitance of a single wire.  Measurements specific to the characterization of the \dword{adc} performance, such as integral nonlinearity (INL) and differential nonlinearity (DNL) determination, are done before the \dword{adc} is mounted on the \dword{femb}.
