The warm interface electronics are housed in warm interface electronics crates (WIECs)
attached directly to the signal flange.  The WIEC shown in Figure~\ref{fig:tpcelec-flange} 
contains one
Power and Timing Card (PTC), five Warm Interface Boards (WIBs) and a passive
Power and Timing Backplane (PTB), which fans out signals and LV power from the PTC to the WIBs. The WIEC must provide a Faraday-shielded housing, robust ground connections from the warm electronics boards to the detector ground, and only optical fiber links to the DAQ and slow control, to mitigate noise introduced at the CE feed-through.

\begin{dunefigure}
[Exploded view of the signal flange for ProtoDUNE-SP.]
{fig:tpcelec-flange}
{Exploded view of the signal flange for ProtoDUNE-SP.}
\includegraphics[width=0.9\linewidth]{tpcelec-flange.png}
\end{dunefigure}

The WIB is the interface between the
DAQ system and four
FEMBs. It receives the system clock and control signals from the
timing system and provides for processing and fan-out of those signals to the four
FEMBs. The WIB also receives the high-speed data signals from the four 
FEMBs and transmits them to the DAQ system over optical
fibers. The data signals are recovered onboard the WIB with commercial equalizers. The WIBs are attached directly to the TPC
CE feedthrough on the signal flange. The feedthrough
board is a PCB with connectors to the cold signal and LV power cables fitted
between the compression plate on the cold side, and sockets for
the WIB on the warm side. Cable strain relief for the cold cables is 
supported from the back end of the feedthrough.

\begin{dunefigure}
[Power and Timing Card (PTC) and timing distribution to the WIB and FEMBs.]
{fig:tpcelec-wib_timing}
{Power and Timing Card (PTC) and timing distribution to the WIB and FEMBs.}
\includegraphics[width=0.8\linewidth]{tpcelec-wib_timing.pdf}
\end{dunefigure}

The PTC provides a bidirectional fiber interface to the
timing system.  The clock and data
streams are separately fanned-out to the five WIBs as shown in
Figure~\ref{fig:tpcelec-wib_timing}. The PTC fans the clocks out to the WIB over the
PTB, which is a passive backplane attached directly to the PTC and
WIBs.  The received clock on the WIB is separated into clock and
data using a clock/data separator.

\begin{dunefigure}
[LV power distribution to the WIB and FEMBs implemented for protoDUNE-SP. This will be modified for DUNE to provide the required voltage or voltages depending on which ASICs are used on the FEMBs.]
{fig:tpcelec-wib_power}
{LV power distribution to the WIB and FEMBs implemented for protoDUNE-SP. This will be modified for DUNE to provide the required voltage or voltages depending on which ASICs are used on the FEMBs.}
\includegraphics[width=0.7\linewidth]{tpcelec-wib_power.pdf}
\end{dunefigure}

The PTC also receives LV power for all cold
electronics connected through the signal flange, approximately 250W at 48V for a
fully-loaded flange (one~PTC, five~WIB, and 20~FEMB). The LV power is then stepped down
to 12V via a DC/DC converter onboard the PTC. The output of the PTC converters is filtered with a common-mode choke and fanned out
on the PTB to each WIB, which provides the necessary 12V DC/DC conversions and fans
the LV power out to each of the cold FEMBs supplied by that WIB, 
as shown in Figure~\ref{fig:tpcelec-wib_power}. The output of the WIB converters are further filtered by a common-mode choke. The 
majority of the 250W drawn by a full flange is dissipated in the LAr
by the cold FEMB.

Each WIB contains a 
unique IP address for its UDP slow control interface. The IP address for the WIB is 
derived from a crate and slot address: the crate address is generated on the PTC 
board via dipswitches and the slot address is generated by the PTB slot, numbered 
from one to five. Note that the WIBs also have front-panel
connectors for receiving LV power; these can be used in place of 
the LV power inputs on the PTB generated by the PTC.

The WIB is also capable of
receiving the encoded system timing signals over bi-directional optical
fibers on the front panel, and processing these using either
the on-board FPGA or clock synthesizer chip to provide the clock required by the cold electronics.
The baseline ASIC design currently uses 8b/10b encoding; if the SLAC CRYO ASIC is selected for
the detector's construction, 12b/14b encoding will be used instead of 8b/10b.

\begin{dunefigure}
[Warm Interface Board (WIB). Note that front panel inputs include a LEMO connector and alternate inputs for LV power.]
{fig:tpcelec-dune_wib}
{Warm Interface Board (WIB). Note that front panel inputs include a LEMO connector and alternate inputs for LV power.}
\includegraphics[width=0.9\linewidth]{tpcelec-dune_wib.jpg}
\end{dunefigure}

The FPGA on the WIB is an Altera Arria V GT variant, which has
transceivers that can drive the high-speed data to the DAQ system up to
10.3125~Gbps per link, implying that all data from
two FEMB (2$\times$5~Gbps) could be transmitted on a single link.
The FPGA has an additional Gbps Ethernet transceiver I/O based on the 125~MHz clock, which 
provides real-time digital data readout to the slow control system.
