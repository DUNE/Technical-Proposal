The CE consortium provides all of the electronics used for reading out the charge
collection and induction signals produced on the APA wires by particles that ionize
the LAr. The APA consortium
is responsible for all of the printed circuit boards mounted on the APA frame
holding the anode wires. These boards provide filtering of the wire bias voltages,
connection of the bias voltages to the wires and AC or direct coupling of the
wire signals to the FEMBs built by the CE consortium.
The cold boxes containing the FEMBs are also mounted on the APAs and are connected
electrically to the CR boards. The CE consortium is responsible for providing the bias
voltage to the APAs, and also the bias voltage for the electron diverters and the field
cage termination electrodes (these last two items are part of the interface with
the HV consortium).

A crucial aspect of the interface between the CE and APA consortia
is the choice of routing for the cables that provide power, control, and readout
for the bottom APA. Studies are ongoing to understand whether it is feasible to
route these cables inside the APA frames of the two stacked APAs, and whether it
is necessary to modify the size of the APA frame in order to achieve this goal.
Mockups of the APA frames will be used for routing tests together with ProtoDUNE
cables, under the assumption that there will be minimal changes in the cross
section of cables between ProtoDUNE and DUNE (we expect to be able to remove one
of the seven pairs of power lines used in ProtoDUNE, when the FPGA on the
FEMB is replaced by the COLDATA ASIC). Before Spring 2019 we expect to perform
a realistic test of the cable insertion in a pair of stacked APAs using
mechanical prototypes (i.e.~without wires) of the APA. We are also investigating
the possibility of routing the cables for the bottom APA outside the field
cage, which increases significantly the length of the cables for the central
APA; this also complicates the installation procedure and the engineering of the
support structures inside the cryostat.
