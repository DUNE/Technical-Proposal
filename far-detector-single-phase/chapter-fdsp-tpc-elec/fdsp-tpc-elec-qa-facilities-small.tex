A small test TPC is essential to qualify the different prototype ASICs, offering quick turn around for changing components and refilling; this allows one to study the response of the electronics to signals from cosmic ray muons.  A new reduced-size APA will be constructed, with many similarities to the DUNE APA design.  The APA will be half the width of a DUNE APA (along the beam direction) or 1.3~m, with half the number of readout channels.  This amounts to 10 FEMBs with a total of 1280 channels.  The height will be significantly reduced from the DUNE APA height of 6~m to about 1.25~m, and the wire lengths will be reduced by the same factor.  The ProtoDUNE CR boards will be used, and  the APA will accommodate a single half-length photon detector.  The TPC will have the APA in the center and a cathode on either end, creating two drift volumes with drift distance 0.5~m each.  The TPC will be installed in the cryostat with the wire planes parallel to the floor to optimize the orientation of the cosmic ray tracks.  It will be instrumented with a full readout chain of ProtoDUNE electronics, specifically  the cables, feedthrough flange, WIB, PTC and warm interface crate.  Initially, ProtoDUNE FEMBs will be used for commissioning, and later FEMBs with prototype ASICs will be swapped in.

The TPC electronics and photodetector will be read out through a ``slice'' of the ProtoDUNE data acquisition system.  This system will provide a low noise environment that will allow one to make detailed comparisons of the performance of the new ASICs. It will also enable the study of interactions between the TPC readout and other systems, including the photodetector readout and the HV distribution.
 
The APA will be housed in new LAr cryostat at Fermilab (in the Proton Assembly Building) that complies with the DUNE grounding and shielding requirements, and connected to an existing recirculation system for argon purification.  The cryostat will be a vertical cylinder, with an inner depth of 185~cm and an inner diameter of 150~cm.  With the cryogenic connections on the upper portion of the cylinder, the flat top plate will have penetrations dedicated to readout and cryogenic instrumentation.  The target date for the fabrication of new front-end motherboards is fall 2018, and these will house the latest version of the FE ASIC, the first prototype of COLDATA and various ADC prototypes (including the SLAC CRYO ASIC).  The new APA and the new cryostat will be completed on the same timescale so that tests in both the ProtoDUNE cold box at CERN and the test TPC at Fermilab can be completed and analyzed prior to the submission of the TDR.  
