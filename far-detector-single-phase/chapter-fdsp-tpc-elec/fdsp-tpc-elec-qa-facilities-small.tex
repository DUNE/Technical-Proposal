A small test TPC is essential to qualify the different prototype \dwords{asic}, offering quick turn around for changing components and refilling; this allows one to study the response of the electronics to signals from cosmic-ray muons.  A new reduced-size \dword{apa} will be constructed with many similarities to the DUNE \dword{apa} design.  The \dword{apa} will be half the width of a DUNE \dword{apa} (along the beam direction) or \SI{1.3}{m}, with half the number of readout channels.  This amounts to \num{10} \dwords{femb} with a total of \num{1280} channels.  The height will be significantly reduced from the DUNE \dword{apa} height of \SI{6}{m} to about \SI{1.25}{m}, and the wire lengths will be reduced by the same factor.  The \dword{pdsp} CR boards will be used, and  the \dword{apa} will accommodate a single half-length \dword{pd}.  The TPC will have the \dword{apa} in the center and a cathode on either end, creating two drift volumes with drift distance \SI{0.3}{m} each.  The TPC will be installed in the cryostat with the wire planes parallel to the floor to optimize the orientation of the cosmic ray tracks.  It will be instrumented with a full readout chain of \dword{pdsp} electronics, specifically  the cables, \fdth flange, \dword{wib}, \dword{ptc}, and warm interface crate.  Initially, \dword{pdsp} \dwords{femb} will be used for commissioning, and later \dwords{femb} with prototype \dwords{asic} will be swapped in.

The TPC electronics and photodetector will be read out through a \textit{slice} of the \dword{pdsp} \dword{daq}.  This system will provide a low-noise environment that will allow one to make detailed comparisons of the performance of the new \dwords{asic}. It will also enable the study of interactions between the TPC readout and other systems, including the \dword{pd} readout and the \dword{hv} distribution, to exclude the possibility that one system generates noise on another one when both are being operated.
 
The \dword{apa} will be housed in new \lar cryostat at \fnal (in the Proton Assembly Building) that complies with the DUNE grounding and shielding requirements, and connected to an existing recirculation system for argon purification.  The cryostat will be a vertical cylinder, with an inner depth of \SI{185}{cm} and an inner diameter of \SI{150}{cm}.  With the cryogenic connections on the upper portion of the cylinder, the flat top plate will have penetrations dedicated to readout and cryogenic instrumentation.  The target date for the fabrication of new \dword{fe} motherboards is fall of 2018, and these will house the latest version of the \dword{fe} \dword{asic}, the first prototype of \dword{coldata}, and various \dword{adc} prototypes (including the SLAC CRYO \dword{asic}).  The new \dword{apa} and the new cryostat will be completed on the same timescale so that tests in both the \dword{pdsp} cold box at CERN and the test TPC at \fnal can be completed and analyzed prior to the submission of the \dword{tdr}.  
