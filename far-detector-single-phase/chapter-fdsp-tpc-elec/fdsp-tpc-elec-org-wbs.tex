A preliminary \dword{wbs} has been prepared for the activities
of the \dword{ce} consortium. The \dword{wbs} is split in a time-ordered fashion
between activities related initially to design, R\&D, and engineering, then
to production setup, and finally to the production, integration, and installation
phases for the \dword{spmod} to be installed in the first DUNE cryostat.
The latter three sets of activities could be repeated for the construction
of additional \dwords{spmod} %detectors to be installed in the following cryostats. 
Physics and simulation
activities are to proceed in parallel to the detector design and construction activities.
Within each phase (starting with the design and ending with the installation),
the \dword{wbs} foresees work packages that cover system engineering, installation of the detector
components inside the cryostat (including all \dwords{asic}, \dwords{femb},
cables, and corresponding support structures and cryostat penetrations),
and the detector components installed on top of the cryostat (including the warm
interface electronic crates with their boards, the \dword{lv} and bias-voltage
power supplies with their crates, and all of the associated cables and infrastructure). This
matches the current plan for the future group structure of the \dword{ce} consortium. 
%In addition, the development and support of the testing facilities and of the related software are assigned a separate work package to ensure that these activities are given the effective supervision that is required in order to meet the reliability requirements of the DUNE experiment.
In addition, a separate work package covers the development and support of the testing facilities and the related software in order to provide these activities with the effective supervision that is required in order to meet the reliability requirements of the DUNE experiment.

All of the institutions currently interested and committed to the construction of
the detector components that are a responsibility of the \dword{ce}
consortium are from the USA and are supported by a single
funding agency, the Department of Energy. For this reason, the exact role of
the individual institutions in the activities of the consortium has not been
defined, except for the currently ongoing \dword{asic} development. The role of
each institution will be defined prior to the submission of the dword{tdr}.
