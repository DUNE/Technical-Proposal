A preliminary work breakdown structure (WBS) has been prepared for the activities
of the Cold Electronics consortium. The WBS is split in a time-ordered fashion
between activities related initially to design, R\&D, and engineering, then
to production setup, and finally to the production, integration, and installation
phases for the Single Phase TPC to be installed in the first DUNE cryostat. The
latter three sets of activities could be repeated for the construction
of detectors to be installed in the following cryostats. Physics and simulation
activities proceed in parallel to the detector design and construction activities.
Within each phase (starting with the design and ending with the installation),
the WBS foresees substructures that cover system engineering, the detector
components installed inside the cryostat (including all the ASICs, the FEMBs,
the cables, and the corresponding support structures and cryostat penetrations),
the detector components installed on top of the cryostat (including the warm
interface electronic crates with their boards, the low voltage and bias
voltage power supplies with their crates, all the cables and infrastructure).
In addition the development and the support of the testing facilities and of
the related software are assigned a separate WBS to ensure that these activities
are given the effective supervision that is required to meet the reliability
requirements of the DUNE experiment.

All the institutions currently interested and committed to the construction of
the detector components that are a responsibility of the Cold Electronics
consortium are from the United States and will be supported by a single
funding agency, the Department of Energy. For this reason the exact role of
the individual institutions in the activities of the consortium has not been
defined, except for the currently ongoing ASIC development. The roles of
each institutions will be defined prior to the submission of the Technical
Design Report.
