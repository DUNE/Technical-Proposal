Plans for the CE consortium are based on the overall schedule for DUNE
that assumes that the first APAs will be fully populated with electronics
and tested toward the beginning of 2022, with the installation of
the APAs inside the cryostat beginning in May 2023. The integration of the
APAs for the first cryostat with the electronics should be finished by
October 2023, and their installation in the cryostat by December 2023.
This requires integrating two APAs per week over a span of 21 months,
which allows for a ramp-up period at the beginning and a contingency of
2-3 months at the end. The APA production needs to be completed
about 12 months later and the installation in the cryostat should be
finished by mid-2022. This defines the time window for the completion of
the R\&D program on the ASICs: a set of ASICs (or a single ASIC) meeting
all the DUNE requirements has to be fully qualified by Fall 2020, such
that preproduction ASICs and the corresponding FEMBs can be assembled and
tested in Spring 2021, launching the full production in Summer 2021.

Meeting this timeline requires that the development of the ASICs, and in particular
of the newly designed ones (the SLAC CRYO ASIC, the joint LBNL-BNL-FNAL ADC,
and COLDATA) are prototyped by the end of Summer 2018, with testing
completed by the end of 2018. This would allow a second round of prototyping,
if necessary, in the first half of 2019. It also leaves room for a possible
third design iteration and qualification of ASICs and FEMBs between the
end of 2019 and Fall 2020. The FEMBs used for protoDUNE will
likely have to be redesigned to house a new ADC and to replace the FPGA used
in protoDUNE with the COLDATA ASIC. Multiple variants of this board will be
necessary, depending on the success of the various ADC R\&D projects being currently
pursued. These design changes will be made in the second half of 2018.
Additional FEMBs prototypes will be designed and fabricated depending on
the outcome of the initial testing of the SLAC CRYO ASIC. A second iteration
of FEMB prototype(s) will be necessary in 2019, when the final ASICs (that
may have a different channel count from the first round of prototypes) will become
available.

It is assumed that apart from the ASICs, where rapid development is still
required, and the FEMBs, which have to be redesigned to accommodate the
new ASICs, most of the detector components to be delivered by the CE consortium
will require only minor changes relative to the protoDUNE components. For
this reason the modifications of these other detector components will
be delayed until 2019 or 2020, which will also help with the availability of funding.
Exceptions will be made for further development in test stands, for cabling
studies, and for conceptual studies of automated testing assemblies, rack space
assignment, and of the interface to the DAQ system.
