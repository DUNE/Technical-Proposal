Plans for the \dword{ce} consortium are based on the overall schedule for DUNE
that assumes that the first \dwords{apa} will be fully populated with electronics
and tested in spring 2022, with the installation of
the \dwords{apa} inside the cryostat beginning in May 2023. Plans are being made
for replacing three of the six \dwords{apa} of \dword{pdsp} with the final DUNE 
\dwords{apa} including final \dwords{asic} and \dwords{femb}, and for a second period of
data-taking in 2021-2022. The integration of the
\dwords{apa} for the first cryostat with the electronics should be finished by
October 2023, and their installation in the cryostat by January 2024.
This requires integrating two \dwords{apa} per week over a span of 21 months,
which allows for a ramp-up period at the beginning and a contingency of
two to three months at the end. This defines the time window for the completion of
the R\&D program on the \dwords{asic}. A set of \dwords{asic} (or a single \dword{asic}) meeting
all the DUNE requirements has to be fully qualified by fall 2020, such
that pre-production \dwords{asic} and the corresponding \dwords{femb} can be assembled and
tested in spring 2021, launching the full production in summer 2021.

Meeting this timeline requires that the development of the \dwords{asic}, and in particular
of the newly designed ones (the SLAC CRYO \dword{asic}, the joint LBNL-BNL-\fnal cold \dword{adc},
and \dword{coldata}) are prototyped by the end of summer 2018, with testing
completed by the end of 2018. This would allow for a second round of prototyping,
if necessary, in the first half of 2019. It also leaves room for a possible
third design iteration and qualification of \dwords{asic} and \dwords{femb} between the
end of 2019 and fall 2020. The \dwords{femb} used for \dword{pdsp} will
likely have to be redesigned to house a new \dword{adc} and to replace the \dword{fpga} used
in \dword{pdsp} with the \dword{coldata} \dword{asic}. Multiple variants of this board will be
necessary, depending on the success of the various \dword{adc} R\&D projects being currently
pursued. These design changes will be made in the second half of 2018.
Additional \dword{femb} prototypes will be designed and fabricated depending on
the outcome of the initial testing of the SLAC CRYO \dword{asic}. A second iteration
of \dword{femb} prototype(s) will be necessary in 2019, when the final \dwords{asic} (that
may have a different channel count from the first round of prototypes) will become
available. 

It is assumed that apart from the \dwords{asic}, where rapid development is still
required, and the \dwords{femb}, which have to be redesigned to accommodate the
new \dwords{asic}, most of the detector components to be delivered by the \dword{ce} consortium
will require only minor changes relative to the \dword{pdsp} components. For
this reason the modifications of these other detector components will
be delayed until 2019 or 2020, which will also help with the funding profile.
Exceptions will be made for further development in test stands, for cabling
studies, and for conceptual studies of automated testing assemblies, rack space
assignment, and the interface to the \dword{daq} system.
