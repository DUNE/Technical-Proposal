\fixme{\bf This section is still under review by the consortium.}

Once there is a final design that clearly satisfies the requirements and constraints, given all the testing at the device and system level described above, it will be necessary to put in place procedures and controls to ensure that the production CE parts will continue to fully satisfy the requirements and constraints. This set of Quality Control procedures can only be sketched at this point without yet knowing, for instance, yield statistics on the final ASICs. Nevertheless, it is possible to put forward a general plan based upon long established good practice.

All the custom ASICs will be packaged by a commercial vendor. All custom printed circuit boards will be produced by qualified vendors to at least IPC class 2 standards and all commercial assembly of those boards will also satisfy at least IPC class 2. All commercial parts will be procured from known reliable vendors and manufacturers. Depending upon the actual yield of the custom packaged chips it may or may not be necessary to individually test those chips prior to assembly onto printed circuits. In any event, all custom printed circuits delivered by the assembler will be run through a full functional test sequence at a DUNE institution and then subject to a powered burn in period of one or more weeks at elevated temperature followed by a second functional test sequence. Depending upon experience with pre-production assemblies this burn in may also include a temperature cycling step. All cables will be fully tested for continuity and lack of shorts either by the assembler or at a DUNE institution prior to installation. Cable testing may also include impedance and electrical length verification. All discrete CE parts will be serial numbered and a production data base will record the specifics for each part and each test sequence.

However, as much of the DUNE Cold Electronics will operate in a cryogenic environment, outside ordinary commercial experience, it will be necessary to add additional cryogenic testing steps to the above list for circuit boards that will be subject to those cryogenic conditions. The expectation is. based upon experience to date, that repeating the functional test sequence with the board under test immersed in liquid nitrogen should be sufficient. It may be useful to subject a small sample of such boards to the stress of multiple cycles between room and liquid nitrogen temperatures to understand how much margin is available against failure due to CTE induced stress for each production batch.

Handling of CE components at DUNE institutions shall follow standard IPC2 good practice for cleanliness, electrostatic discharge protection and environmental conditions. Boards, cables and other CE components shall be stored in qualified storage facilities and all shipments of boards and other components will be in qualified shipping containers via qualified shippers.
