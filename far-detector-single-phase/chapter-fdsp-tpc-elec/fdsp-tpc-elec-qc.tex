Once there is a final design that clearly satisfies the requirements and constraints, given all of the testing at the device and system level described above, it will be necessary to put in place procedures and controls to ensure that the production \dword{ce} parts will continue to fully satisfy the requirements and constraints. This set of \dword{qc} procedures, based on the experience gained with \dword{pdsp}, can only be sketched at this point without yet knowing, for instance, yield statistics on the final \dwords{asic}. Nevertheless, it is possible to put forward a general plan based upon long-established good practice.

All of the custom \dwords{asic} will be packaged by a commercial vendor. 
%All custom printed circuit boards will be produced by qualified vendors to at least IPC class 2 standards and all commercial assembly of those boards will also satisfy at least IPC class 2. 
All custom printed circuit boards will be produced by qualified vendors to at least IPC class 2\footnote{ "Acceptability of Printed Boards", IPC(R), IPCA-600F, Association Connecting Electronics Industries\texttrademark{}, \url{http://www.ipc.org}.} standards; the same standard will apply to all commercial assembly of those boards. 
All commercial parts will be procured from known, reliable vendors and manufacturers and qualified for use at LN temperatures. Depending upon the actual yield of the custom packaged chips it may or may not be necessary to individually test those chips prior to assembly onto printed circuits, and the use of an automated cryogenic test station is being considered. In any event, all custom printed circuits delivered by the assembler will be run through a full functional test sequence at a DUNE institution and then subjected to a powered burn-in period of one or more weeks at elevated temperature followed by a second functional test sequence. Depending upon experience with pre-production assemblies, this burn-in may also include a temperature cycling step. All cables will be fully tested for continuity and lack of shorts either by the assembler or at a DUNE institution prior to installation. Cable testing may also include impedance and electrical length verification. All discrete \dword{ce} parts will be serial-numbered and a production database will record the specifics for each part and each test sequence.

However, as much of the \dword{spmod} %DUNE 
TPC electronics will operate in a cryogenic environment, which differs from typical commercial experience, it will be necessary to add additional cryogenic testing steps to the above list for all of the \dwords{asic} and circuit boards that will be subject to those cryogenic conditions. Based upon current experience, the expectation is that repeating the functional test sequence with the board under test immersed in LN$_2$ should be sufficient. It may be useful to subject a small sample of such boards to the stress of multiple cycles between room and LN$_2$ temperatures; this will allow one to understand how much margin is available against failure due to stress caused by thermal cycling for each production batch.

Handling of \dword{ce} components at DUNE institutions must follow standard IPC class 2 good practice for cleanliness, electrostatic discharge protection, and environmental conditions. Boards, cables, and other \dword{ce} components must be stored in qualified storage facilities and all shipments of boards and other components will be in qualified shipping containers and shipped via qualified shippers.
