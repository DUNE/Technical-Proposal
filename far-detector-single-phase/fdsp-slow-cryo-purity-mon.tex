%%%%%%%%%%%%%%%%%%%%%%%%%%%%%%%%%%%%%%%%%%%%%%%%%%%%%%%%%%%%%%%%%%%%
\subsubsection{Introduction}
\label{sec:prm-intro}
Ideally, all electrons produced in the liquid argon by ionising particles should be drifted to the anode plane, but in practice part of the ionisation charge is lost, due to the presence of electronegative impurities in the liquid. To keep such loss to a minimum, purifying the liquid argon during operation is essential, as it is the monitoring of impurities.

Residual Gas Analysers are an obvious choice when analysing gas argon and can be exploited for the monitoring of the gas in the ullage of the tank. Unfortunately, commercially available mass spectrometers have a detection limit of $\sim$1\,ppm, whereas DUNE requires a sensitivity down to the ppt level. This gives us a case to construct small, ``portable'' devices, called ``purity monitors'', to monitor purity in all the phases of operations. Such tools will be placed inside the vessel, as well as outside the cryostat \textcolor{red}{before and after} the recirculation system.

Purity monitors have already been successfully employed in the ICARUS detector and in the 35-ton prototype detector at Fermilab. Also the ProtoDUNE Single-Phase and Dual-Phase detectors employ purity monitors. A similar design to the ones used in the protoDUNEs will be exploited in the DUNE FD. 
