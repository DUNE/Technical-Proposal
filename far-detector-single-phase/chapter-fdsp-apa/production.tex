%%%%%%%%%%%%%%%%%%%%%%%%%%%%%%%%%%%%%%%%%%%%%%%%%%%%%%%%%%%%%%%%%%%%
\section{Production and Assembly (8 pages)}
\label{sec:fdsp-apa-prod-assy}

The APA Consortium will take a ``factory style'' approach to the construction of APAs with multiple factories being planned in the US and UK. %There will be four factories set up in the US and UK. 
This approach will allow the consortium to produce APAs at the rate required to meet overall construction milestones and at the same time reduce risk to the project if any location encounters problems that slow the pace of production.

%%%%%%%%%%%%%%%%%%%%%%%%%%%%%%%%%%%
\subsection{APA Production Plan}
\label{sec:fdsp-apa-prod-plan}

%Approach:
%The APA Consortium will take a ``factory style'' approach to the construction of APAs. There will be four factories set up in the US and UK. This approach will allow the consortium to produce APAs at the rate required to meet overall construction milestones and at the same time reduce risk to the project if any location encounters problems that slow the pace of production. %, i.e. the other factories can pick up their rate of production to offset the loss of constructed APAs for a period of time. 

The starting point for the APA production plan for DUNE Far Detectors is the experience and lessons learned from ProtoDUNE construction. For ProtoDUNE, APAs have been constructed both at PSL in the US and Daresbury Lab in the UK.  Based on this experience, we estimate that each APA will require approximately 50 shifts (8 hour intervals) of effort to construct. This begins with installation of the APA frame into the winding machine and assumes that all hardware necessary for construction is ready to go at the factory. %(a robotic machine used to construct ProtoDUNE APAs at PSL and Daresbury Lab). 
Currently an APA can be built in 64 shifts. Several improvements to the process and tooling are planned that will bring this down to the required 50 shifts. Assuming construction begins in 2021, a minimum of 6 production lines is required to build 150 APAs within 2.5 years for the first DUNE 10-kton module. This assumes that factories will run two shifts per day and that 2 weeks per year are devoted to maintenance of equipment. 

Each production line is centered around a wire winding robot, or ``winder'', that makes the continuous wrapping of wire on a 6~m long frame possible. The winder can also be used for wire tension measurements by replacing the winding head with a laser photo diode system that then can determine an individual wire's natural frequency and hence its tension. A production line will also require two process carts. These carts support the APA and are used during various steps in the construction process, e.g. continuity testing, board epoxy installation, etc. A production line, therefore, requires a means of lifting the APA in and out of the winder. A gantry-style crane has been used on ProtoDUNE.

%A key aspect to the success of this approach will be the source of material delivered to each factory. Here the consortium institutions will play a pivotal role taking on the responsibility for the delivery of APA sub-elements to each of the factories. For this approach to be a success, a key factor will be for each of these suppliers to deliver on time according to material resource plans. %, i.e. don't starve the factory for parts. Supplier institutions will have responsibility for the sourcing, inspection, cleaning, testing, quality assurance, and delivery of hardware to each of the factories.

\paragraph{Material Supply}  Ensuring the reliable supply of raw materials and parts to each of the factories is critical to keeping APA production on schedule through multiple years of construction. Here the consortium institutions will play a pivotal role taking on the responsibility for the delivery of APA sub-elements to each of the factories. Supplier institutions will have responsibility for the sourcing, inspection, cleaning, testing, quality assurance, and delivery of hardware to each of the factories.   
\begin{itemize}
\item Frame construction: We envision two sources of frames, one in the US and one in the UK. The institutions responsible will rely on many lessons learned from ProtoDUNE. The effort requires specialized resources and skills including a large assembly area, certified welding capability, large scale metrology tools and experience, and large scale tooling and crane support. Two approaches are under consideration for sourcing; one is a total outsource strategy with an industrial supplier, the other is to procure all of the major machined and welded components and then assemble and survey in-house. Material suppliers have been identified and used with good results on ProtoDUNE.
\item Mesh supply and construction: Elsewhere in this proposal we describe the current mesh installation procedure. However, our ProtoDUNE experience leads us to believe that moving to smaller self-supporting ``window screen'' panels will save assembly time and improve overall APA quality. An excellent source of mesh exists and was used on ProtoDUNE.
\item Wire procurement: Wire is a significant element in the assembly of an APA. There is approximately 24 km of wire wound on each unit. Through ProtoDUNE we have worked with an excellent supplier that has worked with us to provide wire that is of high quality and wound on spools that we provide. These spools are then used directly on the winder head with no additional handling or re-spooling required. Wire samples from each spool are strength tested prior to use.
\item Comb procurement: An institution will work with either our existing comb supplier or find additional suppliers that can meet our requirements. The ProtoDUNE supplier has been very reliable.
\item Wire wrapping board procurement: One or more consortium institutions will take on the responsibility of wire wrapping board supply. The side and foot boards are rather unique to suppliers as they have electrical traces and provide wire placement support through a separately bonded tooth strip. There are 276 boards per APA for the X, V, U, G, and covers, or 41,400 needed for 150 APAs. Using costs from ProtoDune, the material cost alone is \$1.8M. The institutions that have responsibility for boards will spend time working with multiple vendors to reduce risk and ensure quality. 
\item Capacitor resistor boards: These boards are rather unique given their thickness, high voltage components, and leakage current requirements. A reliable source of bare boards was found for ProtoDUNE. Assembly and testing was performed at PSL. We will conduct a more exhaustive search of vendors that will be willing to take on assembly and test for the 3000 plus boards needed for DUNE.
\item Winders and tooling: We propose that PSL and Daresbury work together to supply tooling and winding machines for additional production lines at new locations and for additional lines in-house. This is a natural collaboration that has been in place for nearly two years on ProtoDUNE.
\end{itemize}

\paragraph{Planned Improvements to Production Process:} Based on our ProtoDUNE experience we have identified the following improvements to tooling and process that will allow the consortium to construct APAs in a more efficient and reliable manner including:
\begin{itemize}
\item Improved interface arm design: The current winder interface arm design allows wire planes to be wound one-half at a time. The new design will allow the winder head to pass from one side to the other in a nearly continuous fashion without removal from the winding machine. 
\item Change to a modular mesh panel: The current approach to mesh installation is slow and cumbersome. We will improve this aspect of construction by moving toward a modular ``window screen'' design that will improve the reliability of the installed mesh (more uniform tension across the mesh), and allow much easier installation on the APA frame.
\item Improvements to epoxy process: There are many epoxy application steps during the construction process. These steps require careful work that takes many hours between winding each successive wire plane. We already have concepts for improved epoxy application jigs from ProtoDUNE, and will investigate whether epoxy pre-forms or accelerated heat curing can yield time or reliability improvements.
\item Winder head improvements: Efforts to improve winder head performance are already underway at Daresbury Lab. We envision improved tension control, continuous tension feedback, improved clutch, and an improvement to the compensator mechanism all leading to better, more consistent and more reliable winder performance.
\item Automated soldering: Every solder joint on the 6 ProtoDUNE APAs was done by hand. We will investigate automated soldering techniques to improve process and the amount of manual effort required. Commercial products exist and can be quite reliable.
\item Improved wire tension measurement techniques: Verifying wire tension is an important but time consuming process during construction. The current technique utilizes a laser photodiode tool mounted on the winder to measure tension one wire at a time. This takes many hours for each wire plane. Techniques are under development at the University of Manchester to electronically measure groups of 20 or more wires at one time. This technique will provide much faster tension measurements and shorter turnaround between wire planes. 
\item Improved winder maintenance plan: Our current approach to winder maintenance is not well formed. As a result, winding machine problems that can be traced back to lack of routine maintenance occur from time-to-time, which shuts the production line down until a repair  or maintenance is performed. We will formulate a routine and preventive maintenance plan that should minimize winder downtime.
\end{itemize}

%%%%%%%%%%%%%%%%%%%%%%%%%%%%%%%%%%%
\subsection{Facility Plans}
\label{sec:fdsp-apa-facility}

Construction of DUNE Far Detector APAs will take place in both the US and the UK. Daresbury Lab in the UK will house 4 production lines, one of which already exists from ProtoDUNE. In the US, it's anticipated that production lines will be set up at the University of Chicago, Yale University, and the already existing production facility at the University of Wisconsin~-~PSL. %Our plan is to have an additional production line set up and operational in both countries for a total of eight. 
Multiple factories will provide some margin on the production schedule and provide backup in the event that technical problems occur at any particular site. 

%It is currently anticipated that APA production lines will be set up at Daresbury Lab in the UK, University of Wisconsin - PSL, University of Chicago, and Yale University - Wright Lab. 
The space requirements for each production line are driven by the large size of the APA frames and the winding robot used to build them. The approximate dimensions of the clean space to house winder operations and associated tooling is 175~m$^2$. The estimated requirement for inventory, work in progress, and completed APAs is about 600~m$^2$. Each facility will also need temporary access to shipping and crating space of about 200~m$^2$. Possible floor layouts at each institution are shown in Figure~\ref{fig:factories}. Adequate space is available at each site and commitments by the administration of each institution has been made for use on DUNE. 

The University of Wisconsin has space available within the Physical Sciences Lab Rowe Technology Center. A portion of the vault area within PSL's Rowe Technology Center has been setup and has been used for the past two years for the ProtoDUNE project. There is approximately 20,000~ft$^2$ (1,850~m$^2$) available for DUNE and the possibility exists to expand the current clean tent to house another production line if needed. 

%\paragraph{PSL} The University of Wisconsin has space available within the Physical Sciences Lab Rowe Technology Center. A portion of the vault area within PSL's Rowe Technology Center has been setup and has been used for the past two years for the ProtoDUNE project. There is approximately 20,000~ft$^2$ available for DUNE and the possibility exists to expand the current clean tent to house another production line if needed. 

%\paragraph{Chicago}

%\paragraph{Yale}

%\paragraph{UK}
ProtoDUNE construction has also taken place at Daresbury Lab.  The current facility cannot accommodate multiple production lines, but the ``Inner Hall'' on the Daresbury site has been identified as an area that is sufficiently large to be used for DUNE APA construction. It has good access and crane coverage throughout. %, but is currently being used as a stores building. 
Daresbury Laboratory management have agreed that the area is available but it would need investment to establish a safe working environment. Preparation work for the construction area is underway with a \pounds200k investment to clear the current area of existing facilities, obsolete cranes, and ancillary equipment. Also planned is the renovation of a plant room which will be used for storage and as a shipping area. %The plant room needs to have obsolete switch gear removed and the boiler system relocated to provide a clear working space. 
This work is ongoing. The production factory is being designed to hold 4 winding machines and associated process equipment and tooling. The factory will be required for 2020 when it is planned that production will begin. 

\begin{dunefigure}[APA factory layouts]{fig:factories}
{Possible factory layouts at PSL, Chicago, Yale, and Daresbury Lab with two to four winders at each site.}
\includegraphics[height=0.27\textheight]{PSL-schematic.png} 
\includegraphics[height=0.25\textheight]{Chicago-2APA.pdf}
\includegraphics[height=0.27\textheight]{Yale-WL-2APA.png}
\includegraphics[height=0.27\textheight]{UK-production-factory-layout.jpg} 
\end{dunefigure}


%%%%%%%%%%%%%%%%%%%%%%%%%%%%%%%%%%%
\subsection{Wire Winding Machine}
\label{sec:fdsp-apa-winding}

\begin{dunefigure}[APA production photos]{fig:APA-photos}{Left: Partially wired ProtoDUNE APA on the winding machine at PSL, University of Wisconsin. Right: Partially-wound protoDUNE APA frame on the winding machine at Daresbury Lab, UK.}
\setlength{\fboxsep}{0pt}
\setlength{\fboxrule}{0.5pt}
\fbox{\includegraphics[height=0.23\textheight]{APA-photo-tension-testing}}
\fbox{\includegraphics[height=0.23\textheight, trim=4mm 0mm 4mm 0mm, clip]{APAAtDaresbury}}
\end{dunefigure}

\todo{Need a technical description of the current machine before going into changes}

The design of the final wire-winding machines for DUNE far detector APAs will be based on the protoDUNE experience. The protoDUNE wire-winding machine (shown in Fig.~\ref{fig:APA-photos}) is partially automated and currently can only wire half a wire plane in a single operation. The UK and PSL engineering design staff will work to advance the design such that it is fully automated, as far as reasonably practicable.

There are several areas which can be considered for design development:
\begin{itemize}
\item Winding Head - The current winding head which carries the wire spool uses a magnetic clutch mechanism. This is manually adjusted to increase or lower the tension of the wire as it is wound around the APA. This is not an automatic process and regularly needs adjustment as the diameter of the wire on the spool reduces during the winding process. We have also found that if the mechanism is run from a `cold start' the tension changes after 10mins of running. Current experience of winding the ProtoDUNE wires shows that it is difficult to maintain the target tension of 5N +/- 20%.

A solution to this problem is to design a winder head with active tension control. This can be achieved by replacing the magnetic clutch with a servo motor and a introducing a potentiometer on a ``dancer arm'' for the feedback loop. This will only work if there are no signal losses when transferring the winding head to the compensator latching mechanism and back. The system can be driven in torque mode and will compensate for any wire spool changes and will be able to operate from a cold start. This development is well underway with tests currently being carried out. See Fig XX.

\begin{dunefigure}[Winding Head]{fig:winding_head}
{Exploded view of winder head with active tension control}
\includegraphics[width=0.55\textwidth]{Winder-head-with-active-tension-controls.jpg}
\end{dunefigure}

\item Winding Machine - To carry out the wiring operation of one full wire plane we must re-think the way in which we hold the frame in the winding machine. Holding the frame in a `flexible' manner is key to solving the current half wire winding limitations. The current method employs interface frames at the foot and head end of the machine which only allows us to wind half a wire plane. The interface frames must then be removed from the APA frame whilst it is in the process cart and flipped through 180 degs to wind the second half of the wire plane.

Design development work is in progress. The interface frames have been replaced at either end by retractable linear guided shafts. These can be withdrawn systematically to allow passing of the winding head around the frame over the full height of the frame. These shafts have conical ends and locate in shafts that are fixed to the internal frame tube to provide guided location. This design change does not alter the design of the frame, however we do not consider the riv-nuts are required and suggest that these should be omitted. The design also allows for rotation in the winding machine or the process cart for some assembly or process operations such as soldering. This is achieved using the existing features in the APA frame with the addition of a spine support at the head end.  It should also be possible to carry out board installation and gluing \& soldering in the winding machine. This eliminates the need to transfer the APA to the process cart for the whole of the production operation. This is inherently a safer production method as it cuts down the amount of handling of the APA. The development design proposal can be seen in figure??
\end{itemize}

\begin{dunefigure}[Winding Machine Dev]{fig:winding_dev}
{Winding machine design development.}
\includegraphics[width=0.8\textwidth]{Winding-machine-design-development.jpg} 
\end{dunefigure}


%%%%%%%%%%%%%%%%%%%%%%%%%%%%%%%%%%%
\subsection{Assembly Procedures, Tooling, and Documentation}
\label{sec:fdsp-apa-assy}

A subset of Procedures describing how to perform the step-by-step assembly of an APA were originally created prior to the finalization of the ProtoDUNE APA series of drawings, and assigned Drawing Numbers. During subsequent assemblies, these instructions have evolved due to the addition of better tooling, fixtures, jigs and more complete drawing documents.  The process steps contained in each procedure have also been changed to create a better match between the B.O.M. (Bill Of Materials) contained on each finalized drawing level.  The table below explains what documents are related to each assembly level.

\begin{dunetable}[APA assembly documents]{lcc}{tab:assamly_docs}{Procedure documents for APA assembly.}   
APA Assembly Level & \textbf{Drawing No.} & \textbf{Assembly Instructions Doc.} \\ \toprowrule
APA Frame Assembly & 8757 004 & 8752Doc001 \\ 
                   &          & 8752Doc002 \\ \colhline
Comb Base and Mesh & 8757 003 & 8752Doc003 \\
				   &          & 8752Doc004 \\ \colhline
Four Wire Layers   & 8757 002 & ~~~~~8752Doc005 (X) \\
                   &          & ~~~~~8752Doc006 (V) \\
                   &          & ~~~~~8752Doc007 (U) \\
                   &          & ~~~~~8752Doc008 (G) \\ \colhline
Factory APA        & 8757 030 & 8752Doc009 \\
                   &          & 8752Doc010 \\ \colhline
Crating for Shipment & being finalized & being finalized \\
\end{dunetable}

Currently these documents are being revised to reflect the latest evolution of these procedures that were used to assemble US APA 4.

%Traveler System
The original procedures were created with RED TEXT that denoted to the personnel using the procedure that information or specific process data needed to be recorded into the Traveler record.  Each APA has a Traveler where specific assembly information is gathered initially by hand on a paper copy, then entered into an electronic version for longer term storage.  

%\todo{pull in a picture of a portion of the traveler system}

Current ProtoDUNE APA construction uses a few very large pieces of tooling to facilitate movement and assembly of a bare APA frame and a frame loaded with mesh/boards/wires.

%Process Cart(s)
We currently employ 2 process carts to move APAs around the assembly facility, and also moves them to a shipping/packing location area for attachment to specialized crate containers.
The process cart with regular casters remains in the assembly area and is maintained at a particular height that coordinates with other construction tooling such as jack stands, platform ladders during the assembly/build activity.
A second process cart has been fitted with specialized 360° rotating castors that are instrumental in allowing the process cart loaded with a fully assembled APA to maneuver tight corners during the journey from the assembly area to the shipping/packing location.

\begin{dunefigure}[APA on a process cart]{fig:apa-process-cart}{APA being moved around a production facility on the process cart.}
\setlength{\fboxsep}{0pt}
\setlength{\fboxrule}{0.5pt}
\fbox{\includegraphics[height=0.2\textheight]{APA-process-cart.jpg}}
\fbox{\includegraphics[height=0.2\textheight]{comb-base-jig.png}}
\end{dunefigure}

Mesh Attach Jig (and associated Hold-Down Bars)
After a bare APA frame has been delivered to the production assembly area, the first operation is to attach mesh panels to (4) areas of the frame.  This assembly operation is completed by lowering a large Mesh Attach Jig to a frame supported in a horizontal position.  An overhead crane is utilized to lift this large jig while multiple personnel guide it into the appropriate position. Once the jig is leveled sufficiently to the frame, a mesh panel is laid into place and hold down bars are iteratively moved and repositioned until the mesh is as flat and tight as possible.  The outside edge of the mesh panel then gets epoxied; the jig and hold down bars remain in place for a 12 hr epoxy cure cycle.  This process is then repeated for the next (3) shifts until all (4) panels of mesh have been attached to the bare APA frame.  When the Mesh Attach Jig is finally dis-engaged from the APA frame, the first level of X foot and head boards are attached with fasteners, and (8) Comb Base assemblies are installed.

%\begin{dunefigure}[APA on a process cart]{fig:apa-process-cart}{APA being moved around a production facility on the process cart.}
%\setlength{\fboxsep}{0pt}
%\setlength{\fboxrule}{0.5pt}
%\fbox{\includegraphics[width=0.5\textwidth]{mesh-stretching.jpg}}
%\fbox{\includegraphics[width=0.5\textwidth]{comb-base-jig.PNG}}
%\end{dunefigure}

%\todo{Pull in a Figure of a Mesh Attach Jig Installation + Mesh panel being held w hold down Bars}


Comb Base Installation Jig (for X-Layer only)
At (4) cross beam locations per each Side of the APA, a Comb Base w/ X-layer combs must be installed.  Currently there are (2) jigs that can be loaded and installed at a time, and after a 6 hr epoxy cure cycle the (2) jigs are removed from the initial locations and placed in another (2) locations, on the same side of the APA.  Thus (4) Comb Bases are installed within (1) operation shift and the APA is left in this horizontal position to cure overnight.  For the following morning shift the APA is flipped 180$^\circ$, supported horizontally and (2) jigs are loaded and installed onto the APA.  A 6 hr epoxy cure cycle is maintained before the (2) jigs are removed, reloaded and installed at the last (2) cross beam locations on the second side of the APA.  The (2) installation jigs are left for an overnight cure in the horizontal position.  

%\todo{Pull in a Figure of the Comb Base Jig Installation + (2) jigs shown attached onto the APA}

Board Gluing/Epoxying Operations - Various Stencils
Once the X-wire layer has been wound onto the APA and solder joints, tension and electrical testing have been completed, another board layer is installed to cover and protect the first layer of wires.  Board specific stencils are now utilized to control the proper dispensed volume of 2216G epoxy to each different board on the APA that requires epoxy. The process of dispensing, striking and measuring the resulting mass of 2216G is similar for each board type that must receive an epoxied board.  Several matrix documents have been created to ensure the dispensing and installation personnel are using the correct stencil and proper dispensed amount of 2216G to complete these operations.   

\todo{Pull in a few Figs of stencils/boards + patterns on back of boards}

\todo{bring in proposed improvements table}


%%%%%%%%%%%%%%%%%%%%%%%%%%%%%%%%%%%%
\subsection{Quality Assurance and Quality Control in APA Production}
\label{sec:fdsp-apa-qa}

\paragraph{Incoming Inspections}  Upon receipt of the rectangular hollow section steel for the frames, a selection procedure is followed to choose the sections of the steel most suited to achieving the geometrical tolerances. This procedure is documented in document PSL-TN-2013-04.  

All circuit boards that get installed to an APA are 100\% inspected for dimensional accuracy prior to being routed through various epoxy and cleaning processes as they are prepped for assembly. Inspection of boards and some other APA materials are documented in spreadsheet format, and if anomalies are found an electronic Non-Conformance report is written.  Materials that can be re-worked to become conforming are set aside from inventory and re-worked.  %If the material does become conforming, it is used on the APA.  
If the material cannot be made usable, the material is kept in a non-conforming area sequestered from usable inventory.   %The non-conforming reports will be reviewed and updated at the end of US APA 4 assembly to determine what materials remain that may be usable for future APA builds.

\paragraph{APA Acceptance Tests} The following parameters will be tested during production. Test criteria are based on requirements listed in the APA design specification and the APA requirements document. The requirements document can be found on DocDB 6416. 

\begin{itemize}
\item Frame flatness: Perform a laser survey to measure the flatness of the assembled bare frame. Three sets of data are compiled into a map that shows the amount of bow, twist, and fold in the frame. Each of these parameters is compared to an allowable amount that will not cause wire plane-to-plane spacing to be out of tolerance (<0.5 mm).  A visual file will be created for each APA from measured data.
%Supplemental Data gathered
A final frame survey is completed after all electrical components have been installed, and the as-built plane-to-plane separations are measured to verify the distance between adjacent wire planes.

\item Mesh to frame connection: To confirm sufficient electrical contact between these two components a resistance measurement is taken in each of 20 zones of mesh bounded by the outside frame perimeter and the four cross beam ribs. 
%(10) measurements on side A, (10) measurements on side B 
This measurement is completed immediately after mesh install, prior to any winding.

\item Wire tension: Once each wire layer has been wound, the tension of each wire is measured using a laser-based system. (It is hoped that an electrical system will become available in the near future.) Wire tensions are required to be in the range $3.5\textrm{--}\SI{7.5}{N}$ for wires longer than \SI{750}{mm} and in the range $2.0\textrm{--}\SI{7.5}{N}$ for wires shorter than \SI{750}{mm}.

%of each wound wire to 5$\pm$1~N
%A laser photo diode head is installed onto the winding robot.
%Natural frequency of each wire is measured individually and converted to tension. 
%If wire tension falls outside of expected limits they are adjusted and re-measured.

\item Cleanliness: APAs are produced inside a Class 100,000 clean area.  Particle counts are completed daily to verify cleanliness of the assembly area.  If counts are found outside of expected limits, measures are taken to re-clean the affected area and check with a follow-up particle count.

\item CR and G bias board testing: Acceptance tests of these boards include leakage current (<0.5 nA) and continuity on each channel.  This test will be performed at room temperature. Individual components will be screened to TBD limits. ProtoDune was used to perform design validation on over 100 boards that were cycled and tested at LN2 temperature. No failures were seen during these tests. 
\end{itemize}

\paragraph{Issue and Action Reports} As assembly issues arise during the build of an APA, these are gathered in an Issue Log for each APA and separate short reports are created to provide details of what caused the occurrence, how the issue was immediately resolved and what measures will be done in the future to ensure the specific issue has a lower risk of occurring.  These Issues and Actions are currently being updated for US APA 2 and 3, and are currently being gathered for US APA 4.  As the documents are finished they will be uploaded to the docdb site.


%Quality Mapping Matrix A Quality Mapping Matrix has been established to ensure certain quality process steps were being performed during the assembly of an APA. We have provided an excel file with links to the associated test result files that are currently gathered during the assembly of an APA.  

%This document was placed on docdb during the summer of 2017 for test results to be reviewed as they were gathered during the build / assembly of US APA 1.  Further updates will be made to this document format to add links to all of the tests performed on US and UK APAs. 

%Incoming Inspection All circuit boards that get installed to an APA are 100\% inspected for dimensional accuracy prior to being routed through various epoxy and cleaning processes as they are prepped for assembly. Inspection of boards and some other APA materials are documented in spreadsheet format, and if anomalies are found an electronic Non-Conformance report is written.  Materials that can be re-worked to become conforming are set aside from inventory and re-worked.  If the material does become conforming, it is used on the APA.  If the material cannot be made usable, the material is kept in a non-conforming area sequestered from usable inventory.   The non-conforming reports will be reviewed and updated at the end of US APA 4 assembly to determine what materials remain that may be usable for future APA builds.

%Issue and Action Reports  As assembly issues arise during the build of an APA, these are gathered in an Issue Log for each APA and separate short reports are created to provide details of what caused the occurrence, how the issue was immediately resolved and what measures will be done in the future to ensure the specific issue has a lower risk of occurring.  These Issues and Actions are currently being updated for US APA 2 and 3, and are currently being gathered for US APA 4.  As the documents are finished they will be uploaded to the docdb site.

%\begin{dunefigure}[QC Table]{fig:qc_table}
%{placeholder until creating an actual table.}
%\includegraphics[width=0.8\textwidth]{QC-table.png} 
%\end{dunefigure}



%\fixme{Ideas from the QA plan - topics to address}
%Work processes: ensure proper training materials for and training of designers, fabricators, etc. 
%Design validation: APA has had design reviews, and is prototyped in ProtoDUNE-SP...
%Acceptance Testing of procured items? 
%Lessons learned 
%Documents and records for all these things.


