%%%%%%%%%%%%%%%%%%%%%%%%%%%%%%%%%%%%%%%%%%%%%%%%%%%%%%%%%%%%%%%%%%%%
\section{Organization and Management}
\label{sec:fdsp-apa-org}

%%%%%%%%%%%%%%%%%%%%%%%%%%%%%%%%%%%%%
\subsection{APA Consortium Organization}
\label{sec:fdsp-apa-org-consortium}

The \dword{apa} consortium comprises \num{21} institutions, of which \num{13} are from the USA, seven from the UK, and one from the Czech Republic. The consortium is organized along the main deliverables, which are the final design of the \dword{apa} and the \dword{apa} production and installation procedures. Since the two main centers for \dword{apa} construction are expected to be located in the USA and the UK, there are usually two leaders of each working group, representing the main stakeholders (Figure~\ref{fig:apa-consortium-structure}). This is particularly important to ensure that common procedures and tooling are developed. 

\begin{dunefigure}[\dword{apa} Consortium organizational chart]{fig:apa-consortium-structure}
{\dword{apa} Consortium organizational chart}
\includegraphics[width=0.95\textwidth,trim=0mm 0mm 0mm 0mm,clip]{apa-consortium-structure.png}
\end{dunefigure}


%%%%%%%%%%%%%%%%%%%%%%%%%%%%%%%%%%%%%%
\subsection{Planning Assumptions}
\label{sec:fdsp-apa-org-assmp}

The planning assumptions are based on having eight to nine \dword{apa} assembly lines, at different locations in the UK and the USA. 
We assume about one year of setup time for the factories.
It will take of the order \num{50} shifts to construct a single \dword{apa}. Assuming a multi-shift system, we will be able to construct the \num{150} \dwords{apa} required for one \dword{spmod} %\SI{10}{kton} module 
within about two years.

%%%%%%%%%%%%%%%%%%%%%%%%%%%%%%%%%%%%%%
\subsection{WBS and Responsibilities}
\label{sec:fdsp-apa-org-wbs}

Here, we only discuss the top-level WBS elements, which are (1)~design, engineering and R\&D, (2)~production setup, (3)~production, (4)~integration, and (5)~installation.

The validation of the design is mainly a responsibility of the university groups and BNL, while engineering and the production setup will be developed at PSL in Madison (USA) and Daresbury Laboratory (UK), where the \dwords{apa} for \dword{pdsp} have been built, with contributions from university groups. In addition to PSL and Daresbury Laboratory, the University of Chicago and Yale University have been identified as candidate sites for the production. The production sites will require significant contributions from university groups during the production process. 

In total, we expect half of the \dwords{apa} to be produced in the USA and half in the UK. The steel for the frames is most likely to be bought from a single vendor. The assembly of the frames will be performed in the USA and the UK separately. The options to assemble the frames in house or in collaboration with industrial partners are still being explored. 

Other significant components include on-\dword{apa} electronics boards. Design modifications relative to \dword{pdsp} are the responsibility of BNL. The boards will be produced by industry, while the testing will be distributed among consortium institutions. The shipping of the \dwords{apa} is the responsibility of the production factories in the USA and the UK.  The integration and installation are a joint responsibility of the Consortium, with ANL providing the interface with the technical coordination group.


%%%%%%%%%%%%%%%%%%%%%%%%%%%%%%%%%%%%%%%
\subsection{High-level Milestones and Schedule}
\label{sec:fdsp-apa-org-cs}

The high-level milestones for the period 2018 to 2024 are given in Table~\ref{tab:milestones} for the periods before and after the \dword{tdr}. The final design of the \dwords{apa} to be proposed in the \dword{tdr} will be informed by the experience of the \dword{pdsp} \dword{apa} production and performance, which will be reviewed in early 2019. Additional design considerations that cannot be directly tested through \dword{pdsp}, such as the two-\dword{apa} assembly and the related cabling issues, will require a full test with cabling of a two-\dword{apa} assembly also in early 2019. The production schedule, the required number of assembly lines, and the location of the production factories will depend on the improvements of the wire winding procedures, which will formally be reviewed in January 2019. The post-\dword{tdr} milestones are driven by the high-level international project milestones and are based on a schedule with one year factory preparation and about two years of \dword{apa} construction time.

\begin{dunetable}[\dword{apa} design and construction milestones]{ll}{tab:milestones}{\dword{apa} design and construction milestones}
Date &  Milestone   \\ \toprowrule
\multicolumn{2}{c}{Pre-\dword{tdr}}\\
 December 2018 & Test two-\dword{apa} assembly   \\
 January 2019 & Formal review of complete modifications to the winder design\\
 February 2019 & Formal review of \dword{pdsp} \dword{apa} performance \\
February 2019 & Complete assembly test of \dword{fd} prototype \dword{apa}\\
March 2019 & Decision on location of factories and required number of assembly lines \\
March 2019 & \dword{apa} cost estimate for \dword{spmod} module \\
March 2019 & \dword{apa} schedule for \dword{spmod} \\
April 2019 & \dword{apa} section of \dword{tdr} delivered \\
\multicolumn{2}{c}{Post-\dword{tdr}}\\
2020 & Preparation of \dword{apa} factories \\
2021 -- 2023 & Construction of \dwords{apa} \\
2022/3 & Installation of \dwords{apa} in \dword{spmod} 1\\
2024 & Commissioning of \dword{spmod} 1 \\
\end{dunetable}
