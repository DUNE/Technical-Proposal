%%%%%%%%%%%%%%%%%%%%%%%%%%%%%%%%%%%%%%%%%%%%%%%%%%%%%%%%%%%%%%%%%%%%
\section{Safety} % (1 page)}
\label{sec:fdsp-apa-safety}

%As well as the moral obligation to avoid harming anyone, there are laws that require machines to be safe, and sound economic reasons for avoiding accidents. 
Gaining on the experience of ProtoDUNE, a full safety analysis will be performed and a set of safe work procedures developed for all stages of the fabrication process before the start of DUNE APA production.  In the final design of the winding machine, central to the production process, safety must be taken into account right from the design stage and must be kept in mind at all stages in the life of the machine: design, manufacture, installation, adjustment, operation, and maintenance.     

%In the UK, machines have to comply with the Essential Health and Safety Requirements (EHSRs) listed in Annex I of the Machinery Directive (2006/42/EC), thus setting a common minimum level of protection across the EEA (European Economic Area).
%In order for a machine (or other equipment) to be made safe it is necessary to assess the risks that can result from its use. 
%the following describes the modes of operation considered a requirement for the safe operation of the winding machine.
%\todo{tables from the word doc sent by Alan to be incorporated}