\chapter{Design Motivation}
\label{ch:fdsp-apa-design}

%%%%%%%%%%%%%%%%%%%%%%%%%%%%%%%%%%%%%%%%%%%%%%%%%%%%%%%%%%%%%%%%%%%%
\section{Introduction to Single-Phase Far Detector in DUNE}
\label{sec:fdsp-design-intro}

The DUNE \dword{spmod} will be the culmination of several decades
of \lartpc technology development, and once operational, it will open new windows of opportunity in the study
of neutrinos.  DUNE's rich physics program, with discovery
potential for \dword{cpv} in the neutrino sector, and capability to make
significant observations of nucleon decay and astrophysical events, is enabled
by the exquisite resolution of the \lartpc detector technique.

Experience with design, construction, operation, and data
analysis with numerous single-phase \lartpc experiments and prototypes has informed the approach to
realizing the massive DUNE \dwords{spmod}. Each far \dword{detmodule} will feature the largest \lartpc{}s ever
constructed, at approximately \nominalmodsize active volume each.  Aside from the
challenges inherit in such a large undertaking, DUNE presents the added complication of construction and operation in a location
that is one mile underground with limited access.

The design of the DUNE \dword{spmod} that is presented in this document
reflects an approach to achieve the science goals of the experiment, and
address the challenges of constructing and operating a massive detector in a deep
underground environment.


%%%%%%%%%%%%%%%%%%%%%%%%%%%%%%%%%%%%
\section{Single-Phase \lartpc Operational Principle}
\label{sec:fdsp-design-op}

The precision tracking and calorimetry offered by the single-phase \lartpc
technology provides excellent ability to identify interactions of interest
while mitigating sources of background.  The operational principle of a
single-phase \lartpc is summarized here for reference.

Charged particles traversing the active volume of the \lartpc ionize the medium,
while also producing scintillation light.  The ionization drifts along
an \efield that is present throughout the volume, towards a series of
anode layers.  Each anode layer is composed of finely spaced wires arranged at
characteristic angles, and appropriate biasing of these wires allows the
ionization to drift through the successive layers before terminating on a wire
in the Collection layer.  The individual wires in the anode layers can be
instrumented with low-noise electronics that record the current in the wire as
a function of time.  The argon scintillation light, which at \SI{128}{nm} wavelength
is deep in the UV spectrum, can be recorded by photon detectors that shift the
wavelength closer to the visible spectrum and subsequently record the time and
pulse characteristics of the incident light.

%\begin{dunefigure}[\lartpc Single-Phase Operational
 %   Principle]{fig:design_lartpcdiagram}{A diagram depicting the operational   principle of a single-phase \lartpc.  Ionization produced in the TPC will  drift towards the anode, creating signals in the wires that are recorded   by readout electronics.  Scintillation light produced in the TPC can be   captured and recorded by photon detectors integrated into the anode  structure.}
%\includegraphics[width=0.4\textwidth]{lartpc_diagram.jpg}
%\end{dunefigure}

%\fixme{Include diagram of basic single-phase \lartpc operational principle.  Expand previous paragraph.}

The performance of the \lartpc hinges on several key factors.  First, the
purity of the liquid argon must be extremely high in order for ionization to
be able to drift over several meters towards the anode planes.  The levels of
electronegative contaminants (e.g., oxygen, water), must be reduced and
maintained to \dword{ppt} levels in order to achieve minimum charge attenuation
over the longest drift lengths in the \lartpc.   Second, the electronic readout
of the \lartpc requires very low noise levels so that the signal of drifting
ionization is clearly discernible over the baseline of the electronics.  
Third, a uniform \efield must be established over the detector volume, requiring a robust and stable high voltage system.  Finally, the sheer size of the \dword{spmod} means that once it is filled with \lar, all components within the cryostat are inaccessible for decades.  All internal devices must have long operating lifetimes at \lar temperatures.

%%%%%%%%%%%%%%%%%%%%%%%%%%%%%%%%%%%
\section{Motivation of Single-Phase \lartpc Design at DUNE}
\label{sec:fdsp-design-impl}

The DUNE Single-Phase Far Detector builds on several decades of experience in designing, constructing, and operating \lartpc{}s.  It implements unique design features to maximize the capability of the experiment, as well as new features motivated by the unprecedented scale of the Far Detector modules and the deep underground location where construction will occur.

Among the features driven by the underground location of the experiment, all detector components are sized to fit within the constraints of the \surf shafts and access pathways.

A drift time of several milliseconds is typical for ionization to arrive at the anode wires after drifting several meters.  This lengthy duration of time, as well as aspects of the DUNE physics program looking for rare and low-energy processes, makes the deep underground location essential for the \dword{spmod}.  The  $\sim$\SI{1.5}{km} overburden of earth greatly reduces the rate of cosmic rays reaching the active volume of the \dword{spmod}, greatly enhancing the ability to search for rare and low-energy signatures without the influence of cosmic-induced backgrounds.  








