\chapter{Design Motivation}
\label{ch:fdsp-apa-design}

%%%%%%%%%%%%%%%%%%%%%%%%%%%%%%%%%%%%%%%%%%%%%%%%%%%%%%%%%%%%%%%%%%%%
\section{Introduction to Single-Phase Far Detector in DUNE}
\label{sec:fdsp-design-intro}

The DUNE Single-Phase Far Detector will be the culmination of several decades
of LArTPC technology development, and once operational it will open new windows of opportunity in the study
of neutrinos.  DUNE's rich physics program, with discovery
potential for CP-Violation in the neutrino sector, and capability to make
significant observations of nucleon decay and astrophysical events, is enabled
by the exquisite resolution of the LArTPC detector technique.

Experience with design, construction, operation, and data
analysis with numerous single-phase LArTPC experiments and prototypes has informed the approach to
realizing the massive DUNE Single-Phase Far Detectors. Each far detector module will feature the largest LArTPCs ever
constructed, at approximately 10 kTon active volume each.  Aside from the
challenges inherit in such a large undertaking, DUNE presents the added complication of construction and operation in a location
that is one mile underground with limited access.

The design of the DUNE Single-Phase Far Detector module that is presented in this document
reflects an approach to acheive the science goals of the experiment, and
address the challenges of constructing and operating a massive detector in an
underground environment.


%%%%%%%%%%%%%%%%%%%%%%%%%%%%%%%%%%%%
\section{Single-Phase LArTPC Operational Principle}
\label{sec:fdsp-design-op}

The precision tracking and calorimetry offered by the single-phase LArTPC
technology provides excellent ability to identify interactions of interest
while mitigating sources of background.  The operational principle of a
single-phase LArTPC is summarized here for reference.

Charged particles traversing the active volume of the LArTPC ionize the medium,
while also producing scintillation light.  The ionization drifts along
an electric-field that is present throughout the volume, towards a series of
anode layers.  Each anode layer is composed of finely-spaced wires arranged at
characteristic angles, and appropriate biasing of these wires allows the
ionization to drift through the successive layers before terminating on a wire
in the Collection layer.  The individual wires in the anode layers can be
instrumented with low-noise electronics that record the current in the wire as
a function of time.  The argon scintillation light, which at 128 nm wavelength
is deep in the UV spectrum, can be recorded by photon detectors that shift the
wavelength closer to the visible spectrum and subsequently record the time and
pulse characteristics of the incident light.

\fixme{Include diagram of basic single-phase LArTPC operational principle}

The performance of the LArTPC hinges on several key factors.  First, the
purity of the liquid argon must be extremely high in order for ionization to
be able to drift over several meters towards the anode planes.  The levels of
electronegative contaminatns (e.g. oxygen, water), must be reduced and
maintained to $\sim$ppt levels in order to achieve minimum charge attenuation
over the longest drift lengths in the LArTPC.   Second, the electronic readout
of the LArTPC requires very low noise levels so that the signal of drifting
ionization is clearly discernable over the baseline of the electronics.  

%%%%%%%%%%%%%%%%%%%%%%%%%%%%%%%%%%%
\section{Implementation of Single-Phase LArTPC Design at DUNE}
\label{sec:fdsp-design-impl}

The DUNE Single-Phase Far Detector builds on several decades of experience in
designing, constructing, and operating LArPTCs.  It implements several
unique design features to maximize the capability of the experiment, as well
as new features specific to the unprecedented scale of the Far Detector
modules and the deep underground location where construction will occur.

Among the novel features of the Single-Phase Far Detector LArTPC are the
presence of ``wrapped'' anode wires that follow a helical trajectory around
the height of the APA.  This design choice was made to minimize the need to
tile electronic readout around the perimeter of the APA, which would lead to
dead space between neighboring APAs.  This choice also was driven by
reconstruction performance, with the angle of the wrap chosen such that a
given Induction plane wire does not intersect a given Collection plane wire
more than once, which greatly reduces pathologies in pattern recognition. 

Among the features driven by the underground location of the experiment, all
detector components are sized to fit within the constraints of the SURF shafts
and access pathways.








