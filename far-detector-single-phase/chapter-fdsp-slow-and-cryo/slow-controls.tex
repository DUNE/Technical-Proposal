%%%%%%%%%%%%%%%%%%%%%%%%%%%%%%%%%%%%%%%%%%%%%%%%%%%%%%%%%%%%%%%%%%%%
\section{Slow Controls}
\label{sec:fdsp-slow-cryo-ctrl}


\fixme{Include an image of the subsystem, indicating its parts. Show how the system fits into the overall system).}


%%%%%%%%%%%%%%%%%%%%%%%%%%%%%%%%%%%
\subsection{Slow Controls Hardware}
\label{sec:fdsp-slow-cryo-hdwr}

% % % % Alec
\subsubsection{Slow Controls Network Hardware}
\label{sec:fdsp-slow-cryo-slow-network}
The Slow Controls data originates from the sensors discussed in
Sec.~\ref{sec:fdsp-cryo-instr} and
Sec.~\ref{sec:fdsp-slow-cryo-slow-dsp}, then heads towards the central
CISC database housed in the CUC DAQ room
(Sec.~\ref{sec:fdsp-slow-cryo-slow-compute}).  It gets there over
conventional network hardware from any sensors located in the cryogenic
plant.  However, the readouts which are located in the racks on top the
cryostats have to be careful about grounding and noise.  Therefore, each
rack on the cryostat will have a small network switch which will send
any network traffic from that rack to the CUC via a fiber transponder.

Network traffic out of SURF to Fermilab will be primarily database calls
to that central CISC DB: either from monitoring applications, or from
database replication to the offline version of the CISC DB.  This
traffic is of a low enough bandwidth that the proposed general purpose
links both out of the mine and back to Fermilab can accommodate it.

% % % % Alec
\subsubsection{Slow Controls Computing Hardware}
\label{sec:fdsp-slow-cryo-slow-compute}

Up to two racks of space and appropriate power and cooling are available
in the CUC's DAQ server room for CISC usage.  Somewhat less space than
that is currently envisioned: Two servers (a primary server and a
replicated backup) suitable for the needed relational database discussed
in Sec.~\ref{sec:fdsp-slow-cryo-sw} will be there, with an additional
two servers to perform front-end monitoring interface services: for
example, assembling dynamic CISC monitoring web pages from the adjacent
databases.  Any special purpose software, such as iFix or EPICS, would
also run here: two more servers (for a total of six) will accommodate
these programs.\fixme{I am completely making up how many special purpose
  iFix machines there should be: need input from the cryo people}
Replicating this setup on a per-module basis would allow for easier
commissioning and independent operation, accommodate different module
design (and the resulting differences in database tables), and ensure
sufficient capacity.  Including fours sets of networking hardware, this
would fit tightly into one rack or very comfortably into two.

% % % % Alec and Sowjanya
\subsubsection{Slow Controls Signal Processing Hardware}
\label{sec:fdsp-slow-cryo-slow-dsp}
\fixme{Is this the place for the laundry list of Things to Be Monitored?
 ({\it but which are not cryo related})}



%%%%%%%%%%%%%%%%%%%%%%%%%%%%%%%%%%%
% Alec
\subsection{Slow Controls Infrastructure}
\label{sec:fdsp-slow-cryo-slow-infra}

%%%%%%%%%%%%%%%%%%%%%%%%%%%%%%%%%%
% Sowjanya
\subsection{Slow Controls Software}
\label{sec:fdsp-slow-cryo-sw}
\fixme{Glenn's talk from the parallel session @CERN is a great starting
  point here}

%%%%%%%%%%%%%%%%%%%%%%%%%%%%%%%%%%
% Ed T
\subsection{Slow Controls Quantities}
\label{sec:fdsp-slow-cryo-quant}

%%%%%%%%%%%%%%%%%%%%%%%%%%%%%%%%%%
% Sowjanya and Anselmo
\subsection{Local Integration}
\label{sec:fdsp-slow-cryo-slow-loc-integ}

%%%%%%%%%%%%%%%%%%%%%%%%%%%%%%%%%%%  
\subsection{Quality Assurance}
\label{sec:fdsp-slow-cryo-slow-qa}

\fixme{need this one? not assigned}

%%%%%%%%%%%%%%%%%%%%%%%%%%%%%%%%%%%%%%%%%%%%%%%%%%%%%%%%%%%%%%%%%%%%
\section{Production and Assembly}
\label{sec:fdsp-slow-cryo-prod-assy}

\fixme{This section not needed? Not assigned; may be addressed in earlier sections.}

