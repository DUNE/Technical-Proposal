%%%%%%%%%%%%%%%%%%%%%%%%%%%%%%%%%%%%%%%%%%%%%%%%%%%%%%%%%%%%%%%%%%%%
\section{Installation, Integration and Commissioning}
\label{sec:fdsp-slow-cryo-install}
% anselmo, sowjanya

%Then, the installation of cryogenics instrumentation devices inside the cryostat will be done in several phases:
%\begin{enumerate}
%\item Before detector installation: Individual temperature sensors anchored to the cryogenic pipes at the bottom of the cryostat and static T-gradient monitors
%  will be installed right after the installation of the pipes. Additional level meters could be also installed during this phase. 
%\item During detector installation: Temperature sensors anchored to the top ground planes will be installed in several steps.
%  In principle as soon as a CPA enters the cryostat and stored there, the corresponding cables and sensors can be mounted on its ground planes. Cables exceeding the dimensions of
%  the ground planes will be roled and stored temperarilly there until they are put in their final position. At this moment cables will be routed towards their ports.   
%\item After detector installation: dynamic T-gradient monitors and purity monitors will be deployed  once the detector in installed. 
%\end{enumerate}

%Since purity monitors will be most likely not hang from a flange they will have to be installed at an earlier stange ... 


%Special treatment need the long devices as the purity monitor system, and T-gradient monitors.

%The installation of gas analyzers will be inst


%Integration: Initial integration between the two systems will be accomplished at vertical slice and DAQ test stands:
%test of cameras for the detection of sparks, testing SC monitoring/control of HV PS, testing HV interlock status bits, etc.
%Integration/installation of CISC devices (cameras, RTDs) on ground planes will be tested at system integration/assembly sites. Final integration will take place as the two systems are commissioned.


%Some parts of the CISC system will be commissioned prior to HV commissioning, and before cryostat filling with LAr.
%This is the case for RTDs on GPs, cold and inspection cameras and thermal interlocks for PS. Final commissioning of
%those systems will be done once the cryostat is filled, since operation will be different in the presence of LAr.
%Commissioning the control/monitoring of HV PS and any related hardware interlocks could probably be done at an early
%stage as well, provided no real HV is provided to the cathode/field-cages. Final commissioning will be done once HV is
%switched on. This will also require coordination from other groups such as LBNF, DAQ, APA etc. The commissioning of the
%interfacing elements should follow naturally after (successful) integration testing.

\subsection{Cryogenics Internal Piping}
\label{sec:fdsp-slow-cryo-install-pipes}

Installation of internal cryogenic pipes should occur soon after the cryostat is completed.

\subsection{Purity Monitors}
\label{sec:fdsp-slow-cryo-instal-pm}

The purity monitor system will be built in a modular way, such that is can be assembled outside of DUNE-FD cryostat.  The assembly of the purity monitors themselves would occur outside of the cryostat and would include everything described in the previous section.  The installation of the purity monitor system can then be carried out with the least number of steps inside the cryostat.  The assembly itself would come into the cryostat with the three individual purity monitors mounted to the support tubes and no HV cables or optical fibers installed yet.  The support tube at the top and bottom of the assembly would then be mounted to the brackets inside the cryostat that could be attached to the cables trays and/or the detector support structure.  In parallel to this work, the front-end electronics and light source can be installed on the top of the cryostat, along with the installation of the electronics and power supplies into the electronics rack.  

Integration would begin by running the HV cables and optical fibers to the purity monitors, coming from the top of the cryostat.  The HV cables would be attached to the HV feedthroughs with enough length to reach each of the respective purity monitors.  The cables would be run through the port reserved for the purity monitor system, along cable trays inside the cryostat until they reach the purity monitor system, and would then be terminated through the support tube down to each of the purity monitors.  Each purity monitor will have three HV cables that connect it to the feedthrough, and then along to the front-end electronics.  The optical fibers would then be run through the special optical fiber feedthrough, into the cryostat, and would be guided to the purity monitor system either using the cables trays or guide tubes.  Which ever solution is adopted for running the optical fibers from the feedthrough to the purity monitor system, it should protect the fibers from accidental breakage during the remainder of the detector and instrumentation installation process.  The optical fibers would then be run inside of the purity monitor support tube and to the respective purity monitors terminating them at the photocathode of each, protecting them from breakage near the purity monitor system itself.

Integration would continue with the connection of the HV cables between the feedthrough and the system front-end electronics, and then optical fibers to the light source.  The cables connecting the front-end electronics and the light source to the electronics rack would also be run and connected at this point.  This would allow for the system to be turned on and the software to begin testing the various components and connections.  Once it was confirmed that all connections had been successfully made, the integration to the slow controls system would be made, first by establishing communications between the two systems and then transferring mach data between them to ensure successful exchange of important system parameters and measurements.  

Commissioning of the purity monitor system would officially be done once the cryostat had been purged and a gaseous argon atmosphere was present.  At this point the HV for the purity monitors could be ramped up without the fear of discharge through the air, and the light source the turned on.  Although the drift electron lifetime in the gaseous argon would be very large and therefore not really measurable with the purity monitors themselves, the signal strength at both the cathode and anode would give a good indication of how well the light source is generating drift electrons from the photocathode and that they are successfully drifted to the anode by looking at the signal strength at the anode and comparing it to that of the cathode.


\subsection{Thermometers}
\label{sec:fdsp-slow-cryo-instal-th}


Individual temperature sensors on pipes and cryostat membrane should be installed prior to any detector component, right after the installation of the pipes.
First, all cable supports will be anchored to pipes. Then each cable will be routed individually starting from sensor end (with IDC-4 female connector but no sensor)
to the corresponding cryostat port. Once all cables going to the same port have been routed, they will be cut to the same length such that they can be properly soldered
to the pins of the SUBD-25 connectors on the flange. In order to avoid damaging the sensors, those will be installed at a later stage just before unfolding the bottom ground planes.

Static T-gradient monitors should be installed before the outer APAs. Thus, the best moment could be right after the installation of the pipes
and before the installation of individual sensors. This will proceed in several steps: i) installation of the two stainless steel strings to the bottom and top corners of the cryostat,
ii) tension and verticality checks, iii) installation of cable supports in one of the strings, iv) installation of sensor supports in the other string, v) cable routing starting from
the sensor end towards the corresponding cryostat port, vi) cut all cables at the same point in that port, vii) solder cable wires to the pins of the SUBD-25 connectors on the flange,
viii) plug sensors onto IDC-4 connectors at a later stage, just before moving corresponding APA insto its final position. 

Individual sensors on top ground plane will have to be integrated with the ground planes. For each CPA (with its corresponding 4 GP modules)
going inside the cryostat, cable and sensor supports will be anchored to the GP threaded rods as soon as possible.
Once the CPA is moved into its final position and its top GPs are ready to be unfolded, sensors on those GPs will be installed. Once unfolded, cables 
exceeding the GP limits can be routed to the corresponding cryostat port either using neighboring GPs or DSS I-bins. 


Dynamic T-gradient monitors will be installed after the completion of the detector ...


\fixme{Missing text for Dynamic T-gradients}

Commissioning of all thermometers will proceed in several steps. Since in a first stage only cables will be installed,
the readout performance and the noise level inside the cryostat will be
tested with precision resistors. Once sensors are installed the entire chain will be check again at room temperature.
The final commissioning phase will be done during and after cryostat filling.  


\subsection{Gas Analyzers}
\label{sec:fdsp-slow-cryo-install-ga}
 
The Gas Analyzers need to be installed prior to the Piston Purge and Gas recirculation phases of the Cryostat commissioning. They should be installed near the location of the tubing switchyard to minimize tubing runs and for convenience when switching the sampling points and gas analyzers. Since each is a stand alone module, a single rack with shelves, should be adequate to house the modules.

Concerning the integration, the gas analyzers typically have an analog output (4-20 \si{mA} or 0-10\si{V}) which maps to the input range of the analyzers. They also usually have a number of relays that indicate the scale they are currently running. These outputs can be connected to the slow controls for readout. However it is preferred to use a digital readout since this directly gives the analyzer reading at any scale. Currently there are a number of digital output connections, ranging from RS-232, RS-485, USB, and ethernet. At the time of purchase, one can choose the preferred option, since the protocol is likely to evolve. The readout usually responds to a simple set of text commands. Due to the natural time scales of the gas analyzers, and lags in the gas delivery times (depending on the length of the tubing runs), sampling at the minute level most likely is adequate.

For commissioning, before the beginning of the gas phase of the Cryostat commissioning, the analyzers need to be brought online and calibrated. Calibration varies for the different modules, but often requires using Argon gas with both zero contaminants (usually removed with a local inline filter)for the zero of the analyzer, and Argon with a known level of the contaminant to check the scale. Since the start of the Gas phase of the Cryostat begins with normal air, the more sensitive analyzers will be valved off at the switchyard to prevent overloading their inputs (and potentially saturating their detectors). As the Argon Purge and gas recirculation progress, the various analyzers will be valved back in when the contaminant levels reach the upper limits of the analyzer ranges. 

\subsection{Liquid Level Monitoring}
\label{sec:fdsp-slow-cryo-install-llm}


\subsection{Cameras}
\label{sec:fdsp-slow-cryo-install-c}


\subsection{Light emitting system}
\label{sec:fdsp-slow-cryo-install-les}


\subsection{Slow Controls Hardware}
\label{sec:fdsp-slow-cryo-install-sc-hard}



%%%%%%%%%%%%%%%%%%%%%%%%%%%%%%%%%%%%
\subsection{Transport and Handling}
\label{sec:fdsp-slow-cryo-install-transport}

Most instrumentation devices will be shiped to SURF in pieces and mouted onsite. The proper 
Instrumentation devices are in general small except the support structures for Purity Monitors and T-gradient monitors,
which will cover the entire height of the cryostat. Being the load on those structures relatively small (<100 kg) they can be fabricated in parts of less than 3 m,
which can be easily transported to SURF. Those parts can also be movedeasily transported down the shaft and through the tunnels.
All instrumentation devices except the dynamic T-Gradient monitors, which will be introduced into the cryostat through the cryostat port above, can be
moved into the cryostat without the crane.

\fixme{Missing text for  cryogenics internal piping}


%%%%%%%%%%%%%%%%%%%%%%%%%%%%%%%%%%%
%\subsection{Integration Facility Operations}
%\label{sec:fdsp-slow-cryo-install-facil-ops}


%%%%%%%%%%%%%%%%%%%%%%%%%%%%%%%%%%%
%\subsection{Underground Operations}
%\label{sec:fdsp-slow-cryo-install-undergr}


%%%%%%%%%%%%%%%%%%%%%%%%%%%%%%%%%%%
%\subsection{Integration }
%\label{sec:fdsp-slow-cryo-install-integration}

%Slow controls needs to be integrated with HV devices as soon as possible 


%%%%%%%%%%%%%%%%%%%%%%%%%%%%%%%%%%%
%\subsection{Commissioning}
%\label{sec:fdsp-slow-cryo-install-commiss}


