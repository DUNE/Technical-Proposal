%%%%%%%%%%%%%%%%%%%%%%%%%%%%%%%%%%%
\subsection{Purity Monitors}
\label{sec:fdsp-slow-cryo-purity-mon}
% andrew, jianming, laura
A purity monitor (PrM) is a standalone miniature TPC which measures the lifetime of photoelectrons generated by a UV-illuminated gold photocathode to measure a given substance's purity. It has high sensitivity to the purity of liquid argon and does not rely on the LArTPC high voltage, electronics, and data acquisition. PrMs are important detectors for guaranteeing successful commissioning and operation of the LArTPC and meet the requirement to measure position-dependent purity necessary to achieve DUNE's physics goal. The purity monitors also have the potential to be developed as a calibration tool that provides high precision and real-time electron lifetime
measurements for wire-by-wire detector calibration. 

A purity monitor's basic design is based on those used by the ICARUS and LAPD experiments~\cite{Carugno:1990kd, Adamowski:2014daa} (Figure~\ref{fig:prm}). It is a double-gridded ion chamber immersed in the liquid argon volume. It measures the electron drift lifetime between its anode and cathode. The electrons are generated by the purity monitor's UV-illuminated gold photocathode. The UV is generated by a xenon flash lamp. The electron lifetime in liquid argon is inversely proportional to the electronegative impurity concentration. The fraction of electrons generated at the cathode that arrive at the anode ($Q_A/Q_C$) after the electron drift time $t$ is a measure of the electron lifetime $\tau$: $Q_A/Q_C=e^{-t/\tau}$. 
