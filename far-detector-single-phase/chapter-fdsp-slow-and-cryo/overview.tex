%%%%%%%%%%%%%%%%%%%%%%%%%%%%%%%%%%%%%%%%%%%%%%%%%%%%%%%%%%%%%%%%%%%%
\section{Slow Controls and Cryogenics Instrumentation Overview}
\label{sec:fdsp-slow-cryo-ov}

% Glenn and Carmen


%% Anselmo: This is just some text I had for ProtoDUNE and could be useful. Fill free to ignore it completely of you want

%% The purpuse of the cryogenics instrumentation devices is to guaranty a high LAr purity, which is essential for the detector performance.  
%% It has been demonstrated in a small LAr-TPC prototype at Fermilab (the 35-ton prototype) that vertical temperature gradients
%% as low as 0.02 K may result in the liquid argon being stratified, with colder argon staying at the bottom of the cryostat and not mixing
%% with the LAr above it. The lack of proper LAr recirculation causes, in turn, an accumulation of impurities that
%% can severely affect the detector performance through a reduced electron lifetime. The design of the DUNE cryogenics
%% system is such that the LAr stratification problem
%% is in principle solved, as LAr will be injected at the bottom of the cryostat with a slightly higher temperature than the bulk LAr,
%% favouring LAr recirculation. Nevertheless, this concept needs to be demonstrated in the real detector. 
%% For this purpose, a set of high precision temperature sensors will be installed at different heights within the cryostat,
%% together with three purity monitors to measure the electron lifetime.  
