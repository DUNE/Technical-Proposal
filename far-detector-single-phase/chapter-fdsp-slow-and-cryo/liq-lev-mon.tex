%%%%%%%%%%%%%%%%%%%%%%%%%%%%%%%%%%%
\subsection{Liquid Level Monitoring}
\label{sec:fdsp-slow-cryo-liq-lev}
% john L

Experience with level sensing in large liquid argon detectors suggests
different solutions for different goals.  Liquid sensing while filling needs
less accuracy of a much larger range than level sensing in the detector during
static operations.
The goals for the level sensing system are basic level sensing when filling, and precise level sensing during static operations. These goals will be met using pressure, temperature, and capacitive sensors.

%The plan is to use a combination of argon level sensors to provide redundant fault tolerant measurements for the liquid level.  Rough levels can be determined by comparing pressures at various heights below the top of the cavity.  Multiple sensors at a given depth combined with multiple depths can give reliable redundant measurements.  To ensure the top of the argon level is known very accurately a few capacitive level sensors can be used and compared.  A rough cut off for filling can be established using a thermal probe sensitive the temperature difference between the liquid and vapor.

In principle two differential pressure level meters are foreseen for the SP detector. This has 0.1\% precision (so 1.5 cm over 15 m) which is sufficient since the only requirement is that the LAr level was well above the ground planes.
In the DP detector, the level measurement is much more important since the Charge Readout Plane (CRP) should be at a fixed well controlled distance from the LAr surface. This is why they have sub-milimeter capacitive 
devices on the CRP itself, apart of the differential pressure level meters (used by the cryogenics system). 
 
