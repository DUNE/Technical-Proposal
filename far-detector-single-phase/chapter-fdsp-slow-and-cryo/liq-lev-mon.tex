%%%%%%%%%%%%%%%%%%%%%%%%%%%%%%%%%%%
\subsection{Liquid Level Monitoring}
\label{sec:fdsp-slow-cryo-liq-lev}
% john L

The goals for the level monitoring system are basic level sensing when filling, and precise level sensing during static operations. 

For filling the detector the differential pressure between the top of the
detector and known points below it can be converted to depth with the known
density of liquid argon.

During operation, the purpose of liquid level monitoring is twofold: it is required by the cryogenics system to tune the LAr flow and by the detector
to guaranty that the top ground planes are always submerged (otherwise there will be high risk of dielectric breakdown).  
Two differential pressure level meters will be installed as part of the cryogenics system, one in each side of the detector.
Those will have a precision of 0.1\%, which corresponds to 14 mm at the nominal LAr surface.
This precision is sufficient for the detector, since the plan is to kept the LAr surface at
least 20 cm above the ground planes (this is the value used for the HV interlock in
ProtoDUNE-SP). Thus, no additional level meters are required. However, the need of a redundant system, probably based on a different tecnology
(array of temperature sensors or/and capacitive sensors), is under discussion. That system could be used for the HV intelock. 


 
