%%%%%%%%%%%%%%%%%%%%%%%%%%%%%%%%%%%
\subsection{Cryogenic Internal Piping}
\label{sec:fdsp-slow-cryo-int-piping}
% david m

\Fixme{To be written}


%We are talking about almost 300 m of 3” pipe and almost 450 m of 2” pipes, they will take some space.
%The majority of them could probably be transported in bunches. Assuming 6 m as long sections, there would be 50 of 3” pipe and 75 of 2” pipe.
%I can see them packing together 10-15 6-m long sections, which will give us 5 pallets/boxes each (assuming 10 sections of the 3” and 15 of the 2”).
%Each one would be about 6.2 m long x 0.8 m wide x 0.5 m high. They could also just bundle the pipes together, but I think they would still sit on a
%pallet even in this case. These would need to be delivered to the site, stored, transported down, and stored again before they are used.
%Depending on when they are installed they could (or not) be stored inside the cryostat itself or in one of the drifts (assuming that CF is fine with that).

%Then there are the 20 cool down pipes for the top. Those are easier: they are CF DN250, so the OD of the flange is 304 mm.
%The length may vary depending on the height of the feedthrough, but shouldn’t be more than ~2.2 m. The box could be something like 2.6 m x 0.6 m x 0.6 m. 20 of them.
%They could probably be a little smaller, but I am not in the shipping industry. This is just a ROM. They could also put many of them in the same box and save space,
%but they will soon run into shaft limits. These would need to be delivered to the site, stored, transported down, and stored again before they are used.
%The last step could take place on top of the cryostat.
