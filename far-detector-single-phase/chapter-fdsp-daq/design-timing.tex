\subsection{Timing and Synchronization}
\label{sec:fd-daq-timing}

\metainfo{David Cussans \& Kostas Manolopoulos.  This is an SP-specific section.  It's file is \texttt{far-detector-single-phase/chapter-fdsp-daq/design-timing.tex}}

%Describe the generation of timing/synchronisation signals and and distribution to the detectors.

All components of the \dword{spmod} are
synchronized to a common clock.  In order make full use of the
information from the \dword{pds}, the common clock must be
aligned within a single \dword{detunit} with an accuracy of $O\,1$\si{ns}.
In order to form a common trigger for \dword{snb} between
\dwords{detmodule}, the timing between them must be aligned with an
accuracy of $O\,1$\si{ms}.  However, a tighter constraint is the need to
calibrate the common clock to universal time (derived from GPS) in
order to adjust the data selection algorithm inside an accelerator
spill, which requires an absolute accuracy of $O\,1$\si{\micro\s}.

\single and \dual \dwords{detmodule} use different timing systems,
driven by the different technical requirements and development history
of the two technologies. %systems. 
A \dword{spmod} has many more
timing end points than a \dword{dpmod} and many of the end points
are simpler than the end points in the \dual{}, for example a \dword{wib}
versus \dword{utca} crate. Both systems have been sucessfully prototyped.

The DUNE \dword{spmod} uses a development of the \dword{pdsp} 
timing system. Synchronization messages are transmitted over a serial
data stream with the clock embedded in the data. The format is
described in DUNE DocDB-1651~\cite{docdb-1651}. Figure~\ref{fig:daq-readout-timing}
shows the overall arrangement of components within the \single
Timing System (SPTS). 
\fixme{need a gloss term?}
A stable master clock, disciplined with a \SI{10}{\MHz}
reference is used in the SPTS. A \dword{pps} is
also received by the system and is time-stamped onto a counter clocked
by the SPTS master clock, however the periodic synchronization
messages distributed to the \dword{sp} \dword{detmodule} are an exact number
of clock cycles apart even if there is jitter in the \dword{pps}.
%\fixme{Need reference added for DocDB-1651.}  done - ATH, 4/16/18

The GPS signal is encoded onto optical fiber and transmitted to the
\dword{cuc}, where it is converted back to an RF signal on coaxial cable and
used as the input to a GPS displined oscillator. The oscillator module
also houses a IEEE 1588 (PTP) grandmaster and an NTP server. The PTP
grandmaster provides a timing signal for the \dword{dp} \dword{wr}
timing network. The NTP server provides an absolute time for the
\dword{pps}. The SPTS relates its time counter onto GPS time by
timestamping the \dword{pps} onto the SPTS time counter and reading
the time in software from the NTP server.

The latency from the GPS antenna on the surface to the GPS receiver in
the \dword{cuc} will be measured by optical time domain reflectometry at
installation. Given the modest absolute time accuracy required
(sufficient to select data within an accelerator spill) dynamic
monitoring of this delay is not required.

The \dword{wr} synchronization signals from the \dword{dp} \dword{detmodule} are
time-stamped onto the SPTS clock domain and the SPTS synchronization
signals are time stamped onto the \dword{dp} clock domain. This allows
the timing in the \dword{sp} and \dword{dp} \dwords{detmodule} to be
aligned. A similar scheme is used to relate the \dword{pdsp}
 timing domain to the beam instrumentation
\dword{wr} time domain.

In order to provide redundancy, and also the ability to easily detect
issues with the timing path, two independent GPS systems are used. One
with an antenna at the head of the Yates Shaft, the other with an
antenna at the head of the Ross Shaft. The two independent timing
paths are brought together in the same rack in the \dword{cuc}. Using 1:2
fiber splitters one SPTS unit can be left as a hot spare while the
other is active. This also allows testing of new firmware and software
during comissioning without the risk of losing the SPTS if a bug is
introduced.


\begin{dunefigure}[Arrangement of components in DUNE timing system]{fig:daq-readout-timing}
  {Illustration of the components in the DUNE timing system.}
\includegraphics[width=0.8\textwidth]{DUNE_Timing_overall.pdf}
\end{dunefigure}

All the custom electronic components for the SPTS are contained in two
\dword{utca} shelves. At any one time one is active and the other is a
hot spare. The \SI{10}{\MHz} reference clock and the \dword{pps} are received
by a single width \dword{amc} at the center of the \dword{utca} shelf. This
master timing \dword{amc} produces the SPTS signals and encodes them onto a
serial data stream. This serial datastream is distributed over a
standard star-point backplane to the fanout \dwords{amc}, which each drive the
signal onto up to \num{13} SFP cages. The SFP cages are either occupied by
1000Base-BX SFPs, each of which connects to a fiber running to an \dword{apa},
or to a Direct Attach cable which connects to systems elsewhere in the
\dword{cuc},  i.e., the \dword{rce} crates and the data selection system. This
arrangement is shown in Figure~\ref{fig:daq-readout-sp-timing}


\begin{dunefigure}[Arrangement of components in \dlong{sp} timing system]{fig:daq-readout-sp-timing}
  {Illustration of the components in the \single timing system.}
  %\fixme{Add a diagram of the SPTS electronics}
  \includegraphics[width=0.8\textwidth]{DUNE_SP_Timing.pdf}
\end{dunefigure}

\input{far-detector-generic/chapter-generic-daq/design-timing-beam}
