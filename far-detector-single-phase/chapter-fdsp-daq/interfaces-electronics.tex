
%%%%%%%%%%%%%%%%%%%%%%%%%%%%%%%%%%%
\subsection{TPC Electronics}
\label{sec:fd-daq-intfc-elec}

Details about the interfaces between the DAQ and the TPC electronics
are documented for the \dword{sp} \dwords{detmodule}
in~\cite{docdb6742}.
\fixme{Add references to bib.}

In the case of the \dword{sp} \dword{detmodule}, data from the cold
electronics \dwords{femb} are 8b10 encoded and sent to the
\dwords{wib} on copper cables at a bit rate of \SI{1.28}{\Gbps}. There
are two options being considered for the \dwords{wib}. In one, the
data are simply converted to optical signals and transmitted to the
DAQ in the CUC on \SI{1.28}{\Gbps} optical links with a total of 80
fibres per APA. In the second option, the \dwords{wib} aggregate the
data onto links running at $\approx$\SI{10}{\Gbps} before transmission
to the DAQ with a total of ten fibres per APA. In both cases the data
are received on rear transition modules connected to the \dword{cob}
\dword{atca} boards (see Section~{sec:fd-daq-fero}).


%%%%%%%%%%%%%%%%%%%%%%%%%%%%%%%%%%%
\subsection{PD Electronics}
\label{sec:fd-daq-intfc-photon}

Details about the interfaces between the DAQ and \dword{sp}
\dlong{pds} are documented. in~\cite{docdb6727}.

\fixme{Add references to bib.}

For the \dword{sp} \dword{pds} the signal to noise ratio of the
\dword{sipm} signals is high enough to allow \dlong{zs} to be safely
applied to the data. This reduces the data flow so that a bandwidth of
\SI{8}{\Gbps} per \dword{apa} is sufficent to transfer it to the DAQ,
with an order of magnitude safety factor. The link from the \dword{sp}
cryostat to the CUC will be implemented as either eight 1000Base-SX
links or a single 10GBase-LR link per APA.
The data on the links will be encoded using UDP/IP.
