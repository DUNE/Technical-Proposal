
%%%%%%%%%%%%%%%%%%%%%%%%%%%%%%%%%%%
\subsection{Front-end Trigger Primitive Generation}
\label{sec:fd-daq-fetp}

\metainfo{Josh Klein \& J.J. Russel \& Brett Viren.  This section is SP-specific.}

\Dwords{trigprimitive} are generated inside the \dword{fe} readout
hardware associated with each \dword{apa} from TPC data on a per-channel basis.
They are sent along with the waveform data to the \dword{fe} \dword{daq}
computing.
In the alternative design, the data is directly sent from the \dwords{apa} to
the \dwords{fec} and the data are sent to the trigger processing
computers.  
In both designs, the primitives from one \dword{daqfrag} are further
processed to produce \dwords{trigcandidate}. 
As such they represent a localization of activity to the corresponding
\dwords{apa} and for some period of time. 
These candidates are emitted to the the \dword{mtl} which may consider
candidates from other \dwords{detmodule} or external sources before
generating trigger commands.
Section~\ref{sec:fd-daq-sel} describes the selections involved in this
triggering.

Only the \num{480} collection channels associated with each \dword{apa} face are
used for forming \dwords{trigprimitive}. 
Reasons for this limitation include the fact that collection
channels:

\begin{itemize}
\item have higher signal to noise ratio compared to induction channels;
\item are fully and independently sensitive to activity on their \dword{apa} face;
\item have unipolar signals that directly give an approximate measure
  of ionization charge without costly computation required to
  deconvolve the field response functions as required to make use of
  the induction channels;
\item can be easily divided into smaller, independent groups in order
  to better exploit parallel processing.
  % fixme: do we need a statement about efficiency here?
\end{itemize}


Figure~\ref{fig:daq-readout-buffering-baseline} illustrates the connectivity between the
four connectors on each of the five \dwords{wib} and the \dword{fe} readout hardware.
The data is received via eighty \SI{1}{Gbps} fiber optical links by four \dwords{rce}
in the \dword{atca} \dword{cob} system. 

The pattern of connectivity between \dwords{wib} and \dwords{rce} results in the data
from the collection channels covering one contiguous half of one \dword{apa}
face being received by each \dword{rce}.
Each \dword{rce} has two primary functions. 
The first is transmission of all data as described in
Section~\ref{sec:fd-daq-hlt}. 
The second is to produce \dwords{trigprimitive} from its portion of the
collection channel data.
The algorithms to produce the \dwords{trigprimitive} still require
development but can be broadly described, as follows.

\begin{enumerate}
\item On a per channel basis, calculate a rolling baseline and spread
  level that characterizes recent noise behavior such that the result
  is effectively free of influence from actual ionization signals.
\item Locate contiguous runs of \dword{adc} samples that are above a threshold
  defined in terms of the baseline and noise spread.
\item Emit their time bounds and total charge as a \dword{trigprimitive}.
\end{enumerate}

Each \dword{trigprimitive} represents ionization activity localized
(relatively) along the drift direction by the times at which the
signal crosses threshold and by two planes parallel to the collection
wire and located midway between the wire and each of its neighboring
wires.
Depending on the threshold set, these \dwords{trigprimitive} may be
numerous due to $^{39}$Ar decays and noise fluctuations.
If their rate cannot be sustained, the threshold may be raised or
further processing may be done still at the \dword{apa} level,  which
considers more global information.
This may be performed either in the \dwords{rce} or later in the
\dword{fe} computing hosts. 
In either case, the results are in the form of \dwords{trigcandidate},
which are sent to the \dword{mtl}.

Sources of \dword{rf} emission inside the cryostat are minimized by
design. 
Any residual \dword{rf} is expected to be picked up coherently across
some group of channels. 
Depending on its intensity, additional processing of the collection
waveforms must be employed to mitigate this coherent noise and this
must occur before the data is sent for \dword{trigprimitive}
production. 
If the required mitigation algorithms outgrow the nominally specified
\dword{rce} \dword{fpga} it is possible to double the number of
\dwords{cob} per \dword{apa}, which would require a redistribution of fibers. 
Alternatively, or in addition, the higher number of
\dwords{trigprimitive} produced as a result of excess noise can be
passed along for further processing in the \dword{fe} computing. 
This would require reprocessing the raw waveform data.

