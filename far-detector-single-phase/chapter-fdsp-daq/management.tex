
%%%%%%%%%%%%%%%%%%%%%%%%%%%%%%%%%%%%%%%%%%%%%%%%%%%%%%%%%%%%%%%%%%%%
\section{Organization and Management (David Newbold \& Georgia Karagiorgi)}
\label{sec:fdsp-daq-org}

%\metainfo{2 Pages}
At the time of this writing, the DAQ Consortium comprises 30
institutions, including universities and national labs, from five
countries. Ever since its conception, the DAQ Consortium has met 
roughly on a weekly basis, and held two international workshops
dedicated to advancing the DUNE FD DAQ design.

Several key technical and architectural decisions have been made in
the last months, that have formed an agreed basis for the DAQ design
and implementation.  

%%%%%%%%%%%%%%%%%%%%%%%%%%%%%%%%%%%
\subsection{DAQ Consortium Organization}
\label{sec:fdsp-daq-org-consortium}

The DUNE DAQ Consortium is organized in the form of five active
Working Groups (WG) and WG Leaders:
\begin{itemize}
\item Architecture, WG Leaders: Giles Barr and Giovanna Lehman-Miotto 
\item Hardware, WG Leaders: David Cussans and Matthew Graham
\item Data Selection, WG Leader: Josh Klein
\item Back-End, WG Leader: Kurt Biery
\item Integration and Infrastructure, WG Leader: Alec Habig
\end{itemize}

During the ongoing early stages of the design, the Architecture and
Hardware WGs have been holding additional meetings focused on aspects
of the design related to architecture solutions and costing. In
parallel, the DAQ Simulation Task Force effort which was in place at
the time of the Consortium inception has been adopted under the Data
Selection WG, and simulation studies have continued to inform design considerations. 
This working structure is expected to remain in place through at least
the completion of the Technical Proposal.

%%%%%%%%%%%%%%%%%%%%%%%%%%%%%%%%%%
\subsection{Planning Assumptions}
\label{sec:fdsp-daq-org-assmp}


%%%%%%%%%%%%%%%%%%%%%%%%%%%%%%%%%%%
\subsection{WBS and Responsibilities}
\label{sec:fdsp-daq-org-wbs}

%%%%%%%%%%%%%%%%%%%%%%%%%%%%%%%%%%
\subsection{High-level Cost and Schedule}
\label{sec:fdsp-daq-org-cs}

