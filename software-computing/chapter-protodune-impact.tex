\chapter{ProtoDUNE plans and impact }
\fixme{7 pages}

\section{Overview}
The ProtoDUNE experiments will run at CERN in the Fall of 2018.  They are currently under construction.  Software and computing is responsible for storage, cataloging and reconstruction of these data after it is written to disk by the data acquisition systems.  In this section we give a short overview of the current status and what we expect to learn from the ProtoDUNE experiments.


\subsection{Single Phase prototype}

The single-phase prototype (ProtoDUNE-SP) will utilize six prototype APA's with the full drift length envisioned for the full far detector. In the single phase detector, the readout plane are immersed in the liquid Argon and no amplification occurs before the electronics.    ProtoDUNE-SP is being constructed in the NP04 beamline at CERN and should run with tagged test beam for around 6 weeks in the Fall of 2018.  In addition cosmic ray commissioning beforehand and cosmic running after the end of beam are anticipated.  Table \ref{tab:np04_data_rate}
shows the anticipated data rates and sizes. 



\begin{table}[htbp]
  \centering
  \begin{tabular}[h]{l|r}
\hline
    Trigger rate & 25\,Hz \\
    Spill duration & 4.5\,s\\
    SPS Cycle & 22.5\,s \\
    Readout time window & 5\,ms \\
    \# of APAs to be read out & 6 \\
    \hline
    Single readout size (per trigger) & 230.4\,MB \\
    Lossless compression factor & 4 \\
    Instantaneous data rate (in-spill) & 1440\,MB/s \\
    Average data rate & 576\,MB/s \\
    \hline
    3-Day buffer depth & 300\,TB \\
    \hline
  \end{tabular}
  \caption{Parameters for ProtoDUNE-SP run at CERN including both
  the in-spill and out-of-spill data.}
  \label{tab:np04_data_rate}
\end{table}
\subsection{Dual-Phase prototype}

The dual-phase prototype will run in the NP02 beamline in late Fall 2018.  In this detector, electrons drift the full height of the cryostat, emerge from the liquid and are collected - after gas amplification - on an grid of instrumented pads at the top of the detector.  The WA105 3x1x1 m test of this technology ran successfully in the summer of 2017\cite{Murphy:20170516}.  Table \ref{tab:np02_data_rate} shows the expected data rates from ProtoDUNE-DP. 

\begin{table}[htbp]
  \centering
  \begin{tabular}[h]{l|r}
\hline
     Compressed, non-zero-suppressed event size & 15.9 MB\\
    Average beam data flow to offline data storage &   305.3 MB/s  \\
    Average off-spills cosmic data flow to offline data storage &   12.4 MB/s\\
    Planned total statistics of beam triggers in 30 beam days with 50\% efficiency&25M\\
    Planned overall storage size of beam events&   0.4 PB\\
   Cosmic rays in 30  beam days&  2.1  M\\
   Planned storage size for cosmic rays during beam period&  0.04 PB\\
   Requested (in 2016) overall storage envelope for ProtoDUNE-DP&0.7 PB \\
    \hline
  \end{tabular}
 \caption{NP02 data volume}
  \label{tab:np02_data_rate}
\end{table}

\section{Data Challenges}

Computing and software is performing a series of data challenges to ensure that systems will be ready when the detectors become fully operational in the summer of 2018.  To date we have performed challenges using simulated single-phase data with a joint dual/single phase test scheduled for April of 2018.

\subsection{Data Challenge 1.0}

The first Data Challenge (DC1) took place the week of November 6th, 2018.  
Dummy data were produced, transferred through the FTS-Lite system to the CERN EOS Tier-0 storage and from there vis FTS to the CERN Castor and Fermilab Enstore tape-backed file storage systems. The files were cataloged in the sam database.  The data were then re-reconstructed using the full reconstruction chain.
The data quality monitor (DQM) system running on the Tier-0 system, provide monitors of purity, noise and an event display. 

\todo{Cite the DC1.0 note}

Rates of 200 MB/sec were maintained to EOS with 180MB/sec to Fermilab over a period of two days. 
This data challenge built on very significant effort by the collaboration and staff at CERN and Fermilab to setup authentication and transfer systems compatible with both laboratories.  In addition to the demonstration of successful end-to-end transfers at the expected rates, it also led to subsequent improvements in the transfer and reconstruction configurations. 

\subsection{Data Challenge 1.5}

In data challenge 1.5 in mid-January, dummy data based on non-zero-suppressed simulated events were produced at EHN1 and successfully transferred via 10-50 parallel transfers to EOS in the tier-0 at a sustained rate of 2GB/sec.    Transfers to dCache/Enstore at Fermilab achieved rates of 500 MB/sec.  
The rate from EHN1 to EOS is close to the maximum expected during ProtoDUNE-SP running. 









\subsection{Data Challenge 2.0}


\section{Impact of the protoDUNE experience on future plans}
