%%%%%%%%%%%%%%%%%%%%%%%%%%%%%%%%%%%%%%%%%%%%%%%%%%%%%%
%
% Where we describe the data analysis model for the experiment as it
% maps into the different phases of offline computing.
%%%%%%%%%%%%%%%%%%%%%%%%%%%%%%%%%%%%%%%%%%%%%%%%%%%%%%
\chapter{DUNE Reconstruction and Analysis Software Model } \fixme{10 pages}
% \section{Analysis Algorithms and Event Processing Framework} \fixme{10 pages}

%%%%%%%%%%%%%%%%%%%%%%%%%%%%%%%%%%%%%%%%%%%%%%%%%%%%%%
% beam simulation. Schellman

\section{Neutrino Flux Generation}

Neutrino flux calculations are now used to optimize and validate the LBNF beamline design the beamlines.  However, the same codes will likely play a very important role in the results from the full LBNF/DUNE long-baseline neutrino oscillation program over several decades.  As a result a continued program of validation and improvement of these codes will be necessary.

\subsection{Beamline simulation}

Current flux studies use a gaussian beam spot.  Full long term analysis will require integration of the proton beamline simulation into the neutrino flux calculations. 

\subsection{Target Modeling and Simulation}

Simulation and optimization of the LBNF beamlines have been performed and document by the Beam Optimization Task Force in 2017\cite{fields_doc_2901}. 
The LBNF neutrino beam simulation, G4LBNF\cite{g4lbnf}, uses a detailed 
model of the LBNF targets and focusing systems constructed within the GEANT4 framework. A beam of protons hits the target,
resulting in a cascade of physics processes including  particle production, propagation,
absorption and decay through material and the magnetic fields generated by the focusing horns. Different choices of  hadronic interaction models can lead to up to 40\% model uncertainties in the absolute flux.  The ppfx framework \cite{Aliaga} developed for the NuMI beamline allows incorporation of data from multiple  hadro-production experiments,  such as NA49, and NA61, in addition to the default GEANT4 models,  with the ability to add new information as it becomes available. The G4LBNF simulation continues to evolve as realistic designs for the LBNF target region are iterated.

The neutrinos produced by decays in the simulation are projected towards
the near and far detector for flux estimation. Substantial computational efficiencies are achieved in near-far flux studies by using each hadron that decays to neutrinos to calculate the relative acceptances of the near and far detectors analytically instead of tracing the neutrino trajectories directly\cite{Szleper:2001nj}.


\subsubsection{Typical size and time requirements for flux studies}
G4LBNE is a standalone GEANT4/root based product which operates outside the LarSoft and Art frameworks. 
A typical flux study involves 
2000 grid jobs taking around 2 CPU-hrs on FNAL grid nodes and producing 50 MB of output each.  Each job corresponds to 100,000 Protons on target.    


\subsection{Radiation and cooling design}
\subsubsection{MARS} % from the beamline CDR

An essential aspect of beam design work is numerical simulation of the primary and secondary neutrino beams. The MARS Monte Carlo  simulations provide input to detailed engineering calculations for beam heating effects for all components in or near the beams. The same software package gives radiological estimates for prompt and residual doses that dictate the amount of shielding needed along and tranverse to the beamline. At Fermilab, the MARS code \cite{abs_1} is used for all these estimates, with a sufficiently detailed model of the entire beam system.  

The MARS code itself is proprietary %check this
which leads to some restrictions on the computing model for running and distributing that code. 

%%%%%%%%%%%%%%%%%%%%%%%%%%%%%%%%%%%%%%%%%%%%%%%%%%%%%%
\section{Simulation of Physics Processes}
\subsection{Neutrino Interaction Generation}
\subsection{Cosmic Ray Flux Simulation}
\subsection{Proton Decay Modeling}
\subsection{Noise Modeling and Noise Overlays}

%%%%%%%%%%%%%%%%%%%%%%%%%%%%%%%%%%%%%%%%%%%%%%%%%%%%%%
\section{Detector Simulation}
\subsection{Detector Response Modeling}
\subsection{Electronics and Digitization Modeling}

%%%%%%%%%%%%%%%%%%%%%%%%%%%%%%%%%%%%%%%%%%%%%%%%%%%%%%
\section{Calibration of detector responses}

%%%%%%%%%%%%%%%%%%%%%%%%%%%%%%%%%%%%%%%%%%%%%%%%%%%%%%
\section{Event Reconstruction Chain}
\subsection{Digital signal processing systems}
%\section{Signal processing./zero suppression/templates/FFT}

%%%%%%%%%%%%%%%%%%%%%%%%%%%%%%%%%%%%%%%%%%%%%%%%%%%%%%
\section{Liquid Argon Software Suites}

\subsection{Clustering Algorithms}
\subsection{Pattern Recognition}
\subsection{Particle Tracking}
\subsection{Vertexing and Interaction Point Determination}

%%%%%%%%%%%%%%%%%%%%%%%%%%%%%%%%%%%%%%%%%%%%%%%%%%%%%%
\section{Object classification}
\subsection{Cluster classification}
\subsection{Track classification}
\subsection{Event classification}
\subsection{Interaction classification}

%%%%%%%%%%%%%%%%%%%%%%%%%%%%%%%%%%%%%%%%%%%%%%%%%%%%%%
\section{Data reduction and filtering}

%%%%%%%%%%%%%%%%%%%%%%%%%%%%%%%%%%%%%%%%%%%%%%%%%%%%%%
\section{Analysis Sets and Spectra}
\subsection{Normalization Tools}
\subsection{Fitters and Optimizer Interfaces}
\subsection{Monkeys trained to write Phys. Rev. Let. Articles\\ replace with ML by 2020}

%%%%%%%%%%%%%%%%%%%%%%%%%%%%%%%%%%%%%%%%%%%%%%%%%%%%%%
\section{Event Display}  %  add for version 2
