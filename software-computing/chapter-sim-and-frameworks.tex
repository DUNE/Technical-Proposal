%%%%%%%%%%%%%%%%%%%%%%%%%%%%%%%%%%%%%%%%%%%%%%%%%%%%%%
%
% Where we describe the simulation systems and frameworks that are
% needed for the DUNE experiment ranging from generation to detector
% response and simulation.
%%%%%%%%%%%%%%%%%%%%%%%%%%%%%%%%%%%%%%%%%%%%%%%%%%%%%%
\chapter{Simulation Systems and Frameworks}  %% Tom Junk lead editor
 \fixme{10 pages}
%%%%%%%%%%%%%%%%%%%%%%%%%%%%%%%%%%%%%%%%%%%%%%%%%%%%%%
\section{Beam simulation Systems}

%%%%%%%%%%%%%%%%%%%%%%%%%%%%%%%
\subsection{G4LBNF}

\subsection{Beam Spectrometer Simulation}

\subsection{Muon Monitors}

%%%%%%%%%%%%%%%%%%%%%%%%%%%%%%%%%%%%%%%%%%%%%%%%%%%%%%
\section{Event Geneartors}

The event generators provided with LArSoft create MCParticle and MCTruth data products in
the {\it{art}} event store.  Any number of event generators, including multiple instances of the
same generator, may be run for a given event.  This feature is useful in order to overlay cosmic
rays and radiological decays on neutrino scattering events for example.  In addition to the
generators listed below that are integrated into the LArSoft software environment, a text-file generator
can read files with particle types and four-vectors of their momenta and energies to provide access
to additional generators without investing in software integration.

%%%%%%%%%%%%%%%%%%%%%%%%%%%%%%%%%%%%%%%%%%%%%%%%%%%%%%
\subsection{Neutrino Interaction simulations}

%%%%%%%%%%%%%%%%%%%%%%%%%%%%%%%
\subsubsection{GENIE $\nu$ interaction generator}

%%%%%%%%%%%%%%%%%%%%%%%%%%%%%%%
\subsubsection{NEUT $\nu$ interaction generator}

%%%%%%%%%%%%%%%%%%%%%%%%%%%%%%%
\subsubsection{NuWro $\nu$ interaction generator}

%%%%%%%%%%%%%%%%%%%%%%%%%%%%%%%
\subsubsection{NUANCE $\nu$ interaction generator}

%%%%%%%%%%%%%%%%%%%%%%%%%%%%%%%
\subsubsection{GIBUU $\nu$ interaction generator}

%%%%%%%%%%%%%%%%%%%%%%%%%%%%%%%
\subsubsection{MARLEY $\nu$ interaction generator}

%%%%%%%%%%%%%%%%%%%%%%%%%%%%%%%%%%%%%%%%%%%%%%%%%%%%%%
\subsection{Nucleon Decay and Exotic Particle Simulation}

%%%%%%%%%%%%%%%%%%%%%%%%%%%%%%%
\subsubsection{GENIE}  % Hector Mendez will supply something here.

%%%%%%%%%%%%%%%%%%%%%%%%%%%%%%%
\subsubsection{MadGraph/MadEvent}

\subsection{Cosmic-Ray Generators}

%%%%%%%%%%%%%%%%%%%%%%%%%%%%%%%
\subsubsection{CRY}

%%%%%%%%%%%%%%%%%%%%%%%%%%%%%%%
\subsubsection{CORSIKA}

%%%%%%%%%%%%%%%%%%%%%%%%%%%%%%%
\subsubsection{MUSUN/MUSIC}

%%%%%%%%%%%%%%%%%%%%%%%%%%%%%%%
\subsection{Radiological Generator}

The radiological generator is a LArSoft module that produces MCParticle objects corresponding
to the decay products of radionuclides.  The spectra of alpha, beta, and gamma particles emitted
by decaying radionuclides are user-specifialbe in root files.  The radiological generator generates
decays within a rectangular prism in space and in a specified range of time.  For a particular
nuclide, volumes in the geometry whose materials match a regular expression are selected as locations
for possible decays.  This mechanism is chosen over the more complex alternatie of specifying radionuclide
concentrations in the geometry specification because GEANT4 and GENIE will attempt to simulate interactions
of neutrinos and other particles on extremely rare nuclides, for which cross-section data may not exist.
The CPU time required to run the radiological generator is negligible, but its output can dominate
the CPU and storage of the fully simulated samples, as is the case for the far detector data.

%%%%%%%%%%%%%%%%%%%%%%%%%%%%%%%%%%%%%%%%%%%%%%%%%%%%%%
\section{Detector geometry and modeling systems}

LArSoft reads in GDML files that describe the geometries of the far detector modules.  Two versions of these
files are required -- one with the readout wires described, and one without.  The total mass in the readout
wires is very small, but their presence would increase the memory and CPU requirements of the simulation
if they are present in any more than a parameterized manner.  These GDML files are constructed with Perl scripts.
A more sophisticated tool called {\tt{ggd}}~\cite{ggd} is designed to create GDML files from a higher-level
Python description of the detector.

Within LArSoft, two ``parallel-world'' representations of the geometry are created.  One is for particle interactions
modeled by GEANT4, and the other is intended for GEANT4's simulation of photon propagation.

%%%%%%%%%%%%%%%%%%%%%%%%%%%%%%%
\subsection{Fast Monte Carlo}

A Fast Monte Carlo (FMC) program was developed by the LBNE collaboration and was used to
compute oscillation sensitivities~\cite{cdr-vol-2}.  The FMC simulated neutrino interactions
on argon using the LBNF flux spectrum and the GENIE~\cite{GENIE} generator.  The detector
responses to the final-state particles produced by GENIE are approximated with efficiency
fractions as functions of particle energy, energy resolution smearing fractions, and
correct and incorrect particle identification rates.  The events were divided into
$\nu_\mu$CC-like, $\nu_e$-CC-like, and NC-like, for the near and far detectors, and the
inputs provided to a parameter fit, such as CAFANA, GLoBES, or LOAF.  The details of the
detector response and analysis requirements can be found in~\cite{cdr-vol-2}.  The CPU time needed
for the FMC project is 1 to 4~M hours per year, with $<1$~TB of storage required, when
it is fully used.  The emphasis on full simulation and reconstruction results however has
displaced the FMC in 2016 and 2017. 

%%%%%%%%%%%%%%%%%%%%%%%%%%%%%%%
\subsection{Geant4 and Detector Response Modeling}

Interactions of particles with the detector materials, both active and inactive, is simulated with 
GEANT4~\cite{GEANT4:NIM}\cite{GEANT4}.  GEANT4 is also used as the underlying physics model for G4LBNF, the
simulation of the baffle, target, horns, and decay pipe.  GEANT4 simulates particle trajectories in steps,
taking into account electric and magnetic fields, as well as interaction and scattering probabilities in the
matter at each step.  GEANT4 requires a detailed description of the geometry of the materials through which
it simulates particle propagation.

LArSoft~\cite{larsoft} provides an interface to GEANT4 that is optimized for liquid argon time projection
chambers.  Step sizes are limited by voxelizing the active liquid argon volume into cubes 300~$\mu$m on a side.
Ionization and scintillation are modeled with a modified Box function~\cite{box} or with NEST~\cite{nest}, using
the energy deposition on each step provided by GEANT4.  Ionization and delta rays are part of a continuous
spectrum -- GEANT4 tracks delta rays down to an energy of ??~KeV using the default LArSoft paramaterization.
The GEANT4 step of a LArSoft simulation typically runs as a separate job, and the output of this job is the
number of electrons deposited on each wire after LArSoft's custom drift and diffusion simulation, which
takes into account recombination, electron lifetime, space charge, and detector geometry effects.  The charge
is summed over all contributing particle steps and integrated over diffusion, and reported separately for
each particle for each tick on each wire (a particle may have many small GEANT4 steps which contribute
charge to a wire).  The average position of the charge creation points are recorded for each particle for each
tick on each wire as well.  

In the LArSoft interface,  it is possible to specify a magnetic field that will cause charged-particle
trajectories to deflect.

For photon propagation modeling, a library-based method is used to speed up the sampling of tens of thousands of
scintillation photons per MeV of deposited energy.  The photon library is a list of detection probabilities
indexed by the voxel in space corresponding to the origin of the photons and the photon detector channel.
The source voxel sizes are cubes ??~cm on a side.  These detection probabilites are computed with a fully-simulated
run using GEANT4 to propagate photons isotropically-distributed in their production angle through the liquid argon,
taking into account the reflectivity of the field cage walls and the cathode and Rayleigh scattering in the liquid.
The detection probability further includes light attenuation along the length of each photon detector's bar.  An
overall quantum efficiency scaling factor is applied before the photon library lookup is performed, reducing the
number of lookups needed to complete the calculation.  The photon lookup library is typically a 300~MB rootfile.
Cherenkov radiation is currently not simulated or propagated by LArSoft; it may be added in a future version.

The GEANT4 step, along with the LArSoft-supplied drift modeling and lookup-library-based photon propagation, 
is the second-quickest simulation step,
after the generation stage, with ??~sec/event for far detector neutrino interactions, and ??~sec/event for
ProtoDUNE events with cosmic rays.  GEANT4 however requires several GB of auxiliary data files in order
to parameterize the interactions of particles with matter.  The GEANT4 collaboration actively compares
the modeling of published data distributions and tunes the simulation to improve the modeling.

%%%%%%%%%%%%%%%%%%%%%%%%%%%%%%%
\subsection{GeantV}

%%%%%%%%%%%%%%%%%%%%%%%%%%%%%%%%%%%%%%%%%%%%%%%%%%%%%%
\section{FLUKA}

FLUKA~\cite{FLUKA} is a full-featured physics and detector simulation package.  It is capable of
performing the same role as GEANT4 in the beam simulations, as well as in the detector response modeling.
FLUKA also has the ability to model neutrino-nucleus interactions, and thus can substitute for GENIE.
Due to the prohibitive computational load of simulating all of the drifting electrons and propagating
photons, the same solution of a custom LArSoft-based drift and charge deposit model would still be needed
if FLUKA were to take on all other roles.  A project has started to interface FLUKA at several points upstream
of the detector charge deposit modeling.  Comparing FLUKA-based calculations with those based on GEANT4 and
GENIE provides additional abilities to model external data and detector data properly, and to estimate
systematic uncertainties.

%%%%%%%%%%%%%%%%%%%%%%%%%%%%%%%%%%%%%%%%%%%%%%%%%%%%%%
\section{Electronics and DAQ Emulation}

The output of the GEANT4 step, which consists of simulated charge deposits on each wire, requires
additional processing to model the expected signals to be digitized.  The digitizers measure the
current on each wire integrated over the sampling time, chosen to be 2~$\mu$s.  As charge drifts
by an induction-plane wire, it induces a current that is a bipolar function of time, while the
current in a collection-plane wire is a unipolar function of time.  The differences in the arrival
times of charge drifting past the layers due to the finite spacing between the layers are include.
LArSoft parameterizes the response
of the detector as a Green's function -- the response of a packet of charge localized in space and time,
and this funtion is convoluted with the function describing the arrival times of the charges passing by
or collecting on the wires.  This convolution is performed in frequency space after FFT'ing the charge
arrival function.  At this stage, the electronics response transfer function is multiplied in.  An inverse
FFT predicts the time-domain signal.  The signal gain is a separately controllable parameter in LArSoft.
  Noise is then added, either white noise or an exponentially falling
spectrum, and then the result is discretized and truncated to the expected range of the 12-bit ADCs.  

Additional steps, such as modeling of stuck bits or groups of bits are configurable within LArSoft.
Modeling dead channels or groups of channels is frequently performed as an early step
in the reconstruction job so that Monte Carlo samples can be reused with varying fractions and placement
of dead channels.

Similar processing steps are applied to the simulation of the photon detector waveforms.

The output of the electronics simulation is digitized waveforms.  Zero suppression is optionally
applied.  The simulated data are compressed either with ROOT's built-in compression, or a fixed-dictionary
Huffman algorithm.

%%%%%%%%%%%%%%%%%%%%%%%%%%%%%%%%%%%%%%%%%%%%%%%%%%%%%%
\section{Trigger Simulation}
