%%%%%%%%%%%%%%%%%%%%%%%%%%%%%%%%%%%%%%%%%%%%%%%%%%%%%%
%
% Where we describe the simulation systems and frameworks that are
% needed for the DUNE experiment ranging from generation to detector
% response and simulation.
%%%%%%%%%%%%%%%%%%%%%%%%%%%%%%%%%%%%%%%%%%%%%%%%%%%%%%
\chapter{Simulation Systems and Frameworks}  %% Tom Junk lead editor
 \fixme{10 pages}
%%%%%%%%%%%%%%%%%%%%%%%%%%%%%%%%%%%%%%%%%%%%%%%%%%%%%%
\section{Beam simulation Systems}

%%%%%%%%%%%%%%%%%%%%%%%%%%%%%%%
\subsection{G4LBNF}

\subsection{Beam Spectrometer Simulation}

\subsection{Muon Monitors}

%%%%%%%%%%%%%%%%%%%%%%%%%%%%%%%%%%%%%%%%%%%%%%%%%%%%%%
\section{Neutrino Interaction simulations}

%%%%%%%%%%%%%%%%%%%%%%%%%%%%%%%
\subsection{GENIE $\nu$ interaction generator}

%%%%%%%%%%%%%%%%%%%%%%%%%%%%%%%
\subsection{NEUT $\nu$ interaction generator}

%%%%%%%%%%%%%%%%%%%%%%%%%%%%%%%
\subsection{NuWro $\nu$ interaction generator}

%%%%%%%%%%%%%%%%%%%%%%%%%%%%%%%
\subsection{NUANCE $\nu$ interaction generator}

%%%%%%%%%%%%%%%%%%%%%%%%%%%%%%%
\subsection{GIBUU $\nu$ interaction generator}

%%%%%%%%%%%%%%%%%%%%%%%%%%%%%%%
\subsection{MARLEY $\nu$ interaction generator}

%%%%%%%%%%%%%%%%%%%%%%%%%%%%%%%%%%%%%%%%%%%%%%%%%%%%%%
\section{Nucleon Decay and Exotic Particle Simulation}

%%%%%%%%%%%%%%%%%%%%%%%%%%%%%%%
\subsection{GENIE}

%%%%%%%%%%%%%%%%%%%%%%%%%%%%%%%
\subsection{MadGraph/MadEvent}

\section{Cosmic-Ray Generators}

%%%%%%%%%%%%%%%%%%%%%%%%%%%%%%%
\subsection{CRY}

%%%%%%%%%%%%%%%%%%%%%%%%%%%%%%%
\subsection{CORSIKA}

%%%%%%%%%%%%%%%%%%%%%%%%%%%%%%%
\subsection{MUSUN/MUSIC}

%%%%%%%%%%%%%%%%%%%%%%%%%%%%%%%%%%%%%%%%%%%%%%%%%%%%%%
\section{Detector geometry and modeling systems}


%%%%%%%%%%%%%%%%%%%%%%%%%%%%%%%
\subsection{Fast Monte Carlo}

A Fast Monte Carlo (FMC) program was developed by the LBNE collaboration and was used to
compute oscillation sensitivities~\cite{cdr-vol-2}.  The FMC simulated neutrino interactions
on argon using the LBNF flux spectrum and the GENIE~\cite{GENIE} generator.  The detector
responses to the final-state particles produced by GENIE are approximated with efficiency
fractions as functions of particle energy, energy resolution smearing fractions, and
correct and incorrect particle identification rates.  The events were divided into
$\nu_\mu$CC-like, $\nu_e$-CC-like, and NC-like, for the near and far detectors, and the
inputs provided to a parameter fit, such as CAFANA, GLoBES, or LOAF.  The details of the
detector response and analysis requirements can be found in~\cite{cdr-vol-2}.  The CPU time needed
for the FMC project is 1 to 4~M hours per year, with $<1$~TB of storage required, when
it is fully used.  The emphasis on full simulation and reconstruction results however has
displaced the FMC in 2016 and 2017. 

%%%%%%%%%%%%%%%%%%%%%%%%%%%%%%%
\subsection{Geant4}

%%%%%%%%%%%%%%%%%%%%%%%%%%%%%%%
\subsection{GeantV}

%%%%%%%%%%%%%%%%%%%%%%%%%%%%%%%%%%%%%%%%%%%%%%%%%%%%%%
\section{FLUKA}

%%%%%%%%%%%%%%%%%%%%%%%%%%%%%%%%%%%%%%%%%%%%%%%%%%%%%%
\section{Detector response simulation systems}

%%%%%%%%%%%%%%%%%%%%%%%%%%%%%%%%%%%%%%%%%%%%%%%%%%%%%%
\section{Electronics and DAQ Emulation}

%%%%%%%%%%%%%%%%%%%%%%%%%%%%%%%%%%%%%%%%%%%%%%%%%%%%%%
\section{Trigger Simulation}
