% see dune-words.tex for explanation.

\usepackage[acronyms,toc]{glossaries}
\makeglossaries


% \newduneword{label}{term}{description}
\newcommand{\newduneword}[3]{
    \newglossaryentry{#1}{
        text={#2},
        long={#2},
        name={\glsentrylong{#1}},
        first={\glsentryname{#1}},
        firstplural={\glsentrylong{#1}\glspluralsuffix},
        description={#3}
    }
}

%                 1      2     3       4
% \newduneabbrev{label}{abbrev}{term}{description}
\newcommand{\newduneabbrev}[4]{
  % this makes acronym entries even if they neither term or abbrev is
  % referenced in the text.
  % \newacronym[see={[Glossary:]{#1}}]{abbrev-#1}{#2}{#3}
  \newacronym{abbrev-#1}{#2}{#3}
  \newglossaryentry{#1}{
    text={#2},
    long={#3},
    name={\glsentrylong{#1}{} (\glsentrytext{#1}{})},
    first={\glsentryname{#1}},
    firstplural={\glsentrylong{#1}\glspluralsuffix{} (\glsentrytext{#1}\glspluralsuffix{})},
    description={#4}
  }
}
\newcommand{\dshort}[1]{\acrshort{abbrev-#1}}
\newcommand{\dlong}[1]{\acrlong{abbrev-#1}}

\newcommand{\dword}[1]{\gls{#1}}
\newcommand{\dwords}[1]{\glspl{#1}}
\newcommand{\Dword}[1]{\Gls{#1}}
\newcommand{\Dwords}[1]{\Glspl{#1}}

\newduneabbrev{adc}{ADC}{analog digital converter}{A sampling of a voltage
  resulting in a discrete integer count corresponding in some way to
  the input}

\newduneabbrev{fe}{FE}{front-end}{The front-end refers a point which is
  ``upstream'' of the data flow for a particular subsystem. 
  For example the front-end electronics is where the cold electronics
  meet the sense wires of the TPC and the front-end DAQ is where the
  DAQ meets the output of the electronics}

\newduneabbrev{daq}{DAQ}{data aquisition}{The Data Acquisition system
  accepts data from the detector \acrshort{fe} electronics, buffers
  it, performs a \gls{trigdecision}, builds events from the selected
  data and delivers the result to the offline \gls{diskbuffer}}

\newduneword{detmodule}{detector module}{The entire DUNE far detector is
  segmented into four modules, each with a nominal \SI{10}{\kton}
  fiducial mass}

\newduneword{submodule}{sub-detector}{A detector unit of granularity less
  than one \gls{detmodule} such as the TPC of the single-phase
  \gls{detmodule}}


\newduneword{detunit}{detector unit}{A \gls{submodule} may be partitioned
  into a number of similar parts. 
  For example the single-phase TPC \gls{submodule} is made up of APA
  units}

\newduneabbrev{tpm}{TPM}{trigger primitive message}{A message flowing
  up the trigger hierarchy from local to global context}

\newduneabbrev{tcm}{TCM}{trigger command message}{A message flowing
  down the trigger hierarchy global to local context}

\newduneword{l0primitive}{L0 trigger primitive}{Information derived by
  the DAQ \gls{fe} hardware and which describes a region of space (eg,
  one or several neighboring channels) and time (eg, a contiguous set
  of ADC sample ticks) associated with some activity}

\newduneword{l1primitive}{L1 trigger primitive}{Or \textit{L1 trigger}
  or module trigger. 
  Information derived from \gls{l0primitive} information at the level
  of one \gls{detmodule} and sent to \gls{gtl}}


\newduneabbrev{mtl}{MTL}{Module Trigger Logic}{Trigger processing
  which consumes \gls{detunit} level \gls{l0primitive} information and
  emits \gls{l1primitive} information to the \glspl{gtl}. 
  It also accepts \glspl{trigcommand} from the \gls{gtl} and
  interprets and routes them to \gls{detunit} level components}

\newduneabbrev{gtl}{GTL}{Global Trigger Logic}{Trigger processing
  which consumes \gsl{detmodule} level \gls{l1primitive} information
  and other global sources of trigger input and emits
  \gls{trigcommand} information back to the \glspl{mtl}}

\newduneword{trigcommand}{trigger command}{Information derived from
  one or more \glspl{l0primitive} and which consists of addresses at
  the level of \glspl{detunit} and a period of time. 
  These are delivered to and interpreted by \gls{detunit} level
  components for the purpose of reading out the corresponding data
  from the \gls{ringbuffer}}

\newduneword{ringbuffer}{primary DAQ buffer}{A buffer in the
  \dshort{daq} with sufficient size to store data long enough for a
  trigger decision to be made and with sufficient endurance and
  throughput to allow constant flow of full-stream data}


\newduneword{diskbuffer}{secondary DAQ buffer}{A secondary
  \dshort{daq} buffer holds a small subset of the full rate as
  selected by a \dword{trigcommand}. 
  This buffer also marks the interface with the DUNE Offline}

\newduneword{dumpbuffer}{DAQ dump buffer}{This \dshort{daq} buffer
  accepts a high-rate data stream, in aggregate, from an associated
  \dword{submodule} sufficient to collect all data likely relevant to
  a potential Supernova Burst.}

\newduneword{trigdecision}{trigger decision}{The process by which
  trigger primitives are converted into trigger commands}

\newduneword{octant}{octant}{Any of the eight parts into which 4$\pi$
  is divided by three mutually perpendicular axes. 
  In particular in referencing the value for the mixing angle
  $\theta_{23}$}

\newduneabbrev{dqm}{DQM}{data quality monitoring}{Analysis of the raw
  data to monitor the integrity of the data and the performance of the
  detectors and their electronics. This type of monitoring may be
  performed in real time, within the \dword{daq} system, or in later
  stages of processing, using disk files as input}
