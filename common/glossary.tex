% See dune-words.tex for detailed explanation.

\usepackage[acronyms,toc]{glossaries}
\makeglossaries


\newcommand{\dshort}[1]{\acrshort{abbrev-#1}}
\newcommand{\dlong}[1]{\acrlong{abbrev-#1}}

% force the "first time" behavior
\newcommand{\dfirst}[1]{\glsfirst{abbrev-#1}}

\newcommand{\dword}[1]{\gls{#1}}
\newcommand{\dwords}[1]{\glspl{#1}}
\newcommand{\Dword}[1]{\Gls{#1}}
\newcommand{\Dwords}[1]{\Glspl{#1}}


% use this to define terms that do NOT have acronyms.
% \newduneword{label}{term}{description}
\newcommand{\newduneword}[3]{
    \newglossaryentry{#1}{
        text={#2},
        long={#2},
        name={\glsentrylong{#1}},
        first={\glsentryname{#1}},
        firstplural={\glsentrylong{#1}\glspluralsuffix},
        description={#3}
    }
}

% use this to define terms that DO have acronyms.
%                 1      2     3       4 
% \newduneabbrev{label}{abbrev}{term}{description}
\newcommand{\newduneabbrev}[4]{
  % this makes acronym entries even if they neither term or abbrev is
  % referenced in the text.
  % \newacronym[see={[Glossary:]{#1}}]{abbrev-#1}{#2}{#3}
  \newacronym{abbrev-#1}{#2}{#3}
  \newglossaryentry{#1}{
    text={#2},
    long={#3},
    name={\glsentrylong{#1}{} (\glsentrytext{#1}{})},
    first={\glsentryname{#1}},
    firstplural={\glsentrylong{#1}\glspluralsuffix{} (\glsentrytext{#1}\glspluralsuffix{})},
    description={#4}
  }
}

%% If plural needs special spelling besides adding an "s"
%                 1      2     3       4        5
% \newduneabbrev{label}{abbrev}{term}{terms}{description}
\newcommand{\newduneabbrevs}[5]{
  % this makes acronym entries even if they neither term or abbrev is
  % referenced in the text.
  % \newacronym[see={[Glossary:]{#1}}]{abbrev-#1}{#2}{#3}
  \newacronym{abbrev-#1}{#2}{#3}
  \newglossaryentry{#1}{
    text={#2},
    long={#3},
    plural={#4},
    name={\glsentrylong{#1}{} (\glsentrytext{#1}{})},
    first={\glsentryname{#1}},
    firstplural={\glsentryplural{#1} (\glsentrytext{#1}\glspluralsuffix{})},
    description={#5}
  }
}


%%%%%%     START ADDING WORDS, IN ALPHABETICAL ORDER IF POSSIBLE!    %%%%%%%%

\newduneabbrev{nd}{ND}{near detector}{Refers to the detectors or more
  generally the experimental site at Fermilab}

\newduneabbrev{fd}{FD}{far detector}{Refers to the detector or more
  generally the experimental site in or above the Homestake mine in
  Lead, SD}

\newduneabbrev{sp}{SP}{single-phase}{Distinguishes one of the four
  \SI{10}{\kton} \glspl{detmodule} of the DUNE far detector by the
  fact that it operates using argon in just its liquid phase.}


\newduneabbrev{dp}{DP}{dual-phase}{Distinguishes one of the four
  \SI{10}{\kton} \glspl{detmodule} of the DUNE far detector by the
  fact that it operates using argon in both gas and liquid phases.}

\newduneabbrev{pds}{PDS}{photon detection system}{The \gls{submodule}
  system sensitive to light produced in the LAr}

\newduneabbrev{tpc}{TPC}{time projection chamber}{The portion of the
  various DUNE \glspl{submodule} which record ionization electrons
  after they drift away from a cathode, through LAr and potentially
  through GAr as well. 
  The activity is recorded by digitizing the waveforms of current
  induced on anode as the distribution of ionization charge passes by
  or is collected on the electrode}

\newduneabbrevs{apa}{APA}{anode plane assembly}{anode plane assemblies}{One unit the SP
  detector containing the elements sensitive to activity in the LAr. 
  It contains two faces each of three planes of wires, cold
  electronics and photo detection system.} 

\newduneabbrev{cro}{CRO}{charge readout}{The system for detecting
  ionization charge distributions in the DP detector module}

\newduneabbrev{lro}{LRO}{light readout}{The system for detecting
  scintillation photons in the DP detector module}


% front-end
\newduneabbrev{fe}{FE}{front-end}{The front-end refers a point which is
  ``upstream'' of the data flow for a particular subsystem. 
  For example the front-end electronics is where the cold electronics
  meet the sense wires of the TPC and the front-end DAQ is where the
  DAQ meets the output of the electronics}

% analog digital converter
\newduneabbrev{adc}{ADC}{analog digital converter}{A sampling of a voltage
  resulting in a discrete integer count corresponding in some way to
  the input}

% data aquisition
\newduneabbrev{daq}{DAQ}{data aquisition}{The data acquisition system
  accepts data from the detector FE electronics, buffers
  it, performs a \gls{trigdecision}, builds events from the selected
  data and delivers the result to the offline \gls{diskbuffer}}

% detector module
\newduneword{detmodule}{detector module}{The entire DUNE far detector is
  segmented into four modules, each with a nominal \SI{10}{\kton}
  fiducial mass}

% detector unit
\newduneword{detunit}{detector unit}{A \gls{submodule} may be partitioned
  into a number of similar parts. 
  For example the single-phase TPC \gls{submodule} is made up of APA
  units}

% secondary DAQ buffer
\newduneword{diskbuffer}{secondary DAQ buffer}{A secondary
  \dshort{daq} buffer holds a small subset of the full rate as
  selected by a \gls{trigcommand}. 
  This buffer also marks the interface with the DUNE Offline}

% data quality monitoring
\newduneabbrev{dqm}{DQM}{data quality monitoring}{Analysis of the raw
  data to monitor the integrity of the data and the performance of the
  detectors and their electronics. This type of monitoring may be
  performed in real time, within the \gls{daq} system, or in later
  stages of processing, using disk files as input}

% DAQ dump buffer
\newduneword{dumpbuffer}{DAQ dump buffer}{This \dshort{daq} buffer
  accepts a high-rate data stream, in aggregate, from an associated
  \gls{submodule} sufficient to collect all data likely relevant to
  a potential Supernova Burst.}


% Global Trigger Logic
\newduneabbrev{gtl}{ETL}{External Trigger Logic}{Trigger processing
  which consumes \gls{detmodule} level \gls{trignote} information
  and other global sources of trigger input and emits
  \gls{trigcommand} information back to the \glspl{mtl}}

\newduneword{trignote}{trigger notification}{Information provided by
  \gls{mtl} to \gls{gtl} about \gls{trigdecision} its processing}

% trigger primitive
\newduneword{trigprimitive}{trigger primitive}{Information derived by
  the DAQ \gls{fe} hardware and which describes a region of space (eg,
  one or several neighboring channels) and time (eg, a contiguous set
  of ADC sample ticks) associated with some activity}

\newduneword{externtrigger}{external trigger candidate}{Information
  provided to the \gls{mtl} about events external to a
  \gls{detmodule} so that it may be considered in forming
  \glspl{trigcommand}}

\newduneabbrev{daqoob}{OOB dispatcher}{out-of-band trigger command
  dispatcher}{This component is responsible for dispatching a SNB dump
  command to all \glspl{daqfer} in the \gls{detmodule}.}

\newduneabbrev{mtl}{MTL}{Module Trigger Logic}{Trigger processing
  which consumes \gls{detunit}-level \gls{trigcommand} information
  and emits \glspl{trigcommand}. 
  It provides the \gls{gtl} with \glspl{trignote} and receives back any
  \glspl{externtrigger}}

% octant
\newduneword{octant}{octant}{Any of the eight parts into which 4$\pi$
  is divided by three mutually perpendicular axes. 
  In particular in referencing the value for the mixing angle
  $\theta_{23}$}

% primary DAQ buffer
\newduneword{ringbuffer}{primary DAQ buffer}{A buffer in the
  DAQ with sufficient size to store data long enough for a
  trigger decision to be made and with sufficient endurance and
  throughput to allow constant flow of full-stream data}

% sub-detector ??? %%%%%%%%%%%%%%%%%%%      Why not ``subdet''? (Anne)  %%%%%% ???????
\newduneword{submodule}{sub-detector}{A detector unit of granularity less
  than one \gls{detmodule} such as the TPC of the single-phase
  \gls{detmodule}}

\newduneword{trigcandidate}{trigger candidate}{Summary information derived
  from the full data stream and representing a contribution toward
  forming a \gls{trigdecision}}

% trigger command
\newduneword{trigcommand}{trigger command}{Information derived from
  one or more \glspl{trigcandidate} and which directs elements of the
  \gls{detmodule} to read out a portion of the data stream}

% trigger command message
\newduneabbrev{tcm}{TCM}{trigger command message}{A message flowing
  down the trigger hierarchy global to local context}

% trigger decision
\newduneword{trigdecision}{trigger decision}{The process by which
  \glspl{trigcandidate} are converted into \glspl{trigcommand}}

% trigger primitive message
\newduneabbrev{tpm}{TPM}{trigger primitive message}{A message flowing
  up the trigger hierarchy from local to global context}

\newduneabbrev{eb}{EB}{event builder}{A software agent servicing one
  \gls{detmodule} by executing \glspl{trigcommand} by reading out
  the requested data}


% \fixme{Needs improvement}
\newduneabbrev{cob}{COB}{Cluster On Board}{Four \glspl{rce} together
  with networking an other hardware}

% \fixme{Needs improvement}
\newduneabbrev{rce}{RCE}{Reconfigurable Computing Element}{One of four
  nodes in a \gls{cob} which consists of ARM CPU and FPGA resources}

% \fixme{Needs improvement}
\newduneabbrev{atca}{ATCA}{Advanced Telecommunications Computing
  Architecture}{A computer architecture specification developed for
  telecommunication, military and aerospace industry.} 

\newduneabbrev{utca}{$\mu$TCA}{Micro Telecommunications Computing Architecture}{A computer architecture.} 

\newduneabbrev{udp}{UDP}{User Datagram Protocol.}{A simple,
  connectionless Internet Protocol which supports data integrity
  checksums, requires no handshaking and does not guarantee packet
  delivery}

\newduneabbrev{amc}{AMC}{Advanced Mezzanine Card}{Holds digitizing
  electronics and lives in \gls{utca} crates}

\newduneabbrev{rf}{RF}{radio frequency}{Electromagnetic emissions
  which are within the frequency band of sensitivity of the detector
  electronics.}

% \fixme{Needs improvement}
\newduneabbrev{fpga}{FPGA}{Field Programmable Gate Array}{An
  integrated circuit technology which the hardware to be reconfigured
  to execute different algorithms after manufacture}

\newduneabbrev{felix}{FELIX}{Front-End Link eXchange}{A
  high-throughput interface between front-end and trigger electronics
  and the standard PCIe computer bus}

\newduneword{daqpart}{DAQ partition}{A cohesive and
 coherent collection of DAQ hardware and software working together to trigger and readout some portion of one detector module consisting of some integral number of\glspl{daqfrag}. 
 Multiple DAQ partitions may operate simultaneously but each instance operates independently.}
 
\newduneabbrev{fec}{FEC}{front-end computer}{The portion of one
  \gls{daqpart} which hosts the \gls{daqdr}, \gls{daqbuf} and
  \gls{daqds}.  It is connected to the \gls{daqfer} via fiber optic. 
Each \gls{detunit} of a  certain granularity, such as two SP APAs, has one front-end computer
  which receives data from the readout hardware, hosts the primary DAQ
  memory buffer for that data, emits trigger candidates derived from
  that data and satisfies requests for producing subsets of that data
  for egress.}

\newduneword{daqfrag}{DAQ front-end fragment}{The portion of one
  \gls{daqpart} relating to a single \gls{fec} and corresponding to an
  integral number of \glspl{detunit}.  See also \gls{datafrag}}

\newduneword{datafrag}{data fragment}{A block of data read
  out from a single \gls{daqfrag} which covers a contiguous period of time
  as requested by a \gls{trigcommand}}

\newduneabbrev{daqfer}{FER}{DAQ front-end readout}{The portion of a
  \gls{daqfrag} which accepts data from the detector electronics and
  provides it to the \gls{fec}. 
  In the nominal design it is also responsible for generating channel
  level \glspl{trigprimitive}}

\newduneabbrev{daqdr}{DDR}{DAQ data receiver}{The portion of the
  \gls{daqfrag} which accepts data from the \gls{daqfer}, emits
  trigger candidates produced from the input trigger primitives and
  forwards the full data stream to the \gls{daqbuf}}

\newduneabbrev{daqbuf}{primary buffer}{DAQ primary buffer}{The portion
  of the \gls{daqfrag} which accepts full data stream from the
  corresponding \gls{detunit} and retains it sufficiently long for it
  to be available to produce a \gls{datafrag}}

\newduneword{daqds}{data selector}{The portion of the \gls{daqfrag}
  which accepts \glspl{trigcommand} and returns the corresponding
  \gls{datafrag}.}
  
\newduneabbrev{femb}{FEMB}{Front-End Mother Board}{Refers a unit of
  the \gls{sp} cold electronics which contains the front-end amplifier
  and ADC ASICs covering 128 channels}

\newduneword{protodune}{ProtoDUNE}{Two prototype detectors operated in
  a CERN beam test. 
  One prototyping \gls{sp} and the other \gls{dp} technology}

\newduneword{pdsp}{ProtoDUNE-SP}{The single-phase ProtoDUNE detector.}

\newduneword{pddp}{ProtoDUNE-DP}{The dual-phase ProtoDUNE detector.}

\newduneword{wa105}{\SI[product-units=power]{3x1x1}{m} DP
  detector}{The \SI[product-units=power]{3x1x1}{m} WA105 dual-phase
  prototype detector.}

% oh boy, here's that dirty word: "event"
\newduneword{rawevent}{DAQ event block}{The unit of data output by the
  DAQ. 
  It contains trigger and detector data spanning a unique, contiguous
  time period and a subset of the detector channels}

\newduneabbrev{ssd}{SSD}{solid-state disk}{Any storage device which
  may provide sufficient write throughput to receive, collectively and
  distributed, the sustained full rate of data from a \gls{detmodule}
  for many seconds}

% fixme: this needs improvement
\newduneabbrev{hlt}{HLT}{high-level trigger}{A source of triggering at
  the module level.}

\newduneabbrev{pid}{PID}{Particle ID}{Particle identification}

\newduneword{readout window}{readout window}{A fixed, atomic and
  continuous period of time over which data from a \gls{detmodule}, in
  whole or in part, is recorded. 
  This period may differ based on the trigger than initiated the
  readout}

\newduneabbrev{zs}{ZS}{zero-suppression}{To delete some portion of a
  data stream which does not significantly deviate from zero or
  intrinsic noise levels. 
  It may be applied at different granularity from per-channel to per
  \dword{detunit}}

% fixme: maybe another sentence
\newduneabbrev{rc}{RC}{run control}{The system for configuring,
  starting and terminating the DAQ}

\newduneabbrev{snb}{SNB}{supernova neutrino burst}{A prompt and brief
  increase in the flux of low energy neutrinos. 
  Can also refer to a trigger command type which may be due to an SNB
  or detector conditions which can mimic its interaction signature}

\newduneabbrev{snble}{SNB/LE}{supernova neutrino burst and low
  energy}{Supernova neutrino burst and low-energy physics program}

\newduneabbrev{snews}{SNEWS}{SuperNova Early Warning System}{A global
  supernova neutrino burst trigger formed by a coincidence of SNB
  triggers collected from participating experiments}

\newduneabbrev{pps}{1PPS signal}{one-pulse-per-second signal}{An
  electrical signal with a fast rise time and which arrives in real
  time with a precise period of one second}

\newduneabbrev{sls}{SLS}{spill location system}{A system residing at
  the DUNE far detector site which provides information, possibly
  predictive, indicating periods of time when neutrinos are being
  produced by the Fermilab Main Injector beam spills}

\newduneabbrev{wib}{WIB}{warm interface board}{Digital electronics
  situated just outside the SP cryostat which receives digital data
  from the FEMBs over cold copper connections and sends it to the RCE
  FE readout hardware}

\newduneabbrev{sipm}{SiPM}{silicon photomultiplier}{A solid-state
  avalanche photo-diode sensitive to single photo-electron signals}

\newduneabbrev{cisc}{CISC}{cryogenic instrumentation and slow controls}{A DUNE
  consortium responsible for the named components}

\newduneword{consortium}{consortium}{A unit of organization in the
  DUNE project focused on some major component of the experiment}

\newduneword{art}{\textit{art}}{A software framework implementing an
  event based execution paradigm} \newduneabbrev{sam}{SAM}{Sequential
  Access via Metadata}{A data handling system to store and retrieve
  files and associated metadata, including a complete record of the
  processing which has used the files}

\newduneword{order}{$\mathcal{O}(n)$}{of order $n$}

\newduneword{3d}{3D}{3 dimensional (also 1D, 2D, etc.)} % not phys

%.........

\newduneword{beamline}{beamline}{A sequence of control and monitoring devices used for the formation of a directed collection of particles}

\newduneabbrev{cdr}{CDR}{Conceptual Design Report}{A formal project
  document required by funding agencies which describes the experiment
  at a conceptual level}

\newduneabbrev{cf}{CF}{Conventional Facilities}{Pertaining to
  construction and operation of buildings or caverns}

\newduneabbrev{cp}{CP}{Charge Parity}{Product of charge and parity
  transformations}

\newduneabbrev{cpt}{CPT}{Charge Parity Time}{Product of charge, parity
  and time-reversal transformations}

\newduneabbrev{cpv}{CPV}{Charge-Parity symmetry violation}{Lack of
  symmetry in a system before and after charge and parity
  transformations are applied}

\newduneword{doe}{DOE}{U.S. Department of Energy}

\newduneword{dune}{DUNE}{Deep Underground Neutrino Experiment}

\newduneword{esh}{ES\&H}{Environment, Safety and Health}

\newduneabbrev{fgt}{FGT}{fine-grained tracker}{A near detector module}

\newduneabbrev{fscf}{FSCF}{far site conventional facilities}{The
  \gslpl{cf} at the DUNE far detector site}

\newduneabbrev{nscf}{NSCF}{near site conventional facilities}{The
  \gslpl{cf} at the DUNE near detector site}

\newduneabbrev{gut}{GUT}{grand unified theory}{A class of theirs which
  unifies the electro-weak and strong forces}


\newduneabbrev{lar}{LAr}{liquid argon}{The liquid phase of argon}

\newduneabbrev{lartpc}{LArTPC}{liquid argon time-projection chamber}{A
  class of detector technology which forms the basis for the DUNE far
  detector modules. 
  It typically entails observation of ionization activity by
  electrical signals and of scintillation by optical}

\newduneabbrev{lbl}{LBL}{long-baseline}{Refers to the distance between
  neutrino source (or near detector) and the far detector. 
  The ``long'' designation is an approximate and relative distinction}

\newduneabbrev{lbnf}{LBNF}{Long-Baseline Neutrino Facility}{The
  organizational entity responsible for developing the neutrino beam
  used by DUNE}

\newduneabbrev{mh}{MH}{mass hierarchy}{Describes the separation
  between the mass squared differences related to the solar and
  atmospheric neutrino problems}

\newduneabbrev{mi}{MI}{Fermilab Main Injector}{An accelerator at
  Fermilab which provides a beam of high energy protons that upon
  striking a target produce secondaries that decay to provide the
  neutrinos directed toward the DUNE far detector}

\newduneabbrev{pot}{POT}{protons on target}{Typically used as a unit
  of normalization which the number of protons striking the neutrino
  production target}

\newduneabbrev{qa}{QA}{quality assurance}{The process by which a
  certain level of quality is maintained in some system}

\newduneabbrev{qc}{QC}{quality control}{A system of maintaining
  quality through testing products against a specification}

\newduneabbrev{sm}{SM}{Standard Model}{Refers to a theory describing
  the interaction of elementary particles}

\newduneabbrev{tdr}{TDR}{Technical Design Report}{A formal project
  document required by funding agencies which describes the experiment
  at a conceptual level}


%%%%%%%%%%%%% PROJECT AND PHYSICS VOLUME list for acronyms below %%%%%%%%%%%%
\newduneabbrev{ckm}{CKM matrix}{Cabibbo-Kobayashi-Maskawa
  matrix}{Refers to the matrix describing the mixing between mass and
  weak eigenstates of quarks}

\newduneabbrev{cl}{CL}{confidence level}{Refers to a probability
  used to determine the value of a random variable given its
  distribution}

\newduneabbrev{pmns}{PMNS matrix}{Pontecorvo-Maki-Nakagawa-Sakata
  matrix}{Describes the mixing between mass and weak eigenstates of
  the neutrino}



%%%%%%%%%%%%%%%%%.....................

%%%%%%%%%%%%% PROJECT AND DETECTORS VOLUME list for acronyms below %%%%%%%%%%%%

% fixme: should not have degenerate definition.  This also should be an abbrev.
\newduneword{blm}{BLM}{(in Volume 4) beamline measurement (system); (in Volume 3) beam loss monitor}

\newduneabbrev{cpa}{CPA}{cathode plane assembly}{The component of the SP detector module which provides the drift HV cathode}

\newduneabbrev{fc}{FC}{field cage}{The component of a LArTPC which contains and shapes the applied electric field}

\newduneabbrev{topfc}{top FC}{top field cage}{The horizontal portions of the FC on the top}

\newduneabbrev{botfc}{bottom FC}{bottom field cage}{The horizontal portions of the FC on the bottom}

\newduneabbrev{ewfc}{endwall FC}{endwall field cage}{The vertical portions of the FC near the wall}

\newduneabbrev{gp}{GP}{ground plane}{An electrode which is held to be
  electrically neutral relative to Earth ground voltage}

\newduneabbrev{alara}{ALARA}{as low as reasonably
  achievable}{Typically used with regard management of radiation
  exposure but may be used more generally. It means making every
  reasonable effort to maintain e.g., exposures, to as far below the
  limits as practical, consistent with the purpose for which the
  activity is undertaken,}

\newduneabbrev{ecal}{ECAL}{electromagnetic calorimeter}{A detector
  component that measure energy deposition of traversing particles}

\newduneabbrev{hv}{HV}{high voltage}{Generally describes a voltage
  applied to drive the motion of free electrons through some media}

% can also use in the text: \dword{sp} \dword{detmodule} 
\newduneword{spmod}{SP module}{single-phase detector module}
\newduneword{dpmod}{DP module}{dual-phase detector module}

\newduneabbrev{tc}{TC}{technical coordination}{A dedicated DUNE
  project effort responsible for assuring proper interfaces between
  the collaboration consortia}


%%%%%%%%%%%%% PHYSICS AND DETECTORS VOLUME list for acronyms below %%%%%%%%%%%%

\newduneabbrev{cc}{CC}{charged current}{Refers to an interaction
  between elementary particles where a charged weak force carrier
  ($W^+$ or $W^-$) is exchanged}


\newduneabbrev{dis}{DIS}{deep inelastic scattering}{Refers to
  interaction of an elementary charged particle with a nucleus in an
  energy range where the interaction can be modeled as being with
  individual nucleons}

\newduneabbrev{fsi}{FSI}{final-state interactions}{Refers to
  interactions between elementary or composite particles subsequent to
  the initial, fundamental particle interaction, such as may occur as
  the products exit a nucleus}

\newduneabbrev{geant4}{GEANT4}{GEometry ANd Tracking, version 4}{A
  software toolkit for the simulation of the passage of particles
  through matter using Monte Carlo methods}

\newduneabbrev{genie}{GENIE}{Generates Events for Neutrino Interaction
  Experiments}{Software providing an object-oriented neutrino
  interaction simulation resulting in kinematics of the products of
  the interaction}

\newduneabbrev{mc}{MC}{Monte Carlo}{Refers to a method of numerical
  integration which entails the statistical sampling of the integrand
  function. 
  Forms the basis for some types of detector and physics simulations}

\newduneabbrev{qe}{QE}{quasi-elastic}{Refers to interaction between
  elementary particles and a nucleus in an energy range where the
  interaction can be modeled as occurring between constituent quarks
  of one nucleon and resulting in no bulk recoil of the resulting
  nucleus}

%%%%%%%%%%%%%%%%%%%%%%%%% PROJECT VOLUME list for acronyms below %%%%%%%%%%%%%%%

\newduneabbrev{mou}{MOU}{memorandum of understanding}{A document
  summarizing an agreement between two or more parties}

\newduneabbrev{pip2}{PIP-II}{Proton Improvement Plan II}{A plan for
  improving the protons on target delivered for the DUNE neutrino
  production beam. 
  This is revision two of this plan and is followed by a PIP-III}

\newduneabbrev{sdsta}{SDSTA}{South Dakota Science and Technology
  Authority}{The legal entity that manages the Stanford Underground
  Research Facility}

\newduneabbrev{wbs}{WBS}{Work Breakdown Structure}{An organizational
  project management tool by which the tasks to be performed are
  partitioned in a hierarchical manner}

%%%%%%%%%%%%%%%%%%%%%%%%% PHYSICS VOLUME list for acronyms below %%%%%%%%%%%%%%%
\newduneabbrev{br}{BR}{branching ratio}{A fractional probability for a
  decay of a composite particle to occur into some specified set or
  sets of products}

\newduneabbrev{dm}{DM}{dark matter}{The term given to the unknown
  matter or force which explains measurements of motion of galaxies
  which are otherwise inconsistent with the amount of mass associated
  with observed amount of photon production}

\newduneabbrev{dsnb}{DSNB}{Diffuse Supernova Neutrino Background}{The
  term describing the pervasive, constant flux of neutrinos due to all
  past supernova neutrino bursts}

\newduneabbrev{globes}{GLoBES}{General Long-Baseline Experiment
  Simulator}{A software package for simulating energy spectra of
  neutrino flux, interaction and measured (after application of some
  model of a detector response) energy spectra}


% are these really used anywhere?
%\newduneword{l/e}{L/E}{length-to-energy ratio}
%\newduneword{lri}{LRI}{long-range interactions}
%\newduneword{solarmass}{$M_{\odot}$}{solar mass}

\newduneabbrev{nc}{NC}{neutral current}{Refers to an interaction
  between elementary particles where a neutrally charged weak force carrier
  ($Z^0$) is exchanged}

\newduneabbrev{nh}{NH}{normal hierarchy}{Refers to the neutrino mass
  eigenstate ordering whereby the sign of the mass squared difference
  associated with the atmospheric neutrino problem is positive}

\newduneabbrev{ih}{IH}{inverted hierarchy}{Refers to the neutrino mass
  eigenstate ordering whereby the sign of the mass squared difference
  associated with the atmospheric neutrino problem is negative}

\newduneabbrev{nsi}{NSI}{nonstandard interactions}{A general class of
  theory of elementary particles other than the Standard Model}

\newduneabbrev{msw}{MSW}{Mikheyev-Smirnov-Wolfenstein effect}{Explains
  the oscillatory behavior of neutrinos produced inside the Sun as
  they traverse the solar matter}

% \newduneword{sme}{SME}{Standard-Model Extension}

\newduneabbrev{susy}{SUSY}{supersymmetry}{Theoretical symmetry between a Fermion and a Boson}

\newduneabbrev{wimp}{WIMP}{weakly-interacting massive particle}{A
  hypothesized particle which may be a component of dark matter}

%%%%%%%%%%%%%%%%%%%%%%%%% DETECTORS VOLUME list for acronyms below %%%%%%%%%%%%%%%

\newduneabbrev{ce}{CE}{Cold Electronics}{Generally, describes readout electronics that operate at cryogenic temperature}

\newduneabbrev{crp}{CRP}{Charge-Readout Plane}{A collection of
  electrodes in a planar arrangement placed at a particular voltage
  relative to some applied electric field such that drifting electrons
  may be collected and their number and time may be measured}

\newduneabbrev{dram}{DRAM}{dynamic random access memory}{A computer memory technology}

% these should be abbrev and maybe combined.
\newduneword{fermilab}{Fermilab}{Fermi National Accelerator Laboratory (in Batavia, IL, the Near Site)}
\newduneword{fnal}{FNAL}{see Fermilab}


\newduneabbrev{fs}{FS}{full stream}{Relates to a data stream that has not undergone selection, compression or other forms of reduction}

\newduneword{lem}{LEM}{Large Electron Multiplier}

\newduneabbrev{lng}{LNG}{liquefied natural gas}{Pertaining to natural gas in its liquid phase}

% this is pretty generic and isn't referenced.
%\newduneword{mesh}{mesh screen}{A fine mesh screen, glued directly to
%  the steel frame on both sides of each APA in the single-phase TPC,
% creates a uniform ground layer beneath the wire planes.}

\newduneabbrev{mip}{MIP}{minimum ionizing particle}{Refers to a
  momentum traversing some medium such that the particle is losing
  near the minimum amount of energy per distance traversed}

%\newduneword{abc}{MTS}{Materials Test Stand}
%\newduneword{muid}{MuID}{muon identifier (detector)}

%\newduneword{abc}{OPERA}{Oscillation Project with Emulsion-Racking Apparatus (experiment at LNGS)}
%\newduneword{abc}{NND}{(used only in Volume 4 Chapter 7) near neutrino detector, same as ND}
%\newduneword{abc}{OD}{outer diameter}

% there is also pds
\newduneabbrev{pd}{PD}{photon detector}{Refers to the detector
  elements involved in measurement of number and arrival times of
  optical photons produced in a detector volume}

\newduneabbrev{pmt}{PMT}{photomultiplier tube}{A device that makes use
  of the photoelectric effect to produce an electrical signal from the
  arrival of a optical photons}

\newduneabbrev{ppm}{PPM}{parts per million}{A number equal to $10^{-6}$}
\newduneabbrev{ppb}{PPB}{parts per billion}{A number equal to $10^{-9}$}
\newduneabbrev{ppt}{PPT}{parts per trillion}{A number equal to $10^{-12}$}

\newduneword{rio}{RIO}{reconfigurable input output}

\newduneword{rpc}{RPC}{resistive plate chamber}
\newduneword{s/n}{S/N}{signal-to-noise (ratio)}
\newduneword{ssp}{SSP}{SiPM signal processor}
\newduneword{sbn}{SBN}{Short-Baseline Neutrino program (at Fermilab)}
\newduneword{stt}{STT}{straw tube tracker}
%\newduneword{abc}{SURF (also Sanford Lab)}{Sanford Underground Research Facility (in Lead, SD, the Far Site)}
\newduneword{tr}{TR}{transition radiation}
%\newduneword{abc}{W}{Watt (also mW, kW, MW) }
%\newduneword{abc}{WA105}{Single-Phase LArTPC and the Long Baseline Neutrino Observatory Demonstration}
\newduneword{wire board}{wire board}{At the head end of the APA in the single-phase TPCr, stacks of electronics boards referred to as ``wire boards'' are arrayed to anchor the wires.  They also provide the connection between the wires and the cold electronics.}

\newduneabbrev{wls}{WLS}{wavelength shifting}{A material or process by
  which incident photons are absorbed by a material and photons are
  emitted at a different, typically longer, wavelength}

\newduneabbrev{tpb}{TPB}{tetraphenyl butadiene}{A type of wavelength shifting material}

\newduneword{sft}{SFT}{signal feedthrough}

\newduneword{catiroc}{CATIROC}{charge and time integrated readout chip}

\newduneabbrev{wr}{WR}{White Rabbit}{A timing system}

\newduneabbrev{mch}{MCH}{MicroTCA Carrier Hub}{An network switching device}

\newduneabbrev{wrmch}{WR-MCH}{White Rabbit MicroTCA Carrier Hub}{add def}

\newduneabbrev{cmp}{CMP}{Configuration Management Plan}{A project
  management device by for the management and evolution of the
  project configuration}

\newduneabbrev{qap}{QAP}{Quality Assurance Plan}{A project
  management device for planning \gls{qa}}

\newduneabbrev{ieshp}{IESHP}{Integrated Environmental, Safety and Health Plan}{Refers to the LBNF/DUNE project planning instrument}

\newduneabbrev{dmp}{DMP}{Data Management Plan}{A project management device to state how the experimental data will be managed}

\newduneabbrev{qam}{QAM}{Quality Assurance Manager}{The manager of \gls{qa} for the LBNF/DUNE project}

\newduneabbrev{dss}{DSS}{Detector Support System}{A system of supporting the detector}

\newduneabbrev{itf}{ITF}{Integration \& Test Facility}{A facility where various detector components will be tested prior to installation}

\newduneabbrev{tco}{TCO}{Temporary Construction Opening}{An opening in the LArTPC cryostat utilized during construction}

\newduneabbrev{uit}{UIT}{Underground Installation Team}{An organizational unit responsible for installation in the SURF mine}

\newduneabbrev{pdr}{PDR}{Preliminary Design Review}{A project management device by which an early design is reviewed}
\newduneabbrev{fdr}{FDR}{Final Design Review}{A project management device by which a final design is reviewed}
\newduneabbrev{prr}{PRR}{Production Readiness Review}{A project management device by which the production readiness is reviewed}
\newduneabbrev{ppr}{PPR}{Production Progress Review}{A progress management device by which the progress of production is reviewed}
\newduneabbrev{edms}{EDMS}{electronic document management system}{A computerized system by which documents are managed}

\newduneword{wrgm}{WR Grand Master}{White Rabbit Grand Master}

%%%%% Software and computing %%%%

\newduneword{larsoft}{LArSoft}{Liquid Argon Software (LArSoft),  a shared base of physics software across Liquid Argon (LAr) Time Projection Chamber (TPC) experiments.}
\newduneword{nova}{NOvA}{The NOvA off-axis neutrino oscillation experiment at Fermilab}
\newduneword{minerva}{MINERvA}{The MINERvA neutrino cross sections experiment at Fermilab}
\newduneword{microboone}{MicroBooNE}{The Liquid Argon TPC-based MicroBooNE neutrino oscillation experiment at Fermilab}
\newduneword{wirecell}{wire-cell}{Wire-Cell is a tomographic automated 3D neutrino event reconstruction method for LArTPC's}
\newduneword{ftslite}{F-FTS-lite}{Light weight version of the Fermilab File Transfer system used for rapid data transfers out of the online systems}
\newduneword{fts}{FTS}{File Transfer System developed at Fermilab to catalog and move data to permanent storage}

%%% new ones that I haven't categorized (Anne)
\newduneword{35t}{35 ton prototype}{The 35 ton prototype cryostat and \gls{sp} detector built at Fermilab before the \gls{protodune} detectors.}

\newduneabbrev{cuc}{CUC}{central utility cavern}{The central underground location containing utilities such as central cryogenic systems, the underground data center and control room, etc.}
\newduneabbrev{cfd}{CFD}{computational fluid dynamics}{}
\newduneword{vuv}{VUV}{Vacuum Ultra-Violet}