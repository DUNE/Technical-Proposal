% see dune-words.tex for explanation.

\usepackage[acronyms,toc]{glossaries}
\makeglossaries


% \newduneword{label}{term}{description}
\newcommand{\newduneword}[3]{
    \newglossaryentry{#1}{
        text={#2},
        long={#2},
        name={\glsentrylong{#1}},
        first={\glsentryname{#1}},
        firstplural={\glsentrylong{#1}\glspluralsuffix},
        description={#3}
    }
}

%                 1      2     3       4
% \newduneabbrev{label}{abbrev}{term}{description}
\newcommand{\newduneabbrev}[4]{
  % this makes acronym entries even if they neither term or abbrev is
  % referenced in the text.
  % \newacronym[see={[Glossary:]{#1}}]{abbrev-#1}{#2}{#3}
  \newacronym{abbrev-#1}{#2}{#3}
  \newglossaryentry{#1}{
    text={#2},
    long={#3},
    name={\glsentrylong{#1}{} (\glsentrytext{#1}{})},
    first={\glsentryname{#1}},
    firstplural={\glsentrylong{#1}\glspluralsuffix{} (\glsentrytext{#1}\glspluralsuffix{})},
    description={#4}
  }
}
\newcommand{\dshort}[1]{\acrshort{abbrev-#1}}
\newcommand{\dlong}[1]{\acrlong{abbrev-#1}}

\newcommand{\dword}[1]{\gls{#1}}
\newcommand{\dwords}[1]{\glspl{#1}}
\newcommand{\Dword}[1]{\Gls{#1}}
\newcommand{\Dwords}[1]{\Glspl{#1}}

%%%%%%     START ADDING WORDS, BY VOL, IN ALPHABETICAL ORDER IF POSSIBLE!    %%%%%%%%


% Model to copy

%\newduneword{abc}{ABC's real name}{Sentence or two describing ABC. It can use other dunewords defined here, e.g., \gls{fe}.}


%%%%%%     NOT VOLUME SPECIFIC   %%%%%

% front-end (used in detectors and sw-computing)
\newduneabbrev{fe}{FE}{front-end}{The front-end refers a point which is
  ``upstream'' of the data flow for a particular subsystem. 
  For example the front-end electronics is where the cold electronics
  meet the sense wires of the TPC and the front-end DAQ is where the
  DAQ meets the output of the electronics}


