% This holds definitions of macros to enforce consistency in units.

% This file is the sole location for such definitions.  Check here to
% learn what there is and add new ones only here.  

% also see defs.tex for names.


% see
%  http://ctan.org/pkg/siunitx
%  http://mirrors.ctan.org/macros/latex/contrib/siunitx/siunitx.pdf

% Examples:
%  % angles
%  \ang{1.5} off-axis
%
%  % just a unit
%  \si{\kilo\tonne}
%
%  % with a value:
%  \SI{10}{\mega\electronvolt}

%  range of values:
% \SIrange{60}{120}{\GeV}

% some shorthand notation
\DeclareSIUnit \kton {\kilo\tonne}
\DeclareSIUnit \kt {\kilo\tonne}
\DeclareSIUnit \Mt {\mega\tonne}
\DeclareSIUnit \eV {\electronvolt}
\DeclareSIUnit \keV {\kilo\electronvolt}
\DeclareSIUnit \MeV {\mega\electronvolt}
\DeclareSIUnit \GeV {\giga\electronvolt}
\DeclareSIUnit \km {\kilo\meter}
\DeclareSIUnit \kW {\kilo\watt}
\DeclareSIUnit \MW {\mega\watt}
\DeclareSIUnit \MHz {\mega\hertz}
\DeclareSIUnit \mrad {\milli\radian}
\DeclareSIUnit \year {year}
\DeclareSIUnit \POT {POT}
\DeclareSIUnit \sig {$\sigma$}
\DeclareSIUnit\parsec{pc}
\DeclareSIUnit\lightyear{ly}
\DeclareSIUnit\foot{ft}
\DeclareSIUnit\ft{ft}
% for a bare kt-year
\def\ktyr{\si[inter-unit-product=\ensuremath{{}\cdot{}}]{\kt\year}\xspace}
\def\Mtyr{\si[inter-unit-product=\ensuremath{{}\cdot{}}]{\Mt\year}\xspace}
\def\msr{\si[inter-unit-product=\ensuremath{{}\cdot{}}]{\meter\steradian}\xspace}
\def\ktMWyr{\si[inter-unit-product=\ensuremath{{}\cdot{}}]{\kt\MW\year}\xspace}

% used for hyphen, obsolete now: \newcommand{\SIadj}[2]{\SI[number-unit-product = -]{#1}{#2}}
% change command definition Nov 2017 in case people copy e.g., \ktadj from CDR text.
% E.g., \ktadj{10} now renders the same as \SI{10}{\kt}
\newcommand{\SIadj}[2]{\SI{#1}{#2}}

% Adjective form of some common units (Nov 2107 changed to be same as normal form, no hyphen)
% "the 10-kt detector"

\newcommand{\ktadj}[1]{\SIadj{#1}{\kt}}
% "the 1,300-km baseline"
\newcommand{\kmadj}[1]{\SIadj{#1}{\km}}
% "a 567-keV endpoint"
\newcommand{\keVadj}[1]{\SIadj{#1}{\keV}}
% "Typical 20-MeV event"
\newcommand{\MeVadj}[1]{\SIadj{#1}{\MeV}}
% "Typical 2-GeV event"
\newcommand{\GeVadj}[1]{\SIadj{#1}{\GeV}}
% "the 1.2-MW beam"
\newcommand{\MWadj}[1]{\SIadj{#1}{\MW}}
% "the 700-kW beam"
\newcommand{\kWadj}[1]{\SIadj{#1}{\kW}}
% "the 100-tonne beam"
\newcommand{\tonneadj}[1]{\SIadj{#1}{\tonne}}
% "the 4,850-foot depth beam"
\newcommand{\ftadj}[1]{\SIadj{#1}{\ft}}
%

% Mass exposure, people like to put dots between the units
% \newcommand{\ktyr}[1]{\SI[inter-unit-product=\ensuremath{{}\cdot{}}]{#1}{\kt\year}}
% must make usage of \ktyr above consistent with this one before turning on

% Beam x mass exposure, people like to put dots between the units
\newcommand{\ktmwyr}[1]{\SI[inter-unit-product=\ensuremath{{}\cdot{}}]{#1}{\kt\MW\year}}
