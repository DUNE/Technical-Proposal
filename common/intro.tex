% Intro shared by all subsections

\section{An International Physics Program}

The global neutrino physics community is developing a multi-decade
physics program to measure unknown parameters of the Standard Model of
particle physics and search for new phenomena.  The program will be carried out as an international,
leading-edge, dual-site experiment for neutrino science and proton decay studies, which 
is known as the Deep Underground Neutrino Experiment (DUNE).
The detectors for this experiment will be designed, built, commissioned and operated by the international DUNE Collaboration. The facility required to support this experiment, the Long-Baseline Neutrino Facility (LBNF), is hosted by Fermilab and its design and construction is organized as a DOE/Fermilab project incorporating international partners. Together LBNF and DUNE will comprise the world's highest-intensity neutrino beam at Fermilab, in Batavia, IL, a high-precision near detector on the Fermilab site, a massive liquid argon time-projection chamber (LArTPC) far detector installed deep underground at the Sanford Underground Research Facility (SURF) \SI{1300}{\km} away in Lead, SD, and all of the conventional and technical facilities necessary to support the beamline and detector systems. 


The strategy for executing the experimental program presented in this Technical
Design Report (TDR) has been developed to meet the requirements 
set out in the P5 report~\cite{p5report} and takes into account the recommendations of the European Strategy for Particle Physics~\cite{ESPP-2012}. It adopts a model where U.S. and international funding agencies 
share costs on the DUNE detectors, and CERN and other participants provide in-kind contributions 
to the supporting infrastructure of LBNF. LBNF and DUNE have been and will remain tightly coordinated as DUNE collaborators 
develop the detectors and infrastructure that will carry out the scientific program.
  
The scope of LBNF is
\begin{itemize}
\item an intense neutrino beam aimed at the far site,
\item a beamline measurement system
\item conventional facilities at both the near and far sites, and
\item cryogenics infrastructure to support the DUNE detector at the far site.
\end{itemize}

The DUNE detectors include
\begin{itemize}
\item a high-performance neutrino detector 
located a few hundred meters downstream of the neutrino source, and
\item a massive liquid argon time-projection chamber (LArTPC) neutrino detector located deep underground at the far site.
\end{itemize}

With the facilities provided by LBNF and the detectors provided by
DUNE, the DUNE Collaboration proposes to mount a focused attack on the
puzzle of neutrinos with broad sensitivity to neutrino oscillation
parameters in a single experiment.  The focus of the scientific
program is the determination of the neutrino mass hierarchy and the
explicit demonstration of leptonic CP violation, if it exists, by
precisely measuring differences between the oscillations of muon-type
neutrinos and antineutrinos into electron-type neutrinos and
antineutrinos, respectively. Siting the far detector deep underground
will provide exciting additional research opportunities in nucleon
decay, studies utilizing atmospheric neutrinos, and neutrino
astrophysics, including measurements of neutrinos from a core-collapse
supernova should such an event occur in our galaxy during the
experiment's lifetime.

%%%%%%%%%%%%%%%%%%%%%%%%%%%%%%%%%%%%%%%%%%%%%%%%%%%%%%%%%%%%%%%
\section{The LBNF/DUNE Technical Design Report Volumes}

%%%%%%%%%%%%%%%%%%%%%%%%%%%%%%%%%%%
\subsection{A Roadmap of the TDR}

The LBNF/DUNE TDR describes the proposed physics program and \fixme{baselined?}
technical designs.

The TDR is composed of \fixme{some number of} volumes and is supplemented by several annexes that 
provide details on the physics program and technical designs. The volumes are as follows

\begin{itemize}
\item \volintro{}\cite{tdr-vol-1} provides an executive summary of and strategy for the experimental 
program and introduces the TDR.
\item \volphys{}\cite{tdr-vol-2} outlines the scientific objectives and describes the physics studies that 
the DUNE Collaboration will undertake to address them.
\item \vollbnf{}\cite{tdr-vol-3} describes the LBNF Project, which includes design and construction of the 
beamline at Fermilab, the conventional facilities at both Fermilab and SURF, and the cryostat
 and cryogenics infrastructure required for the DUNE far detector.
\item \voldunenear{}\cite{tdr-vol-4}  describes design, construction and 
commissioning of the DUNE near detector. 
\item \voldunefar{}\cite{tdr-vol-5}  describes design, construction and 
commissioning of the DUNE far detectors.
\end{itemize}

More detailed information for each of these volumes is provided in a set of annexes listed on LBNF and DUNE's shared  \href{https://web.fnal.gov/project/LBNF/SitePages/Proposals%20and%20Design%20Reports.aspx}{\textit{Proposals and Design Reports} page}. 

%%%%%%%%%%%%%%%%%%%%%%%%%%%%%%%%%%%
%\subsection{About this Volume}  <----- follows in overview chapter file of indiv volume

