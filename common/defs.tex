% This holds definitions of macros to enforce consistency in names.

% This file is the sole location for such definitions.  Check here to
% learn what there is and add new ones only here.  

% also see units.tex for units.  Units can be used here.

%%% Common terms

% Check here first, don't reinvent existing ones, add any novel ones.
% Use \xspace.

%%%%% Anne adding macros for referencing TDR volumes and annexes Apr 20, 2015 %%%%%
\def\expshort{DUNE\xspace}
\def\explong{The Deep Underground Neutrino Experiment\xspace}

%\def\thedocsubtitle{LBNF/DUNE Technical Design Report (DRAFT)}
\def\thedocsubtitle{Deep Underground Neutrino Experiment (DUNE)} 
\def\tdrtitle{Technical Proposal}

% For the document titles (not italicized)
\def\voltitleexecsumm{Volume 1: Executive Summary\xspace}
\def\voltitlespfd{Volume 2: The Single-Phase Far Detector\xspace}
\def\voltitledpfd{Volume 3: The Dual-Phase Far Detector\xspace}
\def\voltitleswcomputing{Volume 4: Software and Computing\xspace}



% For use within volumes (italicized)
\def\volexecsumm{ \textbf{23 Feb 2018: First draft of the TP volumes due} \\ \vspace{1cm} Volume 1: \textit{Executive Summary\xspace}}

\def\volspfd{ \textbf{23 Feb 2018: First draft of the TP volumes due} \\ 
\vspace{1cm} 
Volume 2: \textit{The Single-Phase Far Detector\xspace}  
}

\def\voldpfd{ \textbf{23 Feb 2018: First draft of the TP volumes due} \\ \vspace{1cm} Volume 3: \textit{The Dual-Phase Far Detector\xspace} }

\def\volswcomputing{ \textbf{23 Feb 2018: First draft of the TP volumes due} \\ \vspace{1cm} Volume 4: \textit{Software and Computing\xspace}}




% Things about oscillation
%
\newcommand{\numu}{$\nu_\mu$\xspace}
\newcommand{\nue}{$\nu_e$\xspace}
\newcommand{\nutau}{$\nu_\tau$\xspace}

\newcommand{\anumu}{$\bar\nu_\mu$\xspace}
\newcommand{\anue}{$\bar\nu_e$\xspace}
\newcommand{\anutau}{$\bar\nu_\tau$\xspace}

\newcommand{\dm}[1]{$\Delta m^2_{#1}$\xspace} % example: \dm{12}

\newcommand{\sinst}[1]{$\sin^2\theta_{#1}$\xspace} % example \sinst{12}
\newcommand{\sinstt}[1]{$\sin^22\theta_{#1}$\xspace}  % example \sinstt{12}

\newcommand{\deltacp}{$\delta_{\rm CP}$\xspace}   % example \deltacp
\newcommand{\mdeltacp}{$\delta_{\rm CP}$}   %%%%%%%%%%  <--- missing something; what's the m for?

\newcommand{\nuxtonux}[2]{$\nu_{#1} \to \nu_{#2}$\xspace}  % example \nuxtonux23 (no {...} )
\newcommand{\numutonumu}{\nuxtonux{\mu}{\mu}}
\newcommand{\numutonue}{\nuxtonux{\mu}{e}}
% Add chi sqd MH?  avg delta chi sqd?

% atmospheric neutrinos and PDK
\newcommand{\ptoknubar}{$p^+ \rightarrow K^+ \overline{\nu}$\xspace}
\newcommand{\ptoepizero}{$p^+ \rightarrow e^+ \pi^0$\xspace}

% Names of expts or detectors
\newcommand{\cherenkov}{Cherenkov\xspace}
\newcommand{\kamland}{KamLAND\xspace}
\newcommand{\kkande}{Kamiokande\xspace}  %%%% <---- changed to make shorter
\newcommand{\superk}{Super--Kamiokande\xspace}
\newcommand{\hyperk}{Hyper--Kamiokande\xspace}
\newcommand{\miniboone}{MiniBooNE\xspace}
\newcommand{\minerva}{MINER$\nu$A\xspace}
\newcommand{\nova}{NO$\nu$A\xspace}
\newcommand{\numi}{NuMI\xspace}
\newcommand{\lariat}{LArIAT\xspace}
\newcommand{\argoneut}{ArgoNeuT\xspace}

% Random
\newcommand{\lartpc}{LArTPC\xspace}
\newcommand{\SURF}{Sanford Underground Research Facility\xspace}
\newcommand{\globes}{GLoBES\xspace}
\newcommand{\larsoft}{LArSoft\xspace}
\newcommand{\snowglobes}{SNOwGLoBES\xspace}

% Isotopes
\def\argon40{$^{40}$Ar}  %%%%%       <--- \Ar40 doesn't work; don't know why
\def\Ar39{$^{39}$Ar}
\def\Cl40{$^{40}$Cl}
\def\K40{$^{40}$K}
\def\B8{$^{8}$B}

% Parameters
\def\driftvelocity{\SI{1.6}{\milli\meter/\micro\second}\xspace}
\def\lartemp{\SI{88}\,K\xspace}
% What other parameters to include?  Wire spacings? Drift lengths? Decay pipe length?
