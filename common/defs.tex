% This holds definitions of macros to enforce consistency in names.

% This file is the sole location for such definitions.  Check here to
% learn what there is and add new ones only here.  

% also see units.tex for units.  Units can be used here.

%%% Common terms

% Check here first, don't reinvent existing ones, add any novel ones.
% Use \xspace.

%%%%% Anne adding macros for referencing TDR volumes and annexes Apr 20, 2015 %%%%%
\def\expshort{DUNE\xspace}
\def\explong{The Deep Underground Neutrino Experiment\xspace}

%\def\thedocsubtitle{LBNF/DUNE Technical Design Report (DRAFT)}
\def\thedocsubtitle{Deep Underground Neutrino Experiment (DUNE)} 
\def\tdrtitle{Technical Proposal}

% For the document titles (not italicized)
\def\voltitleexecsumm{Volume 1: Executive Summary\xspace}
\def\voltitlespfd{Volume 2: The Single-Phase Far Detector\xspace}
\def\voltitledpfd{Volume 3: The Dual-Phase Far Detector\xspace}
%\def\voltitleswcomputing{Volume 4: Software and Computing\xspace}



% For use within volumes (italicized)
\def\volexecsumm{\textbf{Volume 1:  Executive Summary\xspace}}

\def\volspfd{\textbf{Volume 2: The Single-Phase Far Detector Module Design\xspace}  
}

\def\voldpfd{\textbf{Volume 3: The Dual-Phase Far Detector Module Design\xspace} }

\def\volswcomputing{\textbf{Volume 4: Software and Computing\xspace}}




% Things about oscillation
%
\newcommand{\numu}{$\nu_\mu$\xspace}
\newcommand{\nue}{$\nu_e$\xspace}
\newcommand{\nutau}{$\nu_\tau$\xspace}

\newcommand{\anumu}{$\bar\nu_\mu$\xspace}
\newcommand{\anue}{$\bar\nu_e$\xspace}
\newcommand{\anutau}{$\bar\nu_\tau$\xspace}

\newcommand{\dm}[1]{$\Delta m^2_{#1}$\xspace} % example: \dm{12}

\newcommand{\sinst}[1]{$\sin^2\theta_{#1}$\xspace} % example \sinst{12}
\newcommand{\sinstt}[1]{$\sin^22\theta_{#1}$\xspace}  % example \sinstt{12}

\newcommand{\deltacp}{$\delta_{\rm CP}$\xspace}   % example \deltacp
\newcommand{\mdeltacp}{$\delta_{\rm CP}$}   %%%%%%%%%%  <--- missing something; what's the m for?

\newcommand{\nuxtonux}[2]{$\nu_{#1} \to \nu_{#2}$\xspace}  % example \nuxtonux23 (no {...} )
\newcommand{\numutonumu}{\nuxtonux{\mu}{\mu}}
\newcommand{\numutonue}{\nuxtonux{\mu}{e}}
% Add chi sqd MH?  avg delta chi sqd?

% atmospheric neutrinos and PDK
\newcommand{\ptoknubar}{$p^+ \rightarrow K^+ \overline{\nu}$\xspace}
\newcommand{\ptoepizero}{$p^+ \rightarrow e^+ \pi^0$\xspace}

% Names of expts or detectors
\newcommand{\cherenkov}{Cherenkov\xspace}
\newcommand{\kamland}{KamLAND\xspace}
\newcommand{\kkande}{Kamiokande\xspace}  %%%% <---- changed to make shorter
\newcommand{\superk}{Super--Kamiokande\xspace}
\newcommand{\hyperk}{Hyper--Kamiokande\xspace}
\newcommand{\miniboone}{MiniBooNE\xspace}
\newcommand{\microboone}{MicroBooNE\xspace}
\newcommand{\minerva}{MINER$\nu$A\xspace}
\newcommand{\nova}{NO$\nu$A\xspace}
\newcommand{\numi}{NuMI\xspace}
\newcommand{\lariat}{LArIAT\xspace}
\newcommand{\argoneut}{ArgoNeuT\xspace}

% Random
\newcommand{\lartpc}{LArTPC\xspace}
\newcommand{\globes}{GLoBES\xspace}
\newcommand{\larsoft}{LArSoft\xspace}
\newcommand{\snowglobes}{SNOwGLoBES\xspace}
\newcommand{\docdb}{DUNE DocDB\xspace}
% Isotopes
\def\argon40{$^{40}$Ar}  %%%%%       <--- \Ar40 doesn't work; don't know why
\def\Ar39{$^{39}$Ar}
\def\Cl40{$^{40}$Cl}
\def\K40{$^{40}$K}
\def\B8{$^{8}$B}
\newcommand\isotope[2]{\textsuperscript{#2}#1} % use as, e.g.,: \isotope{Si}{28}

% Parameters common to SP DP
\def\driftvelocity{\SI{1.6}{\milli\meter/\micro\second}\xspace} % same for sp and dp?
\def\lartemp{\SI{88}\,K\xspace}
\def\larmass{\SI{17.5}{\kt}\xspace} % full mass in cryostat
\def\tpcheight{\SI{12.0}{\meter}\xspace} % height of SP TPC, APA, CPA and of DP TPC
\def\cryostatht{\SI{14.1}{\meter}\xspace} % height of cryostat
\def\cryostatlen{\SI{62.0}{\meter}\xspace} % height of cryostat
\def\cryostatwdth{\SI{14.0}{\meter}\xspace} % height of cryostat
\def\nominalmodsize{\SI{10}{kt}\xspace} % nominal module size 10 kt

\def\dunelifetime{\SI{20}{year}\xspace} % nominal operational life time of DUNE experiment


% Parameters SP
\def\spmaxfield{\SI{500}{\volt/\centi\meter}} % SPfield strength
\def\spactivelarmass{\SI{10}{\kt}\xspace} % active mass in cryostat
\def\spmaxdrift{\SI{3.53}{\m}\xspace}
\def\sptpclen{\SI{58}{\meter}\xspace} % length of SP TPC, APA, CPA
\def\apacpapitch{\SI{2.32}{\meter}\xspace} % pitch of SP CPAs and APAs
\def\spfcmodlen{\SI{3.5}{\m}} % length of SP FC module
\def\spnumch{\num{384000}\xspace} % total number of APA readout channels 
\def\spnumpdch{\num{6000}\xspace} % total number of PD readout channels 
\def\uvpitch{\SI{4.669}{\milli\meter}\xspace}
\def\xgpitch{\SI{4.790}{\milli\meter}\xspace}
\def\planespace{\SI{4.75}{\milli\meter}\xspace}

% Parameters DP
\def\dpactivelarmass{\SI{12.096}{\kt}\xspace} % active mass in cryostat
\def\dpfidlarmass{\SI{10.643}{\kt}\xspace} % fiducial mass in cryostat
\def\dpmaxdrift{\SI{12}{\m}\xspace} % max drift length
\def\dptpclen{\SI{60}{\meter}\xspace} % length of TPC
\def\dptpcwdth{\SI{12}{\meter}\xspace} % width of TPC
\def\dpswchpercrp{\num{36}\xspace} % number of anode/lem sandwiches per CRP 
\def\dpnumswch{\num{2880}\xspace} % total number of anode sandwiches in module
\def\dptotcrp{\num{80}\xspace} % total number of CRPs in module
\def\dpchpercrp{\num{1920}\xspace} %  channels per CRP
\def\dpnumcrpch{\num{153600}\xspace} % total number of CRP channels in module
\def\dpchperchimney{\num{6400}\xspace} %  channels per chimney
\def\dpnumpmtch{\num{720}\xspace} % number of PMT channels
\def\dpstrippitch{\SI{3.125}{\milli\meter}\xspace} % pitch of anode strips
\def\dpnumfcmod{\num{244}\xspace} % number of FC modules
\def\dpnumfcres{\num{240}\xspace} % number of FC resistors
\def\dpnumfcrings{\num{60}\xspace} % number of FC rings
\def\dptargetdriftvoltpos{\SI{600}{kV}\xspace} % target drift voltage - positive
\def\dptargetdriftvoltneg{\SI{-600}{kV}\xspace} % target drift voltage - negative

% Nominal readout window time
%% SP has 2.25ms drift time.  The readout is 2*dt + 20%*dt extra.
\def\spreadout{\SI{5.4}{\ms}\xspace}
%% DP has 7.5 ms drift time.  The same (over generous) rule gives 16.5ms
\def\dpreadout{\SI{16.4}{\ms}\xspace}
% Supernova Neutrino Burst buffer and readout window time
\def\snbtime{\SI{30}{\s}\xspace}
% interesting amount of time we might have SNB neutrinos but not yet
% enough to trigger.
\def\snbpretime{\SI{10}{\s}\xspace}
% SP SNB dump size
\def\spsnbsize{\SI{45}{\PB}\xspace}

% keep these three numerically in sync
\def\offsitepbpy{\SI{30}{\PB/\year}\xspace}
\def\offsitegbyteps{\SI{1}{\GB/\s}\xspace}
\def\offsitegbps{\SI{8}{\Gbps}\xspace}

\def\surffnalbw{\SI{100}{\Gbps}\xspace}
\newcommand{\fnal}{Fermilab\xspace}
\newcommand{\surf}{SURF\xspace}
\newcommand{\bnl}{BNL\xspace}
\newcommand{\anl}{ANL\xspace}

% New from Anne March/April 2018
%physics terms
\newcommand{\efield}{E field\xspace}
\newcommand{\lbl}{long-baseline\xspace}
\newcommand{\Lbl}{Long-baseline\xspace}
\newcommand{\rms}{RMS\xspace} % Might want this small caps?
\newcommand{\threed}{3D\xspace}
\newcommand{\twod}{2D\xspace}

%detectors and modules
\newcommand{\fardet}{Far Detector\xspace}
\newcommand{\detmodule}{detector module\xspace}
\newcommand{\dual}{DP\xspace}
\newcommand{\Dual}{DP\xspace}
\newcommand{\single}{SP\xspace}
\newcommand{\Single}{SP\xspace}
\newcommand{\dpmod}{DP detector module\xspace}
\newcommand{\spmod}{SP detector module\xspace}

\newcommand{\lar}{LAr\xspace}

%detector components SP and DP
\newcommand{\dss}{DSS\xspace}
\newcommand{\hv}{high voltage\xspace}
\newcommand{\fcage}{field cage\xspace}
\newcommand{\fc}{FC\xspace}
\newcommand{\fdth}{feedthrough\xspace}
\newcommand{\fcmod}{FC module\xspace}  %%%   eon't need?
\newcommand{\topfc}{top FC\xspace}
\newcommand{\botfc}{bottom FC\xspace}
\newcommand{\ewfc}{endwall FC\xspace}
\newcommand{\pdsys}{PD system\xspace}
\newcommand{\phdet}{photon detector\xspace}
\newcommand{\sipm}{SiPM\xspace}
\newcommand{\pmt}{PMT\xspace}
\newcommand{\phel}{photoelectron\xspace}
\newcommand{\pwrsupp}{power supply\xspace}
\newcommand{\pwrsupps}{power supplies\xspace}

%detector components SP only

%detector components DP only
