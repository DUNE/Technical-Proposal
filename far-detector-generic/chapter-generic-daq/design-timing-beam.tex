

\subsubsection{Beam timing}
\label{sec:fd-daq-design-beamtiming}

The neutrino beam is produced at the Fermilab accelerator complex in
spills of \SI{10}{\us} duration. 
A \dword{sls} at the far detector site will locate the time periods in
the data when beam could be present, based on network packets received
from Fermilab containing predictions of the GPS-time of spills soon to
occur or absolute time stamps of spills recently occurring. 
Experience from MINOS and \nova shows that this can provide beam
triggering with high reliability with some small fraction of late or
dropped packets.
To improve further the system outlined here contains an extra layer
of redundancy in the prediction process. 
Several stages prediction based on recent spill behavior will be applied aiming
for an accuracy of better than 10\% of a readout
time (sub-\si{\ms}) in time for the data to be selected from
the \dword{daq} buffers. 
Ultimately, an offline database will match the actual time of the
spill with the data, thus removing any reliance on real-time network
transfer for this crucial stage of the oscillation measurements. The
network transfer of spill-timing information is simply to ensure a
correctly located and sufficiently wide window of data is considered
as beam data. This system is not required, and is not designed to
provide signals accurate enough to measure neutrino time-of-flight.

The precision to which the spill time can be predicted at \fnal
improves as the acceleration process of the protons producing the
spill in question advances.  The spills currently occur at intervals
of \SI{1.3}{\s}; the system will be designed to work with any interval, and
to be adaptable in case the sequence described here changes.  For
redundancy, three packets will be sent to the far detector for each
spill.  The first is approximately \SI{1.6}{\s} before the spill-time, which
is at the point where a \SI{15}{\Hz} booster cycle is selected; from this
point on, there will be a fixed number of booster cycles until the
neutrinos and the time is subject to a few ms of jitter.  The second
is about \SI{0.7}{\s} before the spill, at the point where the main injector
acceleration is no longer coupled to the booster timing; this is
governed by a crystal oscillator and so has a few \si{\us} of jitter.
The third will be at the so called `\texttt{\$74}' signal generated before the beam line kicker magnet fires
to direct the protons at the LBNF target; this doesn't improve the
timing at the far detector much, but serves as a cross check for
missing packets.  This system is enhanced compared to that of
MINOS-\nova, which only use the third of the above timing signals.  The
reason for the larger uncertainty in the time interval from \SI{1.6}{\s} to
\SI{0.7}{\s} is that the booster cycle time is synchronised to the
electricity supply company's \SI{60}{\Hz} which has a variation of about
1\%.

Arrival-time monitoring information from a year of MINOS data-taking
was analysed, and it was found that 97\% of packets arrived within
\SI{100}{\ms} of being sent and 99.88\% within \SI{300}{\ms}.

The \dword{sls} will therefore have estimators of the GPS-times of
future spills, and recent spills with associated data contained in the
\dwords{daqbuf}. These estimators will improve in precision as
more packets arrive.  The \dword{daq} will use data in a wider window than
usual, if, at the time the trigger decision has to be made, the
precision is less accurate due to missing or late packets.  From the
MINOS monitoting analysis, this will be very rare.

