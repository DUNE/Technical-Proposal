%%%%%%%%%%%%%%%%%%%%%%%%%%%%%%%%%%%
\subsection{Offline Computing (Kurt Biery)}
\label{sec:fd-daq-intfc-fnal-cmptg}

The interface between the \dword{daq} and Offline Computing is
described in DUNE-doc-7123~\cite{docdb-7123}.
The \dword{daq} team will be responsible for reducing the data volume
to the level that is agreed upon by all interested parties, and the
raw data files will be transfered from SURF to Fermilab using a
dedicated network connection.
A disk buffer will be provided by the \dword{daq} on or near the SURF
site to hold the data from several days of data taking so that the
operation of the experiment will not be affected if there happens to
be a network disruption between SURF and Fermilab.
%\fixme{Add ref for docdb-7123.}  done - ATH, 4/16/18

During stable running, the data volume that will be produced by the
\dword{daq} systems of all four \dwords{detmodule} will be no larger
than \offsitepbpy.
This maximum data rate is expected to be independent of the number of
\dwords{detmodule} that are operational.
During the construction of the second, third, and fourth
\dwords{detmodule}, the extra rate per \dword{detmodule} will be used
to gather data to aid in the refinement of the data selection
algorithms.
During commissioning, the data rate is expected to be higher than
nominal running and is anticipated that something on order of one year
data volume will be necessary to commission a \dword{detmodule}.

The disk buffer at SURF is planned to be \SI{300}{\TB} in size.
The data link from SURF to Fermilab will support \surffnalbw
(\offsitepbpy corresponds to about \offsitegbps).
The Offline Computing team will be responsible for developing the
software to manage the transfer of files from SURF to Fermilab.
The \dword{daq} team will be responsible for producing a reference
implementation of the software that is used to access and decode the
raw electronics data.
They will also be responsible for providing the framework for
real-time \dword{dqm}, and both groups will have responsibility for
developing the \dword{dqm} algorithms that run inside this framework.

%%%%%%%%%%%%%%%%%%%%%%%%%%%%%%%%%%%
\subsection{Slow Control}
\label{sec:fd-daq-intfc-sc}
\label{sec:fd-daq-intfc-sc}

The \dword{cisc} systems monitor detector hardware and conditions not
directly involved in taking the data described above.
That data will be stored both locally (in CISC database servers in the
CUC) and offline (the databases will be replicated back to Fermilab)
in a relational database indexed by timestamp.
This will allow bi-directional communications between DAQ and CISC by
reading or inserting data into the database as needed for non
time-critical information.  

For prompt, time sensitive status information such as ``Run is in
progress'' or ``Camera is on'', a low-latency software status register
will be available on the local network to both systems.

There is not a hardware interface, aside from the fact that several
racks of CISC servers will be in the /counting room/ in the CUC, and that rack
monitors in DAQ racks will be read out into the CISC data stream.

Note that life and hardware safety critical items will be hardware
interlocked in the standard Fermilab fashion, and fall outside of this
interface's scope.


%%%%%%%%%%%%%%%%%%%%%%%%%%%%%%%%%%%
\subsection{External Systems (Giles \& Alec)}
\label{sec:fd-daq-intfc-ext}

\fixme{Need to receive information on beam spills (Giles) , SNEWS (Alec).}

The DAQ will need to save data based on external triggers, for
example: for times around when the pulse of beam neutrinos from LBNF
arrives at the Far Detector; or if notice of an interesting
astrophysical event is given by \dword{snews}\cite{snews} or LIGO.
This could involve going back in time to save data that has already
been buffered (see Sec.~\ref{sec:fd-daq-fero}), or changing the trigger
or zero suppression criteria for data in the interesting time period.

\fixme{Do we want subsubsections below instead of explicit paragraphs?}

\paragraph{Beam Trigger:} The method for predicting and receiving the
time of the beam spill is described in
Sec.~\ref{sec:fd-daq-design-beamtiming}.
Once that time is known to the DAQ, a high level trigger can be issued
to ensure that the necessary full data can be saved from the buffer
and saved as an event.

\paragraph{Astrophysical Triggers:} \dfirst{snews} is a coincidence
network of neutrino experiments which are individually sensitive to
the burst of neutrinos which would be observed from a core-collapse
supernova somewhere in our galaxy.
While DUNE should be sensitive to such a burst on its own able to
contribute to the coincidence network (Sec.~\ref{sec:fd-daq-sel}) via
a TCP/IP socket, being able to save data based on other observations
adds an additional chance to not miss saving this rare and valuable
data.
A \dword{snews} alert is formed when two or more neutrino experiments
report a potential supernova signal within \SI{10}{\s}.
The earliest time in the coincidence is then sent via running a script
on the \dword{snews} server at BNL provided by the experiment wishing
to receive the alert.
The latency from neutrino burst is set by the response time of the
second fastest detector to report to \dword{snews}: this could be as
short as seconds, but might be tens of seconds.
At latencies larger than \SI{10}{\s}, full data might not be
available, but selected data is expected to be manageable.

There are other astrophysical triggers available for occurrences which
DUNE might not sensitive to individually, but could be either rarely
or if taken as an ensemble.
Gravitational wave triggers will be available (details being worked
out during the current LIGO shutdown), as will high energy photon
transients, most notably gamma ray bursts.
In fact, cooperation between LIGO/VIRGO, the Gamma Ray Coordinates
Network (GCN)\footnote{Described in detail at
  \url{https://gcn.gsfc.nasa.gov/gcn_describe.html}}, and a number of
automated telescopes via network sockets on the time scale of seconds
enabled the discovery that ``short/hard'' gamma ray bursts are caused
by colliding neutron stars\cite{kilonova}.
