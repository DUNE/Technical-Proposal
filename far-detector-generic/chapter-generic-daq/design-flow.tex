
%%%%%%%%%%%%%%%%%%%%%%%%%%%%%%%%%%%
\subsection{Dataflow, Trigger and Event Builder}
\label{sec:fd-daq-hlt}

\metainfo{Giles Barr \& Josh Klein \& Giovanna Miotto \& Kurt Biery \& Brett Viren.  This is a DP/SP shared section.  It's file is \texttt{far-detector-generic/chapter-generic-daq/design-flow.tex}}

In the general data and trigger flow diagrams for the nominal
(Figure~\ref{fig:daq-overview}) and alternative
(Figure~\ref{fig:daq-overview-alt}) designs the dataflow, trigger and
event builder functions take as input data from the \dword{detmodule}
electronics and culminate in files deposited to \dword{diskbuffer} for
transfer to permanent storage by offline computing processes.  
The continuous, uncompressed data rate of the input from one
\dword{detmodule} is on order of \SI{1}{\TB/\s}. 
The final output data rate, for all \dwords{detmodule} operating at
any given time is approximately limited to \offsitegbyteps. 

To accept this high data inflow rate and to apply the substantial
processing needed to achieve the required reduction factor which is on
order of \num{1000} the \dword{daq} follows a distributed design.
The units of distribution for the front end of the \dword{daq} must match up
with natural units of the \dword{detmodule} providing the data. 
This unit is called the \dword{daqfrag} and each accepts input at a
rate of about 10-20 \si{\GB/\s}. 
The exact choice maps to some integral number of physical
\dword{detmodule} units (eg \dword{sp} \dwords{apa} or \dword{dp}
\dwords{cro}/\dwords{lro}).

As described in the previous sections, the nominal and alternative
designs differ essentially in the order and manner in which the
\dword{snb} buffering occurs and the \dwords{trigprimitive} are
formed. 
The overall data flow, higher level triggering and building of
``event'' data blocks for final writing are conceptually very similar.
This processing begins with the data being received by the
\dword{felix} PCIe board hosted in the \dword{fec}. 
The \dword{felix} board performs a DMA transfer of the data into the
\dword{daqbuf} for the \dword{daqfrag} which resides in the
\dword{fec} host system RAM.
This buffer is sized to hold ten seconds of data assuming the maximum,
uncompressed input rate associated with the fragment.
While data is being written to the buffer, data a delayed portion is
also being read in order to dispatch it for various purposes.
Any and all requests to further dispatch a subset of this data from
the \dword{daqbuf} must arrive within this buffer time.
In the nominal design, the only dispatching will be from a request
made by an \dword{eb} (described more below) on its receipt of a
\dword{trigcommand}. 
In the alternative design, a suitable fraction of the data are also
dispatched via high bandwidth (at least 25-50 \si{\Gbps} simplex, less
if data is compressed at this stage) network connections to a trigger
farm so that \dwords{trigprimitive} may be formed. 
Whether the primitives are formed in this manner or extracted from the
stream sent by the \dword{daqfer} (as in the nominal design) these
trigger primitives from one \dword{daqfrag} are collectively sent for
further processing in order to be combined across channels and so to
produce \dwords{trigcandidate}. 
These are finally combined for one \dword{detmodule} in the
\dword{mtl}. 
It is in the \dword{mtl} where \dwords{trigcandidate} from additional
sources are also considered as described more in
section~\ref{sec:fd-daq-sel}.

In both the nominal and alternative design the dispatch of data
initiated by normal (non-\dword{snb}) \dwords{trigcommand} is
identical. 
This dispatch, commonly termed ``event building'' involves collection
of data spanning an identical and continuous period of time from
multiple \dwords{daqbuf} across the \dword{daq}.
As introduced above, each \dword{trigcommand} is consumed by an
\dword{eb} process. 
It will use fragment address information in the \dword{trigcommand} to
query the \dword{daqds} process representing each referenced
\dword{daqfrag} and accept the returned a \dword{datafrag}.
In the exceptional case that the delay of this request was so large
that the \dword{daqbuf} no longer contains the data then an error
return will be supplied and recorded by the \dword{eb} in place of the
lost data. 
Such failures lead to indicators displayed by the detector operation
monitoring system.
The \dword{eb} will finally assemble all responses into a
\dword{rawevent} and write it to file on the \dword{diskbuffer} where
it becomes responsibility of DUNE offline computing.

The \dword{daqds} and \dword{eb} services will be implemented using
art\dword{daq} which is a modern, general-purpose software framework for
distributed data combination and processing. 
It has been used in the ProtoDUNE and other experiments.
It is designed to exploit the parallelism that is possible with
modern multi-core and networked computers. 
It is the principal architecture that will be used for the DUNE \dword{daq}
back-end computing.
The authors of art\dword{daq} have accomodated DUNE-specific feature-addition
requests and a number of libraries have been developed based on
existing parts of art\dword{daq} used to handle incoming data from data
sources. 
It is likely that future DUNE extensions will be by one of these two
routes.

Unlike the dispatch of data initiated by normal \dwords{trigcommand},
a command formed to indicate the possibility of a \dword{snb} is
handled differently between the nominal and alternative designs. 
Such a command will be interpreted to save all data from all channels
for an rather extended time of \snbtime starting from \snbpretime
before the time associated with the \dword{trigcommand}. 
As no data selection is being preformed, the bandwidth needed requires
special buffering to nonvolatile storage in the form of \dword{ssd}. 
Today's technology supplies individual \dword{ssd} in M.2 form factor
which support individual write speeds up to \SI{2.5}{\GB/\s}. 
The two designs differ as to the location of and data source for these
buffers

In the nominal design, these \dwords{ssd} reside in the \dword{daqfer}
as described in Section~\ref{sec:daq-fero}. 
In that location, due to larger granularity of computing units, the
data rate into any one \dword{ssd} is within the quoted write
bandwidth. 
However, and as shown in Figure~\ref{fig:daq-overview}, the data and
trigger flow for \dword{snb} in the nominal case, takes a special
path. 
Instead of an \dword{eb} consuming the \dword{trigcommand} as
described above, it is sent to the \dfirst{daqoob} which dispatches
it to each \dword{daqfer} unit hosting an \dword{ssd}. 
This component is used in order to immediately free up the \dword{mtl}
to continue to process normal triggers.
When the command is received, each host must begin to stream data from
its local RAM supplying at least \snbpretime of buffer to the
\dword{ssd} and continue until the full \snbtime has elapsed. 
While it is performing this dump it must continue to form
\dwords{trigprimitive} and pass them and the full data stream to the
connected \dword{fec}.

In the alternative design the same \dword{daqbuf} provides the
\snbpretime of pre-trigger \dword{snb} buffering. 
Like in the nominal case, it must rely on fast, local \dword{ssd}
storage to sink the dump. 
Current \dword{ssd} technology allows four M.2 \dword{ssd} devices to
be hosted on a PCIe board. 
Initial benchmarks of this technology show such a combination can
achieve \SI{7.5}{\GB/\s} write bandwidth, which is short of linear
scaling. 
To support the maximum of \SI{20}{\GB/\s}, three such boards would be
required.
The alternative design presents a synergy between the need to dump
high rate data and the need to provide CPU to form
\dwords{trigprimitive}. 
With current commodity computing hardware it is expected that each
\dword{fec} must be augmented with about two computers in the trigger
farm. 
These trigger processors will need to accept the entire DP and
$\frac{3}{8}$'s of the SP data stream from their \dword{daqfrag}. 
If they instead accepted the entire stream then they can also provide
RAM buffering and split up the data rate which must be sunk to
\dword{ssd} buffers.


In both designs, the data dumped to \dword{ssd} may contain precious
information about a potential \dword{snb}. 
It must be extracted from the buffer, processed and either discarded
or saved to permanent storage. 
The requirements on these processes are not easy to determine.
The average period between actual \dword{snb} to which DUNE is
sensitive is measured in decades. 
However, to maintain high efficiency to capture such important
physics, the thresholds will be placed as low as feasible, limited
only by the ability to acquire, validate and if passing, write out the
data to permanent storage. 
Even as such, the (largely false positive) \dword{snb} trigger rate is
expected to be minuscule relative to normal triggers.
Understanding the exact rate requires more study, including using
early data, but for planning purposes it is simply assumed that one
whole-detector data dump will occur per month on average.
Using the \dword{sp} \dword{detmodule} as an example, and choosing the
nominal time span for the dump to be \snbtime about \spsnbsize of
uncompressed data would result.
In the nominal \dword{sp} \dword{daq} design, this dump would be spread over
600 \dword{ssd} units leading to \SI{75}{\GB} per \dword{ssd} per dump.
% this number assumes defs.tex hasn't had its numbers changed.
% check once in a while
Thus, typical \dwords{ssd} offer storage to allow any given dump to be
held for at least one half year before it must be purged to assure
storage is available for subsequent dumps.
If every dump were to be sent to permanent storage, it would represent
a sustained \SI{0.14}{\Gbps} (per \dword{detmodule}) which is a small
perturbation on the bandwidth supplied throughout the \dword{daq} network. 
Saved to permanent storage this rate integrates to \SI{0.5}{\PB/\year}
which while substantial is a minor fraction of the total data budget.
The size of each dump is still larger than is convenient to place into
a single file and so the \dword{snb} event building will likely differ from
that for normal triggers in that the entire dump will not be held in a
single \dword{rawevent}. 
Finally, it's important to qualify that these rates assume
uncompressed data. 
At the cost of additional processing elements, lossless compression
can be expected to reduce this data rate by 5--10$\times$ or
alternatively allow lower thresholds leading to that factor more
dumps.
Additional study is required to optimize the costs against expected
increase in sensitivity.


% Lossless compression may be applied either just before or just after
% the data is received resulting in 
% This compression factor reduces the requirements on various bandwidths
% but increases the requirement on processing needed to form triggers.
