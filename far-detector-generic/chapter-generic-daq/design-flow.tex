
%%%%%%%%%%%%%%%%%%%%%%%%%%%%%%%%%%%
\subsection{Dataflow, Trigger and Event Builder}
\label{sec:fd-daq-hlt}

\metainfo{Giles Barr \& Josh Klein \& Giovanna Miotto \& Kurt Biery \& Brett Viren.  This is a DP/SP shared section.  It's file is \texttt{far-detector-generic/chapter-generic-daq/design-flow.tex}}

The data volume that is continuously flowing through the DUNE
\dword{sp} \dword{detmodule} DAQ from one APA is about \SI{10}{\GB/\s}
and from the entire far detector as much as \SI{10}{\TB/\s}.
However, by design a distributed system that follows the natural
granularity of the detector this high rate data is made manageable.
In particular the \dword{ringbuffer} is defined at the APA level for
the \dword{sp} \dword{detmodule}.
The majority of the data corresponds to the untriggered parts and
never leaves the computers hosting the \dwords{ringbuffer}.
The parts that exit do so in one of three circumstances: (a) as data
selected by nominal \dwords{trigcommand}, (b) as \dwords{trignote} or
(c) as previously triggered \dword{snb} candidate data which is
sent out after being dumped to the SSD storage local to
\dwords{rce} and as available bandwidth permits. 
All three of these transfers live entirely in the software domain of a
commodity computer farm and so a variety of techniques can be
considered for each; in this description, we consider each of them as
a data-pull protocol, similar to that in use at ProtoDUNE.
\fixme{Is it true? 
  There has been the statement that SNB dumps will egress via special
  paths direct to a special EB and not via FELIX. 
  If this isn't the case then figure 7.2 needs fixing.}


artDAQ is a modern, general-purpose framework that has been used in
DUNE prototype tests and elsewhere [refs]. 
It is optimised further than other frameworks to expliot the
paralelism that is possible with modern multi-core machines. 
It is the principle architecture that will be used in DUNE.
\fixme{Context here is too vague. 
  ``...used in DUNE DAQ back-end computing''?}
The authors of artDAQ have accomodated DUNE-specific feature-addition
requests and a number of libraries have been developed based on
existing parts of artDAQ used to handle incoming data from data
sources. 
It is likely that future DUNE extensions will be by one of these two
routes.

(Section on FELIX data ingress, error handling, synchronisation,
ring-buffer expiry). 
As noted above, one of the complex data flow control tasks is managing
the data as it enters the \dwords{ringbuffer} on the per-APA \dword{fe}
computing hosts. 
Data throughput on these nodes is governed by the use of the PCIe and
memory busses, and so the software will be written to mimumize
data-copies on the input side of the \dword{ringbuffer}.
%% removed: this is RCE-land, not primary buffer.
%During normal operation, there must be sufficient 'headroom' to allow
%an additional write of data to SSDs during the supernova burst. 
The current model for dataflow control is that each APA will operate
semi-autonomously, data being written directly from hardware into physical
memory and not moved. 
The software will keep track of the free and in-use memory (one way is
with a ring for each data source) and will maintain an index of the
data. 
One strategy that decentralises the error handling is for the writing
to the \dword{ringbuffer} to not cause error handling events, but
for it to flag areas of the data for which problems exist only if/when
it is to be read again. 
Similarly, to avoid memory alloction error handling, the incoming data
can overwrite the older data and read requests for data that has been
overwritten will report the error. 
It is planned to prototype the data transfers for the TDR.

(Section on (a) event building) Requests for event building
(i.e. reading data for succesful triggers from the \dwords{ringbuffer}
will be done using mostly standard features that are now in artDAQ. 
An \dword{eb} node is allocated for each trigger to be read and it
then requests portions of data from the \dwords{ringbuffer} for the
data from that event (this read must be initiated by the request, a
difference with the current ProtoDUNE).
If the data is missing (either due to errors when it was first
received, or because the request arrived so late that the data were
overwritten), the request still proceeds but with flags set in the event
header and which are displayed on the operators console.

(Section on (b) trigger farming) As described above, a \dword{fec}
will receive \dwords{trigprimitive} from its associated \dwords{rce}
which identify channels and periods of time where potentially useful
signal exists. 
In most circumstances, this will local information will be sufficient
to form a \dword{trigcandidate} which on its own will lead to a
\dword{trigcommand}.
In other cases, such as SNB activity, \dwords{trigcandidate} from many
\dwords{fec} must be combined to form a \dword{trigcommand}.
This trigger information will be gathered into a block of data
spanning some time period. 
Our default scheme for the dataflow is to use a similar method as
employed in \dword{protodune}, however several others are under study. 
For the former, a trigger farm node is assigned to each time block,
requests the associated trigger data block and based on its contents
retrieves the indicated detector data from the appropriate
\dwords{fec}.   
The received detector data is then aggregated along with the
associated trigger information into a \dword{rawevent} (using artDAQ
functions).
Scaling methods are being considered to mitigate bottlenecks as may be
required.  
These are based adding an intermediate layer which groups a number of
\dwords{fec} to avoid the possibility of too many small blocks of
data.
\fixme{This next sentence doesn't make sense: ``Another solution is
  to use data-push, since, it is likely that a single node with a lot
  of cores can deal with all the trigger processing.''}

\fixme{We need something about running in a partitioned mode. 
  Eg, 148 APAs working nominally and 2 in their own partition for the
  purpose of calibration or debugging or something.}

(Section on (c) SNB trickling) The average rate of dump
\dwords{trigcommand} such as may be issued by a SNB candidate will be
relatively low. 
The nominal period of the dump is \SI{10}{\s} which will thus produce
about \SI{15}{\TB} across the entire \dword{sp} \dword{detmodule}. 
Spread over the 600 \dword{ssd} units this leads to about \SI{25}{\GB}
per \dword{rce} per dump. 
Today, typical \dwords{ssd} offer storage on the order of \SI{1}{\TB}. 
For any reasonable SNB false-positive trigger rate, there will be
sufficient \dword{ssd} capacity to store dumps for years and to not
directly drive any throughput requirements on the DAQ.
It is thus expected that this dump data will be sent out as a
background process. 
The SNB event building will likely differ somewhat from nominal in
that the entire dump will not be held in a single \dword{rawevent}
object as this would produce a single \SI{15}{\TB} file.

