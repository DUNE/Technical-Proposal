
%%%%%%%%%%%%%%%%%%%%%%%%%%%%%%%%%%%
\subsection{Computing and Network Infrastructure}
\label{sec:fd-daq-infra}

\metainfo{Kurt Biery \& Babak Abi.  This is a DP/SP shared section.  It's file is \texttt{far-detector-generic/chapter-generic-daq/design-compnet.tex}}

The computing and network infrastructure that will be used in each
of the four \dwords{detmodule} is similar, if not identical.
It supports the buffering, data selection, event
building, and data flow functionality described
above, and it includes computing elements that consist of servers that:

\begin{itemize}
\item buffer the data until a \dword{trigdecision}
  is received;
\item host the software processes that
  build the data fragments from the relevant
  parts of the detector into complete events;\fixme{whole detector or module?}
\item host the processes that make the
  \dword{trigdecision};
\item host the data logging processes and
  the disk buffer where the data is written;
\item host the real-time \dlong{dqm} processing;
\item host the control and monitoring processes.
\end{itemize}

The network infrastructure that connects these computers has the following components:

\begin{itemize}
\item subnet(s) for transferring triggered data from the buffer
  nodes to the event builder nodes.  These need to
  connect underground and above-ground computers.
\item a control and monitoring subnet that connects all
  computers in the \dword{daq} system and all \dword{fe}
  electronics that support Ethernet communication.  This
  sub-network must connect to underground and
  above-ground computers.
\item a subnet for transferring complete events from the
  event builder servers to the storage servers.  This subnet
  is completely above-ground.
\end{itemize}
