
%%%%%%%%%%%%%%%%%%%%%%%%%%%%%%%%%%%
\subsection{Computing \& Network Infrastructure}
\label{sec:fd-daq-infra}

\metainfo{Kurt Biery \& Babak Abi.  This is a DP/SP shared section.  It's file is \texttt{far-detector-generic/chapter-generic-daq/design-compnet.tex}}

The computing and network infrastructure that will be used in each
of the four \dwords{detmodule} will be similar, if not identical.
It will support the buffering, data selection, event
building, and data flow functionality described
above, and it will include computing elements which consist of servers that:

\begin{itemize}
\item buffer the data until a \dword{trigdecision}
  is received
\item host the software processes that
  build the data fragments from the relevant
  parts of the detector into complete events
\item host the processes that make the
  \dword{trigdecision}
\item host the data logging processes and
  the disk buffer where the data is written
\item host the real-time \dlong{dqm} processing
\item host the control and monitoring processes
\end{itemize}

The network infrastructure that will connect these computers
will have the following components:

\begin{itemize}
\item subnet(s) for transferring triggered data from the buffer
  nodes to the event builder nodes.  These will need to
  connect underground and above-ground computers.
\item a control and monitoring subnet that will connect all
  computers in the \dword{daq} system and all front-end
  electronics that support Ethernet communication.  This
  sub-network will need to connect underground and
  above-ground computers.
\item a subnet for transferring complete events from the
  event builder servers to the storage servers.  This subnet
  will be completely above-ground.
\end{itemize}
