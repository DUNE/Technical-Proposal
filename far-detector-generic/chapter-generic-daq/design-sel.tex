

%%%%%%%%%%%%%%%%%%%%%%%%%%%%%%%%%%%
\subsection{Data Selection Algorithms}
\label{sec:fd-daq-sel}

\metainfo{Josh Klein \& Brett Viren.  This is a shared DP/SP section.  It's file is \texttt{far-detector-generic/chapter-generic-daq/design-sel.tex}}

Data Selection will follow a hierarchical design. 
It begins with forming \dword{detunit}-level \dwords{trigcandidate}
inside the \dword{daqfrag} front-end computing using channel-level
\dwords{trigprimitive}. 
These will then be used to form \dword{detmodule} \dwords{trigcommand}
in the \dword{mtl}.
When executed, they lead to readout of a small subset of the total
data. 
In addition, \dwords{trigcandidate} will be provided to the
\dword{mtl} from external sources such as the \dword{gtl} in order to
indicate external events such as beam spills, or SNB candidates
detected by the other \dwords{detmodule}. 
In addition to supplying triggers to \dword{snews}, triggers from
\dword{snews} or other cosmological detector sources such as LIGO and
VIRGO can be accepted in order to possibly record low energy or
dispersed activity that would not pass the self-triggering. 
The latency of arrival for these sources must be less than the nominal
\snbpretime buffers used to capture low-level early \dword{snb}
activity.
A \dword{hlt} may also be active within the \dword{mtl}. 
The hierarchical approach is both natural from a design standpoint, as
well as allowing for vertical slice testing and running multiple
\dwords{daqpart} simultaneously during commissioning of the system or
when debugging of individual \dwords{detunit} is required.

As discussed in Sections~\ref{sec:fd-daq-fero}
and~\ref{sec:fd-daq-fetp}, \dwords{trigprimitive} will be generated in
either in \dwords{daqfer} (in the nominal design) or in trigger
processing computers (in the alternative design). 
In both designs, and for both \dword{sp} and \dword{dp}
\dwords{detmodule}, only data from TPC collection channels will feed
the self-triggering as their waveforms directly supply a measure of
ionization activity without computationally costly signal processing.
The \dwords{trigprimitive} will contain summary information for each
channel, such as the time of any threshold-crossing pulse, its
integral charge, and time over threshold. 
A channel with an associated \dword{trigprimitive} is said to be
``hit'' for the time spanned by the primitive. 
Trigger primitives from one \dword{detunit} are then further processed
to produce a \dword{trigcandidate}. 
The candidate represents a cluster of ``hits'' across time and
channel, localized to the \dword{detunit}.
The candidates from all \dwords{daqfrag} are passed to the
\dword{mtl}.
The \dword{mtl} arbitrates between various trigger types, determines
trigger priority and ultimately the time range and detector coverage
for a \dword{trigcommand} which it emits back to the \dwords{fec}.
The \dword{mtl} assures that no \dwords{trigcommand} are issued which
overlap in time nor detector channel space.
It also may employ a \dword{hlt} to reduce or aggregate triggers into
fewer \dwords{trigcommand} so as to optimize the subsequent readout. 
For example, aggregating many small readouts into fewer but larger
ones may allow for more efficient processing.   This can be particularly
important during periods of high rate activity such as due to various
backgrounds or instrumental effects.

When activity leads to the formation of a \dword{trigcommand} this
command is sent down to the \dwords{fec} instructing which slice of
time of its buffered data should be saved. 
The \dword{trigcommand} information is saved along with this data. 
At the start of DUNE data taking, it is anticipated that for any given
single-interaction trigger (a cosmic ray track, for example) waveforms
from all channels in the \dword{detmodule} will be recorded over a one
\dword{readout window} (nominally, \spreadout for \dword{sp} and
\dpreadout for \dword{dp}, chosen to be two drift times with plus
extra 20\%). 
Such an approach is clearly very generous in terms of the amount of
data saved, but it ensures that associated low-energy physics (such as
captures of neutrons produced by neutrino interactions or cosmic rays)
will be recorded without any need to fine-tune \dword{detunit}-level
triggering, and will not depend on the noise environment across
\dwords{detunit}. 
In addition, the wide \dword{readout window} ensures that the data of
all associated activity is recorded.
As generous as it is, it is estimated that this \dword{readout window}
will not produce an unmanageable volume of data. 
We anticipate, however, that as we gain experience running DUNE the
overall data volume will reduced by writing out data from only a
subset of \dwords{fec} for any given \dword{trigcommand}, and possibly
by reducing the size of the readout window.

Other trigger streams---calibrations, random triggers, and prescales
of various trigger thresholds, will also be generated at the
\dword{detmodule} level, and filtering and compression can be applied
based upon the trigger stream. 
For example, a large fraction of random triggers may have \dword{zs}
applied to their waveforms, reducing the data volume substantially, as
the dominant data source for these will be $^{39}$Ar events.
Additional signal-processing can also be done on particular trigger
streams if needed and if the processing is available, such as fast
analyses of calibration data.

At the \dword{detmodule} level, a decision can also be made on whether
a series of interactions are consistent with a \dword{snb}. 
If the number of \dword{detunit} level, low-energy
\dwords{trigcandidate} exceeds a threshold for the number of such
events in a given time, a trigger command is sent from the \dword{mtl}
back to the \dwords{daqfer}, which store up to \SI{10}{\s} of full
waveform data. 
That data is then streamed to non-volatile storage to allow for
subsequent analysis by the supernova working group, perhaps as an
automated process. 
If not rejected it is then sent out of the DAQ to permanent offline
storage.
In addition, the \dword{mtl} passes \dwords{trigcandidate} up to a
detector-wide \dfirst{gtl}, which among other functions can decide
whether, integrated across all modules, enough \dwords{detunit} have
detected interactions to qualify for a \dword{snb}, even if within a
particular module the threshold is not exceeded. 
\Dwords{trigcandidate} from the \dword{gtl} are passed to the
\dword{mtl} for dispatch to the \dwords{fec} (or \dwords{daqfer} in the
case of SNB dump commands in the nominal design). 
That is, to the \dword{mtl}, an \dword{externtrigger} looks like just
one more ``external'' trigger input.

\Dword{detunit} level \dwords{trigcandidate} will be generated within
the context of one \dword{daqfrag}, specifically in each \dword{fec}. 
The trigger decision will be based on the number of nearby channels
hit in a given fragment within a time window (roughly \SI{100}{\us}),
the total charge collected in these adjacent channels and possibly the
union of time-over-threshold for the \dwords{trigprimitive} in the
collection plane.
Our studies show that even for low-energy events (roughly
\SIrange{10}{20}{\MeV}) the reduction in radiological backgrounds is
extremely high with such criteria.
The highest-rate background, $^{39}$Ar, which has a decay rate of
10~MBq within a \SI{10}{\kton} volume of argon, has an endpoint of 500
keV and requires significant pileup in both space and time to get near
a \SI{10}{\MeV} threshold.
One important background source is $^{42}$Ar, which has a 3.5 MeV
endpoint and a decay rate of 1 kBq. 
$^{222}$Rn decays via a highly-quenched $\alpha$ of 5.5 MeV and is
also an important source of background.
\fixme{Need Rn-222 decay rate.  Punted and removed mention of ``decay rate XXXX''.  Someone please fix.}
The radon decays to $^{218}$Po which a few minutes later leads to a
quenched $\alpha$ of 6 MeV, and ultimately a $^{214}$Bi daughter (many
minutes later) which has a $\beta$ decay with endpoint near 3.5 MeV. 
The $\alpha$ ranges are short and will hit at most a few electrodes,
but the charge deposit can be large, and therefore the charge
threshold will have to be well above the $\alpha$ deposits plus any
pileup from $^{39}$Ar and noise.

At the level of one \dword{detunit}, two kinds of local
\dwords{trigcandidate} can be generated.
One is a ``high-energy'' trigger, that indicates local ionization
activity corresponding to more than than \SI{10}{\MeV} has been found.
The per channel thresholds on total charge and time-over-threshold
will be optimized to achieve at least 50\% efficiency at this energy
threshold, with efficiency increasing to 100\% via a turn-on curve
that ensures at least 90\% efficiency at \SI{20}{\MeV}. 
The second type of trigger candidate that will be generated is for
low-energy events between \SI{5}{\MeV} and \SI{10}{\MeV}. 
In isolation, these candidates will not lead to formation of a
\dword{trigcommand}. 
Rather, at the \dword{detmodule} level they will be combined, time
ordered and their aggregate rate compared against a threshold based on
fluctuations due to noise and backgrounds in order to form a
\dword{snb} \dword{trigcommand}.

The \dword{mtl} takes as input \dwords{trigcandidate} (both low-energy
and high-energy) from the participating \dwords{daqfrag}, as well as
\dword{externtrigger} sources, such as the \dword{gtl} which includes
global, detector-wide triggers, external trigger sources such as
\dword{snews}, and information about the time of a Fermilab beam
spill. 
The \dword{mtl} will also generate \dwords{trigcommand} for internal
consumption, such as random triggers and calibration triggers (for
example, telling a laser system to fire at a prescribed time). 
The \dword{mtl} can also generate \dwords{trigcommand} from a
prescaled fraction of trigger types that otherwise do not generate
such commands on their own. 
For example, a prescaled fraction of single, low-energy trigger
commands could be allowed to generate a trigger command, even though
those candidates normally only result in a trigger command when
aggregated (ie, as they would be for a \dword{snb}).


The \dword{mtl} is also responsible for checking candidate triggers
against the current \dword{rc} trigger mask: in some runs, for
example, we may decide that only random triggers are accepted, or that
certain \dwords{trigcandidate} streams should not be considered
because their \dwords{daqfrag} have been producing unreasonably large
rates in the recent past (such as may be due to noise spikes, flaky
hardware or buggy software).
In addition, the \dword{mtl} will count low-energy trigger candidates
and based upon their number and distribution over a long time interval
(e.g., \SI{10}{\s}), decide to generate a \dword{snb} trigger command.
The trigger logic will be optimized to record the data due to at least
90\% of all Milky Way supernovae, and studies of simple low-energy
trigger criteria show that a much higher efficiency can likely be
achieved.

	
The \dword{hlt} can also be applied at this level, particularly if
there are unexpectedly higher rates from instrumental or low-energy
backgrounds, that require some level of reconstruction or pattern
recognition. 
An \dword{hlt} might also allow for efficiently triggering on
lower-energy single interactions, or allow for better sensitivity for
supernovae beyond the Milky Way, by employing a weighting scheme to
individual \dwords{trigcandidate}---higher-energy
\dwords{trigcandidate} receiving higher weights. 
Thus, for example, two \dwords{trigcandidate} consistent with
\SI{10}{\MeV} interactions in \SI{10}{\s} might be enough to create a
\dword{snb} candidate trigger, while 100 \SI{5}{\MeV} trigger
candiates in \SI{10}{\s} might not.
Lastly, the \dword{hlt} can allow for dynamic thresholding; for
example, if a trigger appears to be due to a cosmic-ray muon, the
threshold for single interactions can be lowered (and possibly
prescaled) for a short time after that to identify spallation
products. 
In addition, the \dword{hlt} could allow for a dynamic threshold after
a \dword{snb}, to extend sensitivity beyond the \SI{10}{\s}
\dword{snb} \dword{readout window}, while not increasing the data
volume associated with \dword{snb} candidates linearly. 

All low-energy \dwords{trigcandidate} are also passed upwards to the
\dword{gtl} so that they may be integrated across all \SI{10}{\kton}
\dwords{detmodule} in order to determine that a \dword{snb} may be
occurring. 
This approach increases the sensitivity to trigger on \dwords{snb} by
a factor of four (for \SI{40}{\kton}), thus extending the burst
sensitivity to a distance twice as far as for a single \SI{10}{\kton}
\dword{detmodule}. 

	
The \dword{mtl} is also responsible to include in the
\dword{trigcommand} a global timestamp built from its input
\dwords{trigcandidate}, and information on what type of trigger was
created. 
A log of all \dwords{trigcandidate} will also be kept, whether or not
they contribute to the formation of a \dword{trigcommand}. 
As described above, the \dword{readout window} for nominal
\dwords{trigcommand} (those other than for \dword{snb} candidates)
somewhat more than two times the maximum drift time. 
Further, a nominal readout spans all channels in a \dword{detmodule}. 
The \dword{mtl} is also responsible for sending the trigger commands
that tell the \dwords{daqfer} to stream all data from the past
\snbpretime and for a total of \snbtime in hopes to catch
\dwords{snb}.
This command may be produced based on \dwords{trigcandidate} from
inside the \dword{mtl} itself or it may be produced based on an external
\dword{snb} \dword{trigcandidate} passed to the \dword{mtl} by the
\dword{gtl}.
