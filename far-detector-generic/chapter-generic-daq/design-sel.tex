

%%%%%%%%%%%%%%%%%%%%%%%%%%%%%%%%%%%
\subsection{Data Selection Algorithms}
\label{sec:fd-daq-sel}

\metainfo{Josh Klein \& Brett Viren.  This is a shared DP/SP section.  It's file is \texttt{far-detector-generic/chapter-generic-daq/design-sel.tex}}

%This section describes the strategy for using the trigger/dataflow infrastructure, with example algorithms.

\fixme{Need to add Josh's table of data volumes.}

Data Selection will follow a hierarchical design. 
It begins with forming local, APA-level \dwords{trigcandidate} which
are used to form \dword{detmodule} \dwords{trigcommand} in the
\dword{mtl} which lead to readout of a small subset of the total data. 
In addition, \dwords{trigcandidate} will be provided to the
\dword{mtl} from external sources such as the \dword{gtl} in order to
indicate external events such as beam spills, or SNB candidates
detected by the other \dwords{detmodule} or by SNEWS.
A \dword{hlt} may also be active within the \dword{mtl}. 
The hierarchical approach is both natural from a design standpoint, as
well as allowing for vertical slice testing and partitioning during
commissioning of the system.

As discussed above, \dwords{trigprimitive} will be generated in the
\dwords{rce} from the data stream for each collection channel. 
The \dwords{trigprimitive} will contain summary information for each
collection wire, such as the time of any threshold-crossing pulse, its
integral charge, and time over threshold. 
A collection wire with an associated \dword{trigprimitive} is said to
be ``hit'' for the time spanned by the primitive. 
These trigger primitives are then passed, along with the full data
stream, to the \dword{fec}, each of which handles channels from a pair
of \dwords{apa}. 
An APA-level \dword{trigcandidate} is passed upward to the
\dword{mtl}, which arbitrates between various trigger types,
determines trigger priority and ultimately the time range and detector
coverage for a \dword{trigcommand} which it emits back to the
\dwords{fec}.
The \dword{mtl} assures that no overlapping \dwords{trigcommand} are
issued.
It also may emply a \dword{hlt} to allow for a reduction or an
aggregation of triggers to reduce or optimize the subsequent readout
of high rate activity such as due to various backgrounds, including
instrumental effects, as needed.

A valid \dword{detmodule}-level, single-interaction trigger sends
\dwords{trigcommand} back to the \dwords{fec} telling it which slice
of time should be saved under the particular trigger header. 
At the start of DUNE data taking, it is anticipated that for any given
single-interaction trigger (a cosmic ray track, for example) waveforms
for all wires will be recorded for a full \readout \dword{readout
  window}. 
Such an approach is clearly very conservative, but it ensures that
associated low-energy physics (such as captures of neutrons from
produced by neutrino interactions or cosmic rays) will be recorded
without any need to fine-tune APA-level triggering, and will not
depend on the noise environment across the APAs. 
In addition, the \readout readout window ensures that regardless of when
a given even causes a trigger, the entire track is recorded. 
It is estimated that this generous \dword{readout window} will not
produce an unmanageable data volume. 
We anticipate, however, that as we gain experience running DUNE the
overall data volume will reduced by writing out data from only a
subset of APAs for any given track, and possibly reducing the size of
the readout window.

Other trigger streams---calibrations, random triggers, and prescales
of various trigger thresholds, will also be generated at the
\dword{detmodule} level, and filtering and compression can be applied
based upon the trigger stream. 
For example, a large fraction of random triggers may have \dword{zs}
applied to their waveforms, reducing the data volume substantially, as
the dominant data source for these will be $^{39}$Ar events.
Additional signal-processing can also be done on particular trigger
streams if needed and if the processing is available, such as fast
analyses of calibration data.

At the \dword{detmodule} level, a decision can also be made on whether
a series of interactions are consistent with a \dword{snb}. 
If the number of \dword{apa} level, low-energy \dwords{trigcandidate}
exceeds a threshold for the number of such events in a given time, a
trigger command is sent from the \dword{mtl} back to the \dwords{rce},
which store up to \SI{10}{\s} of unsuppressed, full waveform data. 
That full waveform is then written out as a separate trigger stream
for fast analysis by the supernova working group, perhaps as an
automated process. 
In addition, the \dword{mtl} passes information about APA level
\dwords{trigcandidate} up to a detector-wide \dword{gtl}, which among
other functions can decide whether, integrated across all modules,
enough APAs have detected interactions to qualify for a \dword{snb}
even if within a particular module the threshold is not exceeded. 
\Dwords{trigcandidate} from the \dword{gtl} are passed to the
\dword{mtl} for dispatch to the \dwords{fec} (or \dwords{rce} in the
case of SNB dump commands). 
That is, to the \dword{mtl}, an \dword{externtrigger} looks like just
one more ``external'' trigger input.

APA level \dwords{trigcandidate} will be generated within each server
of the \dwords{fec}. 
The trigger decision will be based on the number of adjacent
collection wires hit in a given APA within a time window (roughly
\SI{100}{\us}), the total charge on these adjacent wires (or possibly
on any hit wire in the collection plane), and possibly the union of
time-over-threshold for the \dwords{trigprimitive} in the collection
plane.
Our studies show that even for low-energy events (roughly
\SIrange{10}{20}{\MeV}) the reduction in radiological backgrounds is
extremely high with such criteria.
The highest-rate background, $^{39}$Ar, which has a decay rate of
10~MBq within a \SI{10}{\kton} volume of argon, has an endpoint of 500
keV and requires significant pileup in both space and time to get near
a \SI{10}{\MeV} threshold.
Other important background sources are $^{42}$Ar, which has a 3.5 MeV
endpoint and a decay rate of 1 kBq, and $^{222}$Rn which has a decay
rate of XXXX and decays via a highly-quenched $\alpha$ of 5.5 MeV.
\fixme{Need Rn-222 decay rate.}
The radon decays to $^{218}$Po which a few minutes later leads to a
quenched $\alpha$ of 6 MeV, and ultimately a $^{214}$Bi daughter (many
minutes later) which has a $\beta$ decay with endpoint near 3.5 MeV. 
The $\alpha$ ranges are short and will hit at most a few collection
wires, but the charge deposit can be large, and therefore the charge
threshold will have to be well above the $\alpha$ deposits plus any
pileup from $^{39}$Ar and noise.

At the APA level, two kinds of local \dwords{trigcandidate} can be
generated.
One is a ``high-energy'' trigger, that indicates that within a given
APA a candidate event with energy more than \SI{10}{\MeV} has been
found.
The thresholds in hit wires, total charge, and time-over-threshold,
will be optimized for at least 50\% efficiency at this threshold, with
efficiency increasing to 100\% via a turn-on curve that ensures at
least 90\% efficiency at \SI{20}{\MeV}. 
A second APA level trigger candiate will be generated for low-energy
events, between \SI{5}{\MeV} and \SI{10}{\MeV}. 
These low-energy APA trigger candidates will not by themselves
generate valid \dword{detmodule} level triggers, but rather be used at
the \dword{detmodule} level to determine whether a burst of events
across many APAs is consistent with a supernova.

The \dword{detmodule} level takes as input both APA level
\dwords{trigcandidate} (both low-energy and high-energy), as well as
\dword{externtrigger} sources, such as the \dword{gtl} which includes
global, detector-wide triggers, external trigger sources such as
SNEWS, and information about the time of a Fermilab beam spill. 
The \dword{mtl} will also generate internal triggers, such as random
triggers and calibration triggers (for example, telling a laser system
to fire at a prescribed time). 
In addition, at the \dword{mtl} prescales of all trigger types that
normally would not generate a \dword{trigcommand} will be made. 
For example, a single low-energy \dword{trigcandidate} does not cause
a \dword{trigcommand} on its own, but at some large prescale fraction
it could.


The \dword{mtl} is responsible also for checking candidate triggers
against the current \dword{rc} trigger mask: in some runs, for
example, we may decide that only random triggers are accepted, or that
certain APA trigger candidates should not be accepted because those
APAs are noisy or misbehaving in some way. 
In addition, the \dword{mtl} will count low-energy trigger candidates
from the APAs and, based upon the number and distribution of those
triggers in a long time interval (e.g., \SI{10}{\s}), decide to
generate a \dword{snb} trigger command.
 The trigger logic will be
optimized to record the data due to at least 90\% of all Milky Way
supernovae, and our studies of simple low-energy trigger criteria show
that we can likely do much better than that.  

	
The \dword{hlt} can also be applied at this level, particularly if
there are unexpectedly higher rates from instrumental or low-energy
backgrounds, that require some level of reconstruction or pattern
recognition. 
An \dword{hlt} might also allow for efficiently triggering on
lower-energy single interactions, or allow for better sensitivity for
supernovae beyond the Milky Way, by employing a weighting scheme to
individual APA level \dwords{trigcandidate}---higher-energy
\dwords{trigcandidate} receiving higher weights. 
Thus, for example, two APA level \dwords{trigcandidate} consistent
with \SI{10}{\MeV} interactions in \SI{10}{\s} might be enough to
create a \dword{snb} candidate trigger, while 100 \SI{5}{\MeV} APA
level trigger candiates in \SI{10}{\s} might not.
Lastly, the \dword{hlt} can allow for dynamic thresholding; for
example, if a trigger appears to be due to a cosmic-ray muon, the
threshold for single interactions can be lowered (and possibly
prescaled) for a short time after that to identify spallation
products. 
In addition, the \dword{hlt} could allow for a dynamic threshold after
a \dword{snb}, to extend sensitivity beyond the \SI{10}{\s}
\dword{snb} \dword{readout window}, while not increasing the data
volume associated with \dword{snb} candidates linearly. 

All low-energy \dwords{trigcandidate} are also passed upwards to the
\dword{gtl}, which can integrate the APA level trigger across all
\SI{10}{\kton} \dwords{detmodule} and determine if enough APAs have
\dwords{trigcandidate} that a \dword{snb} may be occuring. 
This approach increases the sensitivity to trigger on bursts by a
factor of four (for \SI{40}{\kton}), thus extending the burst
sensitivity to a distance twice as far as for a single \SI{10}{\kton}
\dword{detmodule}. 

	
The \dword{mtl} is also responsible for creating a trigger header that
includes a global timestamp for the trigger from its input
\dwords{trigcandidate}, and information on what type of trigger was
created. 
A log of all APA level \dwords{trigcandidate} will also be kept,
whether or not they contribute to the formation of a
\dword{trigcommand}. 
As described above, the \dword{readout window} for nominal
\dwords{trigcommand} (those other than for \dword{snb} candidates) is
\readout.
Thus, a nominal DUNE ``event'' is \readout of data from all wires
in the \dword{sp} \dword{detmodule} in response to a \dword{mtl}
\dword{trigcommand}. 
The \dword{mtl} is also responsible for sending the trigger commands
that tell the RCEs to dump their \SI{10}{\s} ``supernova'' buffer. 
If a supernova burst trigger is created at the \dword{gtl} level, an
\dword{externtrigger} is passed back down to the \dword{mtl}, which
then dispatches a dump \dword{trigcommand} to the RCEs for readout of
the long (\SI{10}{\s}) unsuppressed buffer, the same way it would if a
supernova burst is detected at the \dword{mtl}.

\fixme{There are still a lot of repetition of concepts and statements in the above section.}

	


%\metainfo{Describe Module Trigger Logic and Global Trigger Logic.
%  Describe creation of Trigger Command Messages (TCM) from TPMs.
%  Included external sources of TPMs such as SNEWS, Fermilab Beam,
%  testing triggers, periodic triggers, calibration triggers.}
%
%\metainfo{Here should go the assumptions that went into the thinking
%  for Josh's event category data volumes.  It can include discussion
%  of tighter selection criteria for ``cosmics'' such as Phil's
%  distribution of number of APAs hit by cosmics and arguments for
%  something less than 2+ drift time readouts.}

%
% %%%%%%%%%%%%%%%%%%%%%%%%%%%%%%%%%%%
% \subsection{High-Level Trigger}
% \label{sec:fd-daq-eb}
%
% Data selection after event building.
%
