
%%%%%%%%%%%%%%%%%%%%%%%%%%%%%%%%%%%
\subsection{Run Control and Monitoring}
\label{sec:fd-daq-tcm}

\metainfo{Giovanna Miotto \& Jingbo Wang.  THis is a DP/SP shared section.  The file is \texttt{far-detector-generic/chapter-generic-daq/design-ctrmon.tex}}

% Giovanna used "T\dword{daq}", not "\dword{daq}".
% I need to check if there is a distinction (bv).

The online software constitutes the backbone of the DUNE \dword{daq}
system and provides control, configuration and monitoring of the data taking in
a uniform way.
It can be subdivided logically into four subsystems: the run control,
the management of the \dword{daq} and \dword{detmodule} electronics configuration, the monitoring, and the non-physics data archival.
Each of these subsystems has a distinct %well defined 
function, but their
implementation will share underlying technologies and tools.

In contrast to experiments in which data taking sessions, i.e., runs, are
naturally subdivided into time slots by external conditions (e.g., a
collider fill, a beam extraction period), the DUNE experiment aims to
take data continuously.
Therefore, a classic run control with a coherent state machine and a
predefined and concurrently configured number of active detector and
\dword{daq} elements does not seem adequate. 

The DUNE online software is thus structured according to the
architecture principle of loose coupling: each component has as
little knowledge as possible of other components.
%While in the back-end \dword{daq} components may have the granularity of
%individual software processes, for the front-end \dword{daq} a minimum
%granularity will have to be defined, balancing fault tolerance and
%recovery capability with the requirement of data consistency.
While the granularity of the back-end \dword{daq} components may match the 
individual software processes, for the front-end \dword{daq} a minimum
granularity must be defined, balancing fault tolerance and
recovery capability against the requirement of data consistency.
The smallest independent component is called a \dword{daqfrag}, which
is made up of the \dwords{detunit} associated with a single
\dlong{fec}.
In the nominal design, this corresponds to two \dword{sp} \dwords{apa}
and about ten \dword{dp} \dword{cro} crates.

The concept of a \textit{run} represents a period of time in which the same
\dword{fe} elements are active or the same data selection criteria
are in effect (possibly with maximum lengths for offline processing
reasons). 
More than just orchestrating data taking, the run control
provides the mechanisms allowing \dword{daq} applications to publish
their availability, subscribe to information, and exchange messages. 
In addition, the online software provides a configuration service
for \dword{daq} elements to store their settings and a conditions
archive, keeping track of varying detector electronics settings and
status.

Another important aspect of the online software is the monitoring
service.
Monitoring can be subdivided into two main domains: the monitoring of
the data taking operations (rates, number of \dwords{datafrag}
in flight, error flags, application logs, network bandwidth, %monitoring of 
computing and network infrastructure) and the monitoring of the
physics data.
Both are essential to the success of the experiment and must be %should be
designed and integrated into the \dword{daq} system from the start.
In particular, for such a large and distributed system, the information sharing and archival system is very important, as are 
scalable and easily accessible data visualization tools, which will evolve during the lifetime of the experiment.
The online software provides the glue that holds the
\dword{daq} applications together and enables %allows to operate the 
data taking.
Its architecture guides the
approach to \dword{daq} application design and also shapes the view
that the operators will have of the experiment.

