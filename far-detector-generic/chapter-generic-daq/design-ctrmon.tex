
%%%%%%%%%%%%%%%%%%%%%%%%%%%%%%%%%%%
\subsection{Run Control \& Monitoring}
\label{sec:fd-daq-tcm}

\metainfo{Giovanna Miotto \& Jingbo Wang.  THis is a DP/SP shared section.  The file is \texttt{far-detector-generic/chapter-generic-daq/design-ctrmon.tex}}

% Giovanna used "TDAQ", not "DAQ".
% I need to check if there is a distinction (bv).

The online software constitutes the backbone of the DUNE \dword{daq}
system and allows to control, configure and monitor the data taking in
a uniform way.
It can be subdivided logically into four subsystems: the run control,
the configuration, the monitoring, the non-physics data archival.
Each of those subsystems has a well defined function, but their
implementation will share underlying technologies and tools.

Contrary to experiments in which data taking sessions, i.e. runs, are
naturally subdivided into time slots by external conditions (e.g. a
collider fill, a beam extraction period), the DUNE experiment aims at
taking data continuously.
Therefore, a classic run control with a coherent state machine and a
predefined and concurrently configured number of active detector and
\dword{daq} elements does not seem adequate. 

The DUNE online software will thus be structured following the
architecture principle of loose coupling: each component will have as
little knowledge as possible of other components.
While in the back-end DAQ components may have the granularity of
individual software processes, for the front-end DAQ a minimum
granularity will have to be defined, balancing fault tolerance and
recovery capability with the requirement of data consistency.
The smallest independent component is called a \dword{daqfrag} which
is made up of the \dwords{detunit} associated with a single
\dlong{fec}.
In the nominal design, this corresponds to two \dword{sp} \dwords{apa}
or about ten \dword{dp} \dword{cro} crates.

The concept of ``run'' will represent time slices in which the same
front-end elements are active and/or the same data selection criteria
are used (possibly with a maximum lengths for offline processing
reasons). 

In summary, the run control, more than orchestrating data taking will
provide the mechanisms allowing \dword{daq} applications to publish
their availability, subscribe to information and exchange messages. 

In addition, the online software will provide a configuration service
for \dword{daq} elements to store their settings and a conditions
archive, keeping track of varying detector electronics settings and
status.

Another important aspect of the online software is the monitoring
service.
Monitoring can be subdivided into two main domains: the monitoring of
the data taking operations (rates, number of \dwords{datafrag}
in flight, error flags, application logs, network bandwidth, monitoring
of computing and network infrastructure) and the monitoring of the
physics data.
Both are essential to the success of an experiment and should be
designed and integrated into the \dword{daq} system from the start.
In particular, for such a large and distributed system it will be
important to have a well thought through information sharing and
archival system, and also to put human resources into the development
of scalable and easily accessible data visualization tools, which will
evolve during the lifetime of the experiment.

In summary, the online software provides the glue that holds the
\dword{daq} applications together and allows to operate the data
taking.
While it is not involved in physics data flow nor selection, it has a
large impact on the \dword{daq}: its architecture will guide the
approach to \dword{daq} application design and also shape the view
that the operators will have of the experiment.

