%%%%%%%%%%%%%%%%%%%%%%%%%%%%%%%%
\subsection{Surface Logistics \& Testing}
\label{sec:fdsp-coord-integ-test}

The logistics for integrating and installing the \dword{dune} Far
Detectors and their associated infrastructure face a number of
challenges. Possible difficulties include the size and complexity of
the detector itself, the number of sites around the world that will be
fabricating detector and infrastructure components, the necessity for
protecting components from dust, vibration and shock during their
journey to the deep underground laboratory and the lack of space on
the surface near the Sanford Lab Ross Shaft. One mitigation of these
risks is the establishment of a \dword{dune} Integration and Test
Facility (ITF) somewhere in the vicinity of Sanford Lab. Such a
facility and its associated staff would contribute to \dword{dune} in
the following areas.
\begin{itemize}
  \item {\bf Transport Buffer:} Storage capacity for one month
    material in the vicinity of Sanford Lab. Handle packaging
    materials returned from underground laboratory.
  \item {\bf Re-packaging} Facilitate possible re-packaging of
    components before transport underground.
  \item {\bf Component Fabrication:} Possibly provide a capability for
    fabrication of components near Sanford Lab. Undergraduate science
    and engineering students from the South Dakota School of Mines and
    Technology (SDSM\&T) may contribute low cost effort to these
    fabrication activities.
  \item {\bf Component Integration:} Some integration activities may
    be best accomplished in proximity to Sanford Lab. A possible
    example is connecting photodetectors and cold electronics to APAs.
  \item {\bf Inspection, Testing and Repair:} Consortia will define
    their testing requirements including procedures and
    criteria. Consortia will also specify procedures in cases of test
    failure, for example, repair, return to source or discard.
  \item {\bf Visitor Support:} Consortia will likely send staff to the
    ITF for the integration, testing and installation of the consortia
    detector components. The ITF will provide temporary space,
    computer access, assistance personnel and other infrastructure
    support for \dword{dune} visitors.
\item {\bf Outreach:} The ITF may be well located to support a public
  outreach program. The \dword{dune} Experiment is likely to generate
  considerable public interest and addressing those interests is
  important to long-term public support for \dword{dune} specifically
  and particle physics generally.
\end{itemize}

\subsubsection{Scope}
The scope of the ITF includes several possibly related but mostly
independent tasks. They are:
\begin{itemize}
\item{\bf Cryostat:} The scope of this item is the four cryostats
  planned for installation at the 4850 level of Sanford Lab. Cryostat
  components include the warm steel structure, the stainless steel
  membrane and the insulation. The logistics for the cryostat
  components will be managed by the cryostat installation contractor
  and \dword{lbnf} logistics coordinator.  Most likely this function
  will be met with a commercial warehousing vendor, who will supply
  suitable space, loading and unloading facilities and a commercial
  inventory management and control system. The vendor will provide all
  required personnel effort as part of its contracted
  responsibilities.
\item{\bf Cryogenics Systems:} The cryogenics systems are also an \dword{lbnf}
  responsibility and cryogenics system logistics will likely be
  managed by \dword{lbnf} similarly to the logistics for the cryostat
  components.
%\item{\bf Cryostat Support Structure:} This structure is an \dword{lbnf} responsibility and will likely be addressed similarly to the Cryostats and the Cryogenic Systems.
\item{\bf \dword{dune} Detectors:} The \dword{dune} Detectors are the
  responsibility of the collaboration as implemented by the
  consortia. The role of the ITF will vary for the several consortia
  and a description of these various roles is a major topic of this
  document.
\end{itemize}

\subsubsection{Location }
A reasonable criterion for the location of the ITF is within about an
hour drive from Sanford Lab. That criterion yields the following
possibilities.
\begin{center}
\begin{tabular}{ |c|c| } \hline
{\bf Location} & {\bf 2016 Population}  \\ \hline 
Deadwood & 1,264  \\ 
Lead & 3,010  \\
Rapid City & 74,048  \\
Spearfish & 11,531  \\
Sturgis & 6,832  \\ \hline
\end{tabular}
\end{center}
Since infrastructure is correlated with population, Rapid City would
seem the most likely choice for location with Spearfish as a second
possibility. In addition to overall infrastructure, particular assets
of Rapid City include proximity to SDSM\&T and a business community
that is possibly interested in incorporating a \dword{dune} ITF into
an overall regional development program.
%Black Hills State University is located in Spearfish, but that institution is less technology oriented than SDSM\&T.

%$$$$$$$$$$
\subsubsection{\bf Requests from each consortia} 
In February 2018, questionnaires were distributed to each consortia to seek
their requirements for ITF. Table~\ref{table:leders} 
\begin{table}[htbp]
\caption{Leaders and respondents of each consortia.}
\label{table:leders}
\begin{center}
\begin{tabular}{|l|c|c|} \hline
{\bf Consortium} & {\bf Leaders} &{\bf Respondents} \\\hline
High Voltage & Francesco Pietropaolo, Bo Yu & Bo Yu \\ \hline
APA & Stefan Soldner-Rembold, Alberto Marchionni & Peter Sutcliffe \\ \hline
DAQ & Georgia Karagiorgi, Dave Newbold & Alec Habig \\ \hline
SPCE & David Christian, Marco Verzocchi  & Marco Verzocchi, Matt Worcester\\ \hline
DPCE & D. Autiero, T. Hasegawa &  D.Autiero  \\ \hline
SPPD & & \\ \hline
DPPD & Ines Gil Botella, Dominique Duchesneau & Burak Bilki \\ \hline
CISC & &   \\ \hline
CRP & &  \\   \hline
\end{tabular}
\end{center}
\end{table}
lists leaders of each consortia and names of respondents to the
questionnaires, while Tab.~\ref{table:responses} summarizes their
needs for the ITF.
\begin{table}[htbp]
\caption{Summary of each consortia's needs at ITF..}
\label{table:responses}
\begin{center}
\scalebox{0.95}
{
\begin{tabular}{|l|c|c|c|c|c|c| } 
\hline
{\bf Consortium} & {\bf Transport} &{\bf Re-Packaging}&{\bf Component}
&{\bf Component}&{\bf Inspection,}&{\bf Visitor} \\
 & {\bf Buffer} &{\bf }&{\bf Fabrication}
&{\bf Integration}&{\bf Testing}&{\bf Support} \\ \hline 
High Voltage & Yes & Yes & No & Yes? & Yes & Yes \\ \hline
APA & Yes & Yes & Yes & Yes & Yes & Yes \\ \hline
DAQ & Yes & Yes & Yes & Yes & Yes & Yes \\ \hline
SPCE & Yes & Yes & No & No & Yes & Yes \\ \hline
DPCE & Yes & No & No & No & Yes & Yes \\ \hline
SPPD & & & & & &  \\ \hline
DPPD & Yes & Yes & Yes & No & Yes & Yes \\ \hline
CISC & & & & & &  \\ \hline
CRP & & & & & &  \\   \hline
\end{tabular}
}
\end{center}
\end{table}
\noindent Responses from each consortia follow below.

\paragraph{\bf Transport Buffer}
\begin{itemize}
 \item {\bf High Voltage System} \SI{1000}{m$^2$} maximum needed
   (\num{1} month before the start of TPC installation and \num{1}
   month before the end of the TPC installation) in which
   $\sim$\SI{500}{m$^2$} for dedicated space and \SI{500}{m$^2$} for
   shared space.  Humidity needs to be $<$70\%. Re-packaging area
   needs to be class \num{100000} and no insects. There also needs to be a
   crane coverage between buffer and re-packaging areas.
  \item {\bf APA} For 40--80 APAs, say minimum of
    $\sim$\SI{1000}{m$^2$}, including a place for a cleanroom.  This
    is based on \num{1} year APA production and assume they will be
    transported from the manufacturing facility straight after they
    are made.  The space can be shared.  There need to be crane
    access, large door openings, height enough to lift boxes and allow
    fork lift.  Some APAs will be kept in transport boxes and after
    the PDs and electronic boxes have been added, they will be
    ``hung'' in a clean, dry area, ready for transport to SURF in a
    specialized box.
  \item {\bf DAQ} Area for some boxes and crates needed, but not while
    shipping containers.  And it can be shared..). It should meed
    standard electronics environment as well.
  \item {\bf Single Phase Cold Electronics} A total of \SI{40}{m$^2$}
    of space to be populated with racks and possibly one cabinet with
    dry air storage.  The space can be shared, although we would
    prefer not to have to share the dry air cabinets.  We prefer to
    avoid storage at temperatures below 10$^\circ$C and we would also
    prefer an environment with a controlled humidity level such that
    the dew point in the storage area is below 5$^\circ$C.

For components that will be installed on the APAs at the integration
facility, they need to be stored (after unpacking) in a dry-air
cabinets such that the dew point is significantly below that of the
room temperature (a relative humidity in the dry air cabinets at the
level of 30\% is sufficient to ensure this). We also need these
cabinets to be connected to ground such that we can store the
components minimizing the possibility of having electrostatic
discharge damage.
  \item {\bf Dual Phase Cold Electronics} The largest space will be
    taken by the signal chimneys (box for a chimney
    $2.2\times0.5\times0.5~m^3$), \num{240} chimneys to be installed,
    30\% buffer.  We would need \SI{50}{m$^2$} dedicated space out of
    the total \SI{200}{m$^2$} space.  No particular environmental
    requirements are needed, but we would need handling facilities for
    the chimneys boxes with weight $\sim$\SI{100}{kg}.
   \item {\bf Dual Phase Photon Detection} We would need a dedicated
     space of \SI{45}{m$^2$} with dark room with climate (temperature
     and humidity) control.
\end{itemize}

\paragraph{\bf Re-packaging}

\begin{itemize}
  \item {\bf High Voltage System} Nearly all CPA, FC modules are
    shipped in $20'$ shipping containers with high packing density.
    These units need to be transferred to the UG crates to be provided
    by the HVS.  During this process, some basic inspection and tests
    will be performed either by HVS personnel or trained ITF staff.
    An outer layer of plastic bag/sheet will be removed and replaced
    on the module before mounted into the UG crates.
  \item {\bf APA} We will be using a separate transport box to crane
    the APAs into the SURF facility, therefore will need a crane to
    repackage in a reasonably clean area.
  \item {\bf DAQ} We will likely set some stuff up in conjunction with
    the cold electronics reception/test station.  In which case, that
    would need to be disassembled and shipped out afterwards.  With
    respect to the main volume of Production DAQ stuff, it would come
    in computer boxes, pallets, or possibly electronics racks.  We are
    not sure if this would need repackaged to go down the shaft,
    however.
  \item {\bf Single Phase Cold Electronics} This is hard to predict at
    this point. For examples for detector cables we may want to
    transfer the cables onto spools that can be used to speed up the
    installation of the cables in the APAs once the APAs are brought
    into the toaster in the mine. For other components (crates, power
    supplies) we may need to transfer the components from the original
    packing used for the shipment from the institutions where the
    components were fabricated or tested into a different packing that
    is optimized for the transport in the mine of the set of
    components that are going to be installed in a short time period,
    or that facilitate lifting the components on the top of the
    cryostat. The possibility of fully populating racks or even crates
    prior to the transport in the mine, installation on the top of the
    cryostat, cannot be excluded. Depending on the nature of the work,
    we expect that some monitoring or active participation in the
    re-packaging activities will be provided by members of the
    consortium.
   \item {\bf Dual Phase Cold Electronics} Very likely there will be
     no re-packaging.
   \item {\bf Dual Phase Photon Detection} The original packaging will
     be opened for testing of the equipment inside. Re-packaging will
     be done using the original packaging materials. At this stage,
     additional external attachments might be added in order to make
     the package more suitable for underground transportation. These
     may include vibration dampers, locks, carriage hooks, etc.
\end{itemize}

\paragraph{\bf Component Fabrication} 

\begin{itemize}
  \item {\bf High Voltage System} No need of fabrication capabilities
    of the ITF.
  \item {\bf APA} There is always a need for technical effort, the
    specifics of this is difficult to evaluate at this time, but may
    include simple tooling needs, turning, milling, drilling, grinding
    etc.
  \item {\bf DAQ} We might need for things we didn't anticipate
    beforehand --- do we need new mounting brackets, strain relief,
    etc.
  \item {\bf Single Phase Cold Electronics} We do not expect to do any
    fabrication work at the ITF. We cannot however exclude the need
    for small repairs or the need for the quick fabrication of tooling
    that may be needed either at the ITF or at Sanford Lab. We expect
    to have engineer(s) and technician(s) from the consortium
    institution available for these activities, but we may need to
    resort to the help of local personnel from the SDSM\&T. For small
    repairs we are likely to require a small electronic shop.
  \item {\bf Dual Phase Cold Electronics} Very likely no need.
  \item {\bf Dual Phase Photon Detection} We might choose to perform
    the TPB coating of the PMTs at ITF. In this case, a coating
    facility will be established in a dedicated space at ITF,
    dimensions to be determined at a later stage. The operations will
    be supervised by DPPD and will likely be executed by
    students/engineers.
\end{itemize}

\paragraph{\bf Component Integration}

\begin{itemize}
  \item {\bf High Voltage System} It is possible that the integration
    of the top field cage to the ground plane (attaching the ground
    plane tiles to the top FC modules), or the integration of the
    bottom ground plane (linking the ground plane tiles into larger
    modules) can be carried out at the ITF.
  \item {\bf APA} Skilled technical effort will be needed with APA
    integration assembly, tooling attachment, some cabling
    assistance. Some of this will be because of health and safety
    reasons. Space requirement is \SI{100}{m$^2$} for 1--2 APAs
  \item {\bf DAQ} Possibly, rack stuffing.
  \item {\bf Single Phase Cold Electronics} We expect that the
    installation and testing of the cold electronics onto the APAs
    that will take place at the Integration facility will be performed
    by member of the Cold Electronics consortium stationed there. We
    plan to have a team comprising at least one engineer, one
    technician and several students/postdocs/scientists to perform
    these activities. Students from SDSM\&T could be integrated in
    this team mostly for the testing activities, but we do expect that
    the majority of the team will be composed by member of the Cold
    Electronics consortium at all times.
   \item {\bf Dual Phase Cold Electronics} We do not plan to perform integration at the ITF.
   \item {\bf Dual Phase Photon Detection} DPPD deliverables will not
     require integration with other subsystem elements at the ITF.
\end{itemize}

\paragraph{\bf Inspection, Testing and Repair}

\begin{itemize}
  \item {\bf High Voltage System} During the re-packaging of the
    CPA/FC/GP modules, perform visual inspection of damages,
    electrical continuity test of a set predetermined test points and
    a small number of resistivity measurements.  Test results will be
    logged in the traveler documents accompanying the modules.  Test
    instruments will be provided by HVS.  Repairs will be performed by
    HVS experts. Entire process needs to be in class \num{100000} clean
    space.
  \item {\bf APA} We will need a reasonably clean area for visual
    inspection and possibly the tension of the wires.  And there will
    be a full test of the APA in a cold box and will require liquid
    nitrogen.  We might need minor repairs only.
  \item {\bf DAQ} Testing of components to make sure they arrived ok,
    at the level of ``does it turn on'' or ``there's no link light''.
    More detailed testing could be done remotely by DAQ experts and if
    repairs are needed, it should be shipped back for expert TLC.
  \item {\bf Single Phase Cold Electronics} The Cold Electronics
    consortium plans to use the Integration Facility mostly for
    installing the Front End Motherboards on the APAs and then
    performing tests of the fully populated APA prior to the shipment
    of the APA to Sanford Lab. These activities will be performed
    jointly by the APA, Photon Detector and Cold Electronics
    consortia, using equipment that will also be provided by the Cold
    Instrumentation and Slow Controls consortium and by the DAQ
    consortium. The facility required for the installation and the
    test of the Front End Motherboards onto the APA should be modeled
    on the \dword{protodune} installation area. It requires a crane
    system for lifting the APA from its shipping box, a suspension
    system using rails that can be used to move the APA in and out of
    an area dedicated to the installation of the electronics and in
    and out of a cold box to be used for tests. Scissor lifts or a
    system of platforms should be in place to allow work at
    heights. The team responsible for the \dword{protodune}
    installation should provide feedback in the design of this area. A
    detailed study of the scheduling for the integration of the
    electronics and the photon detector system on the APA should be
    done to understand how many areas where this work is performed in
    parallel are needed (we expect that at least two stations
    operating in parallel are requires). At the moment we do not
    foresee the need to perform other tests at the integration
    facility. We would still prefer to keep the option open for having
    a small laboratory space (\SI{20}{m$^2$}) where we can test Front End
    Motherboards that do not perform as expected after the
    installation on the APAs to decide whether they should be repaired
    locally or sent back to one of the consortium institutions for
    further investigation/repairs.
   \item {\bf Dual Phase Cold Electronics} Just integrity of the
     transportation packaging, no opening of the packaging.
   \item {\bf Dual Phase Photon Detection} We will perform basic
     operation and quality checks on the photodetectors, calibrations
     systems, cables, fibers and high voltage system components.
     Photodetectors will be tested for basic operation, others might
     be as simple as visual inspection.  The laboratory space required
     for these test is minimum \SI{40}{m$^2$}. This laboratory space must
     have climate control, sufficient electrical and cabling
     infrastructure (racks, power, lighting, cable trays) and
     reasonable proximity to the DPPD storage area. The testing
     operations will be supervised by DPPD and will likely be executed
     by students.
\end{itemize}

\paragraph{\bf Visitor Support} 

\begin{itemize}
  \item {\bf High Voltage System} A common shared office space with
    other consortia member would suffice. We expect to have $\sim$\num{2}
    long term (commute between ITF and SURF), up to \num{5} short term
    visitors.
  \item {\bf APA} Likely over a period of \num{2} years with \num{4} people full
    time plus \num{4} people visiting part time 50\% and \num{4} people
    underground (\num{1} engineer, \num{2} physicists and \num{1} tech) for \num{1}
    year.
  \item {\bf DAQ} During commissioning, at least the same size team as
    will be at \dword{protodune}.  During operations, probably one or two
    experts steady-state.
  \item {\bf Single Phase Cold Electronics} We expect that a large
    fraction of the students/postdocs/scientific personnel from the
    consortium will spend long periods of time (between \num{3} months and
    \num{1} year) at the integration facility and at Sanford Laboratory. A
    smaller fraction of the personnel will commute for shorter periods
    of time. We expect a similar pattern for engineers and
    technicians. Overall, we expect to have a team of 12--15 people
    from the Cold Electronics consortium will be present at all times
    at the Integration Facility. A similar number of people (up to
    \num{20}) working on the installation of the detector in Sanford Lab
    may also expect to be able to use any support infrastructure for
    visitors at the Integration Facility.
  \item {\bf Dual Phase Cold Electronics} \num{2} visitors, stay of the order of a few months.
  \item {\bf Dual Phase Photon Detection} Eight visitors for \num{4}
    months/year; four visitors for \num{12} months/year.
\end{itemize}

%$$$$$$$$$$

\subsubsection{Management:}
The management of the ITF should likely be provided by one or more
\dword{dune} Collaborating Institutions. A possible choice is SDSM\&T
because of its physical proximity, its understanding of the local
infrastructure and relationships and its ability to provide some
specialized effort and specialized facilities that might benefit the
ITF.  Some preliminary discussions with SDSM\&T management have
already occurred. These discussions should be ongoing as the
parameters for the ITF become more definite.


\subsubsection{Inventory System:}
Effective inventory management will be essential for all aspects of
\dword{dune} detector development, construction, installation and
operation.  While its relevance and importance go beyond the
Integration and Test Facility, the ITF is the location at which
\dword{lbnf}, \dword{dune} project management, consortia scientific
personnel and SURF operations will interface.  We therefore will
develop standards and protocols for inventory management as part of
the ITF planning.  A critical requirement for the project is that the
inventory management system for procurement, construction and
installation must be compatible with future QA, calibration and
detector performance database systems.  Experience with past large
detector projects, notably NOvA, has demonstrated that the capability
to track component-specific information is extremely valuable
throughout installation, testing, commissioning and routine operation.
Compatibility between separate inventory management and physics
information systems will be maintained for effective operation and
analysis of \dword{dune}.

\dword{dune} will rely on a commercial vendor for warehouse and
logistics services in Rapid City or another location nearby to SURF.
Warehouse vendors have a variety of inventory software packages and
standards, and final specification of the \dword{dune}/\dword{lbnf}
system cannot happen until the project warehouse vendor is selected.
Discussions are being coordinated closely with \dword{lbnf} and
initial visits and meetings with warehouse vendors and software
suppliers have occurred.  \dword{dune} scientific personnel will
continue to evaluate candidate systems and assure interoperability
with a future physics database information systems.

Because of the widely distributed nature of the \dword{dune}
development and construction project and the required compatibility
with a commercial warehouse management system, we plan to develop core
inventory management capabilities based on a service-oriented
architecture.  URL connections will be used to pass data (JSON format)
to RESTful APIs, which have task-specific code written in Python that
communicates with standard PostgresSQL database that will be developed
for \dword{dune} by Fermilab.  Specialized code at remote sites would
also be in Python.

Implementation within a commercial cloud-based computing environment,
well suited to the international \dword{dune} project, is also under
consideration.  A recent visit to Rapid City revealed that Dakota
Warehouse
(\href{https://dakotawarehouse.com}{https://dakotawarehouse.com}), a
leading candidate for providing \dword{lbnf}/\dword{dune} warehouse
services for detector components, including cryostat and cryogenic
systems, uses a cloud-based commercial software package, 3PL Central
(\href{https://3plcentral.com}{https://3plcentral.com}), in which
orders of shipments and stock status are entered and queried through
an internet browser interface.  We will consider the feasibility of
this or a similar platform for \dword{lbnf} and \dword{dune}.
