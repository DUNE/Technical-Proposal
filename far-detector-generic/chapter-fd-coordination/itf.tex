%%%%%%%%%%%%%%%%%%%%%%%%%%%%%%%%
\subsection{Surface Logistics \& Testing}
\label{sec:fdsp-coord-integ-test}

The logistics for integrating and installing the DUNE Far Detectors
and their associated infrastructure face a number of
challenges. Possible difficulties include the size and complexity of
the detector itself, the number of sites around the world that will be
fabricating detector and infrastructure components, the necessity for
protecting components from dust, vibration and shock during their
journey to the deep underground laboratory and the lack of space on
the surface near the Sanford Lab Ross Shaft. One mitigation
of these risks is the establishment of a DUNE
Integration and Test Facility (ITF) somewhere in the vicinity of
Sanford Lab. Such a facility and its associated staff would contribute
to DUNE in the following areas.
\begin{itemize}
  \item {\bf Transport Buffer:} Storage capacity for one month
    material in the vicinity of Sanford Lab. Handle packaging
    materials returned from underground laboratory.
  \item {\bf Re-packaging} Facilitate possible re-packaging of
    components before transport underground.
  \item {\bf Component Fabrication:} Possibly provide a capability for
    fabrication of components near Sanford Lab. Undergraduate science
    and engineering students from the South Dakota School of Mines and
    Technology (SDSM\&T) may contribute low cost effort to these
    fabrication activities.
  \item {\bf Component Integration:} Some integration activities may
    be best accomplished in proximity to Sanford Lab. A possible
    example is connecting photodetectors and cold electronics to APAs.
  \item {\bf Inspection, Testing and Repair:} Consortia will define
    their testing requirements including procedures and
    criteria. Consortia will also specify procedures in cases of test
    failure, for example, repair, return to source or discard.
  \item {\bf Visitor Support:} Consortia will likely send staff to the
    ITF for the integration, testing and installation of the consortia
    detector components. The ITF will provide temporary space,
    computer access, assistance personnel and other infrastructure
    support for DUNE visitors.
%\item {\bf Outreach:} The ITF may be well located to support a public outreach program. The DUNE Experiment is likely to generate considerable public interest and addressing those interests is important to long-term public support for DUNE specifically and particle physics generally.
\end{itemize}

\subsubsection{Scope}
The scope of the ITF includes several possibly related but mostly
independent tasks. They are:
\begin{itemize}
\item{\bf Cryostat:} The scope of this item is the four cryostats
  planned for installation at the 4850 level of Sanford Lab. Cryostat
  components include the warm steel structure, the stainless steel membrane and
  the insulation. The logistics for the cryostat components will be
  managed by the cryostat installation contractor and LBNF logistics
  coordinator.  Most likely this function will be met with a
  commercial warehousing vendor, who will supply suitable space,
  loading and unloading facilities and a commercial inventory
  management and control system. The vendor will provide all required
  personnel effort as part of its contracted responsibilities.
\item{\bf Cryogenics Systems:} The cryogenics systems are also an LBNF
  responsibility and cryogenics system logistics will likely be
  managed by LBNF similarly to the logistics for the cryostat
  components.
\item{\bf Cryostat Support Structure:} This structure is an LBNF
  responsibility and will likely be addressed similarly to the
  Cryostats and the Cryogenic Systems.
\item{\bf DUNE Detectors:} The DUNE Detectors are the responsibility
  of the Collaboration as implemented by the Consortia. The role of
  the ITF will vary for the several Consortia and a description of
  these various roles is a major topic of this document.
\end{itemize}

\subsubsection{Location }
A reasonable criterion for the location of the ITF is within about an
hour drive from Sanford Lab. That criterion yields the following
possibilities.
\begin{center}
\begin{tabular}{ |c|c| } \hline
{\bf Location} & {\bf 2016 Population}  \\ \hline 
Deadwood & 1,264  \\ 
Lead & 3,010  \\
Rapid City & 74,048  \\
Spearfish & 11,531  \\
Sturgis & 6,832  \\ \hline
\end{tabular}
\end{center}
Since infrastructure is correlated with population, Rapid City would seem the most likely
choice for location with Spearfish as a second possibility. In addition to overall infrastructure,
particular assets of Rapid City include proximity to SDSM\&T and a business community that is
possibly interested in incorporating a DUNE ITF into an overall regional development
program.
%Black Hills State University is located in Spearfish, but that institution is less technology oriented than SDSM\&T.

\subsubsection{Management:}
The management of the ITF should likely be provided by one or more
DUNE Collaborating Institutions. A possible choice is SDSM\&T because
of its physical proximity, its understanding of the local
infrastructure and relationships and its ability to provide some
specialized effort and specialized facilities that might benefit
the ITF.  Some preliminary discussions with SDSM\&T management
have already occurred. These discussions should be ongoing as the
parameters for the ITF become more definite.
\begin{center}
\begin{tabular}{|l|c|c|c|c|c|c| } 
\hline
{\bf Consortium} & {\bf Transport} &{\bf Re-Packaging}&{\bf Component}
&{\bf Component}&{\bf Inspection,}&{\bf Visitor} \\
 & {\bf Buffer} &{\bf }&{\bf Fabrication}
&{\bf Integration}&{\bf Testing}&{\bf Support} \\\hline 
High Voltage & Yes & Yes & No & Yes? & Yes & Yes \\ \hline
APA & Yes & Yes & Yes & Yes & Yes & Yes \\ \hline
DAQ & Yes & Yes & Yes & Yes & Yes & Yes \\ \hline
CE-SP & Yes & Yes & No & No & Yes & Yes \\ \hline
CE-DP & Yes & No & No & No & Yes & Yes \\ \hline
PD-SP & & & & & &  \\ \hline
PD-DP & Yes & Yes & Yes & No & Yes & Yes \\ \hline
Slow Control & & & & & &  \\ \hline
CRP & & & & & &  \\   \hline
\end{tabular}
\end{center}
\begin{center}
\begin{tabular}{|l|c|c|} \hline
{\bf Consortium} & {\bf Leaders} &{\bf Respondents} \\ \hline
High Voltage & Francesco Pietropaolo, Bo Yu & Bo Yu \\ \hline
APA & Stefan Soldner-Rembold, Alberto Marchionni & Peter Sutcliffe \\ \hline
DAQ & Georgia Karagiorgi, Dave Newbold & Alec Habig \\ \hline
CE-SP & & \\ \hline
CE-DP & D. Autiero, T. Hasegawa &  D.Autiero  \\ \hline
PD-SP & & \\ \hline
PD-DP & Ines Gil Botella, Dominique Duchesneau & Burak Bilki \\ \hline
Slow Control & &   \\ \hline
CRP & &  \\   \hline
\end{tabular}
\end{center}

\subsubsection{Inventory System:}
Effective inventory management will be essential for all aspects of
DUNE detector development, construction, installation, and operation.
While its relevance and importance go beyond the Integration and Test
Facility, the ITF is the location at which LBNF, DUNE project
management, consortia scientific personnel, and SURF operations will
interface.  We therefore will develop standards and protocols for
inventory management as part of the ITF planning.  A critical
requirement for the project is that the inventory management system
for procurement, construction and installation must be compatible with
future QA, calibration and detector performance database systems.
Experience with past large detector projects, notably NOvA, has
demonstrated that the capability to track component-specific
information is extremely valuable throughout installation, testing,
commissioning, and routine operation.  Compatibility between separate
inventory management and physics information systems will be
maintained for effective operation and analysis of DUNE.

DUNE will rely on a commercial vendor for warehouse and logistics
services in Rapid City or another location nearby to SURF.  Warehouse
vendors have a variety of inventory software packages and standards,
and final specification of the DUNE/LBNF system cannot happen until
the project warehouse vendor is selected.  Discussions are being
coordinated closely with LBNF and initial visits and meetings with
warehouse vendors and software suppliers have occurred.  DUNE
scientific personnel will continue to evaluate candidate systems and
assure interoperability with a future physics database
information systems.

Because of the widely distributed nature of the DUNE development and
construction project and the required compatibility with a commercial
warehouse management system, we plan to develop core inventory
management capabilities based on a service-oriented architecture.  URL
connections will be used to pass data (JSON format) to RESTful APIs,
which have task-specific code written in Python that communicates with
standard PostgresSQL database that will be developed for DUNE by
Fermilab.  Specialized code at remote sites would also be in Python.

Implementation within a commercial cloud-based computing environment,
well suited to the international DUNE project, is also under
consideration.  A recent visit to Rapid City revealed that Dakota
Warehouse
(\href{https://dakotawarehouse.com}{https://dakotawarehouse.com}), a
leading candidate for providing LBNF/DUNE warehouse services for
detector components, including cryostat and cryogenic systems, uses a
cloud-based commercial software package, 3PL Central
(\href{https://3plcentral.com}{https://3plcentral.com}), in which
orders of shipments and stock status are entered and queried through
an internet browser interface.  We will consider the feasibility of
this or a similar platform for LBNF and DUNE.
