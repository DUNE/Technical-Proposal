%%%%%%%%%%%%%%%%%%%%%%%%%%%%%%%%
\section{The Integration and Test Facility}
\label{sec:fdsp-coord-integ-test}

\dword{dune} \dword{tc} is responsible for interfacing with the
\dword{lbnf} logistics team at \surf to coordinate transport of all
detector components into the underground areas.  Due to the limited
availability of surface areas at \surf for component storage, nearby
facilities will be required to receive, store and ship materials to
the Ross Shaft on an as-needed basis. A team within the \dword{tc}
organization will %be needed to 
develop and execute the plan for
receiving detector components at a surface facility and transporting
them to the Ross Shaft in coordination with the on-site \dword{lbnf}
logistics team.  The surface facility will require warehouse space
with an associated inventory system, storage facilities, material
transport equipment and access to trucking.  Basic functions of this
facility will be to receive the detector components arriving from
different production sites around the world and prepare them for
transport into the underground areas, incorporating re-packaging and
testing as necessary. As a substantial facility will %necessarily 
be
required, it can also serve as a location where some detector
components are integrated and undergo further testing prior to
installation.

The logistics associated with integrating and installing the
%\dword{dune} 
\dwords{fd} and their associated infrastructure present
a number of challenges. %Possible difficulties 
These include the size and
complexity of the detector itself, the number of sites around the
world that will be fabricating detector and infrastructure components,
the necessity for protecting components from dust, vibration and shock
during their journey to the deep underground laboratory, and the lack
of space on the surface near the Ross Shaft. To help
mitigate the associated risks, \dword{dune} plans to establish an
\dword{itf} somewhere in the vicinity of \surf. This facility
and its associated staff will need to provide certain functions and
services connected to the \dword{dune} \dword{fd} integration and
installation effort and will have other potential roles that are still under
consideration.  The areas to which the \dword{itf} will and could
contribute are the following:
\begin{itemize}
  \item {\bf Transport buffer:} The \dword{itf} needs to provide
    storage capacity for a minimum one-month buffer of detector
    components required for the detector installation process in the
    vicinity of \surf and be able to accept packaging materials
    returned from the underground laboratory.
  \item {\bf Longer-term storage:} The \dword{itf} needs to provide
    longer-term storage of detector components that need to be
    produced in advance of when they are to be installed and for which
    sufficient storage space does not exist at the production sites.
  \item {\bf Re-packaging} The \dword{itf} needs to have the capability
    to re-package components arriving from the various production
    sites into boxes that can be safely transported through the shaft
    into the underground areas.
  \item {\bf Component fabrication:} It could be convenient to fabricate some
  components %The fabrication of some components could be best suited to being located 
  in the vicinity
    of \surf at the \dword{itf}, taking advantage of
    undergraduate science and engineering students from the South
    Dakota School of Mines and Technology (SDSMT).
  \item {\bf Component integration:} Integration of detector components
  %Detector components 
  received from different production sites %that need to be integrated 
  that must be done prior to transport underground,
   % delivery to the underground area, 
    such as the installation of \dwords{pd} and 
    electronics on the \dwords{apa}, could be done
    prior to re-packaging at the \dword{itf}.    
  \item {\bf Inspection, testing and repair:} In cases where
    components are re-packaged for transport to the underground area,
    \dword{itf} support for performing tests on these components to
    ensure that no damage has occurred during shipping is likely
    desirable.  In addition, components integrated at the \dword{itf}
    would require additional testing prior to being re-packaged.  The
    \dword{itf} could provide facilities needed to repair some of the
    damaged components (others would likely need to be returned to
    their production sites).
  \item {\bf Collaborator support:} The host institution of the \dword{itf}
    will need to %be able to 
    provide support for a significant number
    of \dword{dune} collaborators involved in the above activities
    including services such as housing assistance, office space,
    computing access, and safety training.
  \item {\bf Outreach:} The host institution of the \dword{itf} would be
    ideally situated for supporting an outreach program to build upon
    the considerable public interest in the experiment that exists
    within South Dakota.
\end{itemize}

 The facilities project (\dword{lbnf}) will provide the cryostats that 
 house the \dwords{detmodule} and the cryogenics
systems that support them.  Additional large surface facilities in the vicinity
of \surf will be required to stage the components of these
infrastructure pieces prior to their installation in the underground
areas.  The requirements for these facilities, in contrast to the
\dword{itf}, are relatively straight-forward and can likely be met by
a commercial warehousing vendor, who would provide suitable storage
space, loading and unloading facilities, and a commercial inventory
management and control system.  It is currently envisioned that
operation of the \dword{lbnf} surface facilities and the \dword{dune}
\dword{itf} will be independent from one another.  However, the
inventory systems used at the different facilities will need to %be interconnected with each other to 
ensure proper coordination of all
deliveries being made to the \surf site.

A reasonable criterion for the locations of these surface staging
facilities is to be within roughly an hour's drive of \surf.
Population centers in South Dakota within this radius are Lead,
Deadwood, Sturgis, Spearfish, and Rapid City.  The \dword{lbnf}
surface facilities, which are not expected to have significant
auxiliary functionality, would logically be located as near to the
\surf site as possible.  In the case of the \dword{dune}
\dword{itf}, however, locating the facility in Rapid City to take
advantage of facilities and resources associated with the South Dakota
School of Mines \& Technologies (SDSMT) has a number of potential
advantages.  SDSMT and the local business community in Rapid City
have expressed interest in incorporating the \dword{dune} \dword{itf}
into an overall regional development program.

\subsection{Requirements}

The leadership teams associated with the \dword{dune} %institutional
consortia taking responsibility for the different \dword{fd}
subsystems have provided input to \dword{tc} on which potential
\dword{itf} functions and services would be applicable to their
subsystems.  The consortia have also made preliminary assessments of
the facility infrastructure requirements that would be necessary in
order for these functions and services to be provided for their
subsystems at the \dword{itf}.  An attempt to capture preliminary,
global requirements for the \dword{itf} based on the information
received from the consortia results in the following:
\begin{itemize}
  \item Warehousing space on the order of \SI{2800}{m^2} (\num{30000} square feet); driven by
    potential need to store hundreds of the larger detector components
    needed to construct the TPCs.  The provided space will need to
    maintain some minimal cleanliness requirements (e.g.; no insects)
    and be climate-controlled within reasonable temperature and
    humidity ranges.
  \item Crane or forklift coverage throughout the warehousing
    space to access components as needed for further processing or
    transport to \surf.
  \item Docking area to load trucks with detector components being
    transported to \surf and to receive packaging materials
    returned from the site.
  \item Smaller clean room areas within the warehousing space to
    allow for the re-packaging of detector components for transport
    underground.  Re-packaging of larger components
    will require local crane coverage within the appropriate clean room
    spaces.
  \item Dedicated space for racks and cabinets available for dry-air
    storage and testing of electronics components.  Racks and cabinets
    must be properly connected to the building ground to avoid
    potential damage from electrostatic discharges.
  \item Climate-controlled dark room space for the handling and
    testing \dword{pd} components.
  \item Dedicated clean room on order of \SI{1000}{m^2} (\num{10000} square feet) to facilitate
    integration of electronics and \dwords{pd} on \dwords{apa}.
    %which make up the readout planes of the \dword{sp} \dword{fd}. 
    The
    integrated \dwords{apa} are tested in cold boxes supported by cryogenics
    infrastructure.  Clean room lighting must be UV-filtered to avoid
    damaging the \dwords{pd}.  The height of the clean room must
    accommodate crane coverage needed for movement of the \dwords{apa} in and
    out of the cold boxes.  It will also be necessary to have
    platforms for installation crews to perform work at heights within
    different locations in the clean room.
  \item Access to machine and electronics shops for making simple
    repairs and fabricating unanticipated tooling.
  \item Access to shared office space for up to \num{30} collaborators
    contributing to the activities taking place at the \dword{itf}.
    Assistance for identifying temporary housing in the vicinity of
    the \dword{itf} for the visiting collaborators.
\end{itemize}

\subsection{Management}

Overall management of the \dword{itf} is envisioned to be
responsibility of one or more of the collaborating institutions on
\dword{dune}.  If the \dword{itf} is located in Rapid City, SDSMT
would be a natural choice due to its physical proximity, connections
to the local community and ability to provide access to resources and
facilities that would benefit planned \dword{itf} activities.  Initial
discussions with SDSMT representatives have taken place, in which
they have expressed a clear interest for hosting the \dword{itf}.
Additional discussions will be needed to understand the details of the
\dword{itf} management structure with in the context of a more
finalized set of requirements for the facility.

\subsection{Inventory System}

Effective inventory management will be essential for all aspects of
\dword{dune} detector development, construction, installation and
operation.  While its relevance and importance go beyond the
\dword{itf}, the \dword{itf} is the location at
which \dword{lbnf}, \dword{dune} project management, consortium
scientific personnel and \surf operations will interface.  We therefore
will develop standards and protocols for inventory management as part
of the \dword{itf} planning.  A critical requirement for the project
is that the inventory management system for procurement, construction
and installation be compatible with future \dword{qa}, calibration and
detector performance database systems.  Experience with past large
detector projects, notably \nova, has demonstrated that the capability
to track component-specific information is extremely valuable
throughout installation, testing, commissioning and routine operation.
Compatibility between separate inventory management and physics
information systems will be maintained for effective operation and
analysis of \dword{dune} data.

\subsection{ITF Infrastructure}

\dword{tc} is responsible for providing the common support
infrastructure at the \dword{itf}. This includes the cranes and
forklifts to move equipment in the \dword{itf}, %. This also includes 
any
cryostats and cryogenics systems to enable cold tests of consortium-provided 
components, %. This also includes 
the cleanrooms and cleanroom equipment
to enable work on or testing of consortia components, and UV-filtered
lighting as needed. This also includes racks and cable trays.
