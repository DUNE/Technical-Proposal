\section{Project Support}
\label{sec:fdsp-coord-supp}

The \dword{dune} Project is coordinated by Technical Coordination
(TC). The \dword{dune} Project consists of a \dword{fd} and a
\dword{nd}. The \dword{nd} is at a pre-conceptual state; as the
Conceptual Design and organization emerges, it will become part of the
\dword{dune} Project. Currently the \dword{dune} Project consists of
the \dword{dune} \dword{fd} consortia and Technical Coordination.

[high level schedule here?]

As defined in the \dword{dune} Management Plan (DMP), the \dword{dune}
Technical Board (TB) is the technical decision making body for the
collaboration. It consists of all consortia scientific and technical
leads. It reports through the Executive Board (EB) to Collaboration
Management. The \dword{dune} Technical Board is chaired by the
Technical Coordinator.

[Need the Collaboration management and TC Org charts?]


TC will work with the \dword{lbnf}/\dword{dune} Systems Engineer to
implement the \dword{lbnf}/\dword{dune} Configutation Management Plan
to assure that all aspects of the overall \dword{lbnf}/\dword{dune}
project are well integrated. TC will work with \dword{lbnf} and the
Host Lab to ensure that adequate infrastructure and operations support
are provided during construction, integration, installation,
commissioning and operation of the detectors.

[Need the \dword{lbnf}/\dword{dune} systems engineering Org chart]

Several major project support tasks need to be accomplished in advance
of the TDR:
\begin{itemize}
  \item Assure that each consortia has a well defined
and complete scope, that the interfaces between the consortia are
sufficiently well defined and that any remaining scope can be covered
by TC through Common Fund.
  \item Develop an overall project schedule that includes reasonable
production schedules from each consortia, well developed QA and QC
plans and a well developed installation schedule.
  \item Ensure that appropriate engineering and safety standards are
developed and agreed to by all key stakeholders and that these
standards are conveyed to and understood by each Consortium.
  \item Ensure that all \dword{dune} requirements on \dword{lbnf} for
conventional facilities, cryostat and cryogenics have been clearly
defined and understood by each Consortia.
  \item Ensure that all technical issues associated with scaling from
\dword{protodune} have sufficient resources to converge on decisions that
enable the detector to be fully integrated and installed.
  \item Ensure that the integration and QC processes for each
consortia are fully developed and reviewed and that the requiements on
an Integration and Test Facility are well defined.
\end{itemize}

%%%%%%%%%%%%%%%%%%%%%%%%%%%%%%%%
\subsection{Project Schedule}
\label{sec:fdsp-coord-controls}

Technical Coordination Project Controls (TCPC) maintains a web page
(currently located at
\href{https://web.fnal.gov/collaboration/DUNE/DUNE\%20Project/\_layouts/15/start.aspx\#/})
with links to project documents. TC maintains repositories of project
documents and drawings. These include the WBS, Schedule, risk
register, requirements, milestones, strategy, detector models and
drawings that define the \dword{dune} detector.

[something about DocDB and edms?]

In order to ensure that the \dword{dune} detector remains on schedule, TCPC
will monitor schedule statusing from each consortium, will organize
reviews of schedules and risks as appropriate. TCPC will maintain a
master schedule that links all consortia schedules and contains
appropriate milestones to monitor progress. The master schedule will
go under change control after the TDR is approved.

The consortia have provided preliminary versions of risk analyses that
have been collected on the TCPC webpage. These will be developed into
an overall risk register that will be monitored and maintained by TCPC
in coordination with the consortia.

A schedule of key consortia activity in the period 2018--19 leading up
to the TDR has been developed.

A monthly report with input from all consortia will be published by
TCPC. This will include updates on consortia technical progress and
updates from TC.

Consortia have developed initial interface documents that will be put
under change control and managed by the TC integration engineering
team along with the consortia involved. These are currently in DocDB
and will likely go under change control later in 2018, although they
will continue to be developed through the TDR.

TCPC will maintain approved versions of QA, QC and testing plans,
installation plans, engineering and safety standards,...

A series of tiered milestones are being developed for the \dword{dune}
project. The Tier-0 milestones are held by the Spokespersons and Host
Lab director. Three have been defined and the current target dates
are:
\begin{enumerate}
\item Start main cavern excavation \hspace{2.1in} 2019
\item Start detector \#1 installation \hspace{2.1in} 2022
\item Start operations of detector \#1--2 with beam \hspace{1in} 2026
\end{enumerate}
These dates will be revisited at the time of the TDR review.  Tier-1
milestones will be held by the Technical Coordinator and \dword{lbnf} Project
Manager and will be defined in advance of the TDR review. Tier-2
milestones will be held by the Consortia.


%%%%%%%%%%%%%%%%%%%%%%%%%%%%%%%%
\subsection{Risk}
\label{sec:fdsp-coord-risk}

The successful operation of \dword{protodune} will retire a great many
potential risks to \dword{dune}. This includes most risks associated with the
technical design, production processes, quality assurance, integration
and installation. Residual risks remain relating to design and
production modifications associated with scaling to \dword{dune}, mitigations
to known installation and performance issues in \dword{protodune}, underground
installation at SURF and organizational growth.

[Enumerate remaining technical risks?.... or all risks?.... 600kV, HV
  in general, noise, dead channels, 20 year operation, QC in general,
  ADC/coldata, photon light yield, purity, LAr surface stbility, LEM
  gain, dual-phase LAr surface cleanliness, cathode/FC discharge to
  cryostat, incomplete calibration plan, incomplete connection of
  design to physics; funding, production schedule, integration plan,
  testing, underground installation, ...]

Key risks for TC to manage include the following:
\begin{enumerate}
  \item A key risk for TC is to ensure that sufficient scope is funded
    by the Consortia, such that the deliverables from TC do not grow
    so large as to be unsupportable by Common Fund.
  \item The second key risk is to ensure that key stakeholders to this
    first international mega-science project hosted in the US,
    including TC, FNAL as Host Lab, SURF, DOE and all international
    partners continue to successfully work together to ensure
    appropriate rules and services are provided to enable success of
    the project.
  \item The third key risk is to ensure that TC obtains sufficient
    personnel resources so as to be able to ensure that TC can oversee
    and coordinate all of its project tasks.  While the US has a
    special responsibility towards TC as host country, it is expected
    that personnel resources will be directed to TC from each
    collaborating country. Related to this risk is the fact that
    consortia deliverables are not really stand-alone subsystems; they
    are all parts of a single TPC. This elevates the requirements on
    coordination between consortia.
\end{enumerate}



%%%%%%%%%%%%%%%%%%%%%%%%%%%%%%%%
\subsection{Reviews}
\label{sec:fdsp-coord-reviews}

Technical Coordination is responsible to review all stages of detector
development and works with each consortium to arrange reviews of the
design, production readiness, production progress and operational
readiness of their system.  These reviews provide input for the TB to
make technical decisions.  Review reports are tracked by TC and
provide guidance as to key issues that will require engineering
oversight by the TC integration engineering team. TCPC will maintain a
calendar of \dword{dune} reviews.

TC will work with consortia leaders to review all detector designs,
with an expectation for a preliminary design review, followed by a
final design review. All major technology decisions will be reviewed
prior to down-select.

Start of production of all DUNE detector elements can commence after
successful production readiness review. Consortia should expect
regular production progress reviews once production has commenced.

%%%%%%%%%%%%%%%%%%%%%%%%%%%%%%%%
\subsection{Quality Assurance}
\label{sec:fdsp-coord-qa}


The \dword{lbnf}/\dword{dune} Quality Assurance Plan outlines the QA
requirements for all \dword{dune} Consortia and describes how the
requirements shall be met. The Consortia will be responsible for
implementing a quality plan that meet the requirements of the
\dword{lbnf}/\dword{dune} Quality Assurance Plan.  The Consortia
implement the plan through the development of individual quality
plans, procedures, guides, QC inspection and test requirements and
travelers/test reports.  In lieu of a Consortia Specific Quality Plan,
the Consortia may work under the \dword{lbnf}/\dword{dune} Quality
Assurance Plan and develop Manufacturing/QC Plans, procedures and
documentation specific to their work scope.  The \dword{dune}
Technical Coordinator and Consortia Leaders are responsible for
providing the resources needed to conduct the Project successfully,
including those required to manage, perform and verify work that
affects quality.  The \dword{dune} Consortia Leaders are responsible
for identifying adequate resources to complete the work scope and to
ensure that their team members are adequately trained and qualified to
perform their assigned work.

The Consortia work will be documented on travelers and applicable test
or inspection reports. Records of the fabrication, inspection and
testing will be maintained. When a component has been identified as
being in noncompliance to the design, the nonconforming condition
shall be documented, evaluated and dispositioned as use-as-is (does
not meet design but can meet functionality as is), rework (bring into
compliance with design), repair (will be brought into meeting
functionality but will not meet design) and scrap.

The \dword{lbnf}/\dword{dune} Quality Assurance Manager (QAM) reports
to the \dword{lbnf} Project Manager and \dword{dune} Technical
Coordinator and provides oversight and support to the Consortia
Leaders to ensure a consistent quality program.
\begin{enumerate}
  \item The QAM will plan reviews as independent assessments to assist
    the \dword{dune} Technical Coordinator in identifying opportunities for
    quality/performance-based improvement and to ensure compliance
    with specified requirements.
  \item The QAM is responsible to work with the Consortia in
    developing their QA/QC Plans.
  \item The QAM will review Consortia QA/QC activity, including
    production site visits.
  \item The QAM will participate in Consortia Design Reviews, conduct
    Production Readiness Reviews prior to the start of production,
    conduct Production Progress Reviews on a regular basis, and
    perform follow-up visits to Consortia facilities prior to shipment
    of components to ensure all components and documentation are
    satisfactory.
\item The QAM is responsible for performing assessments at the
  Integration Facility, the Far Site and the Near Site to
  ensure the activities performed at these locations are in accordance
  with the \dword{lbnf}/\dword{dune} QA Program and applicable procedures,
  specifications and drawings.
\end{enumerate}

%%%%%%%%%%%%%%%%%%%%%%%%%%%%%%%%
\subsection{ES\&H}
\label{sec:fdsp-coord-esh}

The \dword{dune} Environmental, Safety and Health (ESH) program is described
in the \dword{lbnf}/\dword{dune} Integrated Environmental, Safety and Health
Plan. This plan is maintained by the \dword{lbnf}/\dword{dune} ESH Manager, who
reports to the \dword{lbnf} Project Manager and the TC. The ESH is responsible
to work with the Consortia in reviewing their hazards and their ESH
plans.  The ESH Manager is responsible to review ESH at production
sites, integration sites and at SURF.
