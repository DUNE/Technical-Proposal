\section{Project Support}
\label{sec:fdsp-coord-supp}

The DUNE Project is coordinated by Technical Coordination (TC). The
DUNE Project consists of a Far Detector (FD) and a Near Detector
(ND). The Near Detector is at a pre-conceptual state; as the
Conceptual Design and organization emerges, it will become part of the
DUNE Project. Currently the DUNE Project consists of the DUNE Far
Detector Consortia and Technical Coordination.

[high level schedule here?]

As defined in the DUNE Management Plan (DMP), the DUNE Technical Board (TB) is
the technical decision making body for the collaboration. It consists
of all consortia scientific and technical leads. It reports through
the Executive Board (EB) to Collaboration Management. The DUNE Technical
Board is chaired by the Technical Coordinator.

[Need the Collaboration management and TC Org charts?]


TC will work with the LBNF/DUNE Systems Engineer to implement the
LBNF/DUNE Configutation Management Plan to assure that all aspects of
the overall LBNF/DUNE project are well integrated. TC will work
with LBNF and the Host Lab to ensure that adequate infrastructure and
operations support are provided during construction, integration,
installation, commissioning and operation of the detectors.

[Need the LBNF/DUNE systems engineering Org chart]

Several major project support tasks need to be accomplished in advance
of the TDR.
\begin{itemize}
  \item One is to assure that each consortia has a well defined
and complete scope, that the interfaces between the consortia are
sufficiently well defined and that any remaining scope can be covered
by TC through Common Fund.
  \item A second major project support task is to
develop an overall project schedule that includes reasonable
production schedules from each consortia, well developed QA and QC
plans and a well developed installation schedule.
  \item A third major area
is to ensure that appropriate engineering and safety standards are
developed and agreed to by all key stakeholders and that these
standards are conveyed to and understood by each Consortium.
  \item A fourth
major area is to ensure that all DUNE requirements on LBNF for
conventional facilities, cryostat and cryogenics have been clearly
defined and understood by each Consortia.
  \item A fifth major area is to
ensure that all technical issues associated with scaling from
ProtoDUNE have sufficient resources to converge on decisions that
enable the detector to be fully integrated and installed.
  \item A sixth
major area is to ensure that the integration and QC processes for each
consortia are fully developed and reviewed and that the requiements on
an Integration and Test Facility are well defined.
\end{itemize}

The successful operation of ProtoDUNE will retire a great many
potential risks to DUNE. This includes most risks associated with the
technical design, production processes, quality assurance, integration
and installation. Residual risks remain relating to design and
production modifications associated with scaling to DUNE, mitigations
to known installation and performance issues in ProtoDUNE, underground
installation at SURF and organizational growth.

[Enumerate remaining technical risks?.... or all risks?.... 600kV, HV
  in general, noise, dead channels, 20 year operation, QC in general,
  ADC/coldata, photon light yield, purity, LAr surface stbility, LEM
  gain, dual-phase LAr surface cleanliness, cathode/FC discharge to
  cryostat, incomplete calibration plan, incomplete connection of
  design to physics; funding, production schedule, integration plan,
  testing, underground installation, ...]

Key risks for TC to manage include the following:
\begin{enumerate}
  \item A key risk for TC is to ensure that sufficient scope is funded by the
Consortia, such that the deliverables from TC do not grow so large as
to be unsupportable by Common Fund.
  \item The second key risk is to ensure
that key stakeholders to this first international mega-science project
hosted in the US, including TC, FNAL as Host Lab, SURF, DOE and all
international partners continue to successfully work together to
ensure appropriate rules and services are provided to enable success
of the project.
  \item The third key risk is to ensure that TC obtains
sufficient personnel resources so as to be able to ensure that TC can
oversee and coordinate all of its project tasks.  While the US has a
special responsibility towards TC as host country, it is expected that
personnel resources will be directed to TC from each
collaborating country.
\end{enumerate}

		%				3 pages

%%%%%%%%%%%%%%%%%%%%%%%%%%%%%%%%
\subsection{Project Controls}
\label{sec:fdsp-coord-controls}

Technical Coordination Project Controls (PC) maintains a web page
(currently located at
{https://web.fnal.gov/collaboration/DUNE/DUNE\%20Project/\_layouts/15/start.aspx\#/})
with links to project documents. Technical Coordination maintains
repositories of project documents and drawings. These include the WBS,
Schedule, risk register, requirements, milestones, strategy, detector
models and drawings that define the DUNE detector.

[something about DocDB and edms?]

In order to ensure that the DUNE detector remains on schedule, TC
project controls will monitor schedule statusing from each Consortium,
will organize reviews of the schedules and risks as appropriate. PC
will maintain a master schedule that links all Consortia schedules and
contains appropriate milestones to monitor progress.

The Consortia have provided preliminary versions of risk analyses that
have been collected on the PC webpage.

A schedule of key Consortia activity in the period 2018--19 leading up
to the TDR has been developed.

A series of tiered milestones are being developed for the DUNE
project. The Tier-0 milestones are held by the Spokespersons and Host
Lab director. Three have been defined and the current target dates
are:
\begin{enumerate}
\item Start main cavern excavation \hspace{2.1in} 2019
\item Start detector \#1 installation \hspace{2.1in} 2022
\item Start operations of detector \#1--2 with beam \hspace{1in} 2026
\end{enumerate}
These dates will be revisited at the time of the TDR review.  Tier-1
milestones will be held by the Technical Coordinator and LBNF Project
Manager and will be defined in advance of the TDR review. Tier-2
milestones will be held by the Consortia.

%%%%%%%%%%%%%%%%%%%%%%%%%%%%%%%%
\subsection{Reviews}
\label{sec:fdsp-coord-reviews}

Technical Coordination is responsible to review all stages of detector
development and works with each Consortium to arrange reviews of the
design, production readiness, production progress and operational
readiness of their system.  These reviews provide input for the TB to
make technical decisions.  Review reports are tracked by TC and
provide guidance as to key issues that will require engineering
oversight by TC.

%%%%%%%%%%%%%%%%%%%%%%%%%%%%%%%%
\subsection{Quality Assurance}
\label{sec:fdsp-coord-qa}


The LBNF/DUNE Quality Assurance Plan outlines the QA requirements for
all DUNE Consortia and describes how the requirements shall be
met. The Consortia will be responsible for implementing a quality plan
that meet the requirements of the LBNF/DUNE Quality Assurance Plan.
The Consortia implement the plan through the development of individual
quality plans, procedures, guides, QC inspection and test requirements
and travelers/test reports.
In lieu of a Consortia Specific
Quality Plan, the Consortia may work under the LBNF/DUNE Quality
Assurance Plan and develop Manufacturing/QC Plans, procedures and
documentation specific to their work scope.  The DUNE Technical
Coordinator and Consortia Leaders are responsible for providing the
resources needed to conduct the Project successfully, including those
required to manage, perform and verify work that affects quality.  The
DUNE Consortia Leaders are responsible for identifying adequate
resources to complete the work scope and to ensure that their team
members are adequately trained and qualified to perform their assigned
work.

The Consortia work will be documented on travelers and applicable test
or inspection reports. Records of the fabrication, inspection and
testing will be maintained. When a component has been identified as
being in noncompliance to the design, the nonconforming condition
shall be documented, evaluated and dispositioned as use-as-is (does
not meet design but can meet functionality as is), rework (bring into
compliance with design), repair (will be brought into meeting
functionality but will not meet design) and scrap.

The LBNF/DUNE Quality Assurance Manager (QAM) reports to the LBNF Project
Manager and DUNE Technical Coordinator and provides oversight and
support to the Consortia Leaders to ensure a consistent quality
program.
\begin{enumerate}
  \item The QAM will plan reviews as independent assessments to assist
    the DUNE Technical Coordinator in identifying opportunities for
    quality/performance-based improvement and to ensure compliance
    with specified requirements.
  \item The QAM is responsible to work with the Consortia in
    developing their QA/QC Plans.
  \item The QAM will review Consortia QA/QC activity, including
    production site visits.
  \item The QAM will participate in Consortia Design Reviews, conduct
    Production Readiness Reviews prior to the start of production,
    conduct Production Progress Reviews on a regular basis, and
    perform follow-up visits to Consortia facilities prior to shipment
    of components to ensure all components and documentation are
    satisfactory.
\item The QAM is responsible for performing assessments at the
  Integration Facility, the Far Site and the Near Site to
  ensure the activities performed at these locations are in accordance
  with the LBNF/DUNE QA Program and applicable procedures,
  specifications and drawings.
\end{enumerate}

%%%%%%%%%%%%%%%%%%%%%%%%%%%%%%%%
\subsection{ES\&H}
\label{sec:fdsp-coord-esh}

The DUNE Environmental, Safety and Health (ESH) program is described
in the LBNF/DUNE Integrated Environmental, Safety and Health
Plan. This plan is maintained by the LBNF/DUNE ESH Manager, who
reports to the LBNF Project Manager and the TC. The ESH is responsible to work with the Consortia
in reviewing their hazards and their ESH plans.  The ESH Manager is
responsible to review ESH at production sites, integration sites and
at SURF.
