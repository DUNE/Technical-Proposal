%%%%%%%%%%%%%%%%%%%%%%%%%%%%%%%%%%%%%%%%%%%%%%%%%%%%%%%%%%%%%%%%%%%%
\section{Interfaces}
\label{sec:fdgen-slow-cryo-intfc}
% anselmo

% Include an image of each interface in appropriate section.  (?????)
% Good idea for TDR, but let's not do this for the TechPro, due to space. [gahs]

The \dword{cisc} consortium interfaces with all other consortia, task forces (calibration), working groups (physics, software/computing) and technical coordination.
This section provides a brief summary.\footnote{ 
 \cite{bib:docdb6745},
\cite{bib:docdb6991},
\cite{bib:docdb6790},
\cite{bib:docdb6787},
\cite{bib:docdb6784},
\cite{bib:docdb6781},
\cite{bib:docdb6760},
\cite{bib:docdb6679},
\cite{bib:docdb6730},
\cite{bib:docdb7126},
\cite{bib:docdb7099},
\cite{bib:docdb7072},
\cite{bib:docdb7045},
\cite{bib:docdb7018}.}  

There are obvious iterfaces with detector consortia since \dword{cisc} provides full rack monitoring (rack fans, thermometers and rack protection system),
interlock status bit monitoring (not the actual interlock mechanism) and monitoring and control for all \pwrsupps. The  \dword{cisc} consortium must maintain close contacts with all other consortia to ensure that specific hardware choices have acceptable slow controls (SC)  solutions.  
Also, installation of instrumentation devices interferes with other devices and must be coordinated with the respective consortia.  
On the software side \dword{cisc} must define, in coordination with other consortia, the quantities to be monitored and controlled by slow controls and the corresponding alarms,
archiving and GUIs. 

%Rack space distribution and interaction between slow controls (SC) modules and other modules (\dwords{apa}, \dword{hv}, PD, \dword{daq}, etc)

%SC signals into main \dword{daq} data stream and viceversa

A major interface is the one with the cryogenics system. As mentioned in Section~\ref{sec:fdgen-slow-cryo-purity-mon} purity monitors and gas analyzers are essential
to mitigate the liquid argon contamination risk. The appropriate interlock mechanism to prevent the cryonenics system from irreversible contamination
must be designed and implemented. 

Another important interface is the one with the \dword{hv} system \cite{bib:docdb6787} since several aspects related with safety must be taken into account. 
For all instrumentation devices inside the cryostat, \efield simulations are needed to guaranty proper shielding is in place.
Although this is a \dword{cisc} responsibility, input from \dword{hv} is crucial.
%The location of cold cameras and lights for inspection of \dword{hv} related devices,
%as well as the requirements of cold/warm cameras (see Sec.~\ref{sec:fdgen-slow-cryo-cameras}) have to be understood
%in cooperation with \dword{hv} consortium.
%Special software may be needed to detect discharges using the camera system,
%which will be the responsibility of the \dword{hv} consortium.
During the deployment of inspection cameras, generation of bubbles must be avoided when \dword{hv} is on, as it can lead to discharges.
%Finally, ground planes will be used as support for temperature sensors. The integration of the two systems must be understood. 

There are also interfaces with the \dword{pds}~\cite{bib:docdb6730}. Purity monitors and the light-emitting system for cameras both emit light that might damage \dwords{pd}.
Although this should be understood and quantified, \dword{cisc} and the \single \dword{pds} may have to define the necessary hardware interlocks
that avoid turning on any other light source accidentally when \dwords{pd} are on.

The \dword{daq}-\dword{cisc} interface~\cite{bib:docdb6790} is described in Section~\ref{sec:fd-daq-intfc-sc}. %As discribed there, 
\dword{cisc} data is stored both locally (in \dword{cisc} database servers in the
\dword{cuc}) and offline (the databases are replicated back to \fnal)
in a relational database indexed by timestamp.
This allows bidirectional communications between \dword{daq} and \dword{cisc} by
reading or inserting data into the database as needed for non-time-critical information.  


%\fixme{Does this x-ref work?  The cut-n-paste below (commented out) from
%  the official interfaces document seems overkill.  Also, there are not
%  currently plans for a \dword{daq} test stand at SURF.  The \dword{daq} as a whole is
%  installed before \dwords{apa} arrive, then the actual \dword{daq} used to commission
%  the \dwords{apa} as they go in.  If the there is a slow controls test facility
%  at SURF, the same software interfaces could be used as for the final
%  form, nothing new needs created.}


%\dword{daq} status.  

\dword{cisc} also interfaces with the beam and cryogenics group since at least the status of these systems must be monitored.
%The interface with the facility: .... Building controls; Detector hall monitoring; ground impedance monitoring

%  Power distribution units monitoring; Computer hardware monitoring


Assuming that the scope of software \& computing \dword{swc} group includes scientific computing support to project activities, there are substancial interfaces with that group~\cite{bib:docdb7126}. 
The hardware interfaces resposibility of the \dword{swc} include networking installation and maintenance,
maintenance of SC servers  and any additional computing hardware needed by instrumentation devices.
\dword{cisc} provides the needed monitoring for power distribution units (PDUs). Regarding software interfaces the \dword{swc} group  provides:
(1) SC database maintenance, (2) API for accessing the SC database offline,
(3) UPS packages, local installation and maintenace of software needed by \dword{cisc}, and (4) \dword{swc} creating and maintaining computer accounts on production clusters. 
Additionally  \dword{cisc} provides the required monitoring and control of \dword{swc} quantities including alarms, archiving, and GUIs, where applicable. 


\dword{cisc} has profound hardware interfaces with the not-as-yet formed calibration consortium \cite{bib:docdb7072}. Indeed, since \dword{cisc} and CTF ports are multi-purpose to enable deploying various devices,
both systems must interact in terms of flange design and sharing space around the ports. Also, \dword{cisc} might use calibration ports to extract cables from \dword{cisc} devices. 
At the software level, \dword{cisc} is responsible for calibration device monitoring (and control to the extent needed) and 
monitors the interlock bit status for laser and radioactive sources. 
%Understand the location of Cold cameras & lights for inspection of CFT related devices, as well as the
%requirements of Cold/Warm cameras: resolution, field of view, light sensitivity, low light operation, frames per second, operation in triggered mode?  etc.

\dword{cisc} indirectly interfaces to physics through the CTF devices. One specific need for physics is to extract
instrumentation or slow controls data to correlate high-level quantities to low-level or calibration data.
This requires tools to extract data from the slow controls database (see \dword{cisc}-\dword{swc} interface document \cite{bib:docdb7126}).
A brief list of what \dword{cisc} data is needed by physics is given in the \dword{cisc}-Physics interface document \cite{bib:docdb7099}. 

Interfaces between \dword{cisc} and technical coordination are detailed in the corresponding interface documents for the facility \cite{bib:docdb6991}, installation \cite{bib:docdb7018}
and integration facility~\cite{bib:docdb7045}.

\fixme{Interfaces with technical coordination has to be expanded}

%\fixme{specify external interface of Cryo Inst. Systems with systems outside the cryostat (with LBNF), detector Interface to LBNF design teams working on the design on cryogenic systems (including cryogenic piping), The switchyard for the gas analyzers ... }


%Interfaces with the Facility are: Location of instrumentation devices, anchoring points, cryostat ports and flanges,
%space needed above cryostat, cable routing, rack space above cryostat, Mezzanine and CUC (Central Utility Cavern), etc 


%\fixme{Describe interface with \dword{daq} system, including Interface with
%  \dword{daq}/Electronics groups for a slow controls test facility at SURF,
%  possibly as part of the \dword{daq} test stand.}  







%%%%%%%%%%%%%%%%%%%%%%%%%%%%%%%%%%%
%\subsection{Interface with Environmental and Building Controls}
%\label{sec:fdgen-slow-cryo-slow-enviro}

%\fixme{describe interface with LBNF on environmental and building controls}

%Building Temperature, humidity and pressure will be monitored and integrated into the slow controls system. 


