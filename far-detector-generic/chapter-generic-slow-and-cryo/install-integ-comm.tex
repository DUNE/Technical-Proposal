%%%%%%%%%%%%%%%%%%%%%%%%%%%%%%%%%%%%%%%%%%%%%%%%%%%%%%%%%%%%%%%%%%%%
\section{Installation, Integration and Commissioning}
\label{sec:fdsp-slow-cryo-install}    % for references from sp volume
\label{sec:fddp-slow-cryo-install} % for references from dp volume
\label{sec:fdgen-slow-cryo-install} % for citations from far-detector-generic/
% anselmo, sowjanya

%Then, the installation of cryogenics instrumentation devices inside the cryostat will be done in several phases:
%\begin{enumerate}
%\item Before detector installation: Individual temperature sensors anchored to the cryogenic pipes at the bottom of the cryostat and static T-gradient monitors
%  will be installed right after the installation of the pipes. Additional level meters could be also installed during this phase. 
%\item During detector installation: Temperature sensors anchored to the top ground planes will be installed in several steps.
%  In principle as soon as a \dword{cpa} enters the cryostat and stored there, the corresponding cables and sensors can be mounted on its ground planes. Cables exceeding the dimensions of
%  the ground planes will be roled and stored temperarilly there until they are put in their final position. At this moment cables will be routed towards their ports.   
%\item After detector installation: dynamic T-gradient monitors and purity monitors will be deployed  once the detector in installed. 
%\end{enumerate}

%Since purity monitors will be most likely not hang from a flange they will have to be installed at an earlier stange ... 


%Special treatment need the long devices as the purity monitor system, and T-gradient monitors.

%The installation of gas analyzers will be inst


%Integration: Initial integration between the two systems will be accomplished at vertical slice and \dword{daq} test stands:
%test of cameras for the detection of sparks, testing SC monitoring/control of \dword{hv} PS, testing \dword{hv} interlock status bits, etc.
%Integration/installation of \dword{cisc} devices (cameras, RTDs) on ground planes will be tested at system integration/assembly sites. Final integration will take place as the two systems are commissioned.


%Some parts of the \dword{cisc} system will be commissioned prior to \dword{hv} commissioning, and before cryostat filling with \lar{}.
%This is the case for RTDs on GPs, cold and inspection cameras and thermal interlocks for PS. Final commissioning of
%those systems will be done once the cryostat is filled, since operation will be different in the presence of \lar{}.
%Commissioning the control/monitoring of \dword{hv} PS and any related hardware interlocks could probably be done at an early
%stage as well, provided no real \dword{hv} is provided to the cathode/field-cages. Final commissioning will be done once \dword{hv} is
%switched on. This will also require coordination from other groups such as LBNF, \dword{daq}, \dword{apa} etc. The commissioning of the
%interfacing elements should follow naturally after (successful) integration testing.

\subsection{Cryogenics Internal Piping}
\label{sec:fdgen-slow-cryo-install-pipes}


The installation of internal cryogenic pipes occurs soon after the cryostat is completed or towards the end of the cryostat completion, depending on how the cryostat work proceeds. A concrete installation plan will be developed by the company designing the internal cryogenics. It depends on how they address the thermal contraction of the long horizontal and vertical runs. We are investigating several options, which each have different installation sequences. %In general though, prefabricated spool pieces (as long as allowed) will be delivered and welded together inside the cryostat. The vertical lines will be vacuum insulated, whereas the horizontal ones will just be bare pipes.
All involve delivery and welding together of prefabricated spool pieces inside the cryostat, and vacuum insulation of the vertical lines. The horizontal lines are bare pipes. 

The cool-down assemblies are installed in dedicated cool-down \fdth{}s at the top, arranged in
%. There are 
two rows of ten each in the long direction of the cryostat. Each one features a \lar line connected to a gaseous argon line via a mixing nozzle and a gaseous argon line with spraying nozzles. The mixing nozzles generate droplets of liquid that are circulated uniformly inside the cryostat by the spraying nozzles. They are prefabricated at the vendor's site and delivered as full pieces, %which will 
then mounted over the \fdth{}s.

The current \threed model of the internal cryogenics is developed and archived at CERN as part of the full cryostat model. CERN is currently responsible for the integration of the detector cavern: cryostat, detector, and proximity cryogenics in the detector cavern, including cryogenics on the mezzanine and main \lar circulation pumps.

The prefabricated spool pieces and the cool down nozzles undergo testing at the vendor  before delivery. The installed pieces are helium leak-checked before commissioning, but no other integrated testing or commissioning is possible after the installation, because the pipes are open to the cryostat volume. The internal cryogenics are commissioned once the cryostat is closed.


\subsection{Purity Monitors}
\label{sec:fdgen-slow-cryo-instal-pm}

The purity monitor system is built in a modular way, such that is can be assembled outside of \dword{detmodule} cryostat.  The assembly of the purity monitors themselves occurs outside of the cryostat and includes everything described in the previous section.  The installation of the purity monitor system can then be carried out with the least number of steps inside the cryostat.  The assembly itself is transported into the cryostat with the three individual purity monitors mounted to the support tubes but before installation of \dword{hv} cables and optical fibers. The support tube at the top and bottom of the assembly is then mounted to the brackets inside the cryostat that could be attached to the cables trays or the detector support structure. \fixme{brackets attached to trays or DSS depending on SP vs DP?} In parallel to this work, the \dword{fe} electronics and light source can be installed on the top of the cryostat, along with the installation of the electronics and power supplies into the electronics rack.  

Integration begins by running the \dword{hv} cables and optical fibers to the purity monitors, coming from the top of the cryostat.  The \dword{hv} cables are attached to the \dword{hv} \fdth{}s with enough length to reach each of the respective purity monitors.  The cables are run through the port reserved for the purity monitor system, along cable trays inside the cryostat until they reach the purity monitor system, and are terminated through the support tube down to each of the purity monitors.  Each purity monitor has three \dword{hv} cables that connect it to the \fdth, and then along to the \dword{fe} electronics.  The optical fibers are run through the special optical fiber \fdth, into the cryostat, and guided to the purity monitor system either using the cables trays or guide tubes.  Whichever solution is adopted for running the optical fibers from the \fdth to the purity monitor system, it must protect the fibers from accidental breakage during the remainder of the detector and instrumentation installation process.  The optical fibers are then run inside of the purity monitor support tube and to the respective purity monitors,  terminating at the photocathode of each. %, protecting them from breakage near the purity monitor system itself.

Integration  continues with the connection of the \dword{hv} cables between the \fdth and the system \dword{fe} electronics, and then the optical fibers to the light source.  The cables connecting the \dword{fe} electronics and the light source to the electronics rack are also run and connected at this point.  This allows for the system to turn on and the software to begin testing the various components and connections.  Once it is confirmed that all connections are successfully made, the integration to the slow controls system is made, first by establishing communications between the two systems and then transferring data between them to ensure successful exchange of important system parameters and measurements.  

The purity monitor system is formally commissioned %would officially be done 
once the cryostat is purged and a gaseous argon atmosphere is present.  At this point the \dword{hv} for the purity monitors is ramped up without the risk of discharge through the air, and the light source is turned on.  Although the drift electron lifetime in the gaseous argon is very large and therefore not measurable with the purity monitors themselves, comparing the signal strength at the cathode and anode gives a good indication of how well the light source is generating drift electrons from the photocathode and whether they drift successfully to the anode. % by looking at the signal strength at the anode and comparing it to that of the cathode.


\subsection{Thermometers}
\label{sec:fdgen-slow-cryo-instal-th}


Individual temperature sensors on pipes and cryostat membrane are installed prior to any detector component, right after the installation of the pipes.
First, all cable supports are anchored to pipes. Then each cable is routed individually starting from the sensor end (with IDC-4 female connector but no sensor)
towards the corresponding cryostat port. Once a port's cables are routed, %all cables going to the same port have been routed, 
they are cut to the same length such that they can be properly soldered
to the pins of the SUBD-25 connectors on the flange. 
To avoid damage, the sensors are installed at a later stage, just before unfolding the bottom \dwords{gp}.

Static T-gradient monitors are installed before the outer \dwords{apa}, e.g., %. Thus, the best moment could be right 
after the installation of the pipes
and before the installation of individual sensors. This proceeds in several steps: (1) installation of the two stainless steel strings to the bottom and top corners of the cryostat,
(2) tension and verticality checks, (3) installation of cable supports in one of the strings, (4) installation of sensor supports in the other string, (5) cable routing starting from
the sensor end towards the corresponding cryostat port, (6) cutting all cables at the same point in that port, and (7) soldering cable wires to the pins of the SUBD-25 connectors on the flange. Then, at a later stage, just before moving corresponding \dword{apa} into its final position, (8) the sensors are plugged into IDC-4 connectors. 

For the \single{}, individual sensors on the top \dword{gp} must be integrated with the \dwords{gp}. For each \dshort{cpa} (with its corresponding four \dshort{gp} modules)
going inside the cryostat, cable and sensor supports are anchored to the \dshort{gp} threaded rods as soon as possible.
Once the \dshort{cpa} is moved into its final position and its top \dword{fc} is ready to be unfolded, sensors on those \dwords{gp} are installed. Once unfolded, cables 
exceeding the \dshort{gp} limits can be routed to the corresponding cryostat port either using neighboring \dwords{gp} or \dshort{dss} I-beams. 
\fixme{prev pgraph needs work}

Dynamic T-gradient monitors are installed after the completion of the detector.
The monitor comes in several segments with sensors and cabling already
in place. Additional slack is provided at segment joints to ease the
installation process. Segments are fed into the flange one at the
time. The segments being fed into the \dword{detmodule} are held at the top
with a pin that prevents the segment from sliding in all the way. Then the next
segment is connected. The pin is removed, and the
segment is pushed down until the next segment top is held with the
pin at the flange. Then this next segment is installed. The
process  continues until the entire monitor is in its place
inside the cryostat. Use of a crane is foreseen to facilitate the process.
Extra cable slack at the top is provided again in order to ease  the
connection to the D-Sub standard connector flange and to allow  vertical movement of the
entire system. Then,  a four-way cross is installed with flange electric \fdth{}s on
one side and a window on the other side. 
\fixme{window, I presume? Also, sentence missing verb} 
The wires are connected to
the D-sub connector on the electric flange \fdth on the side. On the
top of the cross, a moving mechanism is then installed with a crane.
The pinion is connected to the top segment. The moving mechanism will
come reassembled with motor on the side in place and pinion and gear
motion mechanism in place as well. The moving mechanism enclosure  is then connected to top part of the cross and this completes the
installation process of the dynamic T-gradient monitor.
\fixme{prev pgraph needs work}

Commissioning of all thermometers proceeds in several steps. Since in a first stage only cables are installed,
the readout performance and the noise level inside the cryostat are
tested with precision resistors. Once sensors are installed the entire chain is checked again at room temperature.
The final commissioning phase occurs during and after cryostat filling.  


\subsection{Gas Analyzers}
\label{sec:fdgen-slow-cryo-install-ga}
 
Prior to the piston purge and gas recirculation phases of the cryostat commissioning, the gas analyzers are installed near the %location of the 
tubing switchyard. This minimizes tubing runs and is  convenient for switching the sampling points and gas analyzers. Since each is a standalone module, a single rack with shelves, should be adequate to house the modules.

Concerning the integration, the gas analyzers typically have an analog output (\numrange{4}{20} \si{mA} or \numrange{0}{10}\si{V}) that maps to the input range of the analyzers. They also usually have a number of relays that indicate the scale they are currently running. These outputs can be connected to the slow controls for readout. However, using a digital readout is preferred since this directly gives the analyzer reading at any scale. Currently there are a number of digital output connections, ranging from RS-232, RS-485, USB, and Ethernet. At the time of purchase, one can choose the preferred option, since the protocol is likely to evolve. The readout usually responds to a simple set of text commands. Due to the natural time scales of the gas analyzers, and lags in the gas delivery times (depending on the length of the tubing runs), sampling at the minute level most likely is adequate. \fixme{``sampling on timescales of a minute...'' otherwise `minute' may strike reader as pronounced ``minoot''}

%For commissioning, 
Before the beginning of the gas phase of the cryostat commissioning, the analyzers must be brought online and calibrated. Calibration varies for the different modules, but often requires using argon gas with both zero contaminants (usually removed with a local inline filter) for the zero of the analyzer, and argon with a known level of the contaminant to check the scale. Since the start of the gas phase  begins with normal air, the more sensitive analyzers are valved off at the switchyard to prevent overloading their inputs and potentially saturating their detectors. As the argon purge and gas recirculation progress, the various analyzers are valved back in when the contaminant levels reach the upper limits of the analyzer ranges. 

\subsection{Liquid Level Monitoring}
\label{sec:fdgen-slow-cryo-install-llm}

% % from John L:
% The pressure sensitive level monitors will be installed with the general
% instrumentation along the walls of the cryostat. Multiple sensors at the
% same depth along with multiple routing is needed to insure reliability.
% Post installation in situ testing of pressure sensors is still an open question.
% 
% The surface level monitoring with thermometers and capacitive level sensors may
% need to wait until the field cage and \dword{apa}'s are installed. Some redundancy and
% multiple routing are needed since these must be reliable for the duration of
% the experiment. Post installation in situ testing of these types of sensors
% can be accomplished with small dewars of cryogenic liquids, such as perhaps
% liquid nitrogen.
% 
% % from David M:
% Redundant differential pressure level meter will be installed. They
% will be connected to the side penetration (at the bottom) and
% dedicated instrumentation ports at the top. They will have a
% sensitivity of \SI{0.1}{\percent} over the full range (14.0 m). A
% capacitance level meter will also be installed. It will be used as
% hardware interface with the slow control to interlock high voltage,
% etc. based on the liquid argon level. It will measure then nominal
% value (currently 13.5 m -0.6 / + 0.4 m).

% Merge
Multiple differential pressure level monitors are installed in the
cryostat, connected both to the side
penetration of the cryostat at the bottom and to dedicated
instrumentation ports at the top.
% They will have a sensitivity of \SI{0.1}{\percent} over the full range (14.0 m). %<<< This is already in the design section, omit here to avoid having numbers in too many places.
% Post installation in situ testing of pressure sensors is still an open question. %<<< Move this to QC?

The capacitance level sensors are installed at the top of the
cryostat in coordination with the \dword{tpc} installation.  Their
placement relative to the upper ground plane (single phase) or
\dshort{crp} (dual phase) is important as these sensors will be used for a
hardware interlock on the \dword{hv}, and, in the case of the \dword{dpmod}, to measure the \lar level at the millimeter level as required
for \dual operation.
% It will measure then nominal value (currently 13.5 m -0.6 / + 0.4 m). <<< this already stated in the level meter design -- omit here to avoid having numbers in too many places
Post installation in situ testing of the capacitive level sensors can be
accomplished with a small dewar of liquid.


\subsection{Cameras and Light-Emitting System}
\label{sec:fdgen-slow-cryo-install-c}

Fixed camera installation is in principle simple, but involves a
considerable number of interfaces. Each camera enclosure has 
threaded holes to allow bolting it to a bracket. A mechanical
interface is required with the cryostat wall, cryogenic internal
piping, or \dword{dss}. Each enclosure is attached
to a gas line for maintaining appropriate underpressure in the fill
gas; this is an interface with cryogenic internal piping. Each camera has a
cable for the video signal (coax or optical), and a multiconductor
cable for power and control, to be run through cable trays to flanges
on assigned instrumentation \fdth{}s.

The inspection camera is designed to be inserted and removed on any
instrumentation \fdth equipped with a gate valve at any time
during operation.  Installation of the gate valves and purge system
for instrumentation \fdth{}s falls under cryogenic internal
piping.

Installation of fixed lighting sources separate from the cameras would
require similar interfaces as fixed cameras.  However, the current
design has lights integrated with the cameras, which do not require separate
installation.



\subsection{Slow Controls Hardware}
\label{sec:fdgen-slow-cryo-install-sc-hard}

Slow controls hardware installation includes installing multiple
servers, network cables, any specialized cables needed
for device communication, and possibly some custom-built rack
monitoring hardware. The installation sequence is interfaced and
planned with the facilities group and other consortia. The network
cables and rack monitoring hardware are common across many racks
and are installed first as part of the basic rack installation, 
led by the facilities group. The installation of
specialized cables needed for slow controls and servers is done
after the common rack hardware is installed, and will be coordinated
with other consortia and the \dword{daq} group respectively.

%%%%%%%%%%%%%%%%%%%%%%%%%%%%%%%%%%%%
\subsection{Transport, Handling and Storage}
\label{sec:fdgen-slow-cryo-install-transport}

Most instrumentation devices are shipped to \surf in pieces and mounted in situ. 
Instrumentation devices are in general small except the support structures for purity monitors and T-gradient monitors,
which will cover the entire height of the cryostat. Since the load on those structures is relatively small
 (\(<\SI{100}{kg}\)) they can be fabricated in parts of less than \SI{3}{m},
which can be easily transported to \surf. These parts are also easy to transport down the shaft and through the tunnels.
All instrumentation devices except the dynamic T-gradient monitors, which are introduced into the cryostat through a dedicated cryostat port, %above, 
\fixme{above something?}
can be
moved into the cryostat without the crane.

%We are talking about almost , they will take some space.
Cryogenic internal piping needs special treatment given the number of pipes and their lengths.
%There will be 750 m of pipes inside each cryostat (300 m of 3'' pipe and almost 450 m of 2'' pipes).
Purging and filling pipes will be most likely pre-assembled by the manufacturer as much as possible, using the largest  
size that can be shipped and transported down the shaft. Assuming \SI{6}{m} long sections,
pipes could be grouped in bunches of \numrange{10}{15} pipes and stored in five pallets or boxes of about \SI{6.2}{m} $\times$ \SI{0.8}{m} $\times$ \SI{0.5}{m}. 
%(assuming 10 sections of the 3” and 15 of the 2”).
These would be delivered to the site, stored, transported down to the detector cavern,
 and stored again before they are used.
Depending on when they are installed, they could be stored inside the cryostat itself or in one of the drifts. 
Cool-down pipes are easier to handle. They could be transported in \num{20} boxes of \SI{2.2}{m} $\times$ \SI{0.6}{m} $\times$ \SI{0.6}{m}, although
there is room for saving some space using a different packaging scheme. 
%They could also put many of them in the same box and save space, but they will soon run into shaft limits.
%These would need to be delivered to the site, stored, transported down, and stored again before they are used.
Once in the cavern they could be stored on top of the cryostat.



%%%%%%%%%%%%%%%%%%%%%%%%%%%%%%%%%%%
%\subsection{Integration Facility Operations}
%\label{sec:fdgen-slow-cryo-install-facil-ops}


%%%%%%%%%%%%%%%%%%%%%%%%%%%%%%%%%%%
%\subsection{Underground Operations}
%\label{sec:fdgen-slow-cryo-install-undergr}


%%%%%%%%%%%%%%%%%%%%%%%%%%%%%%%%%%%
%\subsection{Integration }
%\label{sec:fdgen-slow-cryo-install-integration}

%Slow controls needs to be integrated with \dword{hv} devices as soon as possible 


%%%%%%%%%%%%%%%%%%%%%%%%%%%%%%%%%%%
%\subsection{Commissioning}
%\label{sec:fdgen-slow-cryo-install-commiss}


