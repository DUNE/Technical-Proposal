%%%%%%%%%%%%%%%%%%%%%%%%%%%%%%%%%%%%%%%%%%%%%%%%%%%%%%%%%%%%%%%%%%%%
\section{Quality Control}
\label{sec:fdgen-slow-cryo-qc}
% carmen
The purpose of  \dword{qc} is to ensure that the equipment is capable of performing its intended function. The \dword{qc} includes post-fabrication tests and also tests to run after shipping and installation. A series of tests should be done by the manufacturer and the institute in charge of the device assembly. In case of a complex system, the whole system performance will be tested before shipping. 
Additional \dword{qc} procedures can be performed at the \dword{itf} and underground after installation if possible. The planned tests for each subsystem are described below.  
\fixme{the organization of prev pgraph is confusing}

\subsection{Purity Monitors}
\label{sec:fdgen-slow-cryo-qc-pm}

The purity monitor system undergoes a series of tests to ensure the performance of the system.  This  starts with testing the individual purity monitors in vacuum after each one is fabricated and assembled.  This test looks at the amplitude of the signal generated by the drift electrons at the cathode and the anode.  This ensures that the photocathode is able to provide a sufficient number of photoelectrons for the measurement %to be made 
with the required precision, and that the field gradient resistors are all working properly to maintain the drift field and hence transport the drift electrons to the anode.  A follow-up test in \lar is then performed for each individual purity monitor, ensuring that the performance expected in \lar is met.  

The next step %after individual testing would be 
is to assemble the entire system and make checks of the connections along the way.  Ensuring that the connections are all proper during this time reduces the risk of having issues once the system is finally assembled and ready for the final test.  %With the full system assembled it would be 
The assembled system is placed into the shipping tube, which serves as a vacuum chamber, and tested. % and a test made with the system in vacuum.  
%This would ensure that the performance seen during the individual purity monitors tests can still be achieved after making the final assembly.  
Next, assuming an adequate \lar test facility is available, a test at \lar temperature is made to ensure the required performance. 
\fixme{and what if no \lar facility exists there? Orig sentence: If there is a \lar test facility with the height or length required for the full purity monitor system and it is available for use, then a final full system test would be made there ensuring that the full system operates in \lar and achieves the required performance.}

%CFT?

\subsection{Thermometers}
\label{sec:fdgen-slow-cryo-qc-th}

\subsubsection{Static T-Gradient Thermometers}
\label{sec:fdgen-slow-cryo-qc-thst}


Three type of tests are carried out at the production site prior to installation. First, the mechanical rigidity of the system is tested such that swinging is minimized (< \SI{5}{cm})
to reduce the risk of touching the \dwords{apa}. This is done with a \SI{15}{m} stainless steel string, strung horizontal anchored to two points; its tension is controlled and measured. 
Second, %the quality of each sensor and its calibration should be understood. A
all sensors are calibrated in the lab, as explained in Section~\ref{sec:fdgen-slow-cryo-therm}.
The main concern is the reproducibility of the results since sensors could potentially change their resistance (and hence their temperature scale)
when undergoing successive immersions in \lar{}. In this case the \dword{qc} is given by the calibration procedure itself since five independent measurements
are planned for each set of sensors. Sensors with reproducibility (based on the \rms of those five measurements) beyond the requirements (\SI{2}{mK} for \dword{pdsp}) are discarded.  
The calibration serves as \dword{qc} for the readout system (similar to the final one) and of the PCB-sensor-connector assembly. Finally, the cable-connector assemblies are tested: sensors must measure the expected values with no additional noise introduced by the cable or connector. 

%If there is a \lar test facility with sufficient height or length to test a good portion of the system and it is available for use,
If the available \lar test facility has sufficient height or length to test a good portion of the system, an integrated system test is conducted there ensuring that the system
operates in \lar and achieves the required performance. Ideally, the laboratory sensor calibration will be compared with the in situ calibration
of the dynamic T-gradient monitors by operating both dynamic and static T-gradient monitors simultaneously.   

The last phase of \dword{qc} takes place after installation. %For each of the arrays being installed the verticality of the system can be checked and the tension of the stainless steel strings can be adjusted to avoid lateral swinging towards the \dwords{apa}. 
The verticality of each array is checked and the tensions in the horizontal strings are adjusted as necessary.
Before soldering the wires to the flange, the entire readout chain is tested with temporary SUBD-25 connectors. 
This allows testing the sensor-connector assembly, the cable-connector assembly and the noise level inside the cryostat.
If any of the sensors gives a problem, it is replaced. If the problem persists, the cable is checked and replaced if needed.


\subsubsection{Dynamic T-Gradient Thermometers}
\label{sec:fdgen-slow-cryo-qc-thdy}

The dynamic T-gradient monitor consists of an array of high-precision temperature sensors mounted on a vertical rod. The rod can move vertically in order to perform cross-calibration of the temperature sensors in situ. Several tests are foreseen to ensure that the dynamic T-gradient monitor delivers vertical temperature gradient measurements with precision at the level of a few \si{mK}.

\begin{itemize}
\item
Before installation, temperature sensors are tested in LN to verify correct operation and to set the baseline calibration for each sensor with respect to the absolutely caibrated reference sensor. 
\item
Warm and cold temperature readings are taken with each sensor after mounting on the PCB board and soldering %of 
the readout cables.
\item
The sensor readout is taken for all sensors after the cold cables are connected to electric \fdth{}s on the flange and the warm cables outside of the cryostat are connected to the temperature readout system.
\item 
The stepper motor is tested before and after connecting to the gear and pinion system.
\item
The fully assembled rod is connected to the pinion and gear, and moved with the stepper motor on a high platform many times to verify repeatability, possible offsets and uncertainty in the positioning. Finally, by repeating the test a large number of times, the sturdiness of the system will be verified.
\item
The full system is tested after installation in the cryostat: both motion and sensor operation are tested by checking % readout for sensors and motion of the system vertically. 
sensor readout and vertical motion of the system.
\end{itemize} 

\subsubsection{Individual Sensors}
\label{sec:fdgen-slow-cryo-qc-is}

The method to address the quality of individual precision sensors is the same as for the static T-gradient monitors.
The \dword{qc} of the sensors is part of the laboratory calibration. After mounting six sensors with their corresponding cables, a
temporary SUBD-25 connector will be added and the six sensors tested at room temperature. All sensors should work and give values within specifications.  
If any of the sensors gives problems, it is replaced.  If the problem persists the cable is checked and replaced if needed.

\subsection{Gas Analyzers}
\label{sec:fdgen-slow-cryo-qc-ga}

The gas analyzers will be guaranteed by the manufacturer. However, once received, the gas analyzer modules are checked for both \textit{zero} and the \textit{span} values using a gas-mixing instrument. This is done using two gas cylinders with both a zero level of the gas analyzer contaminant species and a cylinder with a known percentage of the contaminant gas. This should verify the proper operation of the gas analyzers. When eventually installed at \surf, this process is repeated before the commissioning of the cryostat. It is also important to repeat the calibrations at the manufacturer-recommended periods over the gas analyzer lifetime.


\subsection{Liquid Level Monitoring}
\label{sec:fdgen-slow-cryo-qc-llm}

The differential pressure level meters will require \dword{qc} by the manufacturer.
While the capacitive and thermal sensors can be tested with a modest sample of \lar in the lab,
the differential pressure level meters require testing over a greater range.  While they do not
require testing over the whole range,  lab tests  in \lar 
done over a meter or two can ensure operation
at cryogenic temperatures.  Depth tests can be accomplished using a
pressurization chamber with water.

%The pressure sensors will be purchased requiring a \dword{qc} by the manufacturer.
%While the capacitive and thermal sensors can be tested with a modest sample of \lar in the lab, the pressure sensors require testing over a greater range.  They do not necessarily need to be tested in \lar over the whole range,  lab tests can be done in \lar over a range of a meter or two to ensure operation at cryogenic temperatures.  Depth tests can be accomplished with a pressurization chamber with water.

\subsection{Cameras}
\label{sec:fdgen-slow-cryo-qc-c}

Before transport to \surf, each cryogenic camera unit (comprising the enclosure, camera, and internal thermal control and monitoring) is checked for correct operation of all operating features, for recovery from \SI{87}{K} non-operating mode, for no leakage, and for physical defects. Lighting systems are similarly checked for operation. Operations tests will include verification of correct current draw, image quality, and temperature readback and control. The movable inspection camera apparatus is inspected for physical defects, and checked for proper mechanical operation before shipping. A checklist is completed for each unit, filed electronically in the DUNE logbook, and a hard copy sent with each unit. 

Before installation, each fixed cryogenic camera unit is inspected for physical damage or defects and checked in the cryogenics test facility  for correct operation of all operating features, for recovery from \SI{87}{K} non-operating mode, and for no contamination of the \lar{}. Lighting systems are similarly checked for operation. Operations tests include correct current draw, image quality, and temperature readback and control. After installation and connection of wiring, fixed cameras and lighting are again  checked for operation. The movable inspection camera apparatus is inspected for physical defects and, after integration with a camera unit, tested in facility for proper mechanical and electronic operation and cleanliness, before installation or storage. A checklist is completed for each \dword{qc} check and filed electronically in the DUNE logbook. 

\subsection{Light-emitting System}
\label{sec:fdgen-slow-cryo-qc-les}

The complete system is checked before installation to ensure the functionality of the light emission. 
\fixme{to ensure it meets requirements?}
Initial testing of the light-emitting system (see Figure~\ref{fig:gen-cisc-LED}) is done by first
measuring the current when a low voltage (\SI{1}{V}) is applied, to check
that the resistive \dword{led} failover path is correct. Next, measurement
of the forward voltage is done with the nominal forward current applied, to
check that it is within \SI{10}{\%} of the nominal forward voltage drop of
the \dwords{led}, that all of the \dwords{led} are illuminated, and that each of the
\dwords{led} is visible over the nominal angular range. If the \dwords{led} are
infrared, a video camera with IR filter removed is used for the
visual check. This procedure is then duplicated with the current
reversed for the \dwords{led} oriented in the opposite direction.  

These tests are duplicated during
installation to make sure that the system has
not been damaged in transportation or installation. However, once
the \dwords{led} are in the cryostat a visual check could be difficult or impossible.


\subsection{Slow Controls Hardware}
\label{sec:fdgen-slow-cryo-qc-sc-hard}

Networking and computing systems will be purchased commercially, requiring \dword{qa}. However, the new servers are tested after delivery to confirm no damage during shipping. The new system is allowed to \textit{burn in} overnight or for a few days, 
running a diagnostics suite on a loop. This should turn up anything that escaped the manufacturer's \dword{qa} process.

The system can be shipped directly to the underground, 
\fixme{maybe ``once the system arrives on-site, it can be transported down to the 4850L''}
where an on-site
expert peforms the initial booting of systems and basic
configuration. Then the specific configuration information is pulled over
the network, after which others may log in remotely to do the final
setup, minimizing the number of people underground.

 
