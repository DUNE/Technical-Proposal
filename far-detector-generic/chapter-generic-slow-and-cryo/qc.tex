%%%%%%%%%%%%%%%%%%%%%%%%%%%%%%%%%%%%%%%%%%%%%%%%%%%%%%%%%%%%%%%%%%%%
\section{Quality Control}
\label{sec:fdgen-slow-cryo-qc}
% carmen
The purpose of  \dword{qc} is to ensure that the equipment will meet its intended function. The \dword{qc} should include tests after fabrication and after shipping and installation. A series of tests should be done by the manufacturer and the institute in charge of the device assembly. In case of a complex system, the whole system performance will be tested before shipping. 
An additional \dword{qc} should be applied in the Integration Test Facility (ITF) and underground after installation if possible. The planned tests for each subsystem are described below.  


\subsection{Purity Monitors}
\label{sec:fdgen-slow-cryo-qc-pm}

The purity monitor system will undergo a series of tests to ensure the performance of the system.  This will start with the individual purity monitors being tested in vacuum after each one is fabricated and assembled.  This test will look at the amplitude of the signal generated by the drift electrons at the cathode and the anode.  This will ensure that the photocathode is able to provide a sufficient number of photoelectrons for the measurement to be made with the required precision, and that the field gradient resistors are all working properly to maintain the drift field and hence transport the drift electrons to the anode.  A follow-up test in \lar would then be performed for each of the individual purity monitors, ensuring that the performance expected in \lar is met.  

The next step after individual testing would be to assemble the entire system and make checks of the connections along the way.  Ensuring that the connections are all proper during this time will reduce the risk of having issues once the system is finally assembled and ready for the final test.  With the full system assembled it would be placed into the shipping tube, which can also serve as a vacuum chamber, and a test made with the system in vacuum.  This would ensure that the performance seen during the individual purity monitors tests can still be achieved after making the final assembly.  If there is a \lar test facility with the height or length required for the full purity monitor system and it is available for use, then a final full system test would be made there ensuring that the full system operates in \lar and achieves the required performance.
%CFT?

\subsection{Thermometers}
\label{sec:fdgen-slow-cryo-qc-th}

\subsubsection{Static T-Gradient Thermometers}
\label{sec:fdgen-slow-cryo-qc-thst}


Three type of tests will be carried out at the production site prior to installation. First, the mechanical rigidity of the system will be tested such that swinging is minimized (< \SI{5}{cm})
to reduce the risk of touching the \dwords{apa}. This is will be done with a \SI{15}{m} stainless steel string in horizontal position anchored to two points, being able to control and measure its tension. 
Second, the quality of each sensor and its calibration should be understood. All sensors will be calibrated in the lab, as explained in Section~\ref{sec:fdgen-slow-cryo-therm}.
The main concern is the reproducibility of the results since sensors could potentially change their resistance (and hence their temperature scale)
when undergoing successive immersions in \lar{}. In this case the \dword{qc} is given by the calibration procedure itself since five independent measurements
will be done for each set of sensors. Sensors with reproducibility (\rms of those five measurements) beyond the requirements (\SI{2}{mK} for \dword{pdsp}) will be discarded.  
The calibration will serve as \dword{qc} of the readout system (similar to the final one) and of the PCB-sensor-connector assembly. Finally, the cable-connector assemblies will
be tested: sensors should measure the expected values with no additional noise introduced by the cable or connector. 

%If there is a \lar test facility with sufficient height or length to test a good portion of the system and it is available for use,
If the available \lar test facility has sufficient height or length to test a good portion of the system, an integrated system test would be made there ensuring that the system
operates in \lar and achieves the required performance. Ideally, the laboratory sensor calibration will be compared with the in situ calibration
of the dynamic T-gradient monitors by operating both dynamic and static T-gradient monitors simultaneously.   

The last phase of \dword{qc} will take place after installation. For each of the arrays being installed
the verticality of the system can be checked and the tension of the stainless steel strings can be adjusted to avoid lateral swinging towards the \dwords{apa}. 
Before soldering the wires to the flange, the entire readout chain will be tested with temporary SUBD-25 connectors. 
This will allow testing the sensor-connector assembly, the cable-connector assembly and the noise level inside the cryostat.
If any of the sensors gives a problem, it will be replaced. If the problem persists the cable will be checked and replaced if needed.


\subsubsection{Dynamic T-Gradient Thermometers}
\label{sec:fdgen-slow-cryo-qc-thdy}

Dynamic T-gradient monitor will consist of an array of high-precision temperature sensors mounted on a vertical rod. The rod can move vertically in order to perform cross-calibration of the temperature sensors in situ. Several tests are foreseen to ensure that the dynamic T-gradient monitor will deliver vertical temperature gradient measurement with precision at the level of a few \si{mK}.

\begin{itemize}
\item
Before installation, temperature sensors will be tested in LN$_2$ to verify correct operation and set the baseline calibration for each sensor with respect to the reference absolutely calibrated temperature sensor. \fixme{wrt the reference (period). Not sure I understand ``absolute'' calibration beyond calling it a baseline or reference (anne)}
\item
Warm and cold temperature readings will be taken with each sensor after mounting on the PCB board and soldering %of 
the readout cables.
\item
The sensor readout will be taken for all sensors after the cold cables are connected to electric \fdth{}s on the flange and the warm cables outside of the cryostat are connected to the temperature readout system.
\item 
The stepper motor will be tested before and after connecting to the gear and pinion system.
\item
The fully assembled rod will be connected to the pinion and gear, and moved with the stepper motor on a high platform many times to verify repeatability, possible offsets and uncertainty in the positioning. Finally, by repeating the test a large number of times, the sturdiness of the system will be verified.
\item
The full system will be tested after installation in the cryostat: both motion and sensor operation will be tested by checking readout for sensors and motion of the system vertically. \fixme{by checking sensor readout and vertical motion of the system?}
\end{itemize} 

\subsubsection{Individual Sensors}
\label{sec:fdgen-slow-cryo-qc-is}

The method to address the quality of individual precision sensors is the same as for the static T-gradient monitors.
\dword{qc} of the sensors is part of the laboratory calibration. \fixme{??} After mounting six sensors with their corresponding cables a
temporary SUBD-25 connector will be added and the six sensors tested at room temperature. All sensors should work and give values within specifications.  
If any of the sensors gives problems, it will be replaced.  If the problem persists the cable will be checked and replaced if needed.

\subsection{Gas Analyzers}
\label{sec:fdgen-slow-cryo-qc-ga}

The gas analyzers will be guaranteed by the manufacturer. However, once received, the gas analyzer modules will be checked for both \textit{zero} and the \textit{span} values using a gas-mixing instrument. This is done by using two gas cylinders with both a zero level of the gas analyzer contaminant species and a cylinder with a known percentage of the contaminant gas. This should verify the proper operation of the gas analyzers. When eventually installed in \surf, this process will be repeated before the commissioning of the cryostat . It is also important to repeat the calibrations at the manufacturer-recommended periods over the gas analyzer lifetime.
 

\subsection{Liquid Level Monitoring}
\label{sec:fdgen-slow-cryo-qc-llm}

The pressure sensors will be purchased requiring a \dword{qc} by the manufacturer.
While the capacitive and thermal sensors can be tested with a modest sample of \lar in the lab,
the pressure sensors require testing over a greater range.  But they do not
necessarily need to be tested in \lar over the whole range,  lab tests
can be done in \lar over a range of a meter or two to ensure operation
at cryogenic temperatures.  Depth tests can be accomplished with a
pressurization chamber with water.

\subsection{Cameras}
\label{sec:fdgen-slow-cryo-qc-c}

Before transport to \surf, each cryogenic camera unit (comprising the enclosure, camera, and internal thermal control and monitoring) will be checked for correct operation of all operating features, for recovery from \SI{87}{K} non-operating mode, for no leakage, and for physical defects. Lighting systems will similarly be checked for operation. Operations tests will include correct current draw, image quality, and temperature readback and control. The movable inspection camera apparatus will be inspected for physical defects, and checked for proper mechanical operation before shipping. A checklist will be completed for each unit, filed electronically in the DUNE logbook, and a hard copy sent with each unit. 

Before installation, each fixed cryogenic camera unit will be inspected for physical damage or defects and checked in CTF for correct operation of all operating features, for recovery from \SI{87}{K} non-operating mode, for no contamination of the \lar{}. Lighting systems will similarly be checked for operation. Operations tests will include correct current draw, image quality, and temperature readback and control. After installation and connection of wiring, fixed cameras and lighting will again be checked for operation. The movable inspection camera apparatus will be inspected for physical defects and, after integration with a camera unit, tested in CTF for proper mechanical and electronic operation and cleanliness, before installation or storage. A checklist will be completed for each \dword{qc} check and filed electronically in the DUNE logbook. 

\subsection{Light-emitting System}
\label{sec:fdgen-slow-cryo-qc-les}

The complete system will be checked before installation to ensure the light emission. 
Initial testing of the light-emitting system (see Figure~\ref{fig:gen-cisc-LED}) can be done by first
measuring the current when a low voltage (\SI{1}{V}) is applied, to check
that the resistive \dword{led} failover path is correct. Next, measurement
of the forward voltage with the nominal forward current applied, to
check that it is within \SI{10}{\%} of the nominal forward voltage drop of
the \dwords{led}, that all of the \dwords{led} are illuminated, and that each of the
\dwords{led} is visible over the nominal angular range. If the \dwords{led} are
infrared, a video camera with IR filter removed will be used for the
visual check. This procedure is then duplicated with the current
reversed for the \dwords{led} oriented in the opposite direction.  

These tests will be duplicated during
installation to make sure that the system has
not been damaged in transportation or installation. However, once
the \dwords{led} are in the cryostat the visual check could be difficult or impossible.


\subsection{Slow Controls Hardware}
\label{sec:fdgen-slow-cryo-qc-sc-hard}

Networking and computing systems will be commercially bought, requiring \dword{qa}. However, the new servers will be tested after delivery to confirm no damage during shipping. The new system will be let \textit{burn in} overnight or for a few days, 
\fixme{allowed to?}
running a diagnostics suite on a loop. This should turn up anything that escaped the manufacturer's \dword{qa} process.

The system can be shipped directly to the underground, 
\fixme{maybe ``once it arrives on-site, it can be transported down to the 4850L''}
where an on-site
expert can do the initial booting of systems and basic
configuration. Then the specific configuration information may be pulled over
the network, after which others may log in remotely to do the final
setup, minimizing the number of people underground.

 
