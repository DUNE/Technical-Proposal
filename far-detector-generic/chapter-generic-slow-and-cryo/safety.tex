%%%%%%%%%%%%%%%%%%%%%%%%%%%%%%%%%%%%%%%%%%%%%%%%%%%%%%%%%%%%%%%%%%%%
\section{Safety}
\label{sec:fdgen-slow-cryo-safety}

% anselmo

Several aspects related to safety should be taken into account for the different phases of the \dword{cisc} project, including R\&D, laboratory calibration and testing, mounting tests and installation. 
The initial safety planning for all phases will be reviewed and approved by safety experts as part of the initial design review, and always prior to implementation. 
All component cleaning, assembly, testing  and installation procedure documentation will include a section on safety concerns
relevant to that procedure, and will be reviewed during the appropriate pre-production reviews.

Areas of particular importance to \dword{cisc} include:
\begin{itemize}
\item Hazardous chemicals (e.g., epoxy compounds used to attach sensors to cryostat inner membrane) and cleaning compounds:
  All chemicals used will be documented at the consortium management level, with MSDS (Material safety data sheet) and approved handling and disposal plans in place.

\item Liquid and gaseous cryogens used in calibration and testing: LN$_2$ and \lar will be used for calibration and testing of most of the instrumentation devices.
  Full hazard analysis plans will be in place at the consortium management level for all module or
  module component testing involving cryogenic hazards, and these safety plans will be reviewed in the appropriate pre-production and production reviews

\item \dword{hv} safety:  Purity monitors will have a voltage of $\sim$\SI{2000}{V}. Fabrication and testing plans will demonstrate compliance with local
  \dword{hv} safety requirements at the particular institution or lab where the testing or operation is performed, and this compliance will be reviewed as part of the standard review process.

%\item UV and VUV light exposure:  Some QA and QC procedures used for module testing and qualification may require use of UV and/or VUV light sources, which can be hazardous  to unprotected operators.  Full safety plans must be in place and reviewed by consortium management prior to beginning such testing.

\item Working at heights: Some aspects of the fabrication, testing and installation of \dword{cisc} devices require working at heights. This is the 
  case of T-gradient monitors and purity monitors, which have a considerable length. Temperature sensors installed near the top cryostat membrane and cable routing for all instrumentation devices
  will require working at heights as well. The appropriate safety procedures including lift and harness training will be designed and reviewed. 
  
\item Falling objects: all work at height mentioned above have associated risks of falling objects. The corresponding safety procedures, including the proper helmet ussage 
  and the observation of a well delimited safety areas, will be included in the safety plan. 
\end{itemize}
  

  

