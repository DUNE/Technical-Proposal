%%%%%%%%%%%%%%%%%%%%%%%%%%%%%%%%%%%
\subsection{Liquid Level Monitoring}
\label{sec:fdgen-slow-cryo-liq-lev}
% john L, anselmo

The goals for the level monitoring system are basic level sensing when filling, and precise level sensing during static operations. 

For filling the \dword{detmodule} the differential pressure between the top of
the detector and known points below it can be converted to depth with
the known density of \lar.  The temperatures of \dwords{rtd} at known
heights may also be used to determine when the cold liquid has reached
each \dword{rtd}.

During operation, the purpose of liquid level monitoring is twofold:
it is required by the cryogenics system to tune the \lar flow and by
the \dword{detmodule} to guarantee that the top \dwords{gp} are always
submerged (otherwise there will be high risk of dielectric breakdown).
Two differential pressure level meters will be installed as part of
the cryogenics system, one in each side of the detector.  Those will
have a precision of SI{0.1}{\%}, which corresponds to \SI{14}{mm} at the
nominal \lar surface.  This precision is sufficient for the single
phase detector, since the plan is to kept the \lar surface at least \SI{20}{cm} above the \dwords{gp} (this is the value used for the \dword{hv}
interlock in \dword{pdsp}); thus, no additional level meters are
required for the single phase. However, in the \dual \lar
system the surface level should be controlled at the millimeter level,
which can be accomplished with capacitive monitors. Using the same
capacitive monitor system in each detector reduces design differences
and provides a redundant system for the \single.  Either system
could be used for the \dword{hv} interlock.

Table \ref{tab:fdgen-liq-lev-req} summarizes the
requirements for the liquid level monitor system.

\begin{dunetable}
[Liquid level monitor requirements]
{p{0.45\linewidth}p{0.50\linewidth}}
{tab:fdgen-liq-lev-req}
{Liquid level monitor requirements}   
Requirement & Physics Requirement Driver \\ \toprowrule
 Measurement accuracy (filling) \(\sim \SI{14}{mm}\) & Understand status of detector during filling \\ \colhline
 Measurement accuracy (operation, \dual) \(\sim \SI{1}{mm}\) & Maintain correct depth of gas phase. (Exceeds \single requirements) \\ \colhline
 Provide interlock with \dword{hv} & Prevent damage to \dword{detmodule} from \dword{hv} discharge in gas \\
\end{dunetable}


%\subsubsection{Production and Assembly}
Cryogenic pressure sensors will be purchased from commercial sources.
Installation methods and positions will be determined as part of the
cryogenics internal piping plan.  Sufficient redundancy will be designed in
to ensure that no single point of failure compromises the level measurement.

Multiple capacitive level sensors will be deployed along the top of
the fluid to be used during stable operation and checked against each
other.

%"Challenges associated with the DP liquid level in the 3X1X1 and any lessons learned for DUNE."
During operations of the \num{3}$\times$\num{1}$\times$\SI{1}{\cubic\meter} detector, the cryogenic Programmable Logic Controller (PLC) continuously checked the measurements from one level meter on the charge readout plane (\dword{crp}) in order to regulate the flow from the liquid recirculation to maintain a constant liquid level inside the cryostat. Continuous measurements from the level meters around the drift cage and the \dword{crp} illustrated the stability of the liquid level within the \SI{100}{\micro\meter} intrinsic precision of the instruments. The observation of the level was complemented by live feeds from the custom built cryogenic cameras, hereby providing qualitative feedback on the position and flatness of the surface.

In addition to the installed level meters, the liquid height in the extraction region of the \dword{crp} could be inferred by measuring the capacitance between the grid and the bottom electrode of each \dword{lem}. Averaging over all \num{12} \dwords{lem} the measured values of this capacitance typically ranged from \SI{150}{pF} with the liquid below the grid to around  \SI{350}{pF}  when the \dwords{lem} are submerged.This method offers the potential advantage of monitoring the liquid level in the \dword{crp} extraction region with a \num{50}$\times$\SI{50}{\cm^2} granularity and could be used for the \dword{crp} level adjustment in the future large-scale detectors where, due to the space constraints, placement of the level meters along the \dword{crp} perimeter would not be possible.

