%%%%%%%%%%%%%%%%%%%%%%%%%%%%%%%%%%%
\subsection{Liquid Level Monitoring}
\label{sec:fdgen-slow-cryo-liq-lev}
% john L, anselmo

The goals for the level monitoring system are basic level sensing when filling, and precise level sensing during static operations. 

For filling the detector the differential pressure between the top of
the detector and known points below it can be converted to depth with
the known density of liquid argon.  The temperatures of RTDs at known
heights may also be used to determine when the cold liquid has reached
each RTD.

During operation, the purpose of liquid level monitoring is twofold:
it is required by the cryogenics system to tune the LAr flow and by
the detector to guaranty that the top ground planes are always
submerged (otherwise there will be high risk of dielectric breakdown).
Two differential pressure level meters will be installed as part of
the cryogenics system, one in each side of the detector.  Those will
have a precision of 0.1\%, which corresponds to \SI{14}{mm} at the
nominal LAr surface.  This precision is sufficient for the single
phase detector, since the plan is to kept the LAr surface at least 20
cm above the ground planes (this is the value used for the HV
interlock in ProtoDUNE-SP); thus, no additional level meters are
required for the single phase. However, in the dual phase liquid argon
system the surface level should be controlled at the millimeter level,
which can be accomplished with capacitive monitors. Using the same
capacitive monitor system in each detector reduces design differences
and provides a redundant system for the single phase.  Either system
could be used for the HV interlock.

Table \ref{tab:fdgen-liq-lev-req} summarizes the
requirements for the liquid level monitor system.

\begin{dunetable}
[Liquid level monitor requirements]
{p{0.45\linewidth}p{0.50\linewidth}}
{tab:fdgen-liq-lev-req}
{Liquid level monitor requirements}   
Requirement & Physics Requirement Driver \\ \toprowrule
 Measurement accuracy (filling) \(\sim \SI{14}{mm}\) & Understand status of detector during filling \\ \colhline
 Measurement accuracy (operation, DP) \(\sim \SI{1}{mm}\) & Maintain correct depth of gas phase. (Exceeds SP requirements) \\ \colhline
 Provide interlock with HV & Prevent damage to detector from HV discharge in gas \\
\end{dunetable}


%\subsubsection{Production and Assembly}
Cryogenic pressure sensors will be purchased from commercial sources.
Installation methods and positions will be determined as part of the
cryogenic internal piping plan.  Sufficient redundancy will be designed in
to ensure that no single point of failure compromises the level measurement.

Multiple capacitive level sensors will be deployed along the top of
the fluid to be used during stable operation and checked against each
other.

