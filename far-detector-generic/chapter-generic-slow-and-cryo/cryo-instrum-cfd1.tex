%%%%%%%%%%%%%%%%%%%%%%%%%%%%%%%%%%%%%%%%%%%%%%%%%%%%%%%%%%%%%%%%%%%%
\section{Cryogenics Instrumentation}
\label{sec:fdsp-cryo-instr} % label used for \ref's from single phase sections.
\label{sec:fddp-cryo-instr} % label used for \ref's from dual phase sections.
\label{sec:fdgen-cryo-instr} % label used for \ref's from generic sections.
% Jim G %Carmen

Instrumentation inside the cryostat must ensure that the condition of the \dword{lar} is adequate for operation of the \dshort{tpc}.
This instrumentation includes devices to monitor the impurity level of the argon, e.g., the purity monitors, which provide high precision electron lifetime measurements,
and gas analyzers to ensure that the levels of atmospheric contamination drop below certain limits during the cryostat purging, cooling and filling.
The cryogenics system operation is monitored by temperature sensors deployed in vertical arrays and at the top and bottom of the detector, providing a 
detailed \threed temperature map which can help to predict the \dword{lar} purity across the entire cryostat. The cryogenics instrumentation also includes \lar level monitors and
a system of internal cameras to help in locating sparks in the cryostat and for overall monitoring of the cryostat interior. 
As mentioned in the Introduction, cryogenics instrumentation requires simulation work to identify the proper location for these devices inside the cryostat and
for the coherent analysis of the instrumentation data. 

